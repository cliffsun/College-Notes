\documentclass{article}
\usepackage[left=2cm, right=2cm, top=2cm, bottom=2cm]{geometry}
\usepackage{graphicx}
\usepackage{amsmath}
\usepackage{amssymb}
\usepackage{amsthm}
\usepackage{fancyhdr}
\usepackage{verbatim}
\usepackage{listings}
\usepackage{xcolor}
\usepackage{pgfplots}
\usepackage{physics}

\lstset{
    language=C++,
    basicstyle=\ttfamily\footnotesize,
    keywordstyle=\color{blue},
    commentstyle=\color{green},
    stringstyle=\color{red},
    numbers=left,
    numberstyle=\tiny\color{gray},
    frame=single,
    breaklines=true,
    captionpos=b,
}


\title{PHYS 486: Lecture \# 32}
\author{Cliff Sun}

\newtheorem{theorem}{Theorem}[section]
\newtheorem{lemma}[theorem]{Lemma}
\newtheorem{definition}[theorem]{Definition}
\newtheorem{conjecture}[theorem]{Conjecture}
\newtheorem{proposition}[theorem]{Proposition}
\newtheorem{corollary}[theorem]{Corollary}
\newtheorem{one minute paper}[theorem]{One Minute Paper}

\pagestyle{fancy}
\lhead{\textbf{\thepage}\ \ \nouppercase{\rightmark}}
\chead{PHYS 486: Lecture \# 32}
\rhead{Cliff Sun}

\begin{document}

\maketitle

\section*{Entanglement}

Consider 

\begin{equation}
    \ket{\psi} = \frac{1}{\sqrt{2}}\left( \ket{00} + \ket{11} \right)
\end{equation}
versus
\begin{equation}
    \ket{\psi} = \ket{00}, \;\; \ket{\psi} = \ket{11}
\end{equation}

Consider alice and bob, both of which can measure Bob: $Q = \pm 1$, $R = \pm 1$ and Alice: $S =\pm 1$, $T = \pm 1$. Build a quantity:

\begin{equation}
    Q \cdot S + R \cdot S + R \cdot T - Q \cdot T = \pm 2
\end{equation}

Now study 

\begin{equation}
    E(Q \cdot S + R \cdot S + R \cdot T - Q \cdot T) = \sum_{qrst}p(q,r,s,t)(qs + rs + rt - qt)
\end{equation}

We note several things

\begin{equation}
    E(A_1 + A_2) = E(A_1) + E(A_2) \;\; \text{realism}
\end{equation}

\begin{equation}
    E(A_1(B_1 + B_2)) = E(A_1B_1) + E(A_1B_2) \;\; \text{locality}
\end{equation}

Then 

\begin{equation}
    E(QS) + E(RS) + E(RT) - E(QT) \leq 2\sum_{qrst}p(q,r,s,t)
\end{equation}

\end{document}
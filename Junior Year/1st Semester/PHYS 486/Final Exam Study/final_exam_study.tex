\documentclass{article}
\usepackage[left=2cm, right=2cm, top=2cm, bottom=2cm]{geometry}
\usepackage{graphicx}
\usepackage{amsmath}
\usepackage{amssymb}
\usepackage{amsthm}
\usepackage{fancyhdr}
\usepackage{verbatim}
\usepackage{listings}
\usepackage{xcolor}
\usepackage{pgfplots}
\usepackage{physics}

\lstset{
    language=C++,
    basicstyle=\ttfamily\footnotesize,
    keywordstyle=\color{blue},
    commentstyle=\color{green},
    stringstyle=\color{red},
    numbers=left,
    numberstyle=\tiny\color{gray},
    frame=single,
    breaklines=true,
    captionpos=b,
}


\title{PHYS 486 Final Exam Cheat Sheet}
\author{Cliff Sun}

\newtheorem{theorem}{Theorem}[section]
\newtheorem{lemma}[theorem]{Lemma}
\newtheorem{definition}[theorem]{Definition}
\newtheorem{conjecture}[theorem]{Conjecture}
\newtheorem{proposition}[theorem]{Proposition}
\newtheorem{corollary}[theorem]{Corollary}
\newtheorem{one minute paper}[theorem]{One Minute Paper}

\pagestyle{fancy}
\lhead{\textbf{\thepage}\ \ \nouppercase{\rightmark}}
\chead{PHYS 486 Final Exam Cheat Sheet}
\rhead{Cliff Sun}

\begin{document}

\maketitle

\section*{Important stuff to know!}

\subsection*{Spin}

Bosons have a spin of $\pm 1$ and Fermions have a spin of $\pm 1/2$. As well

\begin{align*}
    S^{2}\ket{s,m} &= \hbar^2s(s+1)\ket{s,m}\\
    S_{z}\ket{s,m} &= \hbar m\ket{s,m} \\
    S_{\pm}\ket{s,m} &= \hbar\sqrt{s(s+1) - m(m\pm 1)}\ket{s,m\pm 1}
\end{align*}

Note that $m \in [-s, s]$. As well, 

\begin{align*}
    [S_x, S_y] &= i\hbar S_z \\
    [S_y, S_z] &= i\hbar S_x \\
    [S_z, S_x] &= i\hbar S_y
\end{align*}

\subsection*{s = $1/2$}

We will work in the eigenbasis of $S_z$. We define the spin up ($\lambda = 1$) as $$\chi_{+}\iff \begin{pmatrix}
    1 \\
    0
\end{pmatrix} = \ket{\frac{1}{2},\frac{1}{2}}$$ and $$\chi_{-} \iff \begin{pmatrix}
    0 \\
    1
\end{pmatrix} = \ket{\frac{1}{2}, -\frac{1}{2}}$$

The relations:

\begin{equation}
    S^2\chi_{\pm} = \frac{3}{4}\hbar^2\chi_{\pm} = \frac{3}{4}\hbar^2\mathbb{I}
\end{equation}

\begin{equation}
    S_z \chi_{\pm} = \pm\frac{\hbar}{2}\chi_{\pm}
\end{equation}

\begin{equation}
    S_x = \frac{\hbar}{2}\sigma_x\;\; S_y = \frac{\hbar}{2}\sigma_y \;\; S_z = \frac{\hbar}{2}\sigma_z
\end{equation}

For two particles, the notation is 

\begin{equation}
    \ket{s_1,s_2,m_1,m_2}
\end{equation}

\begin{equation}
    S_x = \frac{\hbar}{2}\left( \ket{0}\bra{1} +\ket{1}\bra{0} \right)
\end{equation}

\begin{equation}
    S_y = \frac{\hbar}{2}\left( i\ket{0}\bra{1} - i\ket{1}\bra{0} \right)
\end{equation}

\begin{equation}
    S_z = \frac{\hbar}{2}\left( \ket{0}\bra{0} -\ket{1}\bra{1} \right)
\end{equation}

\begin{equation}
    a_{-}\ket{\alpha} = \alpha\ket{\alpha}
\end{equation}
\begin{equation}
    a_{+}\ket{\alpha} = \alpha^{*}\ket{\alpha}
\end{equation}

\begin{equation}
    \hat{x} = \sqrt{\frac{\hbar}{2m\omega}}\left( a_{-} + a_{+} \right)
\end{equation}

\begin{equation}
    \hat{p} = i\sqrt{\frac{\hbar \omega m}{2}}\left( a_{+} - a_{-   } \right)
\end{equation}


\end{document}
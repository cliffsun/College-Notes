\documentclass{article}
\usepackage[left=2cm, right=2cm, top=2cm, bottom=2cm]{geometry}
\usepackage{graphicx}
\usepackage{amsmath}
\usepackage{amssymb}
\usepackage{amsthm}
\usepackage{fancyhdr}
\usepackage{verbatim}
\usepackage{listings}
\usepackage{xcolor}
\usepackage{pdfpages}
\usepackage{cancel}
\usepackage{physics}

\lstset{
    language=C++,
    basicstyle=\ttfamily\footnotesize,
    keywordstyle=\color{blue},
    commentstyle=\color{green},
    stringstyle=\color{red},
    numbers=left,
    numberstyle=\tiny\color{gray},
    frame=single,
    breaklines=true,
    captionpos=b,
}


\title{PHYS 486 Notes}
\author{Cliff Sun}

\newtheorem{theorem}{Theorem}[section]
\newtheorem{lemma}[theorem]{Lemma}
\newtheorem{definition}[theorem]{Definition}
\newtheorem{conjecture}[theorem]{Conjecture}
\newtheorem{proposition}[theorem]{Proposition}
\newtheorem{corollary}[theorem]{Corollary}
\newtheorem{one minute paper}[theorem]{One Minute Paper}

\pagestyle{fancy}
\lhead{\textbf{\thepage}\ \ \nouppercase{\rightmark}}
\chead{PHYS 486 Notes}
\rhead{Cliff Sun}

\begin{document}

\maketitle

\section*{Lecture 9: 9/23}

Recap: 

\begin{enumerate}
    \item $|\psi\rangle = \sum_k c_k |\alpha_k\rangle$ 
    \item $|\psi \rangle = \int dx \psi(x)|x\rangle$
    \item Observables $A = A^\dagger$, real eigenvalues
    \item Commutator (for hermitian operators): 
    \begin{equation}
        [\hat{A}, \hat{B}] = AB - BA 
    \end{equation}
    If $[A,B] = 0$, then a simulataneous eigenbasis exists. This means that $\exists \;|j\rangle$ such that 
    \begin{align}
        A|j\rangle &= \alpha_j |j\rangle \\
        B|j\rangle &= \beta_j |j\rangle
    \end{align}
\end{enumerate}

\subsection*{Finding eigenvalues and eigenvectors}

Let $$A = \begin{bmatrix}
    0 & 1 \\
    1 & 0
\end{bmatrix}$$ We then diagonalize this, and find the eigenvalues. In the continuous case, 

$$\hat{p}\psi = p \psi$$
$$-i\hbar \partial_x \psi = p \psi$$
$$\psi = A\exp(i\frac{p}{\hbar}x)$$

These eigenvectors are never physical since they cannot be normalizable. Finding eigenstates of energy yields
$$-\frac{\hbar^2}{2m}\partial^2_x \psi = E\psi $$
$$(\partial_x - \frac{i \sqrt{2m E}}{\hbar})(\partial_x + \frac{i \sqrt{2m E}}{\hbar})\psi = 0$$
$$\psi = A\exp(ikx) + B\exp(-ikx)$$

This means that if we perform a perfect measurement of energy. Then, we know that our particle moves left or right, and we do not know which. 

\subsection*{Basis Transformations}

Suppose that we know $\psi$ in some basis $\left(|\alpha_k\rangle \right)$. We want to know it in $|\beta_k\rangle$. We know that 

$$c_k = \langle \alpha_k | \psi \rangle$$

and vice versa. Then 

$$d_n = \langle \beta_n | \sum_k c_k | \alpha_k \rangle $$
$$1= \sum_k c_k \langle \beta_n | \alpha_k \rangle $$

Therefore, 

$$|\psi\rangle = \sum_{k,n} c_k \langle \beta_n | \alpha_k \rangle |\beta_n\rangle$$

In the continuous case, for example $$\hat{p} |p\rangle = p |p\rangle$$

Then 

$$\int dx \ket{x} \bra{x}\hat{p}\ket{p} = p\int dx \ket{x} \braket{x|p}$$

Note that $$|p\rangle = \ket{x} \bra{x}\ket{p}$$

Then on the right hand side, we expand out the momentum basis as 

$$\int dx dx' \ket{x} \bra{x}\hat{p}\ket{p}  = \int dx \ket{x} \bra{x}\hat{p}\ket{x'} \bra{x'}\ket{p}$$

Note that $\bra{x}\ket{p} = f_p(x)$, then we analyze 


\begin{align*}
    \bra{x}\hat{p}\ket{x'} &= \text{matrix element of p at x,x'}
\end{align*}

Let $$\bra{x}\hat{p}\ket{x'} = -i\hbar \partial_x \delta(x - x')$$ then we can reobtain the momentum operator. 

Back to the infinite square well. Then  

\begin{enumerate}
    \item Energy Basis: $\bra{n}\ket{\psi} = c_n$
    \item Position Basis: $\bra{x}\ket{\psi_n} = \psi_n(x) = \sqrt{\frac{2}{a}}\sin(\dots)$
    \item Momentum Basis: $\bra{p}\ket{\psi_n} = \psi_n(p)$ 
\end{enumerate}

Example: Two level system (qubit)

$$H = \hbar \omega \begin{pmatrix}
    1  & 0 \\
    0 & -1
\end{pmatrix} \iff \hbar \omega \sigma_z$$ Note that $\sigma_z$ is a Pauli Matrix operator. 

General state of the qubit is 

$$\alpha|0\rangle + \beta |1 \rangle $$

Consider an operator $$\hbar \begin{pmatrix}
    0 & 1 \\
    1 & 0
\end{pmatrix} = \hbar \sigma_x$$

\newpage 

\section*{Lecture 16: 9/25}

Midterm: Up to Lecture 9, Homework 3, and Discussion 3. We discuss the two-level system (TLS). Simple observable:

$$a\sigma_z = a \cdot \begin{pmatrix}
    1 & 0 \\
    0 & -1
\end{pmatrix}$$

The eigenvectors $|0\rangle$ and $|1\rangle$. Another observable:

$$\sigma_x = \begin{pmatrix}
    0 & 1 \\
    1 & 0
\end{pmatrix} = \ket{0}\bra{1} + \ket{1}\bra{0}$$

\subsection*{Double Well}

Consider an infinite well with a finite wall in its middle. If the particle is on the left side, then it is in $\ket{0}$, and else $\ket{1}$. Suppose we measure this particle with $\sigma_x$. Then, suppose we have a tunneling operator $\hat{T}$ such that: 

$$\hat{T}\ket{0} = \ket{1}$$
$$\hat{T}\ket{1} = \ket{0}$$

We can compute the matrix elements from here:

$$T = \sigma_x$$

\subsection*{Uncertainty Principle}

The high level: measuring $\hat{A}$ disrupts the possible measurements of the other observable $\hat{B}$. We note that 

$$\sigma_A^2 = \langle (\hat{A} - \langle  \hat{A}\rangle )^2\rangle$$

\begin{align}
    &= \langle \psi |(\hat{A} -\langle A\rangle)^2 | \psi \rangle \\
    &= \langle (\hat{A} - \langle A \rangle)\psi | (\hat{A} - \langle A \rangle ) \psi \rangle \\
    &= \langle f | f\rangle 
\end{align}


For the second operator: 

$$\sigma_B^2 = \langle g | g \rangle $$ with similar definition. From linear algebra, we know that 

$$|\langle f | g \rangle|^2 \leq \langle f|f\rangle \langle g|g\rangle $$

We get that 

$$\sigma_A^2\sigma^2_B \geq |\langle f | g \rangle|^2 $$

For any complex number $z = z' + iz''$, then 

$$|z|^2 \geq z''^2 = [\frac{1}{2i}(z - z^*)]^2$$

$$\sigma_A^2\sigma_B^2 \geq \left(\frac{1}{2i}(\langle f| g\rangle - \langle g|f\rangle)\right)^2$$

Note, 

$$\langle f |g \rangle = \langle \psi | (\hat{A} - \langle A\rangle) (\hat{B} - \langle B \rangle)|\psi\rangle$$
$$ = \langle \psi|(\hat{A}\hat{B} - \hat{A}\langle B\rangle - \hat{B}\langle A \rangle + \langle A\rangle \langle B\rangle|\psi\rangle $$

\end{document}
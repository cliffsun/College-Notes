\documentclass{article}
\usepackage[left=2cm, right=2cm, top=2cm, bottom=2cm]{geometry}
\usepackage{graphicx}
\usepackage{amsmath}
\usepackage{amssymb}
\usepackage{amsthm}
\usepackage{fancyhdr}
\usepackage{verbatim}
\usepackage{listings}
\usepackage{xcolor}
\usepackage{pgfplots}
\usepackage{physics}

\lstset{
    language=C++,
    basicstyle=\ttfamily\footnotesize,
    keywordstyle=\color{blue},
    commentstyle=\color{green},
    stringstyle=\color{red},
    numbers=left,
    numberstyle=\tiny\color{gray},
    frame=single,
    breaklines=true,
    captionpos=b,
}


\title{PHYS 486: Lecture \# 30}
\author{Cliff Sun}

\newtheorem{theorem}{Theorem}[section]
\newtheorem{lemma}[theorem]{Lemma}
\newtheorem{definition}[theorem]{Definition}
\newtheorem{conjecture}[theorem]{Conjecture}
\newtheorem{proposition}[theorem]{Proposition}
\newtheorem{corollary}[theorem]{Corollary}
\newtheorem{one minute paper}[theorem]{One Minute Paper}

\pagestyle{fancy}
\lhead{\textbf{\thepage}\ \ \nouppercase{\rightmark}}
\chead{PHYS 486: Lecture \# 30}
\rhead{Cliff Sun}

\begin{document}

\maketitle

\section*{Bosons and Fermions}

Given a wave function 

\begin{equation}
    \psi(r_1, r_2)  = \psi_a(r_1)\psi_b(r_2)
\end{equation}

This is a 2-electron system, and not 2x 1 electron systems. Then we can distinguish between the electrons because the wave function can be distinguished. In particular,
we can also distinguish them if they are in fully separate potentials. But if we cannot separate the wave function, then one cannot be a sure that measuring a particle in location $r_1$ is particle a or particle b. Therefore, 

\begin{equation}
    \psi_{\pm}(r_1,r_2) = A \cdot (\psi_a(r_1)\psi_b(r_2) \pm \psi_b(r_1)\psi_a(r_2))
\end{equation}

Note that $+$ corresponds Bosons and $-$ corresponds to Fermions. Then a particle exchange 

\begin{equation}
    \psi_+(r_1, r_2) = \psi_+(r_2, r_1)
\end{equation}

Is obeyed with Bosons. With fermions, 

\begin{equation}
    \psi_{-}(r_1, r_2) = -\psi_{-}(r_2,r_1)
\end{equation}

Let two particles in the same s.s., then $\psi_a = \psi_b$. Then 

\begin{equation}
    \psi_+ = 2A\psi_a(r_1)\psi_a(r_2)
\end{equation}
\begin{equation}
    \psi_{-} = 0
\end{equation}
This is the Pauli exclusion principle. This means that Fermions cannot be in the same quantum state, because the wave function is not normalizable and therefore cannot be physical. 

\subsection*{Including Spin:}

\begin{equation}
    \psi = \psi(r)\chi
\end{equation}

For two electrons, or other Fermions, 

\begin{equation}
    \psi(r_1,r_2)\chi(1,2) = \psi(r_2,r_1)\chi(2,1)
\end{equation}

For a singlet: 

\begin{equation}
    \chi_{-} = \frac{1}{\sqrt{2}}\left( \ket{\uparrow \downarrow} - \ket{\downarrow \uparrow} \right) = -\frac{1}{\sqrt{2}}\left( \ket{\downarrow \uparrow} - \ket{\uparrow \downarrow} \right) 
\end{equation}



\end{document}
\documentclass{article}
\usepackage[left=2cm, right=2cm, top=2cm, bottom=2cm]{geometry}
\usepackage{graphicx}
\usepackage{amsmath}
\usepackage{amssymb}
\usepackage{amsthm}
\usepackage{fancyhdr}
\usepackage{verbatim}
\usepackage{listings}
\usepackage{xcolor}
\usepackage{pgfplots}
\usepackage{physics}

\lstset{
    language=C++,
    basicstyle=\ttfamily\footnotesize,
    keywordstyle=\color{blue},
    commentstyle=\color{green},
    stringstyle=\color{red},
    numbers=left,
    numberstyle=\tiny\color{gray},
    frame=single,
    breaklines=true,
    captionpos=b,
}


\title{PHYS 486: Lecture \# 23}
\author{Cliff Sun}

\newtheorem{theorem}{Theorem}[section]
\newtheorem{lemma}[theorem]{Lemma}
\newtheorem{definition}[theorem]{Definition}
\newtheorem{conjecture}[theorem]{Conjecture}
\newtheorem{proposition}[theorem]{Proposition}
\newtheorem{corollary}[theorem]{Corollary}
\newtheorem{one minute paper}[theorem]{One Minute Paper}

\pagestyle{fancy}
\lhead{\textbf{\thepage}\ \ \nouppercase{\rightmark}}
\chead{PHYS 486: Lecture \# 23}
\rhead{Cliff Sun}

\begin{document}

\maketitle

\section*{Solving the radial equation}

Let 

\begin{equation}
    R = \frac{u}{r}
\end{equation}
Then 
\begin{equation}
    \frac{dR}{dr} = (r\frac{du}{dr} - u)\frac{1}{r^2}
\end{equation}
Then, our radial equation becomes 
\begin{equation}
    -\frac{\hbar}{2m}\frac{d^2u}{dr^2} + [V + \frac{\hbar^2}{2m}\frac{l(l+1)}{r^2}]u = E_n
\end{equation}

As a toy system, we examine the infinite well in 3-D in spherical coordinates. 

\begin{equation}
    V(r) = \begin{cases}
        0 & r \leq a \\
        \infty & r > a
    \end{cases}
\end{equation}

Inside the well, 

\begin{equation}
    \frac{d^2u}{dr^2} = [\frac{l(l+1)}{r^2} - k^2]u
\end{equation}

For $l=0$, then 

\begin{equation}
    u(r) = A\sin(kr) + B\cos(kr)
\end{equation}

Since the WF $= u/r$, then $B=0$. with BC: $\sin(ka) = 0$, then $ka = N\pi$. Thus, 

\begin{equation}
    E = \frac{N^2\pi^2\hbar^2}{2ma^2}
\end{equation}

For $l>0$, generally:

\begin{equation}
    u(r) = Arj_l(kr) + Bru_l(kr)
\end{equation}

\begin{equation}
    j_l(x) = (-x)^l(\frac{1}{x}\frac{d}{dx})^l\frac{\sin(x)}{x}
\end{equation}

Then, 

\begin{equation}
    R(r) = Aj_l(kr) \land j_l(ka) = 0
\end{equation}

The wave function is then 

\begin{equation}
    \psi = A R(r) Y(\phi, \theta)
\end{equation}

\newpage

\section*{The Hydrogen Atom}

Given a Hydrogen Atom with an immobile proton and an electron that is free to move around. Then 

\begin{equation}
    V(r) = -\frac{e^2}{4\pi \epsilon_0}\frac{1}{r}
\end{equation}

Thus, the radial equation is 

\begin{equation}
    -\frac{\hbar^2}{2m}\frac{d^2u}{dr^2} + \left[ -\frac{e^2}{4\pi \epsilon_0}\frac{1}{r} + \frac{\hbar^2}{2m_e}\frac{l(l+1)}{r^2} \right]u = E_u
\end{equation}

\end{document}
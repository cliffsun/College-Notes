\documentclass{article}
\usepackage[left=2cm, right=2cm, top=2cm, bottom=2cm]{geometry}
\usepackage{graphicx}
\usepackage{amsmath}
\usepackage{amssymb}
\usepackage{amsthm}
\usepackage{fancyhdr}
\usepackage{verbatim}
\usepackage{listings}
\usepackage{xcolor}
\usepackage{pgfplots}
\usepackage{physics}

\lstset{
    language=C++,
    basicstyle=\ttfamily\footnotesize,
    keywordstyle=\color{blue},
    commentstyle=\color{green},
    stringstyle=\color{red},
    numbers=left,
    numberstyle=\tiny\color{gray},
    frame=single,
    breaklines=true,
    captionpos=b,
}


\title{PHYS 486: Lecture \# 31}
\author{Cliff Sun}

\newtheorem{theorem}{Theorem}[section]
\newtheorem{lemma}[theorem]{Lemma}
\newtheorem{definition}[theorem]{Definition}
\newtheorem{conjecture}[theorem]{Conjecture}
\newtheorem{proposition}[theorem]{Proposition}
\newtheorem{corollary}[theorem]{Corollary}
\newtheorem{one minute paper}[theorem]{One Minute Paper}

\pagestyle{fancy}
\lhead{\textbf{\thepage}\ \ \nouppercase{\rightmark}}
\chead{PHYS 486: Lecture \# 31}
\rhead{Cliff Sun}

\begin{document}

\maketitle

\section*{Exchange Force}

For distinguishable particles, the wave function is 

\begin{equation}
    \Psi = \psi_a(r_1)\psi_b(r_2)
\end{equation}

But for indistinguishable particles, the wave function is 

\begin{equation}
    \Psi_{\pm} = A(\psi_a(r_1)\psi_b(r_2) \pm \psi_a(r_2)\psi_b(r_1))
\end{equation}

Where $+$ is a boson and $-$ is a fermion. We compute

\begin{equation}
    \langle (x_1 - x_2)^2 \rangle = \langle x_1\rangle ^2 + \langle x_2\rangle ^2  + \langle x_1x_2\rangle 
\end{equation}

For a boson, this is 

\begin{equation}
    \langle (\Delta x)^2 \rangle - 2|\langle x \rangle_{ab}|^2
\end{equation}

For a fermion, this is 

\begin{equation}
    \langle (\Delta x)^2 \rangle + 2|\langle x \rangle_{ab}|^2
\end{equation}

Where 

\begin{equation}
    |\langle x \rangle_{ab}| = \int x \psi_a \psi_b^* dx
\end{equation}

Note that this term only appears if there is an overlap between $\psi_a$ and $\psi_b$. Therefore, for a boson, the particles are a bit closer than distinguishable particles and vice versa for fermions. 

\end{document}
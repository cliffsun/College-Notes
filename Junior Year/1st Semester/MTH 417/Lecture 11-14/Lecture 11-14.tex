\documentclass{article}
\usepackage[left=2cm, right=2cm, top=2cm, bottom=2cm]{geometry}
\usepackage{graphicx}
\usepackage{amsmath}
\usepackage{amssymb}
\usepackage{amsthm}
\usepackage{fancyhdr}
\usepackage{verbatim}
\usepackage{listings}
\usepackage{xcolor}
\usepackage{pdfpages}
\usepackage{cancel}

\lstset{
    language=C++,
    basicstyle=\ttfamily\footnotesize,
    keywordstyle=\color{blue},
    commentstyle=\color{green},
    stringstyle=\color{red},
    numbers=left,
    numberstyle=\tiny\color{gray},
    frame=single,
    breaklines=true,
    captionpos=b,
}


\title{MTH 417 Notes}
\author{Cliff Sun}

\newtheorem{theorem}{Theorem}[section]
\newtheorem{lemma}[theorem]{Lemma}
\newtheorem{definition}[theorem]{Definition}
\newtheorem{conjecture}[theorem]{Conjecture}
\newtheorem{proposition}[theorem]{Proposition}
\newtheorem{corollary}[theorem]{Corollary}
\newtheorem{one minute paper}[theorem]{One Minute Paper}

\pagestyle{fancy}
\lhead{\textbf{\thepage}\ \ \nouppercase{\rightmark}}
\chead{MTH 417 Notes}
\rhead{Cliff Sun}

\begin{document}

\maketitle

\section*{Lecture 11: 9/17}

Recall: the \textbf{SUBGROUP LATTICE} is the set of subgroups of G ordered by $\leq$. 

\subsection*{Cyclic Groups}

Let G be a group, then the cyclic subgroup of G generated by a is 

\begin{equation}
    \langle a \rangle = \{a^k \;\;| \;\;k \in \mathbb{Z}\}
\end{equation}

If $a$ can generate all of G, then G is \textit{cyclic}. Note, that $a^k$ denotes concatenation. 

\subsubsection*{Example}

If $G = \mathbb{Z}_n$, then

\begin{equation}
    \langle [1]_n\rangle = \{[0], \pm [1], \dots\}
\end{equation}

\begin{definition}
    The order of G is defined as $o(a) = |\langle a \rangle |$. This is just saying that the size of the set generated by a is its order. 
\end{definition}

\begin{proposition}
    \begin{enumerate}
        \item If $o(a) = \infty$, then $\langle a \rangle$ is isomorphic to $\mathbb{Z}$. 
        \item if $o(a) = n$, then $\langle a \rangle$ is isomorphic to $\mathbb{Z}_n$. 
    \end{enumerate}
\end{proposition}

When defining functions for isomorphisms, first need to make sure that it is well defined, then check if it is a homorphism. Note, that $\mathbb{Z}^\times$ just means all elements that have multiplicative inverses, this is just the group defined under the multiplication operation, which means that every element must have an inverse. 

Let 

\begin{equation}
    S^1 = \{z \in \mathbb{Z} \;| |z| = 1\}
\end{equation}

\newpage

\section*{Lecture 12: 9/19}

Recall, the circle group is such that 

\begin{equation}
    S^1 = \{z \in \mathbb{Z} \;| |z| = 1\}
\end{equation}

A product in $S'$ is a composition of rotations since $z \sim e^{i\theta}$. We can then define a function 

\begin{equation}
    f(\theta) = \exp(i\theta)
\end{equation}

This function is clearly on-to, but is not injective. 

\begin{proposition}
    Suppose $\theta$ is an angle, then we study 
    \begin{equation}
        \langle e^{i\theta} \rangle \leq S^{1}
    \end{equation}
    This is a finite cyclic group when $\theta$ is a rational multiple of $2\pi$, and is infinite if not. 
\end{proposition}

If $\theta = 2\pi \cdot (a/b)$, then $\langle \exp (i\theta) \rangle$ is ismorphic to $\mathbb{Z}_b$. 

\begin{definition}
    $D_n$ is the group of symmetries of a regular $n$-gon. We claim that 
    \begin{equation}
        D_6 = \{e, \rho, \rho^2, \dots, \tau\rho, \dots\}
    \end{equation}
    Such that $\rho^6 = e$, $\tau^2 = e$, and $\tau\rho^k = \rho^{-k}$. 
\end{definition}

\begin{proposition}
    Let $H \leq \mathbb{Z}$, then let $H = \langle d \rangle = \{dk | k \in \mathbb{Z}\} = d\mathbb{Z}$. 
\end{proposition}

\newpage

\section*{Lecture 13: 9/22}
Recall:
\begin{proposition}
    Let $H \leq \mathbb{Z}$, then let $H = \langle d \rangle = \{dk | k \in \mathbb{Z}\} = d\mathbb{Z}$. 
\end{proposition}

\begin{proposition}
    Let $H \leq \mathbb{Z}_n$, then either $H = \{[0]\}$ or $\exists d \in \mathbb{N}$ such that $H = \langle [d] \rangle$. Note, $H$ is isomorphic to $\mathbb{Z}_{n/d}$
\end{proposition}

\begin{lemma}
    Let $n \in \mathbb{N}$, and let $b \in \mathbb{Z}/\{0\}$, $d = \gcd(n,b)$, then in $\mathbb{Z}_n$, we have 
    \begin{enumerate}
        \item $\langle [b] \rangle = \langle [d] \rangle  \leq \mathbb{Z}_n$ 
        \item $o([b]) = n/d$
    \end{enumerate}
\end{lemma}

We prove this lemma 
\begin{proof}
    Can find $s,t \in Z$ such that $d = sb + tn$, then 
    \begin{equation}
        [d] = [sb] = s[b] \in \langle [b] \rangle
    \end{equation}
    This means that $\langle [d] \rangle \leq \langle [b] \rangle$. But also d divides b, which means that $b = dm$, this means that $\langle [b] \rangle \leq \langle [d] \rangle$. 
\end{proof}

\begin{definition}
    A function $f: G \rightarrow H$ is a homomorphism if 
    \begin{equation}
        f(g_1g_2) = f(g_1)f(g_2)
    \end{equation}
\end{definition}

If $G \leq H$, then $f: G \rightarrow$ is a homomorphism. 

\newpage 

\section*{Lecture 14: 9/26}

If $G \rightarrow H$ is a homomorphism, then 

$$f(g_1g_2) = f(g_1)f(g_2)$$

If $f$ and $g$ are group homomorphisms, then 
$$f\circ g$$

is also a homomorphism. Let $f: G \rightarrow H$, let $A \subseteq G$ and $B \subseteq H$ be subsets. Then 

$$f(A) = \{f(a) | a \in A\} \subseteq H$$

be the image of $A$ under $f$. Similarly

$$f^{-1}(B) = \{g \in G | f(g) \in B\} \subseteq G$$

be the preimage of $B$. Let $f: G \rightarrow H$, and $A,B$ defined similarly. Then if $A \leq G$ (subgroup) then $f(A) \leq H$. Similarly, if $B \leq H$, then $f^{-1}(B) \leq G$. 

We define the kernel of $f$ to be 

$$f^{-1}(e) \leq G$$

This means that which elements of $G$ map to $e$. Note, this kernel is trivial if $f$ is injective. However, such non-trivial kernels can happen, for example:

\begin{equation}
    f^{-1}([0]_n) = nk \;\; k \in \mathbb{Z}
\end{equation}

\begin{definition}
    A subgroup $N \leq G$ is a normal subgroup if for all $g \in G$
    $$gNg^{-1} = N$$
    In other words, $gng^{-1}$ is also called the \textit{conjugate} of n by $g$. 
\end{definition}

\begin{proposition}
    If $f: G \rightarrow H$ is a homomorphism. Then $\ker(f)$ is a normal subgroup of $G$. 
\end{proposition}

\begin{proof}
    Let $g \in G$, then let 
    $$y = gxg^{-1}$$

    Then apply
    $$f(y)= f(gxg^{-1}) = f(g)f(x)f(g)^{-1} = e$$
    Therefore, $y \in G$ is in $\ker(f)$.
\end{proof}

\begin{proposition}
    Let f is injective. Then $$\iff \ker(f) = \{e\}$$
\end{proposition}

\begin{proof}
    $\implies$, f is injective, then let $x \in \ker(f)$. Then if $f(e) = f(x) \implies x = e$. We next prove the opposite direction. Let $\ker(f) = \{e\}$. Suppose $f(x) = f(y)$. Then compute 
    $$f(x^{-1}y) = f(x)^{-1}f(y) = e$$
    Then $x^{-1}y \in \ker(f) \implies x =y$. 
\end{proof}

\end{document}
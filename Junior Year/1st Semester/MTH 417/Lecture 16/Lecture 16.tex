\documentclass{article}
\usepackage[left=2cm, right=2cm, top=2cm, bottom=2cm]{geometry}
\usepackage{graphicx}
\usepackage{amsmath}
\usepackage{amssymb}
\usepackage{amsthm}
\usepackage{fancyhdr}
\usepackage{verbatim}
\usepackage{listings}
\usepackage{xcolor}
\usepackage{pgfplots}
\usepackage{physics}

\lstset{
    language=C++,
    basicstyle=\ttfamily\footnotesize,
    keywordstyle=\color{blue},
    commentstyle=\color{green},
    stringstyle=\color{red},
    numbers=left,
    numberstyle=\tiny\color{gray},
    frame=single,
    breaklines=true,
    captionpos=b,
}


\title{MTH 417: Lecture \# 16}
\author{Cliff Sun}

\newtheorem{theorem}{Theorem}[section]
\newtheorem{lemma}[theorem]{Lemma}
\newtheorem{definition}[theorem]{Definition}
\newtheorem{conjecture}[theorem]{Conjecture}
\newtheorem{proposition}[theorem]{Proposition}
\newtheorem{corollary}[theorem]{Corollary}
\newtheorem{one minute paper}[theorem]{One Minute Paper}

\pagestyle{fancy}
\lhead{\textbf{\thepage}\ \ \nouppercase{\rightmark}}
\chead{MTH 417: Lecture \# 16}
\rhead{Cliff Sun}

\begin{document}

\maketitle

Recall Lagrange's theorem. 

\begin{corollary}
    If $p$ is prime, then $|G| = p$, then $G$ is cyclic of order of $p$. The only subgroups of $G$ is $\{e\}$ and $G$
\end{corollary}

\begin{proof}
    Let $H \leq G$, then $|H|/|G| = p$. So $|H| = 1$ or $p$. 
\end{proof}

More generally, any finite $G$, $g \in G$, the order of $g$ is defined as

\begin{equation}
    o(g) = |\langle g \rangle | \bigg| |G|
\end{equation}

\begin{corollary}
    Let $n \in \mathbb{N}$, $a \in \mathbb{Z}$. Then 
    \begin{equation}
        a^{\varphi(n)} \equiv 1 \; \mod n
    \end{equation}
    Recall that 
    \begin{equation}
        \varphi(n) = \big|\mathbb{Z}^{\times}_n \big|
    \end{equation}
\end{corollary}

\begin{proof}
    Note that $[a]_n \in \mathbb{Z}^{\times}_n$, so 
    \begin{equation}
        o([a]_n) \bigg| \varphi(n)
    \end{equation}
    Therefore, 
    \begin{equation}
        \varphi(n) = o([a]_n)k \;\; k \in \mathbb{N}
    \end{equation}
    However,
    \begin{equation}
        [a_n]^{o([a])} = [1] \in \mathbb{Z}_n^{\times}
    \end{equation}
    The rest of this proof follows trivially. 
\end{proof}

Recall, $N \leq G$ is normal if $\forall g \in G$, then $gNg^{-1} = N$. We can check 

\begin{center}
    $N$ is normal if $G$ $\iff gN = Ng$
\end{center}



\end{document}
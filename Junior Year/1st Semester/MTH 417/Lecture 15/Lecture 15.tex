\documentclass{article}
\usepackage[left=2cm, right=2cm, top=2cm, bottom=2cm]{geometry}
\usepackage{graphicx}
\usepackage{amsmath}
\usepackage{amssymb}
\usepackage{amsthm}
\usepackage{fancyhdr}
\usepackage{verbatim}
\usepackage{listings}
\usepackage{xcolor}
\usepackage{pgfplots}
\usepackage{physics}

\lstset{
    language=C++,
    basicstyle=\ttfamily\footnotesize,
    keywordstyle=\color{blue},
    commentstyle=\color{green},
    stringstyle=\color{red},
    numbers=left,
    numberstyle=\tiny\color{gray},
    frame=single,
    breaklines=true,
    captionpos=b,
}


\title{MTH 417: Lecture 15}
\author{Cliff Sun}

\newtheorem{theorem}{Theorem}[section]
\newtheorem{lemma}[theorem]{Lemma}
\newtheorem{definition}[theorem]{Definition}
\newtheorem{conjecture}[theorem]{Conjecture}
\newtheorem{proposition}[theorem]{Proposition}
\newtheorem{corollary}[theorem]{Corollary}
\newtheorem{one minute paper}[theorem]{One Minute Paper}

\pagestyle{fancy}
\lhead{\textbf{\thepage}\ \ \nouppercase{\rightmark}}
\chead{MTH 417: Lecture 15}
\rhead{Cliff Sun}

\begin{document}

\maketitle

\begin{equation}
    S_{3} = \{e, (12), (13), (23), (123), (132)\}
\end{equation}

All elements in this set partition $S_3$. 

\begin{definition}
    Let $H \leq G$ be a subgroup. Let $g \in G$. A subset of the form 
    \begin{equation}
        gH = \{gh | h \in H\}
    \end{equation}
    is called a left coset of $H$ in $G$. The right coset is defined as 
    \begin{equation}
        Hg = \{hg | h \in H\}
    \end{equation}
    Note, this is also a function of $g$ where $g$ is a fixed element of $G$. 
\end{definition}

In general, the left and right cosets are different. 

\begin{proposition}
    Let $H \leq G$, and $a,b \in G$. The following statements are all equivalent (TFAE):
    \begin{enumerate}
        \item $a \in bH$
        \item $b \in aH$
        \item $b^{-1}a \in H$
        \item $a^{-1}b \in H$
        \item $aH = bH$
    \end{enumerate}
\end{proposition}

\begin{proof}
    Prove that $1 \implies 2$. Suppose $a \in bH$. This means that $a = bh \iff ah^{-1} = b \in aH$. 
\end{proof}

\begin{proof}
    Prove that $2 \implies 3$. Suppose $b = ah \iff h^{-1} = b^{-1}a \in H$.
\end{proof}

\begin{proof}
    Prove that $3 \implies 4$. Suppose $b^{-1}a = h \iff h^{-1} = a^{-1}b \in H$.
\end{proof}

\begin{proof}
    Prove that $4 \implies 5$. Suppose $a^{-1}b  = h$. Then let $a\tilde{h} \in aH$, then $a\tilde{h} = bh^{-1}\tilde{h} \in bH$. A similar exercise for $bH \subseteq aH$.  
\end{proof}

\begin{proof}
    Prove that $5 \implies 1$. Suppose $aH = bH$. In particular, $ae \in aH \iff ae \in bH$.  
\end{proof}

\begin{proposition}
    $H \leq G$, $a, b\in G$. Then, these are consequences:
    \begin{enumerate}
        \item Either $aH = bH$ or $aH \cap bH = \emptyset$
        \item The function $f: aH \rightarrow bH$ such that $f(x) = ba^{-1}x$. This is a bijection.
        \item $G = \cup \text{ Left Cosets}$
    \end{enumerate}
\end{proposition}

\begin{proof}
    We prove $3$. Note that left coset $\subseteq G$. Then let $g \in G$, then in particular, $g \in $ some left coset because every left coset has a the element $ge = g$. 
\end{proof}

\begin{proof}
    We prove $2$. Assume that $aH \cup bH \neq \emptyset$. Then $\exists x \in aH \cup bH$. Then in particular, $x \in aH$ means that $aH = xH$. Similarly, $bH = xH$. The result is trivial. 
\end{proof}

\begin{theorem}
    \underline{Lagrange's Theorem}: Let $G$ be a finite group. Let $H \leq G$. Then the order of $H$ divides the order of $G$ and the quotient is the number of left cosets of $H$ in $G$. 
\end{theorem}

\begin{definition}
    For any $H \leq G$, the \# of left cosets of H in G is the \underline{index} of H in G. Denoted $[G:H]$ 
\end{definition}

Then Lagrange tells us that if $|G| \leq \infty$, then $|G| = |H|[H:G]$

\begin{proof}
    Suppose $a_1,\dots,a_k$ be representatives of the left cosets of $H$ in $G$. Then 
    \begin{equation}
        G = \bigcup_i a_iH \
    \end{equation}
    \begin{equation}
        \implies |G|= \sum_i |a_i H|
    \end{equation}
    because they are disjoint.
    \begin{equation}
        \sum_i |a_iH| = \sum_i |H|
    \end{equation}
    \begin{equation}
        |G| = k|H|
    \end{equation}
    This means that the order of $H$ divides $G$ and the integer is the number of left cosets.
\end{proof}

\end{document}
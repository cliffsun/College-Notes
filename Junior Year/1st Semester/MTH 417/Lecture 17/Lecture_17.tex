\documentclass{article}
\usepackage[left=2cm, right=2cm, top=2cm, bottom=2cm]{geometry}
\usepackage{graphicx}
\usepackage{amsmath}
\usepackage{amssymb}
\usepackage{amsthm}
\usepackage{fancyhdr}
\usepackage{verbatim}
\usepackage{listings}
\usepackage{xcolor}
\usepackage{pgfplots}
\usepackage{physics}

\lstset{
    language=C++,
    basicstyle=\ttfamily\footnotesize,
    keywordstyle=\color{blue},
    commentstyle=\color{green},
    stringstyle=\color{red},
    numbers=left,
    numberstyle=\tiny\color{gray},
    frame=single,
    breaklines=true,
    captionpos=b,
}


\title{MTH 417: Lecture \# 17}
\author{Cliff Sun}

\newtheorem{theorem}{Theorem}[section]
\newtheorem{lemma}[theorem]{Lemma}
\newtheorem{definition}[theorem]{Definition}
\newtheorem{conjecture}[theorem]{Conjecture}
\newtheorem{proposition}[theorem]{Proposition}
\newtheorem{corollary}[theorem]{Corollary}
\newtheorem{one minute paper}[theorem]{One Minute Paper}

\pagestyle{fancy}
\lhead{\textbf{\thepage}\ \ \nouppercase{\rightmark}}
\chead{MTH 417: Lecture \# 17}
\rhead{Cliff Sun}

\begin{document}

\maketitle

Can we give the set of left cosets of $H$ in $G$

\begin{equation}
    G/H = \{aH | a \in H\}
\end{equation}

a group structure with respect to the operation in $G$. We proceed with the equivalence class. For $a \in X$, the equivalence class $\sim$

\begin{equation}
    [a] = \{b \in X | b \sim a\}
\end{equation}

\begin{proposition}
    Suppose $\sim$ be an equivalence relation on $X$. Then, for $x,y \in X$, then 
    \begin{equation}
        x \sim y \iff [x] = [y]
    \end{equation}
    This is proving the equivalence class from the ground up.
\end{proposition}

\begin{corollary}
    Let $\sim$ be an equivalence relation on $X$. Then for $x,y \in X$, either $[x] = [y]$ or $[x] \cap [y] = \emptyset$. 
\end{corollary}

\begin{proof}
    If $[x] \cap [y] \neq \emptyset$, then $z \in [x] \cap [y]$. If $z \in [x]$, then $z \sim x \sim y$. Therefore, $[x] = [y]$. 
\end{proof}

\begin{theorem}
    Let $N$ be a normal subgroup relative to $G$. Then there exists a unique group structure on the set of cosets $G/N$
    which means the quotient function:
    \begin{equation}
        \pi: X \rightarrow X/\sim = \{[x]\}
    \end{equation}  
    Into a group homomorphism. 
\end{theorem}

\end{document}
\documentclass{article}
\usepackage[left=2cm, right=2cm, top=2cm, bottom=2cm]{geometry}
\usepackage{graphicx}
\usepackage{amsmath}
\usepackage{amssymb}
\usepackage{amsthm}
\usepackage{fancyhdr}
\usepackage{verbatim}
\usepackage{listings}
\usepackage{xcolor}
\usepackage{pgfplots}
\usepackage{physics}

\lstset{
    language=C++,
    basicstyle=\ttfamily\footnotesize,
    keywordstyle=\color{blue},
    commentstyle=\color{green},
    stringstyle=\color{red},
    numbers=left,
    numberstyle=\tiny\color{gray},
    frame=single,
    breaklines=true,
    captionpos=b,
}


\title{PHYS 436: Lecture \# 35}
\author{Cliff Sun}

\newtheorem{theorem}{Theorem}[section]
\newtheorem{lemma}[theorem]{Lemma}
\newtheorem{definition}[theorem]{Definition}
\newtheorem{conjecture}[theorem]{Conjecture}
\newtheorem{proposition}[theorem]{Proposition}
\newtheorem{corollary}[theorem]{Corollary}
\newtheorem{one minute paper}[theorem]{One Minute Paper}

\pagestyle{fancy}
\lhead{\textbf{\thepage}\ \ \nouppercase{\rightmark}}
\chead{PHYS 436: Lecture \# 35}
\rhead{Cliff Sun}

\begin{document}

\maketitle

\section*{Lorentz Transformation of E\&M Potentials}

The retarded E\&M potential forms a relativisitic 4-potential, 

\begin{equation}
    A^{\mu} = \begin{pmatrix}
        V/c \\
        \textbf{A}
    \end{pmatrix}
\end{equation}

There, they also follow the same Lorentz Transformations. Moreover, since 

\begin{equation}
    \square^{2}V/c = -\mu_0 \rho c = -\frac{\rho}{\epsilon_0 c}
\end{equation}
\begin{equation}
    \square^{2}A = -\mu_0 J
\end{equation}

It follows that 

\begin{equation}
    J^\mu  = \begin{pmatrix}
        \rho c \\
        \textbf{J}
    \end{pmatrix} = A^\mu
\end{equation}

The Lorenz gauge can therefore be written as 

\begin{equation}
    \nabla \cdot A + \frac{1}{c^2}\partial_t V = \partial_\mu A^\mu =0 
\end{equation}


\end{document}
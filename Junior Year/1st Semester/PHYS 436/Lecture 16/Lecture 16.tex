\documentclass{article}
\usepackage[left=2cm, right=2cm, top=2cm, bottom=2cm]{geometry}
\usepackage{graphicx}
\usepackage{amsmath}
\usepackage{amssymb}
\usepackage{amsthm}
\usepackage{fancyhdr}
\usepackage{verbatim}
\usepackage{listings}
\usepackage{xcolor}
\usepackage{pgfplots}
\usepackage{physics}

\lstset{
    language=C++,
    basicstyle=\ttfamily\footnotesize,
    keywordstyle=\color{blue},
    commentstyle=\color{green},
    stringstyle=\color{red},
    numbers=left,
    numberstyle=\tiny\color{gray},
    frame=single,
    breaklines=true,
    captionpos=b,
}


\title{PHYS 436: Lecture 16}
\author{Cliff Sun}

\newtheorem{theorem}{Theorem}[section]
\newtheorem{lemma}[theorem]{Lemma}
\newtheorem{definition}[theorem]{Definition}
\newtheorem{conjecture}[theorem]{Conjecture}
\newtheorem{proposition}[theorem]{Proposition}
\newtheorem{corollary}[theorem]{Corollary}
\newtheorem{one minute paper}[theorem]{One Minute Paper}

\pagestyle{fancy}
\lhead{\textbf{\thepage}\ \ \nouppercase{\rightmark}}
\chead{PHYS 436: Lecture 16}
\rhead{Cliff Sun}

\begin{document}

\maketitle

\section*{E\&M Waves in Conductors}

Recall that 

\begin{equation}
    \vec{E} = \hat{x}E_0 \exp(-kz)\cos(kz - \omega t + \delta_E)
\end{equation}
\begin{equation}
    \vec{B} = \hat{y}B_0\exp(-kz)\cos(kz - \omega t + \delta_E + \phi)
\end{equation}

Reflection inside a conductor is same as a linear medium, with the boundary conditions:

\begin{align*}
    E^{\parallel}_1 &= E^{\parallel}_2 \\
    \frac{1}{\mu_1}B^{\parallel}_1 &= \frac{1}{\mu_2}B^{\parallel}_2
\end{align*}

But the $\tilde{k}$ value changes. 

\end{document}
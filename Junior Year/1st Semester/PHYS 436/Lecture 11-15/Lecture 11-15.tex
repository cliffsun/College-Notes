\documentclass{article}
\usepackage[left=2cm, right=2cm, top=2cm, bottom=2cm]{geometry}
\usepackage{graphicx}
\usepackage{amsmath}
\usepackage{amssymb}
\usepackage{amsthm}
\usepackage{fancyhdr}
\usepackage{verbatim}
\usepackage{listings}
\usepackage{xcolor}
\usepackage{pdfpages}
\usepackage{cancel}

\lstset{
    language=C++,
    basicstyle=\ttfamily\footnotesize,
    keywordstyle=\color{blue},
    commentstyle=\color{green},
    stringstyle=\color{red},
    numbers=left,
    numberstyle=\tiny\color{gray},
    frame=single,
    breaklines=true,
    captionpos=b,
}


\title{PHYS 436 Lecture Notes}
\author{Cliff Sun}

\newtheorem{theorem}{Theorem}[section]
\newtheorem{lemma}[theorem]{Lemma}
\newtheorem{definition}[theorem]{Definition}
\newtheorem{conjecture}[theorem]{Conjecture}
\newtheorem{proposition}[theorem]{Proposition}
\newtheorem{corollary}[theorem]{Corollary}
\newtheorem{one minute paper}[theorem]{One Minute Paper}

\pagestyle{fancy}
\lhead{\textbf{\thepage}\ \ \nouppercase{\rightmark}}
\chead{PHYS 436 Lecture Notes}
\rhead{Cliff Sun}

\begin{document}

\maketitle

\section*{Lecture 11: 9/17}

What does a monochromatic E\&M plane wave carry? 

\begin{enumerate}
    \item Energy Density $u$
    \item E\&M Wave carries energy, i.e. the enegry current density $S = 1/\mu_0 (E \times B) = \hat{k}cu$
    \item Momentum
\end{enumerate}

Note, light intensity is the magnitude of the averaged energy current density. 

\subsection*{Radiation pressure for perfect absorption}

\begin{enumerate}
    \item Consequence of EM momentum
    \item For cross section A and length of $c \Delta t$. The average momentum is 
    \begin{equation}
        \Delta p = \langle g \rangle (A \cdot c \Delta t) \iff P = |F|/A = \langle g \rangle c
    \end{equation}
    Note, $P$ is pressure. 
\end{enumerate}

\subsection*{E\&M Wave in Isotropic Linear Medium}

Maxwells equations are

\begin{align}
    \nabla \cdot E &= 0 \\
    \nabla \cdot B &= 0 \\
    \nabla \times E &= -\partial_t B \\
    \nabla \times B &= \mu\epsilon\partial_t E
\end{align}

Note, $\mu = \mu_0(1 + \chi_m)$ and etc for $\epsilon$. 

\newpage 

\section*{Lecture 12: 9/19}

E\&M wave in a linear media, has that

\begin{equation}
    \epsilon_0 \rightarrow \epsilon \;\;\; \mu_0 \rightarrow \mu 
\end{equation}

\section*{Lecture 13: 9/22}

Deriving Fresnel equation for normal incidence:
\begin{enumerate}
    \item E, B are parallel to the surface
    \item The boundary conditions are in the lecture notes. 
\end{enumerate}

We now study generalized incidence and try to derive the Fresnel Equation for this. The Electric field is 

\begin{align}
    E_R &= E_{RO} \exp(i(k_R \cdot r - \omega t)) \\
    B_R &= \hat{k}_R \times \frac{n_1}{c}E_R
\end{align}
\begin{align}
    E_I &= E_{IO} \exp(i(k_I \cdot r - \omega t)) \\
    B_I &= \hat{k}_I \times \frac{n_1}{c}E_I
\end{align}

The boundary conditions are 

\begin{align}
    (D_1 - D_2) \cdot n &= 0 \\
    (B_1 - B_2) \cdot n&= 0 \\
    (E_1 - E_2) \times n &= 0 \\
    (H_1 - H_2)\times n &= 0
\end{align}

\section*{Lecture 14: 9/24}

Midterm, 1st question chapter 8, 2 questions on chapter 9. 

\newpage 

\section*{Lecture 15: 9/26}

Electric field in conductors. Maxwell's equations become 

\begin{align}
    \nabla \cdot E &= \rho_f/\epsilon_0 \\
    \nabla \cdot B &= 0 \\
    \nabla \times E &= -\partial_t B \\
    \nabla \times B &= \mu_0 J + \mu_0\epsilon_0\partial_t E
\end{align}

Assuming thqt $J_f = \sigma E$, we can then assume that the free charge is constant. We prove this:

\begin{align*}
    \partial_t\rho &= \nabla \cdot J \;\;\ \text{Ohm's Law} \\
    &= \sigma \nabla \cdot E \;\;\ \text{Gauss's Law} \\
    &= \sigma \frac{\rho}{\epsilon} \\
    \rho(t) &= \rho(t=0)\exp\left(-\frac{\sigma}{\epsilon_0}t\right)
\end{align*}

Note that if $\sigma = 1$, then $1/\epsilon_0 \sim 10^{12}$, which means that the time dependence of the charges can be ignored. Then the modified wave equation becomes:

\begin{align*}
    \nabla \times (\nabla \times E) &= -\partial_t (\nabla \times B) \\
    -\nabla^2 E &= -\partial_t (\mu_0\epsilon_0 \partial_t E + \mu_0 \sigma E) \\
    &= \mu_0\epsilon_0 \partial^2_t E + \mu_0 \sigma \partial_t E 
\end{align*}

Plugging in $E_0\exp(i(kz - \omega t))$, we obtain

\begin{equation}
    k^2 = \mu\epsilon \omega^2 + i\mu\sigma\omega 
\end{equation}

Then our solution becomes 

\begin{equation}
    \exp(-\kappa z)\exp(i(kz - \omega t))
\end{equation}

\end{document}
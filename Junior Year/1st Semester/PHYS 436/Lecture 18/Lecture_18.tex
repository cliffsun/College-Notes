\documentclass{article}
\usepackage[left=2cm, right=2cm, top=2cm, bottom=2cm]{geometry}
\usepackage{graphicx}
\usepackage{amsmath}
\usepackage{amssymb}
\usepackage{amsthm}
\usepackage{fancyhdr}
\usepackage{verbatim}
\usepackage{listings}
\usepackage{xcolor}
\usepackage{pgfplots}
\usepackage{physics}

\lstset{
    language=C++,
    basicstyle=\ttfamily\footnotesize,
    keywordstyle=\color{blue},
    commentstyle=\color{green},
    stringstyle=\color{red},
    numbers=left,
    numberstyle=\tiny\color{gray},
    frame=single,
    breaklines=true,
    captionpos=b,
}


\title{PHYS 436: Lecture \# 18}
\author{Cliff Sun}

\newtheorem{theorem}{Theorem}[section]
\newtheorem{lemma}[theorem]{Lemma}
\newtheorem{definition}[theorem]{Definition}
\newtheorem{conjecture}[theorem]{Conjecture}
\newtheorem{proposition}[theorem]{Proposition}
\newtheorem{corollary}[theorem]{Corollary}
\newtheorem{one minute paper}[theorem]{One Minute Paper}

\pagestyle{fancy}
\lhead{\textbf{\thepage}\ \ \nouppercase{\rightmark}}
\chead{PHYS 436: Lecture \# 18}
\rhead{Cliff Sun}

\begin{document}

\maketitle

\section*{Rectangular wave guide}

\begin{definition}
    Wave guide restricts the transmission of energy to one direction. 
\end{definition}

Allows only frequencies of 

\begin{equation}
    \omega = c\sqrt{k^2 + (m\pi/a)^2 + (n/pi/b)^2}
\end{equation}

\section*{General Wave Guide}

\begin{align}
    E(r,t) &= E(r_{\perp})\exp(i(kz- \omega t)) \\
    B(r,t) &= B(r_{\perp})\exp(i(kz- \omega t)) \\
\end{align}

\end{document}
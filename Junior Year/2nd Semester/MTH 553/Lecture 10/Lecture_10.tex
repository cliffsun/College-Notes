\documentclass{article}
\usepackage[left=2cm, right=2cm, top=2cm, bottom=2cm]{geometry}
\usepackage{graphicx}
\usepackage{amsmath}
\usepackage{amssymb}
\usepackage{amsthm}
\usepackage{fancyhdr}
\usepackage{verbatim}
\usepackage{listings}
\usepackage{xcolor}
\usepackage{pgfplots}
\usepackage{physics}

\lstset{
    language=C++,
    basicstyle=\ttfamily\footnotesize,
    keywordstyle=\color{blue},
    commentstyle=\color{green},
    stringstyle=\color{red},
    numbers=left,
    numberstyle=\tiny\color{gray},
    frame=single,
    breaklines=true,
    captionpos=b,
}


\title{MTH 553: Lecture \# 10}
\author{Cliff Sun}

\newtheorem{theorem}{Theorem}[section]
\newtheorem{lemma}[theorem]{Lemma}
\newtheorem{definition}[theorem]{Definition}
\newtheorem{conjecture}[theorem]{Conjecture}
\newtheorem{proposition}[theorem]{Proposition}
\newtheorem{corollary}[theorem]{Corollary}
\newtheorem{one minute paper}[theorem]{One Minute Paper}

\pagestyle{fancy}
\lhead{\textbf{\thepage}\ \ \nouppercase{\rightmark}}
\chead{MTH 553: Lecture \# 10}
\rhead{Cliff Sun}

\begin{document}

\maketitle

\section*{Lecture Span}
\begin{itemize}
    \item Continue working on non-homogenous wave equation with arbitrary boundary conditions
\end{itemize}

Recall that we reduced the wave equation to 

\begin{align*}
    w_{tt} - c^2w_{xx} &= 0\\
    w(x,0) &= g_2(x) \\
    w_t(x,0) &= h_2(x) \\
    aw(0,t) - bw_x(0,t) &= 0 \\
    c^*w(L,t) + dw_x(L,t) &= 0 
\end{align*}

Assume $g_2 = 0$ which corresponds to zero initial displacement, then 

\begin{equation}
    w(x,t) = \sum_n \left( a_n\sin(\lambda_n x) + b_n\cos(\lambda_n x) \right) \cdot \sin(\lambda_n ct)
\end{equation}

Note that given the $w$ above, we can verify that: 

\begin{equation}
    w_{tt} = c^2w_{xx}
\end{equation}

The, we can indeed check that: 

\begin{equation}
    w(x,0) = 0
\end{equation}

So now, the question is what $\lambda_n$ do we choose? We can determine $\lambda_n$ from the boundary conditions. 
First, assume that $b=d=0$, then define 

\begin{equation}
    \phi_n(x) = a_n\sin(\lambda_n x) + b_n\cos(\lambda_n x)
\end{equation}

Then we have the dirichlet boundary conditions:

\begin{align*}
    \phi_n(0) &= 0 \\
    \phi_n(L) &= 0
\end{align*}

The first boundary condition requires that $b_n = 0$, and the second one requires $\sin(\lambda_n L) = 0$, therefore,

\begin{equation}
    \lambda_n = \frac{n\pi}{L}
\end{equation}

Then, we have that

\begin{equation}
    w(x,t) = \sum_{n=1}^{\infty} a_n\sin\left( \frac{n\pi x}{L} \right)\sin\left( \frac{n\pi ct}{L} \right)
\end{equation}

If $a = c^* = 0$, then we get the neumann boundary conditions, 

\begin{align*}
    \partial_x \phi_n (0) &= 0 \\
    \partial_x \phi_n(L) &= 0
\end{align*}

Then $a_n = 0$ and we obtain that $\sin\left( \lambda_n L \right) = 0$ and 

\begin{equation}
    \lambda_n = \frac{n\pi}{L}
\end{equation}

Therefore, 

\begin{equation}
    w(x,t) = \sum_{n=0}^{\infty}b_n\cos\left( \frac{n\pi x}{L} \right)\sin\left( \frac{n\pi ct}{L} \right)
\end{equation}

Now, we go back to the Dirichlet example, we can determine $a_n$ using the initial conditions. Here, $g_2 \equiv 0$, $\lambda = n\pi/L$, and $b_n = 0$ with 

\begin{equation}
    w(x,t) = \sum_n a_n\sin(\lambda_n x)\sin(\lambda_n ct)
\end{equation}

Then we impose IC, 

\begin{equation}
    h_2(x) = w_t(x,0) = \sum_n \left( \lambda_n ca_n \right)\sin(\lambda_n x)
\end{equation}

To find $a_n$, we multiply by $\sin(\lambda_m x)$ and integrate (i.e. take $L^2$ inner product with the basis elements). Then we obtain,

\begin{equation}
    \int_{0}^{L}h_2(x)\sin\left( \lambda_m x \right)dx = \sum_{n=0}^{\infty}\left( \lambda_n ca_n \right)\int_{0}^{L}\sin\left( \lambda_n x \right)\sin\left( \lambda_m x \right)dx
\end{equation}

But 

\begin{equation}
    \int_{0}^{L}\sin\left( \lambda_n x \right)\sin\left( \lambda_m x \right)dx = \frac{L}{2}\delta_{nm}
\end{equation}

Therefore, we obtain 

\begin{equation}
    \int_{0}^{L}h_2(x)\sin\left( \lambda_m x \right)dx = \lambda_m c a_m \frac{L}{2}
\end{equation}

Therefore, 

\begin{equation}
    a_m = \frac{2}{m\pi c}\int_{0}^{L}h_2(x)\sin\left( \lambda_m x \right)dx
\end{equation}

\section*{Step 3b - solve 2b by characteristics}

Recall that 

\begin{equation}
    w(x,t) = F(x+ct) + G(x-ct)
\end{equation}

On a characteristic parallelogram on the x,t plane with points A,B,C,D, here, $B$ connects to $A$ and $C$ connects to $D$ with the slope $x-ct = d$ where $d$ is a constant. Similarly, $B$ connects to $C$ and $A$ connects to $D$ with a slope of $x + ct = e$ where $e$
is a constant. Then consider the wave equation at point $A$ and $C$, we calculate 

\begin{align*}
    w(A) + w(C) &= F(A) + G(A) + F(C) + F(D) \\
    &= F(D) + G(B) + F(B) + G(D) \\
    &= W(B) + W(D)
\end{align*}

Therefore, 

\begin{equation}
    w(A) + w(C) = w(B) + w(D)
\end{equation}

Find $w$ in triangle $R_1$ (which extends back to the initial condition) using D'Alembert on interval $[0,L]$. Then assume there is a point $A$ that we don't know, then we can pick 3 other points
that form a parallelogram on the x,t plane. Then, we obtain 

\begin{equation}
    w(A) = w(B) + w(D) - w(C)
\end{equation}

Where $B$ rests on the boundary, then $w(B) = 0$.

\end{document}
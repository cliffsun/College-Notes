\documentclass{article}
\usepackage[left=2cm, right=2cm, top=2cm, bottom=2cm]{geometry}
\usepackage{graphicx}
\usepackage{amsmath}
\usepackage{amssymb}
\usepackage{amsthm}
\usepackage{fancyhdr}
\usepackage{verbatim}
\usepackage{listings}
\usepackage{xcolor}
\usepackage{pgfplots}
\usepackage{physics}

\lstset{
    language=C++,
    basicstyle=\ttfamily\footnotesize,
    keywordstyle=\color{blue},
    commentstyle=\color{green},
    stringstyle=\color{red},
    numbers=left,
    numberstyle=\tiny\color{gray},
    frame=single,
    breaklines=true,
    captionpos=b,
}


\title{MTH 553: Lecture \# 5}
\author{Cliff Sun}

\newtheorem{theorem}{Theorem}[section]
\newtheorem{lemma}[theorem]{Lemma}
\newtheorem{definition}[theorem]{Definition}
\newtheorem{conjecture}[theorem]{Conjecture}
\newtheorem{proposition}[theorem]{Proposition}
\newtheorem{corollary}[theorem]{Corollary}
\newtheorem{one minute paper}[theorem]{One Minute Paper}

\pagestyle{fancy}
\lhead{\textbf{\thepage}\ \ \nouppercase{\rightmark}}
\chead{MTH 553: Lecture \# 5}
\rhead{Cliff Sun}

\begin{document}

\maketitle

\section*{Lecture Span}
\begin{itemize}
    \item Shock waves
\end{itemize}

Recap, the jump condition is 

\begin{equation}
    \xi'(y) = \frac{G(u_l) - G(u_r)}{u_l - u_r}
\end{equation}

\section*{Burger's Example}

Consider:

\begin{align*}
    G(u) &= \frac{1}{2}u^2 \\
    \left( \frac{1}{2}u^2 \right)_x + u_y &= 0 \\
    uu_x + u_y &= 0
\end{align*}

Consider $u_r = 0$ and $u_l = 1/2$. There is an intersection between $u_l$ and $u_r$. Clearly, there will not be a smooth solution, so we will look for a weak solution with a 
shock $x = \xi(y)$ with $\xi(0) = 0$ (why?). This is because the shocks start infinitesimally close to $x=0$. 

\begin{equation}
    \xi'(y) = \frac{G(2) - G(0)}{2} = 1
\end{equation}

Therefore, 

\begin{equation}
    \xi(y) = y
\end{equation}

NOTE: we are finding $\xi(y)$. This is a weak solution with a shock. The jump condition singles out a meaningful solution when characteristics cross. For Burger's, we have that 

\begin{equation}
    \frac{G(u_l) - G(u_r)}{u_l - u_r} = \frac{1}{2}(u_l + u_r) = \text{ average of left and right particle velocities}
\end{equation}

Numerically, this shock curve represents a physical path of where the particles collide, and the velocity of the particles along this $\xi(y)$ can be found. 

\section*{Burger's Example (again)}

Now, $u_l = -2$. So now, the characteristic paths leave some sort of gap on the x-y plane. We call this gap region $W$. Here, characteristics give us no information in $W$. 

\subsection*{Solution 1: Try a new shock}

Let $u_l = -2$ and $u_r = 0$, then 

\begin{equation}
    x'(y) = \xi'(y) = -1
\end{equation}

With $\xi(0) = 0$, then we get $x = \xi(y) = -y$. This shock is non-physical since particles are being created at the shock.

\subsection*{Solution 2}

Look for a solution in $W$ of the form 

\begin{equation}
    u(x,y) = v\left( \frac{x}{y} \right)
\end{equation}

Note, $v$ is a function of $x/y$ because $x/y$ is constant along each characteristic for this specific PDE. Substituting into Burger's equation, we obtain 

\begin{align}
    0 &= uu_x + u_y \\
    &= v(x/y)v'(x/y)/y + v'(x/y)(-x/y^2)\\
    &= v(x/y)/y + (-x/y^2)\\
    v(x/y) &= x/y 
\end{align}

So we try this solution:

\begin{equation}
    u(x,y) = \begin{cases}
        0 & x > 0 \\
        x/y & (x,y) \in W \\
        -2 & y < -x/2 \\
    \end{cases}
\end{equation}

We call this type of solution a "fan" or rarefaction wave. 

\section*{Entropy Condition}

Assume $u$ is a weak solution of

\begin{equation}
    G(u)_x + u_y = 0
\end{equation}

This is the conservation law. Also, $u$ satisfies the jump condition 

\begin{equation}
    \xi'(y) = \frac{G(u_l) - G(u_r)}{u_l - u_r}
\end{equation}

along the shock curve $x = \xi(y)$. 

\begin{definition}
    We say that $u$ satisfies the \underline{entropy condition} if $G'(u_l) > \xi'(y) > G'(u_r)$ 
    everywhere on the shock. 
\end{definition}

\subsection*{Meaning}

Characteristics are straight lines of the form $x = G'(u)y + \text{constant}$. So entropy condition means that characteristics can meet at a shock, but \underline{cannot} form from one. 

\begin{center}
    The entropy condition rules out non-physical shocks (like solution 1)
\end{center}

\end{document}
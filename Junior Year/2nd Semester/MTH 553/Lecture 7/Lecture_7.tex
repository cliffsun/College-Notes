\documentclass{article}
\usepackage[left=2cm, right=2cm, top=2cm, bottom=2cm]{geometry}
\usepackage{graphicx}
\usepackage{amsmath}
\usepackage{amssymb}
\usepackage{amsthm}
\usepackage{fancyhdr}
\usepackage{verbatim}
\usepackage{listings}
\usepackage{xcolor}
\usepackage{pgfplots}
\usepackage{physics}

\lstset{
    language=C++,
    basicstyle=\ttfamily\footnotesize,
    keywordstyle=\color{blue},
    commentstyle=\color{green},
    stringstyle=\color{red},
    numbers=left,
    numberstyle=\tiny\color{gray},
    frame=single,
    breaklines=true,
    captionpos=b,
}


\title{MTH 553: Lecture \# 7}
\author{Cliff Sun}

\newtheorem{theorem}{Theorem}[section]
\newtheorem{lemma}[theorem]{Lemma}
\newtheorem{definition}[theorem]{Definition}
\newtheorem{conjecture}[theorem]{Conjecture}
\newtheorem{proposition}[theorem]{Proposition}
\newtheorem{corollary}[theorem]{Corollary}
\newtheorem{one minute paper}[theorem]{One Minute Paper}

\pagestyle{fancy}
\lhead{\textbf{\thepage}\ \ \nouppercase{\rightmark}}
\chead{Generic Title}
\rhead{Cliff Sun}

\begin{document}

\maketitle

\section*{Lecture Span}
\begin{itemize}
    \item Uniqueness of entropy solution
\end{itemize}

\begin{theorem}
    If $G$ is uniformly convex or uniformly concave, then there exists at most one weak solution 
    of the IVP (initial value problem)
    \begin{align*}
        G(u)_x + u_y &= 0 \\
        u(x,0) &= k(x)
    \end{align*}
    that satisfies the entropy condition. In particular, the entropy condition singles out a physically meaningful solution. 
\end{theorem}

\section*{Example - Riemann's problem}

Assume that $G$ is uniformly convex, and $u_l$ and $u_r$ are real numbers. Consider 

\begin{align*}
    G(u)_x + u_y &= 0 \\
    u(x,0) &= \begin{cases}
        u_l & x < 0 \\
        u_r & x > 0
    \end{cases}
\end{align*}

Then the unique entropy solution is 

\subsection*{Case 1}

If $u_l > u_r$ (shock curve, not a fan), then 

\begin{equation}
    u(x,y) = \begin{cases}
        u_l & x < \sigma y \\
        u_r & x > \sigma y
    \end{cases}
\end{equation}

Where 

\begin{equation}
    \sigma = \frac{G(u_l) - G(u_r)}{u_l - u_r}
\end{equation}

\subsection*{Case 2}

If $u_l < u_r$, then this is a fan. Then 

\begin{equation}
    u(x,t) = \begin{cases}
        u_l & x < G'(u_l)y\\
        G'^{-1}\left( \frac{x}{y } \right) &  G'(u_l)y < x < G'(u_r)y\\
        u_r & x > G'(u_r)y
    \end{cases}
\end{equation}

\begin{proof}
    We now prove each limiting case. For part 1, we check the entropy condition, which states 
    \begin{equation}
        G'(u_l) > \xi'(y) > G'(u_r)
    \end{equation}
    This is true for general uniformly convex function. For part 2, we check $u$ is continuous at the fan boundary $x=G'(u_l)y$, 
    $u(x,y) = G'^{-1}(x/y) = G'^{-1}(G(u_l)) = u_l$. Note, $u$ is a smooth solution in the fan and outside of the fan since 
    $u(x,y) = v(x/y)$ where $v = G'^{-1}$. Check
    \begin{align*}
        G(u)_x + u_y \\
        &= G'(v(x/y))v'(x/y)/y  + v'(x/y)(-x/y^2) \\
        &= 0
    \end{align*}
    Since $G'(v(x/y)) = x/y$. We don't have to worry about the entropy condition since this is not a shock. 
\end{proof}

\section*{Example}

Burger's equation 

\begin{equation}
    \left( \frac{1}{2}u^2 \right)_x + u_y = 0
\end{equation}

Consider 

\begin{equation}
    u = \begin{cases}
        0 & x < -1\\
        -1 & -1 < x < 0 \\
        1 & 0 < x < 1 \\
        0 & x > 1
    \end{cases}
\end{equation}

This is symmetrical, so only consider $x > 0$. For $x > 0$, $0 < y \leq 2$, we have a shock on 

\begin{equation}
    x = 1 + \frac{1}{2}y
\end{equation}

We also get a fan $|x| < y$, with 

\begin{equation}
    u = G'^{-1}(x/y) = \frac{x}{y}
\end{equation}

But at $y=2$, the shock hits the fan. So then what is the shock curve? 

\begin{equation}
    \xi'(y) = \frac{1}{2}(u_l + u_r) =\frac{x}{2y} = \frac{dx}{dy}
\end{equation}

Solving this ODE yields 

\begin{equation}
    \frac{dx}{x} = \frac{dy}{2y}
\end{equation}

\begin{equation}
     x = C\sqrt{y}
\end{equation}

Where at $y=2$, $x=2$, so then 

\begin{equation}
    2 = C\sqrt{2} \implies C = \sqrt{2}
\end{equation}

Therefore, 

\begin{equation}
    \xi(y) = \sqrt{2y}
\end{equation}

Now, we brief fully non-linear PDEs. 

\section*{Not examinable fully non-linear equations}

\begin{equation}
    F(x,y,u, u_x, u_y) = 0
\end{equation}

Associated characteristic ODE system 

\begin{align*}
    dx/dt &= F_p(x,y,z,p,q) \\
    dy/dt &= F_q \\
    du/dt &= pF_p + qF_q \\
    dp/dt &= -F_x - pF_z \\
    dq/dt &= -F_y - qF_q 
\end{align*}

\end{document}
\documentclass{article}
\usepackage[left=2cm, right=2cm, top=2cm, bottom=2cm]{geometry}
\usepackage{graphicx}
\usepackage{amsmath}
\usepackage{amssymb}
\usepackage{amsthm}
\usepackage{fancyhdr}
\usepackage{verbatim}
\usepackage{listings}
\usepackage{xcolor}
\usepackage{pgfplots}
\usepackage{physics}

\lstset{
    language=C++,
    basicstyle=\ttfamily\footnotesize,
    keywordstyle=\color{blue},
    commentstyle=\color{green},
    stringstyle=\color{red},
    numbers=left,
    numberstyle=\tiny\color{gray},
    frame=single,
    breaklines=true,
    captionpos=b,
}


\title{MTH 553: Lecture \# 3}
\author{Cliff Sun}

\newtheorem{theorem}{Theorem}[section]
\newtheorem{lemma}[theorem]{Lemma}
\newtheorem{definition}[theorem]{Definition}
\newtheorem{conjecture}[theorem]{Conjecture}
\newtheorem{proposition}[theorem]{Proposition}
\newtheorem{corollary}[theorem]{Corollary}
\newtheorem{one minute paper}[theorem]{One Minute Paper}

\pagestyle{fancy}
\lhead{\textbf{\thepage}\ \ \nouppercase{\rightmark}}
\chead{MTH 553: Lecture \# 3}
\rhead{Cliff Sun}

\begin{document}

\maketitle

\section*{Last time}

Studied quasilinear equations 

\begin{equation}
    a(x,y,u)u_x + b(x,y,u)u_y = c(x,y,u)
\end{equation}

Such that 

\begin{align*}
    dx/dt &= a(x,y,u) \\
    dy/dt &= b(x,y,u) \\
    dz/dt &= c(x,y,u)
\end{align*}

If we define $u(x,y) = z(t)$, then 

\begin{equation}
    c = dz/dt = u_x a(x,y,u) + u_y b(x,y,u)
\end{equation}

\section*{Lecture Span}
\begin{enumerate}
    \item semi-linear example ($u$ does not appear in the coefficients)
\end{enumerate}

\begin{equation}
    xu_x + u_y = u^2
\end{equation}
\begin{equation}
    u(x,0) = k(x)\;\; \text{initial condition is on the $x$ axis}
\end{equation}

We first parameterize $\Gamma$, 

\begin{align*}
    \Gamma: \begin{cases}
        x = s \\
        y = 0 \\
        z = k(s)
    \end{cases}
\end{align*}

We also have to check that 

\begin{equation}
    <f', g'> \nparallel <a,b> \;\in \Gamma
\end{equation}

Here, $f'$ is the derivative of $x$ and $g'$ is the derivative of $y$. Therefore, 

\begin{equation}
    <1,0> \nparallel <x,1> = <s,1>
\end{equation}

This is true. Next, we solve the characteristic system:

\begin{align*}
    dx/dt &= x \\
    dy/dt &= 1 \\
    dz/dt &= z^2
\end{align*}

We can first solve $$y = t + c_2(s)$$ Note, $c_2(s)$ is the constant of coefficient, but can depend on where you start on the initial curve. Then we solve 

$$x=c_1(s)e^t$$

Then we can solve 

$$z = \frac{1}{c_3(s) - t}$$

We then apply the initial condition that $c_2(s) = 0$, $c_1(s) = s$, and $c_3(s) = k(s)$. Therefore, we get 

\begin{align*}
    x &= se^t \\
    y &= t \\
    z &= \frac{1}{1/k(s) - t} = \frac{k(s)}{1 - tk(s)}
\end{align*}

Here,

\begin{equation}
    u(x,y) = z = \frac{k(s)}{1-tk(s)}
\end{equation}

We want to solve for $s,t$ in terms of $x,y$. Here, 

\begin{equation}
    y=t
\end{equation}
\begin{equation}
    s = xe^{-y}
\end{equation}

Therefore, 

\begin{equation}
    u(x,y) = \frac{k(xe^{-y})}{1-yk(xe^{-y})}
\end{equation}

We need this denominator to be $\neq 0$, therefore, the solution exists when 

\begin{equation}
    yk(xe^{-y}) \neq 1
\end{equation}

This curve does contain the initial condition. Moreover, if $k$ is bounded ($|k(x)| \leq A \;\; \forall x$), then the domain contains strip around axis (what does this mean??).

\section*{General practical difficulties}

\begin{enumerate}
    \item Can ODEs (of characteristic equations) be solved explicitly?
    \item Can $s,t$ be found in terms of $x,y$? 
    \item On what domain in the x-y plane is the solution defined? 
\end{enumerate}

\section*{Example: Inviscid burger's equation}

\begin{equation}
    uu_x + u_y = 0
\end{equation}
\begin{equation}
    u(x,0) = k(x)
\end{equation}

Interpretation: stream of particles with constant velocity where $u(x,y)$ is the velocity of particle at location $x$, time $y$. In other words, for all $x,y$, then given $\tau$

\begin{equation}
    u(x + \tau u(x,y), y + \tau) = u(x,y)
\end{equation}

Where $\tau$ has units of time. 

Differentiate w.r.t $\tau$ and let $\tau = 0$, then we get $uu_x + u_y = 0$. Solve 

\begin{align*}
    dx/dt &= z \\
    dy/dt &= 1\\
    dz/dt &= 0
\end{align*}

Note that 

\begin{equation}
    dy/dx = \frac{1}{z}
\end{equation}

In the homework, if $k'(x) \geq 0$, then the particles speeds go up. In particular, then $k(x)$ increasing and $1/k(x)$ is decreasing. The slope of the characteristic decreases from left to right. We have 

\begin{equation}
    u(x_0) = k(x_0)
\end{equation}

Then the slope of the characteristic is $1/k(x_0)$ (FIGURE THIS OUT). These projected characteristics give a global solution. But what if $k'(x) \ngeq 0$, say $k(x_1) > k(x_2)$ for some $x_1 < x_2$. 
Then, some of the characteristic curves will intersect. What do you do? Should $u$ at this intersection point be equal to $k(x_1)$ or $k(x_2)$? These particles are colliding, and we see shock formation. 

\end{document}
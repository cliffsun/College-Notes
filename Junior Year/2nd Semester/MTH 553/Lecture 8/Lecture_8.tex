\documentclass{article}
\usepackage[left=2cm, right=2cm, top=2cm, bottom=2cm]{geometry}
\usepackage{graphicx}
\usepackage{amsmath}
\usepackage{amssymb}
\usepackage{amsthm}
\usepackage{fancyhdr}
\usepackage{verbatim}
\usepackage{listings}
\usepackage{xcolor}
\usepackage{pgfplots}
\usepackage{physics}

\lstset{
    language=C++,
    basicstyle=\ttfamily\footnotesize,
    keywordstyle=\color{blue},
    commentstyle=\color{green},
    stringstyle=\color{red},
    numbers=left,
    numberstyle=\tiny\color{gray},
    frame=single,
    breaklines=true,
    captionpos=b,
}


\title{MTH 553: Lecture \# 8}
\author{Cliff Sun}

\newtheorem{theorem}{Theorem}[section]
\newtheorem{lemma}[theorem]{Lemma}
\newtheorem{definition}[theorem]{Definition}
\newtheorem{conjecture}[theorem]{Conjecture}
\newtheorem{proposition}[theorem]{Proposition}
\newtheorem{corollary}[theorem]{Corollary}
\newtheorem{one minute paper}[theorem]{One Minute Paper}

\pagestyle{fancy}
\lhead{\textbf{\thepage}\ \ \nouppercase{\rightmark}}
\chead{MTH 553: Lecture \# 8}
\rhead{Cliff Sun}

\begin{document}

\maketitle

\section*{Lecture Span}
\begin{itemize}
    \item Wave equation
\end{itemize}

\section*{One dimensional wave equation}

\begin{equation}
    u_{tt} - c^2u_{xx} = 0
\end{equation}

Here, $u$ is the displacement of the string and $c$ is the speed of the wave. \textbf{But why should we expect this equation?}

\subsubsection*{Plausibility}

We consider a concave parabola with a $u_{xx} < 0$. Here, we expect the tension in the string to pull the string down. Here, this tension is a force, which means an acceleration, or $u_{tt}$. 
Therefore, if $u_{xx} < 0$, then $u_{tt} < 0$ since it would be pulling it down. Therefore, we naturally arrive at the wave equation, where the concavity of the wave is proportional to its acceleration.

\begin{equation}
    c^2u_{xx} = u_{tt}
\end{equation}

This wave equation is called the \underline{homogenous wave equation}, or HWE. 

\subsection*{Solution on $\mathbb{R}$}

Change variables, let $\mu = x + ct$ and $\eta = x - ct$. That is 

\begin{align*}
    x &= \frac{\mu + \eta}{2} \\
    t &= \frac{\mu - \eta}{2c}
\end{align*}

Then, the wave equation in these new coordinates becomes

\begin{equation}
    u_{\mu \eta} = 0
\end{equation}

We check: by the chain rule, we have that 

\begin{align*}
    u_{\mu} &= u_x x_\mu +u_y y_\mu\\
    &= \frac{1}{2}u_{x} + \frac{1}{2c}u_{t}
\end{align*}

Similar for $u_{\eta}$. Therefore, 

\begin{equation}
    u_{\mu\eta} = \frac{1}{4c^2}\left( c^2 u_{xx} - u_{tt} \right)
\end{equation}

Hence, 

\begin{equation}
    u = f(\mu) + g(\eta) \iff f(x + ct) + g(x-ct)
\end{equation}

These are traveling waves. That is 

\begin{center}
    $u = $ left-moving wave of shape f $+$ right-moving wave of shape g 
\end{center}

\subsection*{IVP for HWE}

This is the famous d'Alembert solution. Consider a wave equation with initial conditions 

\begin{align*}
    u_{tt} - c^2u_{xx} &= 0 \\
    u(x,0) &= g(x) \\\
    u_t(x,0) &= h(x)
\end{align*}

Using equation $5$, we have that 

\begin{equation}
    g(x) = u(x,0) = F(x) + G(x)
\end{equation}

Then 

\begin{equation}
    h(x) = u_t(x,0) = c\left( F'(x) - G'(x) \right)
\end{equation}

Integrating both sides yields 

\begin{equation}
    \frac{1}{c}\int_{0}^x h(\xi)d\xi + C = F(x) - G(x)
\end{equation}

If we add and subtract these equations, we obtain 

\begin{equation}
    F(x) = \frac{1}{2}g(x) + \frac{1}{2c}\int_{0}^x h(\xi)d\xi + \frac{C}{2}
\end{equation}
\begin{equation}
    G(x) = \frac{1}{2}g(x) - \frac{1}{2c}\int_{0}^x h(\xi)d\xi - \frac{C}{2}
\end{equation}

Then this means that 

\begin{equation}
    u(x,t) = \frac{1}{2}\left[ g(x+ct) + g(x-ct) \right] + \frac{1}{2c}\int_{x-ct}^{x+ct}h(\xi)d\xi 
\end{equation}

So this solution is like the left and right moving "responses" to the initial condition. 

\begin{proposition}
    \underline{Well-posedness}
    \begin{enumerate}
        \item Existence, does a solutione exist? If $g \in C^{k}(\mathbb{R})$, $h \in C^{k-1}(\mathbb{R})$, for some $k \geq 2$, 
        then use equation (5) gives a $C^{k}$ solution of (11). 
        \item Uniqueness, d'Alembert (11) is unique $C^{k}$ smooth solution. 
        \item Continuous dependence. $u$ depends continuously on the data, with respect to the max-norm. This means that if $||g_1 - g_2||_{\infty} \leq \epsilon$, and $||h_1 - h_2||_{\infty} \leq \epsilon$, then 
        $||u_1(\dot, t) - u_2(\dot, t)||_{\infty} \leq (1 + t)\epsilon$. Here $||g||_{\infty}$ is $\max_{x\in \mathbb{R}}|g(x)|$.
    \end{enumerate} 
\end{proposition}

\begin{proof}
    \begin{enumerate}
        \item Check if $u$ is $C^{k}$ smooth (has $k$ derivatives) by using fundamental theorem of Calculus. 
        \item If $u$ and $v$ solve $(4)$, then consider $w = u - v$
        \begin{align*}
            w_{tt} - c^2w_{xx} &= 0 \\
            w(x,0) &= 0 \\
            w_t(x,0)&= 0
        \end{align*} 
        Which implies that $w$ is the zero function which means that $u = v$. 
        \item homework 3, bruh.  
    \end{enumerate}
\end{proof}

\begin{definition}
    Consider a triangle and as cone with slope $x = \pm t$. Domain of dependence, for $u$ is $[x-ct, x+ct]$ for the triangle. And for the cone, the range of influence (points that the wavefunction 
    can later influence) has speed of $\pm c$ at vertex $\xi$.
\end{definition}

\section*{IVP for NHWE on $\mathbb{R}$}

Now, we now consider the non-homoegenous wave equation: 

\begin{align*}
    u_{tt} - c^2u_{xx} &= f(x,t)\\
    u(x,0) &= g(x) \\
    u_t(x,0) &= h(x)
\end{align*}

Decompose $u = w + z$ where $w$ is the homogenous solution and $z$ is the non-homoegenous solution. Note 

\begin{align*}
    z_{tt} - c^2z_{xx} &= f(x,t) \\
    z(x,0) &= 0 \\
    z_t(x,0) &= 0
\end{align*}

\end{document}
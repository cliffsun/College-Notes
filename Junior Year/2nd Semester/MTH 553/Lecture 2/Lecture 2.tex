\documentclass{article}
\usepackage[left=2cm, right=2cm, top=2cm, bottom=2cm]{geometry}
\usepackage{graphicx}
\usepackage{amsmath}
\usepackage{amssymb}
\usepackage{amsthm}
\usepackage{fancyhdr}
\usepackage{verbatim}
\usepackage{listings}
\usepackage{xcolor}
\usepackage{pgfplots}
\usepackage{physics}

\lstset{
    language=C++,
    basicstyle=\ttfamily\footnotesize,
    keywordstyle=\color{blue},
    commentstyle=\color{green},
    stringstyle=\color{red},
    numbers=left,
    numberstyle=\tiny\color{gray},
    frame=single,
    breaklines=true,
    captionpos=b,
}


\title{MTH 553: Lecture \# 2}
\author{Cliff Sun}

\newtheorem{theorem}{Theorem}[section]
\newtheorem{lemma}[theorem]{Lemma}
\newtheorem{definition}[theorem]{Definition}
\newtheorem{conjecture}[theorem]{Conjecture}
\newtheorem{proposition}[theorem]{Proposition}
\newtheorem{corollary}[theorem]{Corollary}
\newtheorem{one minute paper}[theorem]{One Minute Paper}

\pagestyle{fancy}
\lhead{\textbf{\thepage}\ \ \nouppercase{\rightmark}}
\chead{MTH 553: Lecture \# 2}
\rhead{Cliff Sun}

\begin{document}

\maketitle

\section*{Lecture Span}
\begin{itemize}
    \item First order equations
    \item Cauchy problem for quasilinear 1st order equations (coefficients can depend on $u$)1
\end{itemize}

\section*{ODE background}

Suppose autonomous equation (no $t$ variable on the RHS)

\begin{align}
    \frac{dx}{dt} &= a(x,y,z)\\
    \frac{dy}{dt} &= b(x,y,z) \\
    \frac{dz}{dt} &= c(x,y,z)
\end{align}

Suppose a curve modeled by $(x,y,z)$ with the tangent vector $\langle a,b,c \rangle$. With an autonomous system, you don't need to know the time, just where you are. Suppsoe $x=x_0$, and etc at $t=0$. 

\begin{align}
    \frac{dx}{dt} &= a\\
    \frac{dy}{dt} &= 1 \\
    \frac{dz}{dt} &= 0
\end{align}

The solution is 

\begin{align}
    x &= at + x_0 \\
    y &= t + y_0 \\
    z &= z_0
\end{align}

Special case, if $y_0 = 0$, then the path is simply a straight line $x=a(y) + x_0$ and $z=z_0$.

\section*{Cauchy Problem for quasilinear first order PDE}

Given a curve $\Gamma \in \mathbb{R}^3$, find $u(x,y)$ solving 

\begin{equation}
    a(x,y,u)u_x + b(x,y,u)u_y = c(x,y,u)
\end{equation}

Where $a,b$ are coefficients. With $\Gamma \subset \{\text{graph of $u$}\}$. Note, $\Gamma$ represents a curve of initial boundary conditions, which only a part of the "graph of $u$". 

\subsection*{Example - IVP (initial value problem) for transport equation}

\begin{align*}
    au_x + u_y &= 0 \\
    u(x,0) = k(x)
\end{align*}

Therefore, 

\begin{equation}
    \Gamma : \begin{cases}
        z = k(x) \\
        y=0
    \end{cases}
\end{equation}

Note, $z \iff u(x,y)$. This PDE has an associated autonomous system $(1)$. For each point $(x_0, 0, z_0)  \in \Gamma$, we can sketch the solution path
and the resulting surface is the \underline{graph} of the solution $u$.

\begin{proof}
    \begin{align}
        u(x,y) &= k(x_0) \\
        u(x,y) &= k(x-ay) \text{ (traveling wave)}
    \end{align}
    i.e. transport initial height to every point along the path. Note that we are defining an initial 1d curve $z=k(x)$, and then we can move this curve. This is a traveling wave.
\end{proof}

\subsection*{Solution to Cauchy Problem}

Given an autonomous 3-d system. Suppose an initial curve $\Gamma \in \mathbb{R}^3$. Then, we guess that $\Gamma$ would be moved somehow across $\mathbb{R}^3$. For each $(x_0,y_0,z_0) \in \Gamma$,
we sketch the \underline{characeristic curve}. We claim that the resulting surface $u(x,y)$ is a solution to the Cauchy problem. i.e, define $u(x,y)$ by $u(x(t), y(t)) = z(t)$ for all $t$ and for all characeristic curves starting at $\Gamma$. 

\subsubsection*{Check:}

Along each characeristic curve, 
\begin{align}
    c(x(t), y(t), z(t)) &= c = dz/dt\\
    &= u_x(x,y,z)dx/dt + u_x(x,y,z)dy/dt \\
    &= u_x a + u_y b
\end{align}

\textbf{But what can go wrong?:} (1) characeristics may not exist $\forall t$, like blow-up for example like $dz/dt = z^2$ which blows up in finite time.  

(2) Surface may fold, which then no longer becomes a graph (like a carpet that's been rolled up).

(3) $\Gamma$ itself may be a characeristic. In this case, you do not get a surface.

\subsection*{"Local" Theorem}

Our construction does solve the Cauchy problem \underline{near} $\Gamma$ if the projection of $\Gamma$ to the xy-plane is never parallel to 
$<a,b>$. In other words, given 

\begin{equation}
    \Gamma = \begin{cases}
        x=f(s) \\
        y= g(s)\\
        z=  h(s)
    \end{cases}
\end{equation}

We want $<f'(s), g'(s)>$ not parallel to $<a,b>$ (coefficients of the Cauchy problem). The point, the characeristics will truly flow away from $\Gamma$ and generate a surface for at least small $t$.

\begin{proof}
    Be intimidated. 
\end{proof}

\end{document}
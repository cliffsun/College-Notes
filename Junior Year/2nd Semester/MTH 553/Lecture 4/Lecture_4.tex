\documentclass{article}
\usepackage[left=2cm, right=2cm, top=2cm, bottom=2cm]{geometry}
\usepackage{graphicx}
\usepackage{amsmath}
\usepackage{amssymb}
\usepackage{amsthm}
\usepackage{fancyhdr}
\usepackage{verbatim}
\usepackage{listings}
\usepackage{xcolor}
\usepackage{pgfplots}
\usepackage{physics}

\lstset{
    language=C++,
    basicstyle=\ttfamily\footnotesize,
    keywordstyle=\color{blue},
    commentstyle=\color{green},
    stringstyle=\color{red},
    numbers=left,
    numberstyle=\tiny\color{gray},
    frame=single,
    breaklines=true,
    captionpos=b,
}


\title{MTH 553: Lecture \# 4}
\author{Cliff Sun}

\newtheorem{theorem}{Theorem}[section]
\newtheorem{lemma}[theorem]{Lemma}
\newtheorem{definition}[theorem]{Definition}
\newtheorem{conjecture}[theorem]{Conjecture}
\newtheorem{proposition}[theorem]{Proposition}
\newtheorem{corollary}[theorem]{Corollary}
\newtheorem{one minute paper}[theorem]{One Minute Paper}

\pagestyle{fancy}
\lhead{\textbf{\thepage}\ \ \nouppercase{\rightmark}}
\chead{MTH 553: Lecture \# 4}
\rhead{Cliff Sun}

\begin{document}

\maketitle

\section*{Last time}
\begin{itemize}
    \item Inviscid Burger's Equation
    \begin{align}
        uu_x + u_y &= 0 \\
        u(x,0) &= k(x) 
    \end{align}
\end{itemize}

\section*{Lecture Span}
\begin{enumerate}
    \item Method of general solutions
\end{enumerate}

\section*{Method of general solutions}

\textbf{Objective:} Find two expressions of $x,y$ that are constant along each characteristic. Consider an arbitrary function 
of these two expressions. Find $\Gamma$ to determine the solution function. 

\subsection*{Example}
\begin{align}
    uu_x + yu_y &= x \\
    u(x,1) &= 2x
\end{align}

Here, the characteristic system is 

\begin{align}
    dx/dt &= z \\
    dy/dt &= y \\
    dz/dt &= x
\end{align}

Here, we can find 

\begin{equation}
    \frac{dz}{dx} = \frac{x}{z} \iff zdz = xdx \iff \frac{1}{2}z^2 = \frac{1}{x^2} + C_1(s)
\end{equation}

Here, we can then define a constant function (dependent on where on $\Gamma$ you start on) 

\begin{equation}
    \phi(x,y,z) = z^2 - x^2
\end{equation}

Lets try adding $(5) + (7)$, then we obtain 

\begin{equation}
    \frac{d(x+z)}{dt} = (x+z)
\end{equation}

Then consider 

\begin{equation}
    \frac{d(x+z)}{dy} = \frac{x+z}{y} \iff \frac{dy}{y} = \frac{d(x+z)}{x+z}
\end{equation}
\begin{equation}
    \iff x+z = C_2(s)y
\end{equation}

We can then define another constant function (dependent on starting point on $\Gamma$) as 

\begin{equation}
    \psi(x,y,z) = \frac{x + z}{y}
\end{equation}

Then a general solution is 

\begin{equation}
    F(u^2 - x^2, \frac{x+u}{y}) = 0
\end{equation}

Where $F$ depicts a relationship between these values that originates on $\Gamma$.

We can determine $F$ from $\Gamma$, that is $u(x,1) = 2x$. Let $u=2x$, then we obtain 

\begin{equation}
    F(3x^2, 3x) = 0
\end{equation}

Satisfied by $F(a,b) = a - 1/3b^2$, therefore, we obtain 

\begin{equation}
    u-x^2 - \frac{1}{3}\left( \frac{x+u}{y} \right)^2 = 0
\end{equation}

Solving for $u$, we obtain 

\begin{equation}
    u = \frac{x \pm 3xy^2}{3y^2-1}
\end{equation}

To satisfy the initial condition, we choose $+$. Therefore, the final solution is 

\begin{equation}
    u = \frac{x + 3xy^2}{3y^2 - 1}
\end{equation}

The solution exists for $y > 1/\sqrt{3}$ and $\forall x$. \textbf{NOTE:} There is no uniqueness theory (yet) for non-linear equations.

\section*{Weak Solutions}

\subsection*{Jump condition}

This determines where "the shock goes". Given some smooth function $G(z)$, then define 

\begin{equation}
    G(u)_x + u_y = 0
\end{equation}

Similarly, 

\begin{equation}
    G'(u)u_x + u_y = 0
\end{equation}


For the conservation law, consider

\begin{equation}
    G(z) = \frac{1}{2}z^2 \implies G'(z) = z
\end{equation}

Then this is the burger's equation. We call $u$ a \underline{weak solution} of (19) if it satisifies the x-integrated version of (19). That is, $u$ satisfies:

\begin{equation}
    G(u(b,y)) - G(u(a,y)) + \frac{d}{dy}\int_{a}^{b}u(x,y)dx = 0
\end{equation}

for all $a<b$ and all $y$. But what's so cool about this? 

\begin{center}
    You can have discontinuities w.r.t $x$
\end{center}

If $u$ is a smooth solution that satisfies (19), then it satisfies (22), by using the fundamental theorem of Calculus. But the converse is false, because $u$ can have discontinuities and still be a weak solution.

\subsection*{Example}

Specifically, suppose $x= \xi(y)$ as our shock curve, which defines where this discontinuity occur. Let $u=u_l(y)$ be to the left of the shock curve and $u=u_r(y)$ be similarly defined for the right. 
Now, suppose $u(x,y)$ jumps across a $C^1$-smooth curve $x=\xi(y)$. But everywhere else, $u$ is a smooth solution. Then suppose $u$ solves the conservation law to the left and right of $x = \xi(y)$. Then, we plug this equation into $(22)$. 

\begin{equation}
    G(u(b,y)) - G(u(a,y)) + \frac{d}{dy}\left[ \int_{a}^{\xi(y)}u_l(x,y)dx + \int_{\xi(y)}^{b}u_r(x,y)dx \right] 
\end{equation}

$\forall a,b$ with $a<\xi(y)<b$ for all $y$. Need not check intervals of $[a,b]$ that don't contain $\xi(y)$. 

\begin{equation}
    \iff 0 = G(u(b,y)) - G(u(a,y)) + \int_{a}^{\xi(y)}u_y(x,y)dx + \int_{\xi(y)}^{b}u_y(x,y)dx + \xi'(y)u_l - \xi'(y)u_r
\end{equation}

By fundamental theorem of Calculus and chain rule. Where $$u_l = \lim_{x\rightarrow \xi-}u$$ and $u_r$ defined similarly. Moreover, $u_y = -G(u)_x$

\begin{equation}
    \iff 0 = -G(u_l) + G(u_r) + \xi'(y)\left( u_l - u_r \right)
\end{equation}

By evaluating the integrals. Then we obtain 

\begin{equation}
    \xi'(y) = \frac{G(u_l) - G(u_r)}{u_l - u_r}
\end{equation}

\end{document}
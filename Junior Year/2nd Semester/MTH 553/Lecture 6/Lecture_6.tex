\documentclass{article}
\usepackage[left=2cm, right=2cm, top=2cm, bottom=2cm]{geometry}
\usepackage{graphicx}
\usepackage{amsmath}
\usepackage{amssymb}
\usepackage{amsthm}
\usepackage{fancyhdr}
\usepackage{verbatim}
\usepackage{listings}
\usepackage{xcolor}
\usepackage{pgfplots}
\usepackage{physics}

\lstset{
    language=C++,
    basicstyle=\ttfamily\footnotesize,
    keywordstyle=\color{blue},
    commentstyle=\color{green},
    stringstyle=\color{red},
    numbers=left,
    numberstyle=\tiny\color{gray},
    frame=single,
    breaklines=true,
    captionpos=b,
}


\title{MTH 553: Lecture \# 6}
\author{Cliff Sun}

\newtheorem{theorem}{Theorem}[section]
\newtheorem{lemma}[theorem]{Lemma}
\newtheorem{definition}[theorem]{Definition}
\newtheorem{conjecture}[theorem]{Conjecture}
\newtheorem{proposition}[theorem]{Proposition}
\newtheorem{corollary}[theorem]{Corollary}
\newtheorem{one minute paper}[theorem]{One Minute Paper}

\pagestyle{fancy}
\lhead{\textbf{\thepage}\ \ \nouppercase{\rightmark}}
\chead{MTH 553: Lecture \# 6}
\rhead{Cliff Sun}

\begin{document}

\maketitle

\section*{Lecture Span}
\begin{itemize}
    \item Traffic Flow
\end{itemize}

\section*{Traffic Flow}

Let $\rho(x)$ be the density of cars on the highway per unit length and $q(r,t)$ is the rightward flux of the cars per unit time. The conservation 
of cars between $a,b$ is 

\begin{equation}
    q(b,t) - q(a,t) = -\frac{d}{dt}\int_{a}^{b}\rho(x,t)dx
\end{equation}

Assume that the traffic speed is a function of $\rho$. Therefore, the flux is density times speed and therefore, a function of density. 

That is, $q(x,t) = G(\rho(x,t))$. Then the conservation law states that $\rho$ is a weak solution of 

\begin{equation}
    G(\rho)_x + \rho_t = 0
\end{equation}

The above is a conservation law. 

\subsection*{Example}

Assumption: Traffic speed is $c$ (constant), a reasonable assumption for low density. Then 

\begin{equation}
    G(\rho) = c\rho
\end{equation}

We then get the \underline{transport equation}. 

\begin{equation}
    c\rho_x + \rho_t = 0
\end{equation}

The solutions are $x=ct+x_0$. So the initial density just gets translated along with the traffic. 

\subsection*{Example}

Under heavier traffic conditions, assume traffic speed is $$c\left( 1 - \frac{\rho}{\rho_{\max}} \right)$$ i.e. the speed decreases linearly with respect to density. Flux is density times speed. Therefore, 

\begin{equation}
    q = c\rho\left( 1 - \frac{\rho}{\rho_{\max}} \right)
\end{equation}

This is a concave function. 

\subsection*{Example}

Traffic flow after a red light turns green. Therefore, 

\begin{equation}
    \rho(x,0) = \begin{cases}
        \rho_{\max} & x < 0 \\
        0 & x > 0
    \end{cases}
\end{equation}

Characteristic: $x = G'(\rho)t + x_0$. Solution: 

\begin{equation}
    G'(\rho) = c\left( 1- \frac{2\rho}{\rho_{\max}} \right)
\end{equation}

Then $G'(\rho_{\max}) = -c$ and $G'(0) = c$. In the fan, we try some function $\rho = v(x/t)$, note that $x/t$ is constant out of the origin. 
Substituting into the conservation equation yields 

\begin{equation}
    c\left( 1 - \frac{2\rho}{\rho_{\max}} \right)v'(x/t)/t + v'(x/t)(-x/t^2) = 0
\end{equation}

Therefore, 

\begin{equation}
    G'(\rho) = c\left( 1-\frac{\rho}{\rho_{\max}} \right) = \frac{x}{t}
\end{equation}

We then get 

\begin{equation}
    \rho =\frac{\rho_{\max}}{2}\left( 1 - \frac{x}{ct} \right)
\end{equation}

\subsection*{Interpretation}

The first car in the line travels at speed $c$. But any car at $x_0 = -ct_0$ must wait until time $t_0$ in order to move. What path does it follow? 

\textbf{Note:} the characteristics are NOT the trajectory of the vehicles, but are rather the lines of constant density. 

\begin{definition}
    $G$ is uniformly convex if $G'' \geq a > 0$ for some constant $a$. Similar definition for concave $G'' \leq b < 0$ for $b$ is a constant.  
\end{definition}

If $G$ is uniformly convex, then $G'$ is strictly increasing. Then $u_l > u_r$. Vice versa for $G$ uniformly concave. 

\end{document}
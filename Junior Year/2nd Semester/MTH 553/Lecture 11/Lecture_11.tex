\documentclass{article}
\usepackage[left=2cm, right=2cm, top=2cm, bottom=2cm]{geometry}
\usepackage{graphicx}
\usepackage{amsmath}
\usepackage{amssymb}
\usepackage{amsthm}
\usepackage{fancyhdr}
\usepackage{verbatim}
\usepackage{listings}
\usepackage{xcolor}
\usepackage{pgfplots}
\usepackage{physics}

\lstset{
    language=C++,
    basicstyle=\ttfamily\footnotesize,
    keywordstyle=\color{blue},
    commentstyle=\color{green},
    stringstyle=\color{red},
    numbers=left,
    numberstyle=\tiny\color{gray},
    frame=single,
    breaklines=true,
    captionpos=b,
}


\title{MTH 553: Lecture \# 11}
\author{Cliff Sun}

\newtheorem{theorem}{Theorem}[section]
\newtheorem{lemma}[theorem]{Lemma}
\newtheorem{definition}[theorem]{Definition}
\newtheorem{conjecture}[theorem]{Conjecture}
\newtheorem{proposition}[theorem]{Proposition}
\newtheorem{corollary}[theorem]{Corollary}
\newtheorem{one minute paper}[theorem]{One Minute Paper}

\pagestyle{fancy}
\lhead{\textbf{\thepage}\ \ \nouppercase{\rightmark}}
\chead{MTH 553: Lecture \# 11}
\rhead{Cliff Sun}

\begin{document}

\maketitle

\section*{Lecture Span}
\begin{itemize}
    \item Reflection Method
\end{itemize}

\section*{Reflection Method}

Similar to the parallelogram method. Let $g,h$ be defined for $[0,L]$ 

\subsection*{Dirichlet Boundary Conditions}

Extend $g,h$ by odd reflection over $[-L,L]$ with period $2L$ and extend periodically over $\mathbb{R}$. Solve with d'Alembert's formula. Here, imagine a picture where $h \equiv 0$ and $g_2$ is a rectangular pulse. 
Now, imagine performing an odd reflection on $g$ so now there is an inverted rectangular pulse over the $y$ axis. Now, imagine that there is a period of $2L$.    

You can imagine that as $t$ progresses, the rectangular pulses cancel out at $x=0$. So now, to the observer, it looks like the square has been reflected upside down. 

\subsection*{Neumann Boundary Conditions}

Now, let's do an even extension over $[-L,L]$ and extend periodically. Then, the square is reflected the same way up, but reflected over the $y$ axis. (like a parity transformation)

\textbf{Note:} for physical intuition of the BCs, look at Farlow handout on Canvas (week 4). 

\section*{Adjoints and weak solutions}

\begin{theorem}
    Divergence theorem: 
    \begin{equation}
        \int_{\Omega}\nabla \cdot \vec{F} dx = \int_{\partial\Omega}\vec{F} \cdot \vec{v}dS
    \end{equation}
    Where $\vec{v}$ is a normal vector to $\partial \Omega$. This is a version of the fundamental theorem of Calculus but in three dimensions.
\end{theorem}

Next, we apply the div theorem to derive a generalized integration by parts formula with $u,r \in C^1(\overline{\Omega})$ and $\vec{F} = \langle 0,\dots, uv, \dots, 0\rangle$. So 

\begin{equation}
    \nabla \cdot \vec{F} = \partial_{x_n}\left( uv \right) = (\partial_{x_n}u)v + u(\partial_{x_n}v)
\end{equation}

Next, we have that 

\begin{equation}
    F \cdot v = uv v_k
\end{equation}

Therefore, we can derive an integration by parts formula

\begin{equation}
    \int_{\Omega}\frac{\partial u}{\partial x_k}v dx = -\int_{\Omega}u\frac{\partial v}{\partial x_k}dx + \int_{\partial\Omega}uv v_k dS
\end{equation}

If $v = 0$ on $\partial \Omega$, then we have that 

\begin{equation}
    \int_{\Omega}\frac{\partial u}{\partial x_k} v dx = -\int_{\Omega}u\frac{\partial v}{\partial x_k}dx
\end{equation}

Repeating, then we get 

\begin{equation}
    \int_{\Omega}(D^{\alpha}u)v dx = (-1)^{|\alpha|}\int_{\Omega}u (D^{\alpha}v)dx
\end{equation}

For all $u \in C^{|\alpha|}(\Omega)$ and for all $v \in C_{0}^{|\alpha|}(\Omega)$ where $v$ is zero for some neighborhood around the boundary. Here, $\alpha = (\alpha_1, \alpha_2, \dots)$ and $\alpha_i \in \mathbb{Z}^{+} \geq 0$. 

Where $\alpha_i$ determines the number of times that you take the partial derivative with respect to $x_i$. In other words, 

\begin{equation}
    D^{\alpha} \equiv \Pi^{n}\left( \partial_{x_i} \right)^{\alpha_i}
\end{equation}

And 

\begin{equation}
    |\alpha| = \alpha_1 + \dots + \alpha_n
\end{equation}

is the order of the multi-index $\alpha$. As an example,

\begin{equation}
    u_{x_1,x_2} = D^{(1,1)}u
\end{equation}

Where $\alpha = (1,1)$. Then 

\begin{equation}
    \sum_{\alpha : |\alpha| = 2} a_{\alpha}D^{\alpha}u = a_{2,0}u_{x_1,x_1} + a_{1,1}u_{x_1,x_2} + a_{0,2}u_{x_2,x_2}
\end{equation}

Is a linear combination of all partial derivatives. Consequence, let $m \leq 1$, then 

\begin{equation}
    Lu = \sum_{\alpha : |\alpha| \leq m} a_{\alpha}(x)D^{\alpha}u
\end{equation}

Where $m$ denotes the maximum number of partial derivatives you have to take in any direction. Then 

\begin{equation}
    \int_{\Omega}(Lu)v dx = \int_{\Omega}u(L'v)dx
\end{equation}

Where $L'$ is the adjoint operator such that 

\begin{equation}
    L' v = \sum_{|\alpha| \leq m}(-1)^{|\alpha|}D^{\alpha}(a_\alpha(x)v)
\end{equation}

Proof, use the integration by parts formuala, but let $v = a_{\alpha}v$. Note that $v \in c^{m}_{0}(\Omega)$ where $v$ vanishes near $\partial \Omega$. 

\subsection*{Examples}

\begin{enumerate}
    \item $Lu = u_{x_1,x_1} \implies L'v = v_{x_1,x_1}$. Here, $L = L'$, so it is "formally self-adjoint".  
    \item $\nabla^2 u = u_{x_1,x_1} + u_{x_2,x_2}$. It is clear that $\nabla^2 = \nabla'^2$
    \item $Lu = u_{x_1}$, then $L'v = v_{x_1}$. So $L \neq L'$. (because $(-1)^{|\alpha|} = -1$ since $\alpha = (1,0)$)
\end{enumerate}
 
\begin{definition}
    Let $f \in L'(\Omega)$ be an integrable function. (integral of $f$ is finite over $\Omega$). Then we call $u$ a \underline{classical solution} of $Lu = f$ if 
    \begin{equation}
        u \in C^{m}(\Omega) \text{ and } Lu = f
    \end{equation}
    Call $u$ a \underline{weak} solution of $Lu = f$ if 
    \begin{equation}
        u \in L^{1}(\Omega) \text{ and } \int_{\Omega}u(L'v) = \int_{\Omega}fv dx
    \end{equation}
    for all $v \in C^{m}_{0}(\Omega)$. 
\end{definition}

\textbf{Property:} Classical $\implies$ weak. i.e. if $Lu = f$ classically then $Lu = f$ weakly. 

\begin{proof}
    Substitute $Lu = f$ and use the integration by parts theorem. 
\end{proof}

\textbf{Property (2):} Weak + smooth function $\implies$ classical solution. i.e. if $u$ is a weak solution and is also differentiable enough times, then it is also a classical solution. 

\begin{proof}
    Use integration by parts and also rearrange terms. 
\end{proof}

\end{document}
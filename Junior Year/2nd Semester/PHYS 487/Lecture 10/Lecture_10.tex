\documentclass{article}
\usepackage[left=2cm, right=2cm, top=2cm, bottom=2cm]{geometry}
\usepackage{graphicx}
\usepackage{amsmath}
\usepackage{amssymb}
\usepackage{amsthm}
\usepackage{fancyhdr}
\usepackage{verbatim}
\usepackage{listings}
\usepackage{xcolor}
\usepackage{pgfplots}
\usepackage{physics}

\lstset{
    language=C++,
    basicstyle=\ttfamily\footnotesize,
    keywordstyle=\color{blue},
    commentstyle=\color{green},
    stringstyle=\color{red},
    numbers=left,
    numberstyle=\tiny\color{gray},
    frame=single,
    breaklines=true,
    captionpos=b,
}


\title{PHYS 487: Lecture \# 10}
\author{Cliff Sun}

\newtheorem{theorem}{Theorem}[section]
\newtheorem{lemma}[theorem]{Lemma}
\newtheorem{definition}[theorem]{Definition}
\newtheorem{conjecture}[theorem]{Conjecture}
\newtheorem{proposition}[theorem]{Proposition}
\newtheorem{corollary}[theorem]{Corollary}
\newtheorem{one minute paper}[theorem]{One Minute Paper}

\pagestyle{fancy}
\lhead{\textbf{\thepage}\ \ \nouppercase{\rightmark}}
\chead{PHYS 487: Lecture \# 10}
\rhead{Cliff Sun}

\begin{document}

\maketitle

\section*{Lecture Span}
\begin{itemize}
    \item Angular Momentum
\end{itemize}

\section*{Recall}

The perturbation to the Hamiltonian is $$H' \propto J_1 \cdot J_2$$

Where $J_i$ is the angular momentum of a particle. Moreover, 

$$[H', J_1] \neq 0 \neq [H', J_2]$$

So, then we found that we want to understand this Hamiltonian from the basis of the \underline{total angular momentum} such that $J = J_1 + J_2$. For individual A.M, we have that 

\begin{equation}
    \ket{j_1, m_1, j_2, m_2} = \ket{j_1,m_1} \otimes \ket{j_2, m_2}
\end{equation}

This means that the total number of states is $$j_1(j_1 + 1) \cdot j_2(j_2 + 1) \rightarrow (2j_1 + 1) \cdot (2j_2 + 1)$$

\subsection*{Kronecker Products}

Suppose two vectors $$\ket{\alpha} \otimes \ket{\beta} = \begin{pmatrix}
    \alpha_1 \\
    \alpha_2
\end{pmatrix} \otimes \begin{pmatrix}
    \beta_1 \\
    \beta_2
\end{pmatrix} = \begin{pmatrix}
    \alpha_1\beta_1 \\
    \alpha_1\beta_2 \\
    \alpha_2 \beta_1 \\
    \alpha_2 \beta_2
\end{pmatrix}$$

For example, suppose $S_1 \cdot S_2$. Clearly, $S_1$ only acts on the first particle and so on. Both live in different subspaces. Formally, this is the same as 

\begin{equation}
    S_1 \cdot S_2 =\vec{ S_1} \otimes \vec{\mathbb{I}} \cdot \vec{\mathbb{I}} \otimes S_2
\end{equation}
\begin{equation}
    = \begin{pmatrix}
        S_{1,x} \otimes \mathbb{I} \\
        S_{1,y} \otimes \mathbb{I} \\
        S_{1,z} \otimes \mathbb{I} \\
    \end{pmatrix} \cdot \begin{pmatrix}
        \mathbb{I} \otimes S_{2,x} \\
        \mathbb{I} \otimes S_{2,y} \\
        \mathbb{I} \otimes S_{2,z} \\
    \end{pmatrix}
\end{equation}

\section*{Correct quantum numbers}

Commuting operators:

\begin{equation}
    [J^2, J_z] = 0
\end{equation}

But, 

\begin{equation}
    [J_{1,z}, J^2] \neq 0
\end{equation}

\begin{equation}
    [J^2, J^2_{1 \lor 2}] = 0
\end{equation}

But we said that 

\begin{equation}
    J_1 \cdot J_2 \propto (J^2 - J_1^2 - J_2^2)
\end{equation}

Now, now there is more quantum numbers, so define a state 

\begin{equation}
    \ket{j, m_1, j_1, j_2}
\end{equation}

We now conduct perturbation theory with these states. Formally, this is a basis change. In other words: 

\begin{align*}
    \ket{j,m_j, j_1, j_2} &= \sum_{m_1,m_2}\ket{j_1,m_1,j_2,m_2}\braket{j_1,m_1,j_2,m_2}{j,m_j,j_1,j_2} \\
    &= \sum_{m_1,m_2}\braket{j_1,m_1,j_2,m_2}{j,m_j,j_1,j_2}\ket{j_1,m_1,j_2,m_2}
\end{align*}

These are the Clebsch-Gordon coefficients. We note several things. Firstly, $j_1,j_2$ are always fixed (like charge of a particle fixed). Therefore, for brevity

\begin{equation}
    \ket{j,m_j,j_1,j_2} \rightarrow \ket{j,m_j}
\end{equation}

For each $j$, we have that $m_j = -j,-j+1,\dots, +j$. Moreover, $j$ (total angular momentum) can take on values of $|j_1 - j_2|, |j_1-j_2| + 1, \dots, j_1 + j_2$. From this, we note that $m$ in general is not unique, therefore, 
the Clebsch gordon coefficient tables become disgusting. Moreover:

\begin{equation}
    J_z \ket{j_1,m_1,j_2,m_2} = (J_{1,z} + J_{2,z})\ket{j_1,m_1,j_2,m_2} = \hbar(m_1 + m_2)\ket{j_1,m_1,j_2,m_2}
\end{equation}

So now, the question is how can I move the basis from $\ket{j,m_j, j_1,j_2}$ to $\ket{j_1,m_1,j_2,m_2}$. The procedure is the following:

\begin{enumerate}
    \item $j= j_1 + j_2$ and $m_j = -j, \dots, +j$
    \item $\ket{(j_1+j_2), j_1 + j_2} = \ket{j_1,j_1,j_2,j_2}$ [!] (basis transformatin)
    \item Apply ladder operators $J_{-} = J_{1,-} + J_{2,-}$ (iteratively) Recall that $$J_{\pm}\ket{j,m} = \sqrt{j(j+1) - m(m\pm 1)}\hbar \ket{j, m \pm 1}$$ 
    Then, find all $m_1,m_2$ such that $m_1 + m_2 = m$.  
    \item Next, find $j = j_1 + j_2 - 1$
\end{enumerate}

\subsection*{Example, two spin 1/2's (electrons)}

Let $\vec{S} = \vec{S_1} + \vec{S_2}$. Now, consider 

\begin{equation}
    \ket{s=1, m_s=1} = \ket{s_1 = 1/2, m_1 = 1/2, s_2 = 1/2, m_2 = 1/2}
\end{equation}

Now, use the ladder operator. Consider 

\begin{align*}
    S_{-}\ket{11} &= \left( S_{1,-} + S_{2,-} \right)\left( \ket{\frac{1}{2},\frac{1}{2}} \otimes \ket{\frac{1}{2}, \frac{1}{2}} \right) \\
    &= \hbar \left( \ket{\frac{1}{2}, -\frac{1}{2}} \otimes \ket{\frac{1}{2}, \frac{1}{2}}  + \ket{\frac{1}{2}, \frac{1}{2}} \otimes \ket{\frac{1}{2}, -\frac{1}{2}}\right)
\end{align*}

Or equivalently, 

\begin{equation}
    \ket{10} = \frac{1}{\sqrt{2}}\left( \ket{\downarrow} \otimes \ket{\uparrow} + \ket{\uparrow} \otimes \ket{\downarrow} \right)
\end{equation}

Then, 

\begin{equation}
    S_{-}\ket{10} = \ket{1,-1} = \ket{\frac{1}{2},-\frac{1}{2}} \otimes \ket{\frac{1}{2}, -\frac{1}{2}}
\end{equation}

Finally, the starting point $\ket{00}$, orthogonal to $\ket{10}$. This has to be 

\begin{equation}
    \frac{1}{\sqrt{2}}\left( \ket{\frac{1}{2}, -\frac{1}{2}} \otimes \ket{\frac{1}{2}, \frac{1}{2}}  - \ket{\frac{1}{2}, \frac{1}{2}} \otimes \ket{\frac{1}{2}, -\frac{1}{2}}\right)
\end{equation}



\end{document}
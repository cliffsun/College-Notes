\documentclass{article}
\usepackage[left=2cm, right=2cm, top=2cm, bottom=2cm]{geometry}
\usepackage{graphicx}
\usepackage{amsmath}
\usepackage{amssymb}
\usepackage{amsthm}
\usepackage{fancyhdr}
\usepackage{verbatim}
\usepackage{listings}
\usepackage{xcolor}
\usepackage{pgfplots}
\usepackage{physics}

\lstset{
    language=C++,
    basicstyle=\ttfamily\footnotesize,
    keywordstyle=\color{blue},
    commentstyle=\color{green},
    stringstyle=\color{red},
    numbers=left,
    numberstyle=\tiny\color{gray},
    frame=single,
    breaklines=true,
    captionpos=b,
}


\title{PHYS 487: Lecture \# 2}
\author{Cliff Sun}

\newtheorem{theorem}{Theorem}[section]
\newtheorem{lemma}[theorem]{Lemma}
\newtheorem{definition}[theorem]{Definition}
\newtheorem{conjecture}[theorem]{Conjecture}
\newtheorem{proposition}[theorem]{Proposition}
\newtheorem{corollary}[theorem]{Corollary}
\newtheorem{one minute paper}[theorem]{One Minute Paper}

\pagestyle{fancy}
\lhead{\textbf{\thepage}\ \ \nouppercase{\rightmark}}
\chead{PHYS 487: Lecture \# 2}
\rhead{Cliff Sun}

\begin{document}

\maketitle

\section*{Lecture Span}
\begin{itemize}
    \item Symmetries; invariance under transformations
    \item transformations in space
    \item generators
    \item translational invariance
\end{itemize}

\section*{What is a symmetry?}

A symmetry exists if applying a transformation to an object will not change anything observable of the object. There exists discrete symmetries (flipping, rotating by $\pi/2$) and continuous
symmetries (circle). 

\section*{Describing transformations}

Suppose 

\begin{equation}
    A\ket{\psi} = \ket{\psi'}
\end{equation}

Where $A$ is some operator that acts on the wave function. If $A$ acts on a symmetry of the wave function, then its expectation value shouldn't change. That is 

\begin{equation}
        \langle O \rangle = \langle \psi | O | \psi \rangle = \langle \psi | A^{\dagger} O A | \psi \rangle 
\end{equation}

If this is true, then we say that $\langle O \rangle$ is unchanged under the transformation of $A$. This formalism is equivalent to a relative transformation, either transforming $\ket{\psi}$ or $\hat{O}$. Eitherway, they are equivalent. 
We call a transformation of the wavefunction an \underline{"active transformation"} and the transformation of the observable as a \underline{"passive transformation"}. 

\section*{Symmetries + Hamiltonian}

A wavefunction is translationally invariant if its Hamiltonian is also translationally invariant. 

\begin{center}
    Invariance: $A^{\dagger}HA = H' = H$
\end{center}

We first define "unitarity", that is $A^{\dagger}A = AA^{\dagger} = \mathbb{I}$. That is $A^{\dagger} = A^{-1}$. 
\begin{align*}
    A^{\dagger}HA &= H \\
    AA^{\dagger}HA &= AH \text{ (assume $A$ is unitary)}\\
    HA &= AH \\
    [\hat{H}, \hat{A}] &= 0
\end{align*}

This proof is saying what (by default $A$ is unitary), that if $\hat{H}$ is invariant under $A$, then $[\hat{H}, \hat{A}] = 0$.

A unitary is reversible and energy conserving. They also preserve inner product: 

\begin{equation}
    \langle \psi | A^{\dagger}A | \psi \rangle = \braket{\psi}{\psi}
\end{equation}

\section*{Transformations in space}

\begin{enumerate}
    \item Translation
    \begin{equation}
        \hat{T}(a)\psi(x) = \psi'(x) = \psi(x-a)
    \end{equation}
    \item Rotation
    \begin{equation}
        R_z(\varphi)\psi(r,\theta,\phi) = \psi(r,\theta,\phi - \varphi)
    \end{equation}
    \item Parity: reflection about the origin
    \begin{equation}
        \hat{\Pi}\psi(r) = \psi(-r)
    \end{equation}
\end{enumerate}

\section*{Translation operator}

\begin{equation}
    T(a)\psi(x) = \psi(x-a)
\end{equation}

Maybe momentum? Recall, momentum: $\hat{p} = -i\hbar \partial_x$. Let's try to taylor expand $T\psi$ by expanding $\psi(x-a)$. Recall that a taylor expansion about $a$ is 

\begin{equation}
    f(x) \approx \sum_{n}\frac{f'(a)}{n!}(x-a)^n
\end{equation}

Then let $h = x-a$, then we obtain 

\begin{equation}
    f(a + h) \approx \sum_n \frac{f'(a)}{n!}(h)^n
\end{equation}

Note that $a$ can be $x$. 

\begin{align*}
    &\sum_{n=0}^{\infty}\frac{1}{n!}\frac{d^n}{dx^n}\psi(x) ((-a)^{n}) \\
    &= \sum_n \frac{1}{n!}\left( -\frac{ia}{\hbar}\hat{p} \right)^n \psi \\
    &= T(a) = \exp\left( -\frac{ia}{\hbar}\hat{p} \right)
\end{align*}

We then call $\hat{p}$ a generator.

\subsection*{Properties}

\begin{enumerate}
    \item Unitarity: $T^{-1}(a) = T(-a) = T(a)^{\dagger}$
    \begin{center}
        if $\hat{Q}$ is hermitian, then $u = \exp(iQ)$ is unitary.
    \end{center}
    \item Action on $\hat{x}$
    \begin{align}
        T^{\dagger}(a)\hat{x}T(a) &= \hat{x} + a = \hat{x}' \\
        \hat{x}'\psi(x) &= T^{\dagger}(a)\hat{x}T(a)\psi(x)\\
        &= T(-a)x\psi(x-a) \\
        &= (x+a)\psi(x) \text{ (assuming that $\psi$ is invariant under $T$)}
    \end{align}
\end{enumerate}

In general, $T^{\dagger}Q(x,p)T = Q(x + a, p)$. Note, in more dimensions, stepping $\Delta x_i$ in any dimension wouldn't change the observable of the object. 
If $[A,B] = 0$, then $\exp(A+B) =\exp(A)\exp(B)$. But if $[A,B] \neq 0$, then there is a \underline{geometric phase} acquired when stepping $\Delta x_i$. 

\section*{Translational Symmetry}

\begin{enumerate}
    \item Continuous: free particle
    \item Periodic (and infinite)
\end{enumerate}

Consider a infinitely periodic potential with a period. Then consider 

\begin{align}
    H' &= T^{\dagger}(a)HT(a) = T^{\dagger}(a)\left( \frac{p^2}{2m} + V(x) \right)T(a)\\
    &= \frac{p^2}{2m} + T^{\dagger}(a)V(x)T(a) = \frac{p^2}{2m}V(x+a)
\end{align}

\section*{Bloch's Theorem}

Know: $[\hat{H}, \hat{T}] = 0$. And assume, 

\begin{equation}
    H\psi(x) = E\psi(x)
\end{equation}

Then because $H$ and $T$ commute, they share a simulatenous set of eigenvalues. Therefore, assume 

\begin{equation}
    T(a)\psi(x) = \lambda\psi(x)\;\; \lambda \in \mathbb{C}
\end{equation}

But since $T$ preserves the inner product, 

\begin{equation}
    \langle \psi | T^{\dagger}T | \psi \rangle = \braket{\psi}{\psi}
\end{equation}
\begin{equation}
    \lambda^{2} = \braket{\psi}{\psi} \implies \lambda = \exp(i\phi)\;\; \phi \in \mathbb{R}
\end{equation}

Let $\phi = qa$, then 

\begin{equation}
    \psi(x-a) = e^{-iqa}\psi(x)
\end{equation}

We can call $\hbar q$ the "crystal momentum". Then let $\psi(x) = e^{iqx}u(x)$ 

\begin{equation}
    e^{-iqa}e^{iqx}u(x) = e^{iq(x-a)}u(x)
\end{equation}
Then shift by $a$
\begin{equation}
    \psi(x) = e^{iqx}u(x+a)
\end{equation}
Thus, we obtain 
\begin{equation}
    e^{iqx}u(x) = e^{iqx}u(x+a)
\end{equation}

This means that $u$ is periodic in $a$. 

\end{document}
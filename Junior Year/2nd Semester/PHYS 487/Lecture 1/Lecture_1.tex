\documentclass{article}
\usepackage[left=2cm, right=2cm, top=2cm, bottom=2cm]{geometry}
\usepackage{graphicx}
\usepackage{amsmath}
\usepackage{amssymb}
\usepackage{amsthm}
\usepackage{fancyhdr}
\usepackage{verbatim}
\usepackage{listings}
\usepackage{xcolor}
\usepackage{pgfplots}
\usepackage{physics}

\lstset{
    language=C++,
    basicstyle=\ttfamily\footnotesize,
    keywordstyle=\color{blue},
    commentstyle=\color{green},
    stringstyle=\color{red},
    numbers=left,
    numberstyle=\tiny\color{gray},
    frame=single,
    breaklines=true,
    captionpos=b,
}


\title{PHYS 487 Lecture \# 1}
\author{Cliff Sun}

\newtheorem{theorem}{Theorem}[section]
\newtheorem{lemma}[theorem]{Lemma}
\newtheorem{definition}[theorem]{Definition}
\newtheorem{conjecture}[theorem]{Conjecture}
\newtheorem{proposition}[theorem]{Proposition}
\newtheorem{corollary}[theorem]{Corollary}
\newtheorem{one minute paper}[theorem]{One Minute Paper}

\pagestyle{fancy}
\lhead{\textbf{\thepage}\ \ \nouppercase{\rightmark}}
\chead{PHYS 487 Lecture \# 1}
\rhead{Cliff Sun}

\begin{document}

\maketitle

\section*{PHYS 487 Plan}

We will use techniques and examples to analyze realistic simple systems.

\subsection*{Outline}

\begin{enumerate}
    \item Symmetries + conservation laws 
    \begin{enumerate}
        \item Central symmetric problems (rotational symmetries)
        \item Periodic potentials (crystal)
    \end{enumerate}
    \item Time-independent Perturbation Theory
    \begin{enumerate}
        \item Fine-structure of hydrogen
    \end{enumerate}
    \item Dynamics
    \begin{enumerate}
        \item Emission and absorption of light
    \end{enumerate}
    \item Multi-particle quantum systems: approximations to place boundaries 
    \begin{enumerate}
        \item Ground state energies estimation
    \end{enumerate}
    \item Perhaps some bonus topics on some cool stuff
\end{enumerate}

\section*{QM Speedrun Recap}

\begin{enumerate}
    \item Physical system $\ket{\psi} \in \mathcal{H}$
    \item Linearity/Superposition principle $\ket{\psi} = \sum_{k}c_k\ket{\psi_k}$ where $\ket{\psi_k}$ are the basis vectors of the same operator.
    \item Inner product: $\braket{\varphi}{\psi} = \braket{\psi}{\varphi}^{*}$ such that $\braket{\psi}{\psi} \leq 0$
    \item Observables: $\hat{O}$ operators; Linearity $\hat{O}\ket{\psi} = \hat{O}\left( \sum_kc_k\ket{\psi_k} \right) = \sum_kc_k\left( \hat{O}\ket{\psi_k} \right)$; Hermitian: $\hat{O}^{\dagger} = \hat{O}$, 
    defined as $\braket{\psi}{\hat{O}^{\dagger}\psi} = \braket{\hat{O}\psi}{\psi}$
    \item Eigenvalues/vectors via diagonalization: $\hat{O}\ket{\alpha_n} = a_n\ket{\alpha_n}$. Where $a_n$ are the possible measurement outcomes and $\ket{\alpha_n}$ is a complete basis set for $\ket{\psi}$.
    \item Born's Rule: $|c_k|^2$ is the probability for outcome $c_k$. 
    \item Wavefunction Collapse: $\ket{\psi} \rightarrow \ket{\psi_k}$ after measuring. 
    \item Incompatibility: Observables $A,B$. If $[A,B] \neq 0$, then there doesn't exist an eigenbasis where both $A$ and $B$ are simulatenously diagonal. This means that the measurement of $A$ will disturb that of $B$ and vice versa. 
    \item Uncertainty principle: recall $\langle A \rangle = \braket{\psi}{A\psi}$ and $\sigma_A^2 = \langle A^2 \rangle - \langle A \rangle^2$, then 
    \begin{equation}
        \sigma_A^2\sigma_B^2 \leq \left( \frac{1}{2i}\langle [A,B]\rangle \right)^2 
    \end{equation}
\end{enumerate}

\subsection*{Example Systems}

\begin{enumerate}
    \item Continuous systems: $\mathcal{H} = L^2$. $\braket{\psi}{\varphi} = \int dx \varphi(x)^{*}\psi(x)$. In the 1-d position basis, $\psi(x) = \braket{x}{\psi}$
\end{enumerate}

When expressing an operator acting on a function in the position basis, consider (note that $\braket{x}{\hat{x}\psi}$ can be thought of as (first act $\hat{x}$ on $\psi$, then find this vector in the position basis))

\begin{align*}
    \braket{x}{\hat{x}\psi} &= \int dx' \langle x | \hat{x} | x' \rangle \braket{x'}{\psi} \\
    &= \int dx' x'\delta(x - x')\psi(x) \\
    &= x'\psi(x')
\end{align*}

The expectation value is

\begin{equation}
    \int dx \psi^{*}(x)x\psi(x)
\end{equation}

Momentum is 

\begin{equation}
    \langle x | \hat{p} | \psi \rangle = -i\hbar \frac{\partial}{\partial x}\psi(x)
\end{equation}

This is interpretated as acting the momentum operator on $\psi$ (abstract), then making it more concrete by turning this abstract concept into the position basis. Then, the Hamiltonian is 

\begin{equation}
    \hat{H} = \frac{\hat{p}^2}{2m} + V(x)
\end{equation}

\subsubsection*{Discrete Systems}

$\mathcal{H} = \mathbb{C}^n$, $$\vec{c} = \begin{pmatrix}
    c_1\\
    c_2\\
    \dots\\
    c_n
\end{pmatrix}$$

And $\braket{c}{d} = \sum_k c^{*}_k d_k$

\subsection*{Simplest Physical System - Spin 1/2}

\begin{equation}
    \ket{\chi} = c_1 \underbrace{\ket{\uparrow}}_{\begin{pmatrix}
        1 \\
        0
    \end{pmatrix}} + c_2 \underbrace{\ket{\downarrow}}_{\begin{pmatrix}
        0\\
        1
    \end{pmatrix}}
\end{equation}

Recall that the electron is $\ket{\psi}\ket{\chi}$. Operators for spin-1/2:

\begin{equation}
    S_z = \frac{\hbar}{2}\sigma_z = \frac{\hbar}{2}\begin{pmatrix}
        1 & 0 \\
        0 & -1
    \end{pmatrix}
\end{equation}

$S_x, S_y$ are defined similarly. 

\subsubsection*{Angular momentum}

\begin{equation}
    L = r \times p
\end{equation}

Where $r = \begin{pmatrix}
    \hat{x} \\
    \hat{y} \\
    \hat{z}
\end{pmatrix}$ and $p = -i\hbar\begin{pmatrix}
    \partial_x \\
    \partial_y \\
    \partial_z
\end{pmatrix}$. For angular momentum, $[J_i, J_j] = i\hbar \epsilon_{ijk}J_k$. Eigenstates $\ket{j,m}$ where $j$ corresponds to the total angular momentum ($J^2$) and $m$ is the z-projection of the angular momentum $J_z$. Then 

\begin{equation}
    J^2 \ket{j,m} = \hbar^2j(j+1)\ket{j,m}
\end{equation}
\begin{equation}
    J_z\ket{j,m} = \hbar m \ket{j,m}
\end{equation}

Note, $j = 0, 1/2, 1, 3/2, \dots$ and $m = -j, -j+1, \dots, j$. Ladder operators $J_{\pm} = J_x \pm i J_y$ raise and lower $m$ by $\pm 1$. 

\subsection*{Dynamics}

\begin{equation}
    i\hbar \partial_t \ket{\psi} = \hat{H}\ket{\psi}
\end{equation}

And the stationary states are the eigenstates of the Hamiltonian: $\hat{H}\ket{\psi_n} = E_n \ket{\psi_n}$ where $\ket{\psi_n}$ is the stationary state. Given $\ket{\psi(0)}$, then 
\begin{enumerate}
    \item Find $\ket{\psi_n}$
    \item $\ket{\psi(0)} = \sum_n c_n \ket{\psi_n}$
    \item $\ket{\psi(t)} = \sum_n c_n e^{-i E_n t /\hbar}\ket{\psi_n}$
\end{enumerate}

\subsection*{Examples to recall}

\begin{enumerate}
    \item 1D $\infty$ square well
    \item QHO
    \item 3D coulomb potential 
    \item Spin 1/2 (qubit) 
\end{enumerate}
 
\end{document}
\documentclass{article}
\usepackage[left=2cm, right=2cm, top=2cm, bottom=2cm]{geometry}
\usepackage{graphicx}
\usepackage{amsmath}
\usepackage{amssymb}
\usepackage{amsthm}
\usepackage{fancyhdr}
\usepackage{verbatim}
\usepackage{listings}
\usepackage{xcolor}
\usepackage{pgfplots}
\usepackage{physics}

\lstset{
    language=C++,
    basicstyle=\ttfamily\footnotesize,
    keywordstyle=\color{blue},
    commentstyle=\color{green},
    stringstyle=\color{red},
    numbers=left,
    numberstyle=\tiny\color{gray},
    frame=single,
    breaklines=true,
    captionpos=b,
}


\title{PHYS 487: Lecture \# 3}
\author{Cliff Sun}

\newtheorem{theorem}{Theorem}[section]
\newtheorem{lemma}[theorem]{Lemma}
\newtheorem{definition}[theorem]{Definition}
\newtheorem{conjecture}[theorem]{Conjecture}
\newtheorem{proposition}[theorem]{Proposition}
\newtheorem{corollary}[theorem]{Corollary}
\newtheorem{one minute paper}[theorem]{One Minute Paper}

\pagestyle{fancy}
\lhead{\textbf{\thepage}\ \ \nouppercase{\rightmark}}
\chead{PHYS 487: Lecture \# 3}
\rhead{Cliff Sun}

\begin{document}

\maketitle

\section*{Recap}
\begin{enumerate}
    \item Symmetry: suppose $\hat{A}$ is a transformation that leaves $\hat{H}$ unchanged, then $[A,H] = 0$
    \item $\langle O\rangle = \langle \psi | O | \psi \rangle = \langle \psi | A O A | \psi \rangle$. (interpret as active or passive transformation)
    \item Translation in space: 
    \begin{equation}
        T(a)\psi(x) = \psi(x-a)
    \end{equation}
    Here, 
    \begin{equation}
        T(a) = \exp(-ia\hat{p}/\hbar)
    \end{equation}
    Where $T(a)$ is unitary, or in other words, $T^{-1} = T^{\dagger}$. You can also consider a passive transformation:
    \begin{equation}
        T^{\dagger}(a)\hat{x}T(a) = \hat{x} + a
    \end{equation}
    \item Bloch's theorem: SS: $\psi(x)\exp(iqx)u(x)$ where $u(x)$ is periodic in $a$. 
\end{enumerate}

\section*{Lecture Span}
\begin{itemize}
    \item Symmetries
\end{itemize}

\section*{Conservation of Momentum}

We expect momentum to be conserved if there is a constant potential. In other words, $[H, T(a)] = 0$ for all $a$. We consider an infinitesimal translation $a = \delta$. Then 
\begin{equation}
    T(\delta) = e^{-i\delta \hat{p}/\hbar} \approx 1 - i\frac{\delta}{\hbar}\hat{p} \implies [\hat{H}, \hat{p}] = 0
\end{equation}
\begin{equation}
    \frac{d}{dt}\langle p \rangle = \frac{i}{\hbar}\langle [H, p]\rangle = 0
\end{equation}
Generally: symmetry implies conserved quantities. Say $\hat{O}$ with $$\ket{\psi(t)} = \sum_k c_k(t)\ket{\varphi_k}$$ with $$\hat{O}\ket{\varphi_k} = \lambda_k \ket{\varphi_k}$$

Say $[O,H] = 0$, then $\partial_t \langle O \rangle = 0$ (Ehrenfest theorem). Then cofficients 

\begin{equation}
    P(\lambda_k) = |\braket{\varphi_k}{\psi(t)}|^2
\end{equation}

Then 

\begin{equation}
    \ket{\psi} = \sum_m a_m \exp(-iE_mt/\hbar)\ket{\psi_m}
\end{equation}
\begin{equation}
    P(\lambda_k) = |\sum_m a_m \exp(-iE_mt/\hbar) \braket{\varphi_k}{\psi_m}|^2
\end{equation}

Then, because $\varphi_k$ and $\psi_m$ have the same eigenbasis, then let $\varphi_k = \psi_m$, then we get 
\begin{equation}
    = |\sum_m a_m \exp(-iE_mt/\hbar) \braket{\varphi_k}{\varphi_m}|^2 = |a_k|^2
\end{equation}

\section*{Parity}

Parity is defined as: 

\begin{equation}
    \hat{\Pi}\psi(x) = \psi(-x)
\end{equation}

\begin{enumerate}
    \item $\hat{\Pi}$ is an observable. Therefore, it is Hermitian and unitary. 
    \item Eigenvalues of $\hat{\Pi}$ are $\pm 1$.
    \item In inversion-symmetric potentials: $V(x) = V(-x)$. Then $[H, \Pi] = 0$. 
    \item For the case of inversion-symmetric potentials, $\hat{\Pi}\psi_n(x) = \psi_n(-x) = \pm \psi(x)$ 
    \item Operator transformations: $Q' = \Pi^{\dagger}Q\Pi$. Consider 
    \begin{equation}
        \Pi^{\dagger}\hat{x}\Pi = -\hat{x}
    \end{equation}
    \begin{equation}
        \Pi^{\dagger}\hat{p}\Pi = -\hat{p}
    \end{equation}
    In general, $Q'(x,p) = Q(-x,-p)$. 
\end{enumerate}

\section*{Selection Rule}

Recall: $\langle i | Q | j \rangle = 0$. Selection rule: Matrix elements must be equal to $0$ due to a symmetry. (assume $i$ and $k$ are orthogonal)

Consider the electric dipole $\vec{p}_e = q \cdot \vec{r}$. We can apply this parity study to Hydrogen-like atoms, since their potentials are centro-symmetric or parity symmetric. Then consider 

\begin{equation}
    \langle n'l'm' | \hat{p}_e | nlm \rangle = -\langle n'l'm' | \hat{\Pi}^{\dagger}\hat{p}_e \hat{\Pi}| nlm \rangle
\end{equation}

If this matrix element is not zero, then a transition can occur. The magnitude of this matrix element shows how "hard" it is to enforce a transition from $\ket{nlm}$ to $\ket{n'l'm'}$.

\subsection*{Meaning of n,l,m}

\begin{equation}
    \text{SS: } \psi_{nlm}(r,\theta,\phi) = R_{nl}(r)Y^{m}_{l}(\theta, \phi)
\end{equation}

Important: $l$ is the only one changed by parity. 

\begin{equation}
    \hat{\Pi}\psi_{nlm}(r,\theta,\phi) = (-1)^{l}\psi_{nlm}(r,\theta, \phi)
\end{equation}

Then 

\begin{equation}
    \langle n'l'm' | \hat{p}_e | nlm \rangle = -\langle n'l'm' | (-1)^{l'}p_e(-1)^{l}| nlm\rangle 
\end{equation}
\begin{equation}
    = (-1)^{l + l' + 1}\langle n'l'm' | p_e | nlm \rangle 
\end{equation}

If $l + l'$ is even, then this matrix element must be $0$. (Laporte's Rule). Generally, 
\begin{equation}
    \langle \psi_f | \hat{\mu} | \psi_i \rangle = 0 \;\; \text{ if $\psi_f\mu \psi_i$ is odd}
\end{equation}

Where $\mu$ is the transition moment operator.

\section*{Rotational Symmetry}

\begin{enumerate}
    \item Generator is probably angular momentum
    \item If rotational symmetry is conserved, then so is angular momentum probably
\end{enumerate}

Generating rotations: we can start at the $z$ axis. 

\begin{equation}
    R_z(\varphi)\psi(r,\theta, \phi) =\psi'(r,\theta, \phi) = \psi(r,\theta, \phi - \varphi)
\end{equation}

Expect:

\begin{equation}
    R_z(\varphi) = e^{-i\varphi\hat{L_z}/\hbar}
\end{equation}

Where $L_z = xp_y - yp_x$. We can study this formula infinitesimally: $\hat{x} \rightarrow \delta y$ and $y \rightarrow y + \delta x$ (remember, rotation around $z$ axis). Then 

\begin{equation}
    R_z(\delta) \sim 1 - \frac{i\delta}{\hbar}L_z
\end{equation}

Sanity check:
\begin{equation}
    R_z^{\dagger}\hat{x}R_z = x + \frac{i\delta}{\hbar}[L_z, x] = x - \delta y
\end{equation}
Similarly:
\begin{equation}
    R_z^{\dagger}\hat{y}R_z = y + \delta x
\end{equation}

Going to finite: 

\begin{equation}
    R_z(\varphi + \delta) = R_z(\varphi)R_z(\delta) = R_z(\delta)R_z(\varphi)
\end{equation}

Then we get 

\begin{equation}
    R_z(\varphi + \delta) - R_z(\varphi) = R_z(\varphi)R_z(\delta) - R_z(\varphi)
\end{equation}
\begin{equation}
    = R_z(\varphi)\left( -\frac{i\delta}{\hbar}L_z \right)
\end{equation}
Write as a differential equation:
\begin{equation}
    \frac{\partial R}{\partial \varphi} = -\frac{i}{\hbar}RL_z
\end{equation}

\section*{Rotational Symmetries}

Analogous to translational symmetry with momentum conservation, this means that 

\begin{equation}
    [\hat{H}, \hat{L}] = 0
\end{equation}
Also 
\begin{equation}
    \frac{d}{dt}\langle L \rangle = \frac{i}{\hbar} \langle [\hat{H}, \hat{L}] \rangle  = 0
\end{equation}

This means that we have a common basis between $\hat{H}$, $\hat{L^2}$, and $\hat{L}_z$. 

\begin{equation}
    H\psi_{nlm} = E_n \psi_{nlm}
\end{equation}
\begin{equation}
    L_z \psi_{nlm} = m\hbar \psi_{nlm}
\end{equation}
\begin{equation}
    L^2 \psi_{nlm} = l(l+1)\hbar^2\psi_{nlm}
\end{equation}

\end{document}
\documentclass{article}
\usepackage[left=2cm, right=2cm, top=2cm, bottom=2cm]{geometry}
\usepackage{graphicx}
\usepackage{amsmath}
\usepackage{amssymb}
\usepackage{amsthm}
\usepackage{fancyhdr}
\usepackage{verbatim}
\usepackage{listings}
\usepackage{xcolor}
\usepackage{pgfplots}

\lstset{
    language=C++,
    basicstyle=\ttfamily\footnotesize,
    keywordstyle=\color{blue},
    commentstyle=\color{green},
    stringstyle=\color{red},
    numbers=left,
    numberstyle=\tiny\color{gray},
    frame=single,
    breaklines=true,
    captionpos=b,
}


\title{PHYS 326: Lecture 2}
\author{Cliff Sun}

\newtheorem{theorem}{Theorem}[section]
\newtheorem{lemma}[theorem]{Lemma}
\newtheorem{definition}[theorem]{Definition}
\newtheorem{conjecture}[theorem]{Conjecture}
\newtheorem{proposition}[theorem]{Proposition}
\newtheorem{corollary}[theorem]{Corollary}
\newtheorem{one minute paper}[theorem]{One Minute Paper}

\pagestyle{fancy}
\lhead{\textbf{\thepage}\ \ \nouppercase{\rightmark}}
\chead{PHYS 326: Lecture 2}
\rhead{Cliff Sun}

\begin{document}

\maketitle

\section*{Coupled Oscillators (2 degrees of freedom)}

Imagine 2 masses $m_A$ and $m_B$, with 3 springs, one spring $k_1$ connecting $m_A$ to the wall, $k_2$ connecting $m_A$ and $m_B$, and $k_3$ connecting $m_B$ to the wall.
We have that $x_1$ and $x_2$ describe the motion for $m_A$ and $m_B$ respectively. How to we find the equations of motion? Then 
\begin{equation}
    KE = \frac{1}{2}m_1x_1^2 + \frac{1}{2}m_2x_2^2
\end{equation} 
\begin{equation}
    PE = \frac{1}{2}k_1x_1^2 + \frac{1}{2}k_3x_2^2 + \frac{1}{2}k_2(x_1-x_2)^2
\end{equation}

We find E.L equations for this problem:

\begin{equation}
    m_1\ddot{x_1} = -k_1x_1 + k_2(x_2 - x_1)
\end{equation}
\begin{equation}
    m_2\ddot{x_2} = -k_3x_2 - k_2(x_2 - x_1)
\end{equation}

We write this in matrix form:

\begin{equation}
    \begin{pmatrix}
        m_1 & 0 \\
        0 & m_2
    \end{pmatrix}\begin{pmatrix}
        \ddot{x_1} \\
        \ddot{x_2}
    \end{pmatrix} + \begin{pmatrix}
        k_1 + k_2 & -k_2 \\
        -k_2 & k_2 + k_3 
    \end{pmatrix}\begin{pmatrix}
        x_1 \\
        x_2
    \end{pmatrix} = 0
\end{equation}

We rewrite this equation as 

\begin{equation}
    M\ddot{\vec{x}} + K\vec{x} = 0
\end{equation}

Note that $M$ and $K$ are symmetric matrices, that is $M^{t} = M$ and $K^{t} = K$. Ansatz: normal mode solution. We guess 
\begin{equation}
    \vec{x}(t) = \vec{a}e^{i\omega t}
\end{equation}
Then 
\begin{equation}
    (-M\omega^2 + K)\vec{a}e^{i\omega t} = 0
\end{equation}

We expand 

\begin{equation}
    (-M\omega^2 + K)\vec{a} = \begin{pmatrix}
        k_1 + k_2 - m_1\omega^2 & -k_2 \\
        -k_2 & k_1 + k_3 -m_2\omega^2
    \end{pmatrix}\begin{pmatrix}
        a_1 \\
        a_2
    \end{pmatrix} = 0
\end{equation}

Then 

\begin{equation}
    (k_1 + k_2 -m_1\omega^2)a_1 - k_2a_2 = 0
\end{equation}
\begin{equation}
    -k_2a_1 + (k_1 + k_3 - m_2\omega^2)a_2 = 0
\end{equation}

If we brute force solve this, we get $a_1 = a_2 = 0$. To find the non-trivial solution, we demand that 
\begin{equation}
    \det(K - M\omega^2) = 0
\end{equation}
So that both equations are just multiples of each other. We solve for $\omega$, and then find $\vec{a}$ or eigenvectors. Then the general solutions must be a linear combination of the 
eigenfrequncies. We demand this determinant condition because then it implies that some of the vectors are linearly dependent, which we can then use to construct a non-trivial vector that finds this linear dependence.  

\section*{Normal Coordinates}

Define new coordinates for describing the motion:

\begin{equation}
    \zeta_1 = \frac{1}{2}(x_1 + x_2)
\end{equation}
\begin{equation}
    \zeta_2 = \frac{1}{2}(x_1 - x_2)
\end{equation}
\begin{equation}
    \zeta = \frac{1}{2}\begin{pmatrix}
        1 & 1 \\
        1 & -1
    \end{pmatrix}\vec{x}
\end{equation}

We can write E.O.M of equal springs and equal masses in $\zeta$, 
\begin{equation}
    m\ddot{x_1} = -kx_1 + k(x_2 - x_1)
\end{equation}
\begin{equation}
    m\ddot{x_2} = -kx_2 - k(x_2 - x_1)
\end{equation}

We sum the equations 
\begin{equation}
    m(\ddot{x_1} + \ddot{x_2}) = -k(x_1 + x_2)
\end{equation}
\begin{equation}
    \zeta_1 = -\frac{k}{m}\zeta_1
\end{equation}
We then subtract the equations and yield 
\begin{equation}
    \zeta_2 = -\frac{3k}{m}\zeta_2
\end{equation}



\end{document}

\documentclass{article}
\usepackage[left=2cm, right=2cm, top=2cm, bottom=2cm]{geometry}
\usepackage{graphicx}
\usepackage{amsmath}
\usepackage{amssymb}
\usepackage{amsthm}
\usepackage{fancyhdr}
\usepackage{verbatim}
\usepackage{listings}
\usepackage{xcolor}
\usepackage{pgfplots}

\lstset{
    language=C++,
    basicstyle=\ttfamily\footnotesize,
    keywordstyle=\color{blue},
    commentstyle=\color{green},
    stringstyle=\color{red},
    numbers=left,
    numberstyle=\tiny\color{gray},
    frame=single,
    breaklines=true,
    captionpos=b,
}


\title{PHYS 326: Lecture 9}
\author{Cliff Sun}

\newtheorem{theorem}{Theorem}[section]
\newtheorem{lemma}[theorem]{Lemma}
\newtheorem{definition}[theorem]{Definition}
\newtheorem{conjecture}[theorem]{Conjecture}
\newtheorem{proposition}[theorem]{Proposition}
\newtheorem{corollary}[theorem]{Corollary}
\newtheorem{one minute paper}[theorem]{One Minute Paper}

\pagestyle{fancy}
\lhead{\textbf{\thepage}\ \ \nouppercase{\rightmark}}
\chead{PHYS 326: Lecture 9}
\rhead{Cliff Sun}

\begin{document}

\maketitle

\section*{Non-linear Dynamics and Chaos}

Linear systems are solvable and intuitive, but are only an approximation. Non-linear systems can be simple, but show complex and unpredictable behavior. 

\subsection*{The Logistic Map}

Define logistic map 

\begin{equation}
    x_{j+1} = \alpha x_j(1 - x_j)
\end{equation}

The points of stability is 

\begin{equation}
    x=0, x = 1 - \frac{1}{\alpha}
\end{equation}

We search for equilibrium points, stay close to $1 - \frac{1}{\alpha} + \epsilon$, then 

\begin{equation}
    x_{j+1} = \alpha x_j(1 - x_j)
\end{equation}

\begin{equation}
    \iff \alpha(1 - \frac{1}{\alpha} + \epsilon)(1 - 1 + \frac{1}{\alpha} + \epsilon)
\end{equation}

\begin{equation}
    \iff 1 - \frac{1}{\alpha} + \epsilon(2 - \alpha) + O(\epsilon^2)
\end{equation}
\begin{equation}
    \iff 1 - \frac{1}{\alpha} + \epsilon'
\end{equation}

Note 

\begin{equation}
    \epsilon' < \epsilon \implies \epsilon(2 - \alpha) < \epsilon \iff 1 < \alpha < 3
\end{equation}

\begin{definition}
    Chaos is defined as 
    \begin{enumerate}
        \item Attractor = fractal 
        \item Attract is exponentially sensitive to initial conditions. 
    \end{enumerate}
\end{definition}

Suppose two populations $x_n$ and $x_n'$, then for sufficiently large $n$, then 
\begin{equation}
    |x_n - x'_n| \sim \epsilon e^{\lambda n}
\end{equation}

Where $\lambda$ is the Lyapunov exponent. 

\end{document}
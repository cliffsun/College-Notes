\documentclass{article}
\usepackage[left=2cm, right=2cm, top=2cm, bottom=2cm]{geometry}
\usepackage{graphicx}
\usepackage{amsmath}
\usepackage{amssymb}
\usepackage{amsthm}
\usepackage{fancyhdr}
\usepackage{verbatim}
\usepackage{listings}
\usepackage{xcolor}
\usepackage{pgfplots}

\lstset{
    language=C++,
    basicstyle=\ttfamily\footnotesize,
    keywordstyle=\color{blue},
    commentstyle=\color{green},
    stringstyle=\color{red},
    numbers=left,
    numberstyle=\tiny\color{gray},
    frame=single,
    breaklines=true,
    captionpos=b,
}


\title{PHYS 326: Lecture \# 12}
\author{Cliff Sun}

\newtheorem{theorem}{Theorem}[section]
\newtheorem{lemma}[theorem]{Lemma}
\newtheorem{definition}[theorem]{Definition}
\newtheorem{conjecture}[theorem]{Conjecture}
\newtheorem{proposition}[theorem]{Proposition}
\newtheorem{corollary}[theorem]{Corollary}
\newtheorem{one minute paper}[theorem]{One Minute Paper}

\pagestyle{fancy}
\lhead{\textbf{\thepage}\ \ \nouppercase{\rightmark}}
\chead{PHYS 326: Lecture \# 12}
\rhead{Cliff Sun}

\begin{document}

\maketitle

\section*{Rigid Body Mechanics}

\begin{definition}
    A rigid body is a collection of $n$ particles with constant distance between them. That is, the shape does not change. 
\end{definition}

\begin{definition}
    The center of mass is defined as 
    \begin{equation}
        \vec{R} = \frac{1}{M}\sum_{a=1}^{N}m_a r_a
    \end{equation}
    This is the mass weighted with respect to its vector. The continuous limit is 
    \begin{equation}
        \vec{R} = \frac{1}{M}\int dm r
    \end{equation}
\end{definition}

\begin{definition}
    The momentum is 
    \begin{equation}
        \sum_{a} m_a \dot{r}_a \iff M\dot{\vec{R}}
    \end{equation}
    The external force is 
    \begin{equation}
        F_{ext} = M\ddot{\vec{R}}
    \end{equation}
\end{definition}

\begin{definition}
    The total angular momentum is 
    \begin{equation}
        L_O = \sum_{a} \vec{r}_a \times m_a\dot{\vec{r}}_a
    \end{equation}
    Total external torque is 
    \begin{equation}
        R_O = \sum_{a} \vec{r}_a \times F_{a,\; ext}
    \end{equation}
\end{definition}

\begin{theorem}
    \begin{equation}
        \frac{dL_O}{dt} = R_O
    \end{equation}
\end{theorem}

\begin{proof}
    Calculate $dL_O/dt$. 
    \begin{equation}
        \frac{dL_O}{dt} = \frac{d}{dt}\left[\sum_{a} \vec{r}_a \times m_a \dot{\vec{r}}_a\right]
    \end{equation}
    \begin{equation}
        = \sum_a \dot{\vec{r}}_a \times m_a \dot{\vec{r}}_a + \sum_a \vec{r}_a \times m_a \ddot{\vec{r}}_a
    \end{equation}
    \begin{equation}
        = \sum_a \vec{r}_a \times \left( F_a^{ext} + F^{int}_a \right)
    \end{equation}
    \begin{equation}
        = \sum_a \vec{r}_a \times F_a^{ext}
    \end{equation}
\end{proof}

\begin{theorem}
    \begin{equation}
        L_O = \vec{R} \times (M\dot{\vec{R}}) + L^{CM}
    \end{equation}
    Where $L^{CM}$ is the angular momentum around the center of mass. 
\end{theorem}

\begin{proof}
    Define $\rho = r_a - R$ where $R$ is the location of the center of mass, then calculate torque and note that $R$ has no $a$ subscript.
\end{proof}

\begin{theorem}
    \begin{equation}
        \frac{dL^{CM}}{dt} = \vec{\Gamma}^{CM}
    \end{equation}
    The change in the angular momentum around the center of mass is the torque around the center of mass. 
\end{theorem}

\begin{theorem}
    \begin{equation}
        \frac{d}{dt}L_p = \Gamma_p^{\text{actual}} - \vec{r}_{p \rightarrow COM} \times M \vec{a}_p
    \end{equation}
\end{theorem}

\begin{definition}
    The Levi-Civita symbol is 
    \begin{equation}
        \epsilon_{ijk} = -1^p
    \end{equation}
    Where $p$ is the parity of the permutation of $ijk$. 
\end{definition}

\end{document}
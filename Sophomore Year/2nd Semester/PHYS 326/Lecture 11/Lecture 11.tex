\documentclass{article}
\usepackage[left=2cm, right=2cm, top=2cm, bottom=2cm]{geometry}
\usepackage{graphicx}
\usepackage{amsmath}
\usepackage{amssymb}
\usepackage{amsthm}
\usepackage{fancyhdr}
\usepackage{verbatim}
\usepackage{listings}
\usepackage{xcolor}
\usepackage{pgfplots}

\lstset{
    language=C++,
    basicstyle=\ttfamily\footnotesize,
    keywordstyle=\color{blue},
    commentstyle=\color{green},
    stringstyle=\color{red},
    numbers=left,
    numberstyle=\tiny\color{gray},
    frame=single,
    breaklines=true,
    captionpos=b,
}


\title{PHYS 326: Lecture 11}
\author{Cliff Sun}

\newtheorem{theorem}{Theorem}[section]
\newtheorem{lemma}[theorem]{Lemma}
\newtheorem{definition}[theorem]{Definition}
\newtheorem{conjecture}[theorem]{Conjecture}
\newtheorem{proposition}[theorem]{Proposition}
\newtheorem{corollary}[theorem]{Corollary}
\newtheorem{one minute paper}[theorem]{One Minute Paper}

\pagestyle{fancy}
\lhead{\textbf{\thepage}\ \ \nouppercase{\rightmark}}
\chead{PHYS 326: Lecture 11}
\rhead{Cliff Sun}

\begin{document}

\maketitle

\section*{Non-linear dynamics - Continued}

\subsection*{Driven Damped Pendulum}

\begin{equation}
    mL^2\ddot{\theta} = -bc^2\dot{\theta} - mgL\sin\theta + LF(t)
\end{equation}

Choose harmonic forcing $F(t) = F_0\cos(\omega t)$. The phase space is $\{\theta, \dot{\theta}, \phi = \omega t\}$. 

\begin{definition}
    \underline{Poincare sections} are defined as the following:
    \begin{enumerate}
        \item Wait for transients to decay 
        \item Plot Phase-Plane positions $(\theta, \dot{\theta})$ at 1 cycle intervals. 
    \end{enumerate}
    You plot \textbf{points} every $\frac{n2\pi}{\omega}$.
\end{definition}

\begin{definition}
    Chaos is 
    \begin{enumerate}
        \item Divergence from initial conditions
        \item The attractor is a fractal, ("strange attractor")
    \end{enumerate}
\end{definition}

\begin{definition}
    A \underline{fractal} is an object that is 
    \begin{enumerate}
        \item Self-similar (same features at all scales)
        \item Its dimension is a non integer. 
    \end{enumerate}
\end{definition}

Whats the dimension of a fractal? 

\begin{definition}
    The dimension is defined by how mass scales with linear size.
\end{definition}

For example, for a square. If you split the square in fourths, etc. How does the mass increase?
For $r$ sections, it follows that the mass would increase by $r^2$. Thus the dimension 
\begin{equation}
    \dim = \frac{\ln(r^2)}{\ln(r)} = 2
\end{equation}

For the Koch curve, for $r=3$, the mass is $m=4$. Then 
\begin{equation}
    \dim =\frac{\ln(m)}{\ln(r)} = 1.26
\end{equation}

\end{document}
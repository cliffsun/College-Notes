\documentclass{article}
\usepackage[left=2cm, right=2cm, top=2cm, bottom=2cm]{geometry}
\usepackage{graphicx}
\usepackage{amsmath}
\usepackage{amssymb}
\usepackage{amsthm}
\usepackage{fancyhdr}
\usepackage{verbatim}
\usepackage{listings}
\usepackage{xcolor}
\usepackage{pgfplots}

\lstset{
    language=C++,
    basicstyle=\ttfamily\footnotesize,
    keywordstyle=\color{blue},
    commentstyle=\color{green},
    stringstyle=\color{red},
    numbers=left,
    numberstyle=\tiny\color{gray},
    frame=single,
    breaklines=true,
    captionpos=b,
}


\title{PHYS 326: Lecture \# 14}
\author{Cliff Sun}

\newtheorem{theorem}{Theorem}[section]
\newtheorem{lemma}[theorem]{Lemma}
\newtheorem{definition}[theorem]{Definition}
\newtheorem{conjecture}[theorem]{Conjecture}
\newtheorem{proposition}[theorem]{Proposition}
\newtheorem{corollary}[theorem]{Corollary}
\newtheorem{one minute paper}[theorem]{One Minute Paper}

\pagestyle{fancy}
\lhead{\textbf{\thepage}\ \ \nouppercase{\rightmark}}
\chead{PHYS 326: Lecture \# 14}
\rhead{Cliff Sun}

\begin{document}

\maketitle

\section*{Rigid Body Motion (cont'd)}

\subsection*{Principal Axis of Inertia}

Recall, 

\begin{equation}
    \vec{L} = I\vec{\omega} \neq L \nparallel \omega
\end{equation}

But, it's always possible to define $3$ orthogonal directions in which $I$ is diagonal. These $3$ orthogonal axis are called the "3 principal axis of inertia". That is 

\begin{equation}
    I = \begin{bmatrix}
        I_{xx} & 0 & 0 \\
        0 & I_{yy} & 0 \\
        0 & 0 & I_{zz}
    \end{bmatrix}
\end{equation}

\subsection*{Parallel Axis Theorem}

For each $m_a$, we have that 

\begin{equation}
    x_a = x_{cm} + x'_a
\end{equation}
\begin{equation}
    y_a = y_{cm} + y'_a
\end{equation}
\begin{equation}
    z_a = z_{cm} + z'_a
\end{equation}

Then 

\begin{equation}
    I_{zz} = \sum_a m_a(x_a^2 + y_a^2)
\end{equation}
\begin{equation}
    = \sum_a m_a ((x_{cm} + x_a')^2 + (y_{cm} + y'_a)^2)
\end{equation}
\begin{equation}
    = \sum_a m_a (x_{cm}^2 + y_{cm}^2) + \underbrace{2\sum_a m_a (x'_a x_cm + y'_a y_cm)}_{0} + \underbrace{\sum_a m_a ((x'_a)^{2} + (y'_a)^{2})}_{I_{zz}}
\end{equation}

Thus 

\begin{equation}
    I_{zz}^O = M(x_{CM}^2 + y_{CM}^2) + I_{zz}^{CM}
\end{equation}
\begin{equation}
    I_{xx}^{O} = M(z_{CM}^2 + y_{CM}^2) + I_{xx}^{CM}
\end{equation}
\begin{equation}
    I_{yy}^{O} = \dots
\end{equation}

\subsection*{Rigid Body Motion: Torque-Free}

\begin{equation}
    \Gamma_{o}(t) = \frac{dL_O}{dt} = \frac{d}{dt}\left(I_O \omega\right)
\end{equation}
Note, "O" is a fixed point. Let's work in a coordinate system where $I_O$ is diagonal. However, we are now working in a rotating coordinate system. Then 

\begin{equation}
    \vec{v} = \sum v_i \hat{e}_i(t)
\end{equation}

Then 

\begin{equation}
    \frac{dv}{dt} = \sum \frac{dv_i}{dt}\hat{e}(t) + \omega \times v
\end{equation}

Then, choose that $I_{CM}$ is diagonal, but 

\begin{equation}
    \Gamma(t) = \frac{d}{dt}\left( I \omega\right) = I \frac{d\omega}{dt} + \omega \times (I\omega) = 0
\end{equation}
\end{document}
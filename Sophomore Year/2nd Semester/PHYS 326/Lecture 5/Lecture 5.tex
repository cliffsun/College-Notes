\documentclass{article}
\usepackage[left=2cm, right=2cm, top=2cm, bottom=2cm]{geometry}
\usepackage{graphicx}
\usepackage{amsmath}
\usepackage{amssymb}
\usepackage{amsthm}
\usepackage{fancyhdr}
\usepackage{verbatim}
\usepackage{listings}
\usepackage{xcolor}
\usepackage{pgfplots}

\lstset{
    language=C++,
    basicstyle=\ttfamily\footnotesize,
    keywordstyle=\color{blue},
    commentstyle=\color{green},
    stringstyle=\color{red},
    numbers=left,
    numberstyle=\tiny\color{gray},
    frame=single,
    breaklines=true,
    captionpos=b,
}


\title{PHYS 326: Lecture 5}
\author{Cliff Sun}

\newtheorem{theorem}{Theorem}[section]
\newtheorem{lemma}[theorem]{Lemma}
\newtheorem{definition}[theorem]{Definition}
\newtheorem{conjecture}[theorem]{Conjecture}
\newtheorem{proposition}[theorem]{Proposition}
\newtheorem{corollary}[theorem]{Corollary}
\newtheorem{one minute paper}[theorem]{One Minute Paper}

\pagestyle{fancy}
\lhead{\textbf{\thepage}\ \ \nouppercase{\rightmark}}
\chead{PHYS 326: Lecture 5}
\rhead{Cliff Sun}

\begin{document}

\maketitle

\section*{Orthogonality of Eigenmodes}

\begin{definition}
    The inner product in 
    \begin{enumerate}
        \item Cartesian Coordinates: $\{a,b,c, \dots\} \cdot \{x,y,z,\dots\} = ax + by + cz + \dots$
        \item In vector notation
        \begin{equation}
            \vec{v} \cdot \vec{u} = \vec{v}^{T}\vec{u} = \vec{u}^{T}\vec{v}
        \end{equation}
    \end{enumerate}
    We define the \underline{generalized inner product} as 
    \begin{equation}
        \vec{v}\cdot \vec{u} = \vec{v}^{T}\textbf{M}\vec{u} = \vec{u} \cdot \vec{v}
    \end{equation}
    \textbf{Example:}
    \begin{equation}
        \vec{v} \cdot \vec{v} = \vec{v}^{T}\textbf{M}\vec{v} = 2T \geq 0
    \end{equation}
\end{definition}

\begin{center}
    \textbf{Claim:} Different eigenmodes $\vec{u}$ are orthogonal. 
\end{center}

That is 
\begin{equation}
    \vec{u}^{1} \cdot \vec{u}^{2} = \vec{u}^{1} \textbf{M} \vec{u}^{2} = 0
\end{equation}

\begin{proof}
    We have that 
    \begin{equation}
        K\vec{u}^{1} = \omega^2 M\vec{u}^{1}
    \end{equation}
    \begin{equation}
        K\vec{u}^{2} = \omega^2 M\vec{u}^{2}
    \end{equation}
    We multiply the top part by $\vec{u}_{2}^{T}$ and the bottom part with $\vec{u}_{1}^{T}$:
    \begin{equation}
        \vec{u}_{2}^{T}K\vec{u}^{1} = \omega^2 \vec{u}_{2}^{T}M\vec{u}^{1}
    \end{equation}
    \begin{equation}
        \vec{u}_{1}^{T}K\vec{u}^{2} = \omega^2 \vec{u}_{1}^{T}M\vec{u}^{2}
    \end{equation}
    We take the transpose of equation 8 and minus it from equation 7 to yield 
    \begin{equation}
            0 = (\omega_{2}^2 - \omega_{1}^2)u_{2}^{T}M\vec{u}_{1}
    \end{equation}
\end{proof}

Normalizing eigenmodes:

\begin{equation}
    \vec{u} = \frac{\vec{u}}{(\vec{u}^{T}M\vec{u})^{\frac{1}{2}}}
\end{equation}

After normalization

\begin{equation}
    \vec{u}_{i}^{T}M\vec{u}_{j} = \delta_{ij}
\end{equation}

\section*{General Initial conditions problem}

The general solution is 

\begin{equation}
    \vec{x}(t) = \sum A_{s}\vec{u}^{s}\cos\left(\omega_s t\right) + \sum B_{s}\vec{u}^{s}\sin\left(\omega_s t\right)
\end{equation}

$A_s$ and $B_s$ are determined by initial conditions. Define 
\begin{equation}
    \vec{x}_0 = \vec{x}(t=0) = \sum_S A_S \vec{u}^{S}
\end{equation}
\begin{equation}
    \vec{v}_0 = \vec{v}(t=0) = \sum_S B_S \omega_S \vec{u}^{S}
\end{equation}

Then 
\begin{equation}
    \vec{u}_r^{T}\textbf{M}\vec{x}_0 \sum_S A_S \vec{u}_r^{T}\textbf{M}\vec{u}^{S} = A_S \delta_{rS} = A_r
\end{equation}
\begin{equation}
    \vec{u}_r^{T}\textbf{M}\vec{v}_0 = \sum_S B_S \omega_S \vec{u}_r^{T}\textbf{M}\vec{u}^{S} = B_S\omega_S\delta_{rS} = B_r\omega_r
\end{equation}
Thus 
\begin{equation}
    A_r = \vec{u}_{r}^{T}\textbf{M}\vec{x}_0
\end{equation}
\begin{equation}
    B_r = \frac{\vec{u}_r^{T}\textbf{M}\vec{v}_0}{\omega_r}
\end{equation}

\section*{The Modal Matrix}

\begin{definition}
    Define 
    \begin{equation}
        \cup_{\alpha R} = u^{R}_{\alpha}
    \end{equation}
    Where $R$ dictates which mode we are at, and $\alpha$ shows which coordinate we are in. This is a matrix of all the modes. 
\end{definition}

We have that 
\begin{equation}
    \textbf{U}^{T}\textbf{M}\textbf{U} = I \implies \textbf{U}^{-1} = \textbf{U}^{T}
\end{equation}

\end{document}
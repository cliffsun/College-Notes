\documentclass{article}
\usepackage[left=2cm, right=2cm, top=2cm, bottom=2cm]{geometry}
\usepackage{graphicx}
\usepackage{amsmath}
\usepackage{amssymb}
\usepackage{amsthm}
\usepackage{fancyhdr}
\usepackage{verbatim}
\usepackage{listings}
\usepackage{xcolor}
\usepackage{pgfplots}

\lstset{
    language=C++,
    basicstyle=\ttfamily\footnotesize,
    keywordstyle=\color{blue},
    commentstyle=\color{green},
    stringstyle=\color{red},
    numbers=left,
    numberstyle=\tiny\color{gray},
    frame=single,
    breaklines=true,
    captionpos=b,
}


\title{PHYS 326: Lecture \# 15}
\author{Cliff Sun}

\newtheorem{theorem}{Theorem}[section]
\newtheorem{lemma}[theorem]{Lemma}
\newtheorem{definition}[theorem]{Definition}
\newtheorem{conjecture}[theorem]{Conjecture}
\newtheorem{proposition}[theorem]{Proposition}
\newtheorem{corollary}[theorem]{Corollary}
\newtheorem{one minute paper}[theorem]{One Minute Paper}

\pagestyle{fancy}
\lhead{\textbf{\thepage}\ \ \nouppercase{\rightmark}}
\chead{PHYS 326: Lecture \# 15}
\rhead{Cliff Sun}

\begin{document}

\maketitle

\section*{Symmetrical Top, Subject to Torque}

In the presence of torque, we work in a fixed lab frame. Write the Lagrangian. Define generalized coordinates: Euler Angles. We define Orientation/Rotation Matrices $Q$. Consider a coordinate 
system $\hat{i}, \hat{j}, \hat{k}$, then transform to new unit vectors: $\hat{i}', \hat{j}', \hat{k}'$. Then write 

\begin{equation}
    \hat{i}' = Q_{xx}\hat{i} + Q_{xy}\hat{j} + Q_{xz}\hat{k}
\end{equation}
$\hat{j}'$ and $\hat{k}'$ are defined similarly. Note, the elements of $Q$ are just angles, that is 

\begin{equation}
    \hat{i}' \cdot \hat{j} = \underbrace{|\hat{i}'||\hat{j}|}_{1}\cos\theta_{i'j} = Q_{xy}
\end{equation}

Write 

\begin{equation}
    \hat{e}_{i}' = \sum_{j} Q_{ij}\hat{e}_{j}
\end{equation}

Require that $\hat{e}_{i}'$ be orthonormal basis, then 

\begin{equation}
    \delta_{ik} = \hat{e}'_{i} \cdot \hat{e}'_{k} = \sum_{j}Q_{ij}\hat{e}_{j}\sum_{m}Q_{km}\hat{e}'_{m}
\end{equation}
\begin{equation}
    \iff \sum_{jm} Q_{ij}Q_{km}\delta_{jm}
\end{equation}
\begin{equation}
    \iff \sum_{j} Q_{ij}Q_{km} \iff \sum_{j} Q_{ij}Q^T_{jk} \implies I = QQ^{T}
\end{equation}

Note, the eigenvalues of $Q$ is $\lambda = 1$, and $\lambda = e^{\pm i \alpha}$. Where $\alpha$ is the angle of rotation. Consider infinitesimal rotations by small angle $\omega dt$, then 

\begin{equation}
    Q = Q + \epsilon
\end{equation}
Then consider 

\begin{equation}
    Q^{T}Q = I \iff (I + \epsilon^{T})(I + \epsilon) \implies \epsilon^{T} + \epsilon = 0
\end{equation}
This means $\epsilon$ is anti-symmetric. Then consider 
\begin{equation}
    Q(t + dt) = (I + \omega dt)Q(t)
\end{equation}
Then 
\begin{equation}
    \frac{dQ}{dt} = \frac{Q(t + dt) - Q(t)}{dt} = \omega Q
\end{equation}
Then 
\begin{equation}
    \omega = \frac{dQ}{dt}Q^{T}
\end{equation}
\end{document}
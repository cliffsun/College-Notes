\documentclass{article}
\usepackage[left=2cm, right=2cm, top=2cm, bottom=2cm]{geometry}
\usepackage{graphicx}
\usepackage{amsmath}
\usepackage{amssymb}
\usepackage{amsthm}
\usepackage{fancyhdr}
\usepackage{verbatim}
\usepackage{listings}
\usepackage{xcolor}
\usepackage{pgfplots}

\lstset{
    language=C++,
    basicstyle=\ttfamily\footnotesize,
    keywordstyle=\color{blue},
    commentstyle=\color{green},
    stringstyle=\color{red},
    numbers=left,
    numberstyle=\tiny\color{gray},
    frame=single,
    breaklines=true,
    captionpos=b,
}


\title{PHYS 326: Lecture 3}
\author{Cliff Sun}

\newtheorem{theorem}{Theorem}[section]
\newtheorem{lemma}[theorem]{Lemma}
\newtheorem{definition}[theorem]{Definition}
\newtheorem{conjecture}[theorem]{Conjecture}
\newtheorem{proposition}[theorem]{Proposition}
\newtheorem{corollary}[theorem]{Corollary}
\newtheorem{one minute paper}[theorem]{One Minute Paper}

\pagestyle{fancy}
\lhead{\textbf{\thepage}\ \ \nouppercase{\rightmark}}
\chead{PHYS 326: Lecture 3}
\rhead{Cliff Sun}

\begin{document}

\maketitle

\section*{2 pendula connected by spring}

Assume 2 pendula connected by a spring of constant $k$. Each spring has a length $L$ and mass $m$. $\theta_i$ describes the $i$-th pendulum. We describe 
its equation of motion using torque. On the first mass, we have 

\begin{equation}
    \tau_{g} = mgL\sin\theta_1
\end{equation}
\begin{equation}
    \tau_k = (k\Delta x)L\sin\left(\frac{\pi}{2} - \theta_i\right)
\end{equation}

Using small angle approximation, we have 

\begin{equation}
    \tau_{g} = mgL\theta_1
\end{equation}
\begin{equation}
    \tau_{k} = kL^{2}(\theta_1 - \theta_2)
\end{equation}

We use

\begin{equation}
    I\ddot{\theta} = \sum \tau
\end{equation}
\begin{equation}
    \iff mL^2\ddot{\theta_1} - mgL\theta_1 - kL^2(\theta_1 - \theta_2) = 0
\end{equation}

Matrix form: 

\begin{equation}
    \begin{pmatrix}
        mL^2 & 0 \\
        0 & mL^2
    \end{pmatrix}\begin{pmatrix}
        \ddot{\theta_1} \\
        \ddot{\theta_2}
    \end{pmatrix}
    + \begin{pmatrix}
        mgL + kL^2 & -kL^2 \\
        -kL^2 & mgL + kL^2
    \end{pmatrix}\begin{pmatrix}
        \theta_1 \\
        \theta_2
    \end{pmatrix} = \begin{pmatrix}
        0 \\
        0
    \end{pmatrix}
\end{equation}

Let $m=g=L=1$, then define 
\begin{equation}
    \epsilon = \frac{kL}{mg} \iff k = \epsilon\frac{mg}{L}
\end{equation}
Then 
\begin{equation}
    M = \begin{pmatrix}
        1 & 0 \\
        0 & 1
    \end{pmatrix}
\end{equation}
\begin{equation}
    K = \begin{pmatrix}
        1 + \epsilon & -\epsilon \\
        -\epsilon & 1 + \epsilon
    \end{pmatrix}
\end{equation}

Then we have that when $\omega^2 = 1$, then $a_1 = a_2$. And when $\omega^2 = 1 + 2\epsilon$, then $a_1 = -a_2$. These motions originate from the initial placement of the pendulum. 

\subsection*{Weak Coupling}

We assume weak coupling, that is 
\begin{equation}
    \epsilon = \frac{kL}{mg} << 1
\end{equation}
Then $\omega_1  = 1$ and $\omega_2 = \sqrt{1 + 2\epsilon} \approx 1 + \epsilon$. With initial conditions, $\theta_1 = 1$, $\dot{\theta_1} = 0$, $\theta_2 = 0$, $\dot{\theta_2} = 0$, then we obtain 
\begin{equation}
    \begin{pmatrix}
        1 \\ 
        0
    \end{pmatrix} = A_1\begin{pmatrix}
        1 \\
        1
    \end{pmatrix} + A_2\begin{pmatrix}
        1 \\
        -1
    \end{pmatrix}
\end{equation}
\begin{equation}
    \begin{pmatrix}
        0 \\
        0
    \end{pmatrix} = (1 + 2\epsilon)\left(B_1\begin{pmatrix}
        1 \\
        1
    \end{pmatrix} + B_2\begin{pmatrix}
        1 \\
        -1
    \end{pmatrix}\right)
\end{equation}
THen $A_1 = A_2 = \frac{1}{2}$ and $B_1 = B_2 = 0$. Then

\begin{equation}
    \theta_1 = \frac{1}{2}\left[\cos(t) + \cos\left(1 + \epsilon\right)t\right]
\end{equation}
\begin{equation}
    \theta_2 = \frac{1}{2}\left[\cos(t) - \cos\left(1 + \epsilon\right)t\right] 
\end{equation}

We convert them:

\begin{equation}
    \theta_1(t) = \cos(\left(1 + \frac{\epsilon}{2}\right)t)\cos(\frac{\epsilon}{2}t)
\end{equation}
\begin{equation}
    \theta_2(t) = \sin(\left(1 + \frac{\epsilon}{2}\right)t)\sin(\frac{\epsilon}{2}t)
\end{equation}

\section*{3 degrees of freedom}

Assume 3 masses all connected together by strings of length L. Assume that the strings don't stretch and that there is constant tension. Assume that the 
masses are connected to a pole with springs of constant $k_i$. Then for the first mass, we have 
\begin{equation}
    m\ddot{x}_1 = T\sin\theta_2 - T\sin\theta_1 
\end{equation}
We have
\begin{equation}
    \sin\theta_2 \approx \frac{x_2 - x_1}{L}
\end{equation}
\begin{equation}
    \sin\theta_1 \approx \frac{x_1}{L}
\end{equation}

\end{document}
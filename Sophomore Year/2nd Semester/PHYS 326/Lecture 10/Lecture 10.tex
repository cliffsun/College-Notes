\documentclass{article}
\usepackage[left=2cm, right=2cm, top=2cm, bottom=2cm]{geometry}
\usepackage{graphicx}
\usepackage{amsmath}
\usepackage{amssymb}
\usepackage{amsthm}
\usepackage{fancyhdr}
\usepackage{verbatim}
\usepackage{listings}
\usepackage{xcolor}
\usepackage{pgfplots}

\lstset{
    language=C++,
    basicstyle=\ttfamily\footnotesize,
    keywordstyle=\color{blue},
    commentstyle=\color{green},
    stringstyle=\color{red},
    numbers=left,
    numberstyle=\tiny\color{gray},
    frame=single,
    breaklines=true,
    captionpos=b,
}


\title{PHYS 326: Lecture 10}
\author{Cliff Sun}

\newtheorem{theorem}{Theorem}[section]
\newtheorem{lemma}[theorem]{Lemma}
\newtheorem{definition}[theorem]{Definition}
\newtheorem{conjecture}[theorem]{Conjecture}
\newtheorem{proposition}[theorem]{Proposition}
\newtheorem{corollary}[theorem]{Corollary}
\newtheorem{one minute paper}[theorem]{One Minute Paper}

\pagestyle{fancy}
\lhead{\textbf{\thepage}\ \ \nouppercase{\rightmark}}
\chead{PHYS 326: Lecture 10}
\rhead{Cliff Sun}

\begin{document}

\maketitle

\section*{Non-linear Dynamics}

Consider dynamics using 

\begin{equation}
    \frac{d\bar{x}}{dt} = \bar{F}(x)
\end{equation}

Where $\bar{x}$ is the set of coordinates in $n$-dimensional "phase space" (or "state space"). Sufficient to determine the future state. 

\subsection*{Example: Non-linear pendulum}

Study 

\begin{equation}
    \ddot{\theta} + \omega_r^2 \sin\theta = 0
\end{equation}

No approximation. Denote $x = \theta$ and $y = \dot{\theta}$. Then $\bar{x} = \{x, y\} \iff \{\theta, \dot{\theta}\}$. Then 

\begin{equation}
    \frac{dx}{dt} = y
\end{equation}
\begin{equation}
    \frac{dy}{dt} = -\omega_0^2 \sin x
\end{equation}

Let $\omega_0 = 1$, then 

\begin{equation}
    \frac{dx}{dt} = y
\end{equation}
\begin{equation}
    \frac{dy}{dt} = \sin x
\end{equation}

To find fixed points, $y=0$ and $x = n\pi$. 

\subsection*{Van Der Pol Oscillator}

\begin{equation}
    \frac{d^2y}{dt^2} + E(y^2 - 1)\frac{dy}{dt} + y = 0
\end{equation}

Map to phase plane, define $v = \frac{dy}{dt}$, then 

\begin{equation}
    \dot{y} = v
\end{equation}
\begin{equation}
    \dot{v} = -y - E(y^2 - 1)v
\end{equation}

Then $\bar{x} = \{y, v\}$. At small $y$, then the ODE turns into
\begin{equation}
    \frac{d^2y}{dt^2} - E\frac{dy}{dt} + y = 0
\end{equation}
Which means the voltage grows. At large $y$, then the ODE turns into 

\begin{equation}
    \frac{d^2y}{dt^2} + Ey^2\frac{dy}{dt} + y = 0
\end{equation}

Which means the voltage shrinks. This ODE exhibits a cool behavior that all initial conditions converge to the same limit cycle. 

\subsection*{Numerical Solutions to Differential Equations}

Simple Harmonic Oscillator. Note, 

\begin{equation}
    \frac{dx}{dt} = v
\end{equation}
\begin{equation}
    \frac{dv}{dt} = -\frac{k}{m}x
\end{equation}

Use taylor expansion to approximate. 

\subsection*{Runge-Kutta Method}

Let 

\begin{equation}
    \dot{x} = f(x)
\end{equation}

Then 

\begin{equation}
    x_{n+1} = x_n + \frac{1}{6}\left( k_1 + 2k_2 + 3k_3 + k_4 \right)
\end{equation}

Where

\begin{equation}
    k_1 = f(x_n)\Delta t, k_2 = f(x_n + \frac{1}{2}k_1)\Delta t, \dots
\end{equation}

\end{document}
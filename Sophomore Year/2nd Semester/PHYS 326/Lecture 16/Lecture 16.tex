\documentclass{article}
\usepackage[left=2cm, right=2cm, top=2cm, bottom=2cm]{geometry}
\usepackage{graphicx}
\usepackage{amsmath}
\usepackage{amssymb}
\usepackage{amsthm}
\usepackage{fancyhdr}
\usepackage{verbatim}
\usepackage{listings}
\usepackage{xcolor}
\usepackage{pgfplots}

\lstset{
    language=C++,
    basicstyle=\ttfamily\footnotesize,
    keywordstyle=\color{blue},
    commentstyle=\color{green},
    stringstyle=\color{red},
    numbers=left,
    numberstyle=\tiny\color{gray},
    frame=single,
    breaklines=true,
    captionpos=b,
}


\title{PHYS 326: Lecture \#16}
\author{Cliff Sun}

\newtheorem{theorem}{Theorem}[section]
\newtheorem{lemma}[theorem]{Lemma}
\newtheorem{definition}[theorem]{Definition}
\newtheorem{conjecture}[theorem]{Conjecture}
\newtheorem{proposition}[theorem]{Proposition}
\newtheorem{corollary}[theorem]{Corollary}
\newtheorem{one minute paper}[theorem]{One Minute Paper}

\pagestyle{fancy}
\lhead{\textbf{\thepage}\ \ \nouppercase{\rightmark}}
\chead{PHYS 326: Lecture \#16}
\rhead{Cliff Sun}

\begin{document}

\maketitle

\section*{Euler Angles}

\textbf{Euler Angles describe any type of orientation for any object.} Specify 3 principle axis of the body, called $(\hat{e}_1, \hat{e}_2, \hat{e}_3)$. Use lab frame $(\hat{x}, \hat{y}, \hat{z})$. Assume that both coordinate systems are initially aligned. 
\begin{enumerate}
    \item Then rotate away from the lab frame by angle $\phi$ around $\hat{z}$. 
    \item Rotation by angle $\theta$ around $\hat{e}_2$. Then $(\hat{e}_1, \hat{e}_2, \hat{e}_3) \rightarrow (\hat{e}'_1, \hat{e}'_2, \hat{e}'_3)$
    \item Rotation by angle $\Psi$ around $\hat{e}'_3$. Then $(\hat{e}'_1, \hat{e}'_2, \hat{e}'_3) \rightarrow (\hat{e}''_1, \hat{e}''_2, \hat{e}''_3)$
\end{enumerate}

\subsection*{Interpretation}

\begin{enumerate}
    \item $\phi = $ precession around lab vertical 
    \item $\theta = $ nutation away from lab vertical 
    \item $\Psi = $ spin around body axis
\end{enumerate}

Write $Q$ matrix corresponding to this "Euler Procedure". 

\begin{equation}
    Q = Q_3 Q_2 Q_1
\end{equation}

\[= \begin{bmatrix}
    \cos\Psi & \sin\Psi & 0 \\
    -\sin \Psi & \cos\Psi & 0 \\
    0 & 0 & 1
\end{bmatrix}
\begin{bmatrix}
    \cos\theta & 0 & -\sin\theta \\
    0 & 1 & 0 \\
    \sin\theta & 0 & \cos\theta
\end{bmatrix}
\begin{bmatrix}
    \cos\phi & \sin\phi & 0 \\
    -\sin\phi & \cos\phi & 0 \\
    0 & 0 & 1
\end{bmatrix}\]

Express $\vec{\omega}$ vector (instantaneous vector) in terms of Euler Angles. 

\begin{equation}
    \vec{\omega} = \dot{\phi} \hat{e}_3 + \dot{\theta}\hat{e}'_2 + \dot{\Psi}\hat{e}''_3
\end{equation}

Will convert this vector into the double prime coordinate system. Note 

\begin{equation}
    \hat{e}_2'' = Q_3 \hat{e}'_2 \iff \hat{e}'_2 = Q_{3}^T \hat{e}''_2 
\end{equation}
and 
\begin{equation}
    \hat{e}'_3 = Q_2\underbrace{Q_1}_{I}\hat{e}_3 \iff \hat{e}_3 = Q_{2}^T \hat{e}_3' = Q_{3}^TQ_{2}^T\hat{e}_3''
\end{equation}

Then 

\begin{equation}
    \vec{\omega} = (\dot{\theta}\sin\Psi - \dot{\phi}\sin\theta\cos\Psi)\hat{e}''_1 + (\dot{\phi}\sin\theta\sin\Psi + \dot{\theta}\cos\Psi)\hat{e}''_2 + (\dot{\Psi} + \dot{\phi}\cos\theta)\hat{e}''_3
\end{equation}



\end{document}
\documentclass{article}
\usepackage[left=2cm, right=2cm, top=2cm, bottom=2cm]{geometry}
\usepackage{graphicx}
\usepackage{amsmath}
\usepackage{amssymb}
\usepackage{amsthm}
\usepackage{fancyhdr}
\usepackage{verbatim}
\usepackage{listings}
\usepackage{xcolor}
\usepackage{pgfplots}

\lstset{
    language=C++,
    basicstyle=\ttfamily\footnotesize,
    keywordstyle=\color{blue},
    commentstyle=\color{green},
    stringstyle=\color{red},
    numbers=left,
    numberstyle=\tiny\color{gray},
    frame=single,
    breaklines=true,
    captionpos=b,
}


\title{PHYS 326: Lecture 6}
\author{Cliff Sun}

\newtheorem{theorem}{Theorem}[section]
\newtheorem{lemma}[theorem]{Lemma}
\newtheorem{definition}[theorem]{Definition}
\newtheorem{conjecture}[theorem]{Conjecture}
\newtheorem{proposition}[theorem]{Proposition}
\newtheorem{corollary}[theorem]{Corollary}
\newtheorem{one minute paper}[theorem]{One Minute Paper}

\pagestyle{fancy}
\lhead{\textbf{\thepage}\ \ \nouppercase{\rightmark}}
\chead{PHYS 326: Lecture 6}
\rhead{Cliff Sun}

\begin{document}

\maketitle

\section*{Symmetries} 

\begin{definition}
    A symmetry is defined as a change of coordinates that leaves the lagrangian unchanged. $\textbf{S}$ is a symmetric matrix if 
    \begin{equation}
        \textbf{S}^{T}K\textbf{S} = K
    \end{equation}
    \begin{equation}
        \textbf{S}^{T}M\textbf{S} = M
    \end{equation}
    We define a new set of coordinates $\vec{q} = \textbf{S}\vec{p}$, then 
    \begin{equation}
        L(q) = L(p)
    \end{equation}
\end{definition}

Then
\begin{equation}
    PE = \frac{1}{2}q^{T}\textbf{K}q
\end{equation}
\begin{equation}
    = \frac{1}{2}(Sp)^{T}\textbf{K}Sp
\end{equation}
\begin{equation}
    = \frac{1}{2}p\textbf{K}p
\end{equation}

We have various types of symmetries, namely cyclic and mirror symmetries. 

\begin{theorem}
    An eigenvector of the system is also an eigenvector of the symmetry matrix. 
\end{theorem}

Then if 

\begin{equation}
    \textbf{K}\vec{u} = \omega^2 \textbf{M}\vec{u}
\end{equation}
Then 
\begin{equation}
    \textbf{S}\vec{u} = \lambda \vec{u}
\end{equation}

\begin{proof}
    Given
    \begin{equation}
        \textbf{K}\vec{u} = \omega^2 \textbf{M}\vec{u}
    \end{equation}
    Multiply by $\textbf{S}^{T}$ and insert $\textbf{S}\textbf{S}^{-1}$:
    \begin{equation}
        \textbf{S}^{T}\textbf{K}\textbf{S}\textbf{S}^{-1}\vec{u} = \omega^2 \textbf{S}^{T}\textbf{M}\textbf{S}\textbf{S}^{-1}\vec{u}
    \end{equation}
    \begin{equation}
        \textbf{K}\textbf{S}^{-1}\vec{u} = \omega^2\textbf{M}\textbf{S}^{-1}\vec{u}
    \end{equation}
    Therefore, 
    \begin{equation}
        S^{-1}u
    \end{equation}
    obeys the E.O.M. with same eigenvalue as $\vec{u}$. In other words, the 2 vectors are same up to a constant. Therefore 
    \begin{equation}
        \textbf{S}u = \lambda u
    \end{equation}
\end{proof}

We look into reflection, mirror, or parity symmetries. When applied twice, they recover the original state. That is $S^{2} = I$. Then 
\begin{equation}
    Sx = \lambda x
\end{equation}
\begin{equation}
    S^2 x = \lambda^2 x
\end{equation}
Thus
\begin{equation}
    \lambda = \pm 1
\end{equation}


\end{document}
\documentclass{article}
\usepackage[left=2cm, right=2cm, top=2cm, bottom=2cm]{geometry}
\usepackage{graphicx}
\usepackage{amsmath}
\usepackage{amssymb}
\usepackage{amsthm}
\usepackage{fancyhdr}
\usepackage{verbatim}
\usepackage{listings}
\usepackage{xcolor}
\usepackage{pgfplots}

\lstset{
    language=C++,
    basicstyle=\ttfamily\footnotesize,
    keywordstyle=\color{blue},
    commentstyle=\color{green},
    stringstyle=\color{red},
    numbers=left,
    numberstyle=\tiny\color{gray},
    frame=single,
    breaklines=true,
    captionpos=b,
}


\title{PHYS 326: Lecture 8}
\author{Cliff Sun}

\newtheorem{theorem}{Theorem}[section]
\newtheorem{lemma}[theorem]{Lemma}
\newtheorem{definition}[theorem]{Definition}
\newtheorem{conjecture}[theorem]{Conjecture}
\newtheorem{proposition}[theorem]{Proposition}
\newtheorem{corollary}[theorem]{Corollary}
\newtheorem{one minute paper}[theorem]{One Minute Paper}

\pagestyle{fancy}
\lhead{\textbf{\thepage}\ \ \nouppercase{\rightmark}}
\chead{PHYS 326: Lecture 8}
\rhead{Cliff Sun}

\begin{document}

\maketitle

\section*{Forced motion of many d.o.f system}

Normal coordinates 

\begin{equation}
    x = \sum_i \eta_i u^{(i)}
\end{equation}

That is 

\begin{equation}
    x = \textbf{U}\eta
\end{equation}

Where $\textbf{U}$ is the modal matrix. Then 

\begin{equation}
    \textbf{M}\textbf{U}\ddot{\eta} + \textbf{K}\textbf{U}\eta = Q
\end{equation}
\begin{equation}
    \iff \textbf{U}^{T}\textbf{M}\textbf{U}\ddot{\eta} + \textbf{U}^{T}\textbf{K}\textbf{U}\eta = \textbf{U}^{T}Q
\end{equation}

The solution is 

\begin{equation}
    \eta_i = \begin{cases}
        \frac{P_2\sin(\omega_i t)}{\omega_i} & t \geq 0\\
        0 & t < 0
    \end{cases}
\end{equation}

\end{document}
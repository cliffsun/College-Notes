\documentclass{article}
\usepackage[left=2cm, right=2cm, top=2cm, bottom=2cm]{geometry}
\usepackage{graphicx}
\usepackage{amsmath}
\usepackage{amssymb}
\usepackage{amsthm}
\usepackage{fancyhdr}
\usepackage{verbatim}
\usepackage{listings}
\usepackage{xcolor}
\usepackage{pgfplots}

\lstset{
    language=C++,
    basicstyle=\ttfamily\footnotesize,
    keywordstyle=\color{blue},
    commentstyle=\color{green},
    stringstyle=\color{red},
    numbers=left,
    numberstyle=\tiny\color{gray},
    frame=single,
    breaklines=true,
    captionpos=b,
}


\title{PHYS 326: Lecture 7}
\author{Cliff Sun}

\newtheorem{theorem}{Theorem}[section]
\newtheorem{lemma}[theorem]{Lemma}
\newtheorem{definition}[theorem]{Definition}
\newtheorem{conjecture}[theorem]{Conjecture}
\newtheorem{proposition}[theorem]{Proposition}
\newtheorem{corollary}[theorem]{Corollary}
\newtheorem{one minute paper}[theorem]{One Minute Paper}

\pagestyle{fancy}
\lhead{\textbf{\thepage}\ \ \nouppercase{\rightmark}}
\chead{PHYS 326: Lecture 7}
\rhead{Cliff Sun}

\begin{document}

\maketitle

\section*{Symmetries}

\subsection*{Crystal Lattice}

Infinite number of degrees of freedom. All masses $m$ and spring constants $k$. Consider the spatial coordinates $q_{n}$ where $q_{n}$ is the middle one. Each 
mass represents a single atom, springs are the forces between neighboring atoms. Then 
\begin{equation}
    L = \frac{m}{2}\left(\dots + \dot{q}^2_{n-2} + \dots \right) - \frac{k}{2}\left(\dots + (q_{n-2} - q_{n-1})^2 + (q_{n-1} - q)^{2} + \dots\right)
\end{equation}

Thus 
\begin{equation}
    m\ddot{q}_{n} + k(2q_{n} - q_{n+1} - q_{n-1}) = 0 \text{ for each $n$ }
\end{equation}

Seek solution, let 

\begin{equation}
    \bar{q}(t) = \bar{u}e^{i\omega t} 
\end{equation}

Solve 

\begin{equation}
    \left[\textbf{K} - \omega^2\textbf{M}\right]\bar{u} = 0
\end{equation}

Using parity symmetry. Move left or right and reproduce the exact same system. Let 

\begin{equation}
    \bar{u} = \begin{bmatrix}
        \dots \\
        \dots \\
        1 \\ 
        a\\
        b \\
        c \\
        \dots
    \end{bmatrix}
\end{equation}

Apply a symmetry matrix: 

\begin{equation}
    \textbf{S}\bar{u} = \begin{bmatrix}
        \dots \\
        \dots \\
        \dots \\ 
        1\\
        a \\
        b \\
        \dots
    \end{bmatrix} = \lambda \bar{u}
\end{equation}

Thus 
\begin{equation}
    1 = \lambda a \iff a = \frac{1}{\lambda}
\end{equation}
\begin{equation}
    a = \lambda b \iff b = \frac{1}{\lambda^2}
\end{equation}

Thus 

\begin{equation}
    \bar{u} = \begin{bmatrix}
        \dots \\
        1 \\
        \frac{1}{\lambda} \\
        \frac{1}{\lambda^2} \\
        \dots
    \end{bmatrix}
\end{equation}

For infinite system, $|\lambda| = 1$, otherwise $\bar{u}$ is not physical. If $\lambda = 1$, then all atoms are moving together, or $\omega = 0$. If $\lambda = 1$, then 
\begin{equation}
    u_{n+1} = u_{n}
\end{equation} 

Generally, $\lambda = e^{-i \phi}$. Thus 

\begin{equation}
    q_{n} = e^{i(\omega t - n \phi)}
\end{equation}

We find the frequencies:

\begin{equation}
    m\ddot{q}_{n} + k(2q_{n} - q_{n-1} -q_{n+1}) = 0
\end{equation}
Then 
\begin{equation}
    -m\omega^2 + k(-e^{i\phi} - e^{i \phi} ) = 0
\end{equation}
\begin{equation}
    \omega^2 = \frac{2k}{m}(1 - \cos\phi)
\end{equation}

\end{document}
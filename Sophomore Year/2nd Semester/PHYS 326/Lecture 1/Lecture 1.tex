\documentclass{article}
\usepackage[left=2cm, right=2cm, top=2cm, bottom=2cm]{geometry}
\usepackage{graphicx}
\usepackage{amsmath}
\usepackage{amssymb}
\usepackage{amsthm}
\usepackage{fancyhdr}
\usepackage{verbatim}
\usepackage{listings}
\usepackage{xcolor}
\usepackage{pgfplots}

\lstset{
    language=C++,
    basicstyle=\ttfamily\footnotesize,
    keywordstyle=\color{blue},
    commentstyle=\color{green},
    stringstyle=\color{red},
    numbers=left,
    numberstyle=\tiny\color{gray},
    frame=single,
    breaklines=true,
    captionpos=b,
}


\title{PHYS 326: Lecture 1}
\author{Cliff Sun}

\newtheorem{theorem}{Theorem}[section]
\newtheorem{lemma}[theorem]{Lemma}
\newtheorem{definition}[theorem]{Definition}
\newtheorem{conjecture}[theorem]{Conjecture}
\newtheorem{proposition}[theorem]{Proposition}
\newtheorem{corollary}[theorem]{Corollary}
\newtheorem{one minute paper}[theorem]{One Minute Paper}

\pagestyle{fancy}
\lhead{\textbf{\thepage}\ \ \nouppercase{\rightmark}}
\chead{PHYS 326: Lecture 1}
\rhead{Cliff Sun}

\begin{document}

\maketitle

\section*{Review: Simple Harmonic Oscillator}

For a particle moving with potential $V(x)$, then 
\begin{equation}
    L = T - U = \frac{1}{2}mv^2 - V(x)
\end{equation}
Then
\begin{equation}
    \frac{d}{dt}\frac{\partial L}{\partial \dot{x}} = \frac{\partial L}{\partial x}
\end{equation}
\begin{equation}
    \implies m\ddot{x} = -V'(x) = F(x)
\end{equation}
To find the equilibrium points, we have 
\begin{equation}
    \ddot{x} = 0 \implies V'(x) = 0
\end{equation}
For a stable equilibrium, you need
\begin{equation}
    \frac{\partial F}{\partial x} \leq 0 \text{ retrieving force }
\end{equation}

\subsection*{Double Well potential}

Picture a graph with 2 stable equilibrium. We look at small deviations at the stable equilibrium. Define
\begin{equation}
    y(t) = x(t) - x_{eq}
\end{equation}
\begin{equation}
    L = \frac{1}{2}m\dot{x}^2 - V(x)
\end{equation}
\begin{equation}
    L(y) = \frac{1}{2}m\dot{y}^2 - V(x_{eq} + y)
\end{equation}
Then 
\begin{equation}
    V(x_{eq} + y) \approx V(x_{eq}) + \frac{\partial V}{\partial x}y + \frac{1}{2}\frac{\partial^2 V}{\partial^2 x}y^2
\end{equation}
The 2nd term drops out, so then 
\begin{equation}
    L(y) = \frac{1}{2}m\dot{x}^2 - V(x_{eq}) - \frac{1}{2}ky^2
\end{equation}

We find Euler Lagrange for y:

\begin{equation}
    \frac{d}{dt}(\frac{\partial L}{\partial \dot{y}}) - \frac{\partial L}{\partial y} = 0
\end{equation}
\begin{equation}
    \implies m\ddot{y} + ky = 0
\end{equation}
Let our ansatz be 
\begin{equation}
    y(t) = ae^{i\omega t}
\end{equation}
then
\begin{equation}
    \left[k - m\omega^2\right]ae^{i\omega t} = 0
\end{equation}
General non-trival solution:
\begin{equation}
    y(t) = a_{+}e^{i\omega t} + a_{-}e^{-i\omega t}
\end{equation}

\subsection*{Forced Motion of S.H.O.}

\subsubsection*{Harmonic forcing}

\begin{equation}
    m\ddot{y} + ky = F(t)
\end{equation}

If $F(t) = f_0e_{i\omega t}$, 

\begin{equation}
    (-m\omega^2 + k)y_0e^{i\omega_d t} = f_0e_{i\omega t} 
\end{equation}

Then 
\begin{equation}
    y_0 = \frac{f_0}{m}\frac{1}{\omega_0^2 - \omega_d^2}
\end{equation}

\subsubsection*{Impulsive Forcing}

\begin{equation}
    f(t) = I_0\delta(t)
\end{equation}
Assume $y = \dot{y} = 0$ at $t = 0$, then the change in momentum is
\begin{equation}
    \Delta P = \int f(t) dt = I_0
\end{equation}
Initial conditions:
\begin{equation}
    y(t = 0^{+}) = 0
\end{equation}
\begin{equation}
    y(t=0^{+}) = \frac{\Delta P}{m} = \frac{I_0}{m}
\end{equation}
We use these initial conditions to solve the SHO to get:
\begin{equation}
    y(t) = \frac{I_0}{m\omega_0}\sin(\omega_0 t)\Theta(t)
\end{equation}
Where $\Theta(t)$ is the heavyside function. 
\begin{equation}
    y(t) \iff I_0G(t)
\end{equation}
Where $G(t)$ is the impulse response to the system. 

\subsubsection*{Arbitrary Forcing}

\begin{equation}
    m\ddot{y} + ky = f(t)
\end{equation}

Solution is
\begin{equation}
    y(t) = \int_{-\infty}^{t}f(\tau)G(t-\tau)d\tau
\end{equation}

\subsubsection*{SHO Generalization}
\begin{enumerate}
    \item Non-linearity
    \item Damping
    \item Next: Multiple Coupled Oscillators
\end{enumerate}

\end{document}
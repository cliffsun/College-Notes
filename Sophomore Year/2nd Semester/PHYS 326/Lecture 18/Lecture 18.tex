\documentclass{article}
\usepackage[left=2cm, right=2cm, top=2cm, bottom=2cm]{geometry}
\usepackage{graphicx}
\usepackage{amsmath}
\usepackage{amssymb}
\usepackage{amsthm}
\usepackage{fancyhdr}
\usepackage{verbatim}
\usepackage{listings}
\usepackage{xcolor}
\usepackage{pgfplots}

\lstset{
    language=C++,
    basicstyle=\ttfamily\footnotesize,
    keywordstyle=\color{blue},
    commentstyle=\color{green},
    stringstyle=\color{red},
    numbers=left,
    numberstyle=\tiny\color{gray},
    frame=single,
    breaklines=true,
    captionpos=b,
}


\title{PHYS 326: Lecture \# 18}
\author{Cliff Sun}

\newtheorem{theorem}{Theorem}[section]
\newtheorem{lemma}[theorem]{Lemma}
\newtheorem{definition}[theorem]{Definition}
\newtheorem{conjecture}[theorem]{Conjecture}
\newtheorem{proposition}[theorem]{Proposition}
\newtheorem{corollary}[theorem]{Corollary}
\newtheorem{one minute paper}[theorem]{One Minute Paper}

\pagestyle{fancy}
\lhead{\textbf{\thepage}\ \ \nouppercase{\rightmark}}
\chead{PHYS 326: Lecture \# 18}
\rhead{Cliff Sun}

\begin{document}

\maketitle

\section*{Hamiltonian Mechanics}

Begin with $L(q_i, \dot{q}_i; t) = T - U$ for $i = 1,\dots,n$. Then use Euler Lagrangian equations to study mechanics. Define generalized momentum
\begin{equation}
    p_i = \frac{\partial L}{\partial \dot{q}_i}
\end{equation}  

Hamilton aims to describe the state of the system in the phase space ($q_i$, $p_i$). Define the Hamiltonain:

\begin{equation}
    H = \sum_{i=1}^n p_i \dot{q}_i - L \iff \text{ Jacobi Function, Legendre Transformation}
\end{equation}

Next, find the equations of motion in terms of $p_i, q_i$. 

\begin{equation}
    H(\vec{q}, \vec{p}, t) = \sum_i p_i\dot{q}_i(\vec{q}, \vec{p}; t) - L(\vec{q}, \dot{\vec{q}}(\vec{q}, \vec{p}; t); t)
\end{equation}

Then 

\begin{equation}
    \frac{\partial H}{\partial p_j} = \dot{q}_j + \sum_i p_i \frac{\partial \dot{q}_i}{\partial p_j} - \sum_i \frac{\partial L}{\partial \dot{q}_i}\frac{\partial \dot{q}_i}{\partial p_j} = \dot{q}_j
\end{equation}

This is because $p_i = \partial L/\partial \dot{q}_i$. Similarly 

\begin{equation}
    \frac{\partial H}{\partial q_j} = \sum_i p_i \frac{\partial \dot{q}_i}{\partial q_j} - \frac{\partial L}{\partial q_j} - \sum_i \frac{\partial L}{\partial \dot{q}_i}\frac{\partial \dot{q}_i}{\partial q_i} = -\frac{d}{dt}\left(\frac{\partial L}{\partial \dot{q}_i} \right) = -\dot{p}_j
\end{equation}

\subsection*{Conservation Laws}

If $H$ does not depend explicitly on $t$, then $H$ is a constant of motion. 

\begin{proof}
    \[\frac{dH}{dt} = \sum_{i} \underbrace{\frac{\partial H}{\partial q_i}}_{=-\dot{p}_i}\dot{q}_i + \sum_i \underbrace{\frac{\partial H}{\partial p_i}}_{=\dot{q}_i}\dot{p}_i + \underbrace{\frac{\partial H}{\partial t}}_{=0} = 0\]
\end{proof}

If a coordinate $q$ does not appear in $H$, then its conjugate momentum is conserved. 

\begin{proof}
    \[\dot{p}_{q} = -\frac{\partial H}{\partial q} = 0\]
\end{proof}

\subsubsection*{Poisson Brackets}

\begin{definition}
    $\{A,B\}$ is defined as 
    \[\{A,B\} = \sum_i \left[ \frac{\partial A}{\partial q_i}\frac{\partial B}{\partial p_i} - \frac{\partial B}{\partial q_i}\frac{\partial A}{\partial p_i} \right]\]
\end{definition}

\end{document}
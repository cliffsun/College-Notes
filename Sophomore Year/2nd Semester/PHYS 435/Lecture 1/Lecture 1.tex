\documentclass{article}
\usepackage[left=2cm, right=2cm, top=2cm, bottom=2cm]{geometry}
\usepackage{graphicx}
\usepackage{amsmath}
\usepackage{amssymb}
\usepackage{amsthm}
\usepackage{fancyhdr}
\usepackage{verbatim}
\usepackage{listings}
\usepackage{xcolor}
\usepackage{pgfplots}

\lstset{
    language=C++,
    basicstyle=\ttfamily\footnotesize,
    keywordstyle=\color{blue},
    commentstyle=\color{green},
    stringstyle=\color{red},
    numbers=left,
    numberstyle=\tiny\color{gray},
    frame=single,
    breaklines=true,
    captionpos=b,
}


\title{PHYS 435: Lecture 1}
\author{Cliff Sun}

\newtheorem{theorem}{Theorem}[section]
\newtheorem{lemma}[theorem]{Lemma}
\newtheorem{definition}[theorem]{Definition}
\newtheorem{conjecture}[theorem]{Conjecture}
\newtheorem{proposition}[theorem]{Proposition}
\newtheorem{corollary}[theorem]{Corollary}
\newtheorem{one minute paper}[theorem]{One Minute Paper}

\pagestyle{fancy}
\lhead{\textbf{\thepage}\ \ \nouppercase{\rightmark}}
\chead{PHYS 435: Lecture 1}
\rhead{Cliff Sun}

\begin{document}

\maketitle

Coulomb's Law:

\begin{equation}
    \vec{E}(\vec{r}) = \frac{1}{4\pi\epsilon_0}\frac{q_0}{|r-r'|^3}(r-r')
\end{equation}

Because this is a linear phenomenon, that is doubling the charge yields a double in the electric field (because you are just adding another charge to the same position), this implies superposition.
Then 
\begin{equation}
    \vec{E}(\vec{r}) = \sum\frac{1}{4\pi\epsilon_0}\frac{q_{0,i}}{|r-r'|^3}(r-r')
\end{equation}
\begin{equation}
    \implies \frac{1}{4\pi\epsilon_0}\int d^{3}r'\frac{(r-r')}{|r-r'|^3}\rho(r')
\end{equation}

Gauss's law states:

\begin{equation}
    \frac{1}{\epsilon_0}\int_V d^3r' \rho(r') = \int_{\partial V} d\vec{a} \cdot \vec{E}(\vec{r})
\end{equation}

Then, 

\begin{equation}
    \int_V d^3r' \vec{\nabla} \cdot \vec{E}(\vec{r}) = \int_{\partial V}d\vec{a} \cdot \vec{E}(\vec{r})
\end{equation}

Then 

\begin{equation}
    \vec{\nabla} \cdot \vec{E}(\vec{r}) = \frac{\rho(\vec{r})}{\epsilon_0}
\end{equation}

Faraday's Law: the \underline{circulation} of any E field around any given point is equivalent to $-1$ times the time partial derivative of the B field. 

\begin{equation}
    \int_{\partial S}dl \cdot E = -\frac{d}{dt}\int_S da \cdot B
\end{equation}

This is a non-local statement, that is calculating the value at specific point depends on the value at other points. 

We use stokes theorem:

\begin{equation}
    \int_{\partial S}dl \cdot E = \int_{S} d\vec{a} \cdot \vec{\nabla} \times \vec{E}
\end{equation}

Then 

\begin{equation}
    \int_{S} d\vec{a} \cdot \vec{\nabla} \times \vec{E} = -\int_S da \cdot \frac{dB}{dt}
\end{equation}

Thus 

\begin{equation}
    vec{\nabla} \times \vec{E} = -\frac{\partial B}{\partial t}
\end{equation}

Similarly,

\begin{equation}
    \vec{\nabla} \times \vec{B} = \mu_0 \vec{J}
\end{equation}
\end{document}
\documentclass{article}
\usepackage[left=2cm, right=2cm, top=2cm, bottom=2cm]{geometry}
\usepackage{graphicx}
\usepackage{amsmath}
\usepackage{amssymb}
\usepackage{amsthm}
\usepackage{fancyhdr}
\usepackage{verbatim}
\usepackage{listings}
\usepackage{xcolor}
\usepackage{pgfplots}

\lstset{
    language=C++,
    basicstyle=\ttfamily\footnotesize,
    keywordstyle=\color{blue},
    commentstyle=\color{green},
    stringstyle=\color{red},
    numbers=left,
    numberstyle=\tiny\color{gray},
    frame=single,
    breaklines=true,
    captionpos=b,
}


\title{PHYS 435: Lecture 8}
\author{Cliff Sun}

\newtheorem{theorem}{Theorem}[section]
\newtheorem{lemma}[theorem]{Lemma}
\newtheorem{definition}[theorem]{Definition}
\newtheorem{conjecture}[theorem]{Conjecture}
\newtheorem{proposition}[theorem]{Proposition}
\newtheorem{corollary}[theorem]{Corollary}
\newtheorem{one minute paper}[theorem]{One Minute Paper}

\pagestyle{fancy}
\lhead{\textbf{\thepage}\ \ \nouppercase{\rightmark}}
\chead{PHYS 435: Lecture 8}
\rhead{Cliff Sun}

\begin{document}

\maketitle

To develop well posed problems of potential energy, we need 

\begin{enumerate}
    \item Cartesian Coordinates: $V(r)$ inside box and $\partial V$ on box. 
    \item Spherical Coordinates: $V(r)$ inside and outside of sphere and $\partial V$ on sphere. 
    \item Cylindrical Coordinates: $V(r)$ inside cylinder. $\partial V$ on cylinder, and if finite surface, on caps. 
\end{enumerate}

We solve the box problem, finding $V(r)$ on the inside and $\nabla^2 V(r) = 0$ in the inside (no charges). Then guess 
\begin{equation}
    V(r) = X(x)Y(y)Z(z)
\end{equation}
Then finding the laplacian, we yield 

\begin{equation}
    \frac{1}{X(x)}\partial^2_x X(x) + \dots = 0 
\end{equation}
This implies 
\begin{equation}
    \frac{1}{X(x)}\partial^2_x X(x) = a
\end{equation}
\begin{equation}
    \frac{1}{Y(y)}\partial^2_y Y(y) = b
\end{equation}
\begin{equation}
    \frac{1}{Z(z)}\partial^2_z Z(z) = c
\end{equation}



\end{document}
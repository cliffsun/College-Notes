\documentclass{article}
\usepackage[left=2cm, right=2cm, top=2cm, bottom=2cm]{geometry}
\usepackage{graphicx}
\usepackage{amsmath}
\usepackage{amssymb}
\usepackage{amsthm}
\usepackage{fancyhdr}
\usepackage{verbatim}
\usepackage{listings}
\usepackage{xcolor}
\usepackage{pgfplots}

\lstset{
    language=C++,
    basicstyle=\ttfamily\footnotesize,
    keywordstyle=\color{blue},
    commentstyle=\color{green},
    stringstyle=\color{red},
    numbers=left,
    numberstyle=\tiny\color{gray},
    frame=single,
    breaklines=true,
    captionpos=b,
}


\title{PHYS 435: Lecture 1}
\author{Cliff Sun}

\newtheorem{theorem}{Theorem}[section]
\newtheorem{lemma}[theorem]{Lemma}
\newtheorem{definition}[theorem]{Definition}
\newtheorem{conjecture}[theorem]{Conjecture}
\newtheorem{proposition}[theorem]{Proposition}
\newtheorem{corollary}[theorem]{Corollary}
\newtheorem{one minute paper}[theorem]{One Minute Paper}

\pagestyle{fancy}
\lhead{\textbf{\thepage}\ \ \nouppercase{\rightmark}}
\chead{PHYS 435: Lecture 1}
\rhead{Cliff Sun}

\begin{document}

\maketitle

\section*{Infinite Line of charge}

Suppose a infinite line of charge. How do we calculate the E field at a specific point? We take an Gaussian cylinder around some length
$L$ around the Line. We obtain that at a distance $s$, we get
\begin{equation}
    2\pi E(s)sL = \frac{\lambda \cdot L}{\epsilon_0}
\end{equation}
We solve for
\begin{equation}
    E(s) = \frac{1}{2\pi\epsilon_0}\frac{\lambda}{s}
\end{equation}

Solving for potential energy gets 

\begin{equation}
    \frac{\lambda}{2\pi\epsilon_0}\ln\frac{a}{s}
\end{equation}
This is the weakest analytical divergence. We can derive a PDE for the potential energy function:
\begin{equation}
    \vec{\nabla} \cdot \vec{E} = \frac{\rho}{\epsilon_0}
\end{equation}
Let $E = -\vec{\nabla}V$, then 
\begin{equation}
    \nabla^{2}V(r) = -\frac{\rho}{\epsilon}
\end{equation}

The total work done by moving point charges into each other is 

\begin{equation}
    U = \frac{1}{2}\sum_{i=1}^{N}\sum_{j\neq i}\frac{1}{4\pi\epsilon_0}\frac{q_iq_j}{|\vec{r_2} - \vec{r_1}|}
\end{equation}
The continuous analogue of this is 
\begin{equation}
    U = \frac{1}{2}\int_{S}d^{3}rd^3r'\frac{1}{4\pi\epsilon_0}\frac{\rho(r)\rho(r')}{|r-r'|}
\end{equation}

Let the potential energy function be 

\begin{equation}
    V(r') =  \frac{1}{4\pi\epsilon_0}\int d^{r}\frac{\rho(r)}{|r - r'|}
\end{equation}

Then 

\begin{equation}
    U = \frac{1}{2}\int d^{3}r' \rho(r')V(r')
\end{equation}
Since 
\begin{equation}
    \rho(r') = -\epsilon_0\nabla^{2}V(r')
\end{equation}
Then our integral turns into 
\begin{equation}
    U = -\frac{\epsilon_0}{2}\int d^{3}r'V(r')\nabla^{2}V(r')
\end{equation}
First evaluating the x partial derivative, then we can generalize:
\begin{equation}
    U = -\frac{\epsilon_0}{2}\int d^{r}V(r)\partial_x\left[\partial_x V(r)\right]
\end{equation}
We first evaluate:
\begin{equation}
    \partial_x\left[V(r)\partial_xV(r)\right] = \partial_xV(r) \cdot \partial_xV(r) + V\partial^{2}_xV
\end{equation}
\begin{equation}
    \partial_x\left[V(r)\partial_xV(r)\right] - \partial_xV(r) \cdot \partial_xV(r) = V\partial^{2}_xV
\end{equation}
\begin{equation}
    \implies -\frac{\epsilon_0}{2}\int d^3r \left[\partial_x(V(r)\partial_x(V(r))) - E_x^2\right]
\end{equation}
We generalize this:
\begin{equation}
    \implies \frac{\epsilon_0}{2}\int  d^{3}r \left[\vec{\nabla} \cdot \left(V(r)E(r)\right) + \vec{E} \cdot \vec{E}\right]
\end{equation}
We note that 
\begin{equation}
    \vec{\nabla} \cdot \left(V(r)E(r)\right) = 0
\end{equation}
Then 
\begin{equation}
    u = \int d^{r}\frac{\epsilon_0}{2}E^{2}
\end{equation}
\end{document}
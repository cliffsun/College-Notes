\documentclass{article}
\usepackage[left=2cm, right=2cm, top=2cm, bottom=2cm]{geometry}
\usepackage{graphicx}
\usepackage{amsmath}
\usepackage{amssymb}
\usepackage{amsthm}
\usepackage{fancyhdr}
\usepackage{verbatim}
\usepackage{listings}
\usepackage{xcolor}
\usepackage{pgfplots}

\lstset{
    language=C++,
    basicstyle=\ttfamily\footnotesize,
    keywordstyle=\color{blue},
    commentstyle=\color{green},
    stringstyle=\color{red},
    numbers=left,
    numberstyle=\tiny\color{gray},
    frame=single,
    breaklines=true,
    captionpos=b,
}


\title{PHYS 435: Lecture 6}
\author{Cliff Sun}

\newtheorem{theorem}{Theorem}[section]
\newtheorem{lemma}[theorem]{Lemma}
\newtheorem{definition}[theorem]{Definition}
\newtheorem{conjecture}[theorem]{Conjecture}
\newtheorem{proposition}[theorem]{Proposition}
\newtheorem{corollary}[theorem]{Corollary}
\newtheorem{one minute paper}[theorem]{One Minute Paper}

\pagestyle{fancy}
\lhead{\textbf{\thepage}\ \ \nouppercase{\rightmark}}
\chead{PHYS 435: Lecture 6}
\rhead{Cliff Sun}

\begin{document}

\maketitle

\section*{Dirichlet Electrostatic boundary problems}

Find solution for $V(\vec{r})$ inside a volume, then you must 
\begin{enumerate}
    \item Supply $V(\vec{r})$ on boundary $\partial V$
    \item Supply $\rho(\vec{r})$ inside $V$
\end{enumerate}
Then this a well-posed problem. You may pretend that the boundary is a grounded conductor. We solve $V = 0$ on boundary, then we utilize point-source response functions to solve this problem. We have that 
the greens function is $G(r,r')$, then 
\begin{equation}
    V(r) = \frac{1}{4\pi\epsilon_0}\int_V d^3 r' G(r,r')\rho(r')
\end{equation}
In free space without conductors, we have 
\begin{equation}
    V(r) = \frac{1}{4\pi\epsilon_0}\int_v d^3r' \frac{\rho(r')}{|r-r'|}
\end{equation}
Where 
\begin{equation}
    G(r,r')= \frac{1}{|r-r'|}
\end{equation}
Today, we solve the problem of the potential energy of a charge placed above a grounded conductor at $(0,0,z_0)$. We first use the delta function in order to solve this problem. In a grounded conductor, 
we have that the E field is 0 inside the conductor. We can invent an \underline{image solution} to solve this more complicated problem. We consider the system 
\begin{equation}
    S = \{(0,0,z_0), (0,0,-z_0)\}
\end{equation} 
We construct an integral of the potential energy of this problem:
\begin{equation}
    V(r) = \frac{1}{4\pi\epsilon_0}\int_{V}d^3r'\rho(r')\left[\frac{1}{|r-r'|} - \frac{1}{|r-r''|}\right]
\end{equation}
We have that 
\begin{equation}
    r'' = r' - 2\hat{z}(r'\cdot \hat{z})
\end{equation}
\begin{equation}
    V(r) = \frac{1}{4\pi\epsilon_0}\int_{V}d^3r'\rho(r')\left[\frac{1}{|r-r'|} - \frac{1}{|r-r' - 2\hat{z}(r'\cdot \hat{z})|}\right]
    \implies 
\end{equation}

Then 

\begin{equation}
    V(r) = \frac{1}{4\pi\epsilon_0}\int_V d^3r'\rho(r'\left[\frac{1}{\sqrt{(x-x_0)^2 + (y-y_0)^2 + (z-z_0)^2}} - \frac{1}{\sqrt{(x-x_0)^2 + (y-y_0)^2 + (z+z_0)^2}}\right])
\end{equation}

Note, by constructing this symmetrical system, based purely on symmetry, we have that the boundary condition at the $x-y$ plane has a zero potential different. Then, by the well-posed problem theorem, we have that this problem can be applicable to 
our original solution. This solution pretends as if for every particle that exists outside the conductor, there must be an anti-particle that exists inside the conductor. To solve for the surface charge density on the conductor we solve the problem of 
\begin{equation}
    E(x,y,0) = \frac{\sigma(x,y)}{\epsilon_0} = -\nabla V(r)
\end{equation}
Note that 
\begin{equation}
    \frac{\partial }{\partial x}V(r) = \frac{\partial}{\partial y}V(r) = 0
\end{equation}
Since 
\begin{equation}
    \partial V = 0
\end{equation}
Therefore, 
\begin{equation}
    E(x,y,0) = \frac{\sigma(x,y)}{\epsilon_0} = -\frac{\partial }{\partial z} V(r)
\end{equation}

\end{document}
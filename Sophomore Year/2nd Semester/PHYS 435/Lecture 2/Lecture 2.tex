\documentclass{article}
\usepackage[left=2cm, right=2cm, top=2cm, bottom=2cm]{geometry}
\usepackage{graphicx}
\usepackage{amsmath}
\usepackage{amssymb}
\usepackage{amsthm}
\usepackage{fancyhdr}
\usepackage{verbatim}
\usepackage{listings}
\usepackage{xcolor}
\usepackage{pgfplots}

\lstset{
    language=C++,
    basicstyle=\ttfamily\footnotesize,
    keywordstyle=\color{blue},
    commentstyle=\color{green},
    stringstyle=\color{red},
    numbers=left,
    numberstyle=\tiny\color{gray},
    frame=single,
    breaklines=true,
    captionpos=b,
}


\title{PHYS 435: Lecture 2}
\author{Cliff Sun}

\newtheorem{theorem}{Theorem}[section]
\newtheorem{lemma}[theorem]{Lemma}
\newtheorem{definition}[theorem]{Definition}
\newtheorem{conjecture}[theorem]{Conjecture}
\newtheorem{proposition}[theorem]{Proposition}
\newtheorem{corollary}[theorem]{Corollary}
\newtheorem{one minute paper}[theorem]{One Minute Paper}

\pagestyle{fancy}
\lhead{\textbf{\thepage}\ \ \nouppercase{\rightmark}}
\chead{PHYS 435: Lecture 2}
\rhead{Cliff Sun}

\begin{document}

\maketitle

\section*{Potential Theory}

We restate Maxwell's Equations:

\begin{equation}
    \vec{\nabla} \cdot \vec{E}(\vec{r}) = \frac{\rho(\vec{r})}{\epsilon_0}
\end{equation}
\begin{equation}
    \vec{\nabla} \times \vec{E}(\vec{r}) = -\frac{\partial }{\partial t}\vec{B}
\end{equation}
\begin{equation}
    \vec{\nabla} \cdot \vec{B} = 0
\end{equation}
\begin{equation}
    \vec{\nabla} \times \vec{B} = \mu_0 \vec{J} + \mu_0\epsilon_0\frac{\partial }{\partial t}\vec{E}
\end{equation}

Assume static conditions, then all $\frac{\partial }{\partial t}f = 0$ for all functions $f$. Then if 
\begin{equation}
    \vec{\nabla} \times \vec{E} = 0
\end{equation} 
Then we can find $\vec{E}$ by defining a potential function. The issue is that potential functions are not uniquely defined. Then 
\begin{equation}
    \int_{P_i}d\vec{l} \cdot \vec{E}
\end{equation}
is path independent for all paths $P_i$. Then let $P_1$ go from a point $O$ to a point $R$ from $P_2$ goes from $R$ to $O$. Then 
\begin{equation}
    \int_{P_1} + \int_{P_2} = \oint_{\partial S} d\vec{l} \cdot \vec{E} = \int_S \vec{\nabla} \times \vec{E} = 0
\end{equation} 
Thus 
\begin{equation}
    \int_{P_1} = -\int_{P_2}
\end{equation}
Without loss of generality. Then we define a vector potential function relative to a point $O$ to be 
\begin{equation}
    V(\vec{r}) = -\int_{O}^{\vec{r}}d\vec{l'} \cdot \vec{E}
\end{equation}
Note this is a potential function. We define the potential energy function of a charged particle in the presence of an E and B field as 
\begin{equation}
    u(\vec{a}) - u(\vec{b}) = -\int_{a}^{b}d\vec{l} \cdot \vec{F}
\end{equation}
Where $F$ is the lorentz force. We note that the magnetic field does no work on the particle. 
\begin{proof}
    Let $F_{B}$ be the force exerted by the B Field. That is 
    \begin{equation}
        F_{B} = q\vec{v} \times \vec{B}
    \end{equation}
    Then the work done is 
    \begin{equation}
        W = q\int_{a}^{b}d\vec{l} \cdot \vec{v} \times \vec{B}
    \end{equation}
    We note that 
    \begin{equation}
        \frac{d\vec{l}}{dt} = \vec{v}
    \end{equation}
    Then the work turns into  
    \begin{equation}
        q\int_{a}^{b}d\vec{l} \cdot \frac{d\vec{l}}{dt} \times \vec{B}
    \end{equation}
    We multiply this integral by $\frac{dt}{dt}$ to obtain 
    \begin{equation}
        q\int_{a}^{b}dt\frac{d\vec{l}}{dt} \cdot \frac{d\vec{l}}{dt} \times \vec{B}
    \end{equation}
    But 
    \begin{equation}
        \frac{d\vec{l}}{dt} \perp \frac{d\vec{l}}{dt} \times \vec{B}
    \end{equation}
    Thus this integral evaluates to $0$.
\end{proof} 

We first define 
\begin{equation}
    V(\vec{r}) = -\int_{O}^{\vec{r}}d\vec{l'} \cdot \vec{E}
\end{equation}
Let 
\begin{equation}
    \vec{r} \rightarrow \vec{r} + dx \hat{x}
\end{equation}
Then 
\begin{equation}
    V(\vec{r} + dx\hat{x}) \approx V(\vec{r}) + dx \hat{x} \cdot \vec{E}
\end{equation}
Because this is how work is defined. Then 
\begin{equation}
    \frac{V(\vec{r} + \hat{x}dx) - V(\vec{r})}{dx} = E_{x}(\vec{r})
\end{equation}
\end{document}
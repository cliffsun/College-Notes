\documentclass{article}
\usepackage[left=2cm, right=2cm, top=2cm, bottom=2cm]{geometry}
\usepackage{graphicx}
\usepackage{amsmath}
\usepackage{amssymb}
\usepackage{amsthm}
\usepackage{fancyhdr}
\usepackage{verbatim}
\usepackage{listings}
\usepackage{xcolor}
\usepackage{pgfplots}

\lstset{
    language=C++,
    basicstyle=\ttfamily\footnotesize,
    keywordstyle=\color{blue},
    commentstyle=\color{green},
    stringstyle=\color{red},
    numbers=left,
    numberstyle=\tiny\color{gray},
    frame=single,
    breaklines=true,
    captionpos=b,
}


\title{MTH 447: Lecture 4}
\author{Cliff Sun}

\newtheorem{theorem}{Theorem}[section]
\newtheorem{lemma}[theorem]{Lemma}
\newtheorem{definition}[theorem]{Definition}
\newtheorem{conjecture}[theorem]{Conjecture}
\newtheorem{proposition}[theorem]{Proposition}
\newtheorem{corollary}[theorem]{Corollary}
\newtheorem{one minute paper}[theorem]{One Minute Paper}

\pagestyle{fancy}
\lhead{\textbf{\thepage}\ \ \nouppercase{\rightmark}}
\chead{MTH 447: Lecture 4}
\rhead{Cliff Sun}

\begin{document}

\maketitle

$\mathbb{Q}$ has holes. \underline{Completeness axiom for $\mathbb{R}$}. 

\begin{definition}
    Let $S$ be a non-empty set of numbers. IF $\exists M \in S$ s.t. for any $x \in S$, if 
    \begin{equation}
        x \leq M
    \end{equation}
    Then we say that $M$ is the maximum of $S$. Similarly, $\exists m \in S$ s.t.
    \begin{equation}
        m \leq x
    \end{equation}
    Then $m$ is the minimum of $S$. Note that $m, M$ must be in $S$. 
\end{definition} 

\begin{definition}
    Let $S$ be a set of numbers. Then if $\exists M \in \mathbb{R}$ s.t. 
    \begin{equation}
        x \leq M
    \end{equation}
    For all $x \in S$, we say that $M$ is an \underline{upper bound} for $S$. If $S$ has an upper bound, then we say that $S$ is \underline{bounded above}.
\end{definition}

\begin{definition}
    $S$ is \underline{bounded} if bounded above and bounded below. 
\end{definition}

We want to define a minimum of out of all upper bounds.  

\begin{definition}
    Let $S$ be bounded above. If there is a least upper bound, then we call this the \underline{supremum} of $S$. 
\end{definition}

Given $S$ bounded above:
\begin{equation}
    U = \{\text{ all upper bounds of $S$ }\}
\end{equation}
Then 
\begin{equation}
    \sup S = \min U
\end{equation}

\begin{definition}
    Let $S$ be bounded below, then if there is a greatest lower bound, then we call that the \underline{infimum} of $S$. Denoted as $\inf S$. 
\end{definition}

\begin{definition}
    BIG AXIOM (Completeness axiom of real numbers): If $S$ is a set of real numbers, bounded above. Then $\sup S$ exists and is a real number. 
\end{definition}

Let's say that $S$ is not bounded above, then we say that $\sup S = +\infty$. 
\end{document}
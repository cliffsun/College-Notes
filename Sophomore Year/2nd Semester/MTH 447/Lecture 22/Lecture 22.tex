\documentclass{article}
\usepackage[left=2cm, right=2cm, top=2cm, bottom=2cm]{geometry}
\usepackage{graphicx}
\usepackage{amsmath}
\usepackage{amssymb}
\usepackage{amsthm}
\usepackage{fancyhdr}
\usepackage{verbatim}
\usepackage{listings}
\usepackage{xcolor}
\usepackage{pgfplots}

\lstset{
    language=C++,
    basicstyle=\ttfamily\footnotesize,
    keywordstyle=\color{blue},
    commentstyle=\color{green},
    stringstyle=\color{red},
    numbers=left,
    numberstyle=\tiny\color{gray},
    frame=single,
    breaklines=true,
    captionpos=b,
}


\title{MTH 447: Lecture \# 22}
\author{Cliff Sun}

\newtheorem{theorem}{Theorem}[section]
\newtheorem{lemma}[theorem]{Lemma}
\newtheorem{definition}[theorem]{Definition}
\newtheorem{conjecture}[theorem]{Conjecture}
\newtheorem{proposition}[theorem]{Proposition}
\newtheorem{corollary}[theorem]{Corollary}
\newtheorem{one minute paper}[theorem]{One Minute Paper}

\pagestyle{fancy}
\lhead{\textbf{\thepage}\ \ \nouppercase{\rightmark}}
\chead{MTH 447: Lecture \# 22}
\rhead{Cliff Sun}

\begin{document}

\maketitle

\section*{Power Series}

\begin{definition}
    Let $a_n$ be a sequence. Then 
    \begin{equation}
        \sum_{n=0}^\infty a_n x^n
    \end{equation}
    is called a \underline{Power Series} with coefficients $a_n$. 
\end{definition}

\begin{theorem}
    Let $\sum_{n=0}^{\infty}a_nx^n$. Define 
    \begin{equation}
        \beta = \lim \sup |a_n|^{\frac{1}{n}}
    \end{equation}
    Let $R = 1/\beta$, then for $|x| < R$, then 
    \begin{equation}
        \sum a_nx^n \text{ converges absolutely }
    \end{equation}
    For all $|x| > R$, then 
    \begin{equation}
        \sum a_nx^n \text{ diverges }
    \end{equation}
    For $x = R, -R$, we don't know. 
\end{theorem}

\begin{proof}
    $\lim \sup |a_n x^n|^{1/n} = \lim \sup |a_n|^{1/n}|x|$.
    \[ \rightarrow |x| \beta\]
    If $\beta$ is finite, then if $x > 1/\beta$, then it diverges. If $x < 1/\beta$, then it converges. 
\end{proof}

\end{document}
\documentclass{article}
\usepackage[left=2cm, right=2cm, top=2cm, bottom=2cm]{geometry}
\usepackage{graphicx}
\usepackage{amsmath}
\usepackage{amssymb}
\usepackage{amsthm}
\usepackage{fancyhdr}
\usepackage{verbatim}
\usepackage{listings}
\usepackage{xcolor}
\usepackage{pgfplots}

\lstset{
    language=C++,
    basicstyle=\ttfamily\footnotesize,
    keywordstyle=\color{blue},
    commentstyle=\color{green},
    stringstyle=\color{red},
    numbers=left,
    numberstyle=\tiny\color{gray},
    frame=single,
    breaklines=true,
    captionpos=b,
}


\title{MTH 447: Lecture 3}
\author{Cliff Sun}

\newtheorem{theorem}{Theorem}[section]
\newtheorem{lemma}[theorem]{Lemma}
\newtheorem{definition}[theorem]{Definition}
\newtheorem{conjecture}[theorem]{Conjecture}
\newtheorem{proposition}[theorem]{Proposition}
\newtheorem{corollary}[theorem]{Corollary}
\newtheorem{one minute paper}[theorem]{One Minute Paper}

\pagestyle{fancy}
\lhead{\textbf{\thepage}\ \ \nouppercase{\rightmark}}
\chead{MTH 447: Lecture 3}
\rhead{Cliff Sun}

\begin{document}

\maketitle

\begin{definition}
    We say 
    \begin{equation}
        \frac{p}{q} \leq \frac{r}{s}
    \end{equation}
    iff 
    \begin{equation}
        ps \leq rq
    \end{equation}
\end{definition}

In the rational numbers, we have 
\begin{enumerate}
    \item $a,b \in \mathbb{Q}$, $a \leq b$ or $b \leq a$
    \item $a \leq b \land b \leq a \iff a = b$
    \item $a \leq b \land b \leq c \implies a \leq c$
    \item $a \leq b \implies a + c \leq b + c$
    \item $a \leq b \land c \geq 0 \implies ac \leq bc$
\end{enumerate}

Properties in $\mathbb{Q}$:
\begin{enumerate}
    \item $a + c = b + c \implies a = b$
    \item $a \cdot 0 = 0$
    \item $(-a)(b) = -ab$
    \item $(-a)(-b) = ab$
    \item $ac = bc \land c \neq 0 \implies a = b$
    \item $ab = 0 \implies a = 0 \lor b = 0$
    \item $a \leq b \implies b \leq a$
    \item $a \leq b \land c \leq 0 \implies ac \leq bc$
    \item $a \geq 0, b \geq 0 \implies ab \geq 0$
    \item $a^2 \geq 0$
    \item $0 < 1$
    \item $a > 0 \implies a^{-1} > 0$
    \item $0 < a < b \implies 0 < b^{-1} < a^{-1}$
\end{enumerate}

\begin{proof}
    Show that $a \cdot 0 = 0$, then 
    \begin{equation}
        a \cdot 0 \iff a \cdot (0 + 0) \iff a\cdot 0 + a\cdot 0 = a\cdot 0
    \end{equation}
    \begin{equation}
        \iff a \cdot 0 = 0
    \end{equation}
\end{proof}

\begin{definition}
    \begin{equation}
        |a| = \begin{cases}
            a & a > 0 \\
            0 & a = 0 \\
            -a & a < 0
        \end{cases}
    \end{equation}
\end{definition}

\begin{theorem}
    \begin{enumerate}
        \item $|a| > 0$ \& $|a| = 0 \iff a = 0$
        \item $|ab| = |a| \cdot |b|$
        \item $|a + b| \leq |a| + |b|$
    \end{enumerate}
\end{theorem}

Note:
\begin{equation}
    -|a| \leq a \leq |a|
\end{equation}
\begin{equation}
    -|b| \leq b \leq |b|
\end{equation}
Then 
\begin{equation}
    -(|a| + |b|) \leq a + b \leq |a| + |b|
\end{equation}
\begin{lemma}
    If $M > 0$ and 
    \begin{equation}
        -M \leq x \leq M
    \end{equation}
    Then
    \begin{equation}
        |x| \leq M
    \end{equation}
\end{lemma}
Thus we see that 
\begin{equation}
    |a + b| \leq |a| + |b|
\end{equation}

\end{document}
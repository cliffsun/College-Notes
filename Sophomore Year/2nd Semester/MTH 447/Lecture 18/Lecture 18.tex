\documentclass{article}
\usepackage[left=2cm, right=2cm, top=2cm, bottom=2cm]{geometry}
\usepackage{graphicx}
\usepackage{amsmath}
\usepackage{amssymb}
\usepackage{amsthm}
\usepackage{fancyhdr}
\usepackage{verbatim}
\usepackage{listings}
\usepackage{xcolor}
\usepackage{pgfplots}

\lstset{
    language=C++,
    basicstyle=\ttfamily\footnotesize,
    keywordstyle=\color{blue},
    commentstyle=\color{green},
    stringstyle=\color{red},
    numbers=left,
    numberstyle=\tiny\color{gray},
    frame=single,
    breaklines=true,
    captionpos=b,
}


\title{MTH 447: Lecture \#18}
\author{Cliff Sun}

\newtheorem{theorem}{Theorem}[section]
\newtheorem{lemma}[theorem]{Lemma}
\newtheorem{definition}[theorem]{Definition}
\newtheorem{conjecture}[theorem]{Conjecture}
\newtheorem{proposition}[theorem]{Proposition}
\newtheorem{corollary}[theorem]{Corollary}
\newtheorem{one minute paper}[theorem]{One Minute Paper}

\pagestyle{fancy}
\lhead{\textbf{\thepage}\ \ \nouppercase{\rightmark}}
\chead{MTH 447: Lecture \#18}
\rhead{Cliff Sun}

\begin{document}

\maketitle

\section*{Functions}

\begin{definition}
    A \underline{function} is defined as 
    \begin{equation}
        \mathbb{R} \rightarrow \mathbb{R}
    \end{equation}
\end{definition}

\begin{definition}
    Let $F: A \rightarrow B$. Let $x_0 \in A$. We say that $f(x)$ is \underline{continuous at $x_0$} if for any sequence $(x_n)$, $x_n \in A$ and $\forall n$ and 
    \begin{equation}
        \lim_{n\rightarrow \infty}x_n = x_0 \implies \lim_{n\rightarrow \infty} f(x_n) = f(x_0)
    \end{equation}
    This definition can be modified to use the $\epsilon - \delta$ definition. 
\end{definition}

\begin{definition}
    If $F$ is continuous $\forall x_0 \in S \subseteq A$, then $F$ is \underline{continuous on S}. 
\end{definition}

\begin{theorem}
    $f: [a,b] \rightarrow \mathbb{R}$, $f$ is cont. Then 
    \begin{enumerate}
        \item $f$ is bounded 
        \item $f$ attains its maximum \& minimum. That is $\exists x_{max}, x_{min} \in [a,b]$ s.t. $\forall x\in [a,b]$
        \begin{equation}
            f_{x_{min}} \leq f(x) \leq f(x_{max})
        \end{equation}  
    \end{enumerate}
\end{theorem}

\begin{proof}
    (1). Let $f$ be continuous, $f: [a,b] \rightarrow \mathbb{R}$. And assume $\{f(x)\}$ is unbounded above. This means for any $n$, $\exists x$ such that $f(x) > n$. Then $\lim f(x_n) = \infty$. But $(x_n)$ is bounded, in particular, between $[a,b]$. Then $x_{n_k}$ is a convergent subsequence. Then $x_{n_k} \rightarrow x_0$. Then $f(x_{n_k}) \rightarrow \infty$. This is a contradiction. This concludes the proof.  
\end{proof}

\begin{proof}
    (2). Let $M = \sup(f(x))$. This exists and is finite. Then for all $n$, there exists $y_n$ such that 
    \begin{equation}
        M - \frac{1}{n} \leq f(y_n) \leq M
    \end{equation}
    Note, $y_{n_k} \rightarrow y_0$. 
\end{proof}

\end{document}
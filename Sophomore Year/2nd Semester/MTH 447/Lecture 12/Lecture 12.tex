\documentclass{article}
\usepackage[left=2cm, right=2cm, top=2cm, bottom=2cm]{geometry}
\usepackage{graphicx}
\usepackage{amsmath}
\usepackage{amssymb}
\usepackage{amsthm}
\usepackage{fancyhdr}
\usepackage{verbatim}
\usepackage{listings}
\usepackage{xcolor}
\usepackage{pgfplots}

\lstset{
    language=C++,
    basicstyle=\ttfamily\footnotesize,
    keywordstyle=\color{blue},
    commentstyle=\color{green},
    stringstyle=\color{red},
    numbers=left,
    numberstyle=\tiny\color{gray},
    frame=single,
    breaklines=true,
    captionpos=b,
}


\title{MTH 447: Lecture \# 12}
\author{Cliff Sun}

\newtheorem{theorem}{Theorem}[section]
\newtheorem{lemma}[theorem]{Lemma}
\newtheorem{definition}[theorem]{Definition}
\newtheorem{conjecture}[theorem]{Conjecture}
\newtheorem{proposition}[theorem]{Proposition}
\newtheorem{corollary}[theorem]{Corollary}
\newtheorem{one minute paper}[theorem]{One Minute Paper}

\pagestyle{fancy}
\lhead{\textbf{\thepage}\ \ \nouppercase{\rightmark}}
\chead{MTH 447: Lecture \# 12}
\rhead{Cliff Sun}

\begin{document}

\maketitle

\begin{definition}
    Given a sequence $(x_n)$, a sequence is called \underline{Cauchy} if 
    \begin{equation}
        \forall \epsilon >0, \exists N \text{ s.t. } n,m > N \implies |x_n - x_m| < \epsilon
    \end{equation}
    That is, $x_n$ get close to each other. 
\end{definition}

\begin{theorem}
    Convergent $\implies$ Cauchy 
\end{theorem}

\begin{proof}
    Let $\lim x_n = L$, then 
    \begin{center}
        $\forall \epsilon >0$, $\exists N$ such that $|x_n - L| < \epsilon$
    \end{center}
    Pick $n,m > N$, then 
    \begin{equation}
        |x_n - x_m| = |x_n - L + L - x_m| \leq |x_n - L| + |x_m - L| < \epsilon
    \end{equation}
\end{proof}

\begin{theorem}
    Cauchy $\implies$ bounded
\end{theorem}

\begin{proof}
    Pick $\epsilon = 1$, then
    \begin{center}
        $\exists N$ such that $n,m > N$, $|x_n - x_m| < 1$
    \end{center}
    Then 
    \begin{center}
        $\forall n > N \implies |x_n - x_{N+1}| < 1$
    \end{center}
    Then 
    \begin{equation}
        |x_n| \leq |x_{N+1}| + 1
    \end{equation}
    For all $n > N$. Then choose $M = \max(|x_1|, |x_2|, \dots)$, then 
    \begin{equation}
        |x_n| \leq M 
    \end{equation}
    For all $n$. 
\end{proof}

\begin{theorem}
    If $x_n$ is a sequence of real numbers, and $x_n$ is Cauchy. Then $x_n$ converges. 
\end{theorem}

\begin{proof}
    $\forall \epsilon >0$, $\exists \tilde{N}$ such that $n,m > \tilde{N}$, then $|x_n - x_m| < \epsilon$. Then choose $m = \tilde{N} + 1$, then 
    \begin{equation}
        |x_n - x_{\tilde{N} + 1}| < \epsilon \;\; \forall n > \tilde{N}
    \end{equation}
    Then 
    \begin{equation}
        x_n \leq x_{\tilde{N} + 1} + \epsilon
    \end{equation}
    and 
    \begin{equation}
        x_n \geq x_{\tilde{N} + 1} - \epsilon
    \end{equation}
    Then 
    \begin{equation}
        \lim \sup x_{n} = \lim_{\tilde{N} \rightarrow \infty} \sup_{k > \tilde{N}} \{x_k\}
    \end{equation}
    \begin{equation}
        = \lim_{\tilde{N} \rightarrow \infty} b_{\tilde{N}}
    \end{equation}
    Namely, 
    \begin{equation}
        b_{\tilde{N}} \leq x_{\tilde{N} + 1} + \epsilon
    \end{equation}
    Then 
    \begin{equation}
        \lim \sup x_{n} \leq b_{\tilde{N}} \leq x_{\tilde{N} + 1}
    \end{equation}
    Because $\lim \sup$ is a decreasing function. See lecture notes for completion of proof. 
\end{proof}

\section*{Subsequences}

Let $\left( x_n \right)$ be a sequence. Choose a sequence of natural numbers 
\begin{equation}
    1 \leq n_1, n_2, \dots
\end{equation}
Where $n_k < n_{k+1}$, then we define a new subsequence 
\begin{equation}
    y_n = x_{n_k}
\end{equation}

\begin{definition}
    Let $(x_n)$ be a sequence. Let $P(x)$ be a boolean that is true or false for all $x$. Then a subsequenc of $(x_n)$ satisfying $P$ is 
    \begin{equation}
        n_1 = \min_{k}\{P(x_{k}) \text{ true }\}
    \end{equation}
    \begin{equation}
        n_2 = \min_{k > n_1}\{P(x_k) \text{ true }\}
    \end{equation}
    \begin{equation}
        n_3 = \min_{k > n_2}\{P(x_k)\}
    \end{equation}
\end{definition}

\end{document}
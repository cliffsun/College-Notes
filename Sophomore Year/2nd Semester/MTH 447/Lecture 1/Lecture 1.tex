\documentclass{article}
\usepackage[left=2cm, right=2cm, top=2cm, bottom=2cm]{geometry}
\usepackage{graphicx}
\usepackage{amsmath}
\usepackage{amssymb}
\usepackage{amsthm}
\usepackage{fancyhdr}
\usepackage{verbatim}
\usepackage{listings}
\usepackage{xcolor}
\usepackage{pgfplots}

\lstset{
    language=C++,
    basicstyle=\ttfamily\footnotesize,
    keywordstyle=\color{blue},
    commentstyle=\color{green},
    stringstyle=\color{red},
    numbers=left,
    numberstyle=\tiny\color{gray},
    frame=single,
    breaklines=true,
    captionpos=b,
}


\title{MTH 447: Lecture 1}
\author{Cliff Sun}

\newtheorem{theorem}{Theorem}[section]
\newtheorem{lemma}[theorem]{Lemma}
\newtheorem{definition}[theorem]{Definition}
\newtheorem{conjecture}[theorem]{Conjecture}
\newtheorem{proposition}[theorem]{Proposition}
\newtheorem{corollary}[theorem]{Corollary}
\newtheorem{one minute paper}[theorem]{One Minute Paper}

\pagestyle{fancy}
\lhead{\textbf{\thepage}\ \ \nouppercase{\rightmark}}
\chead{MTH 447: Lecture 1}
\rhead{Cliff Sun}

\begin{document}

\maketitle

Natural number Axioms (Peano Axioms):
\begin{enumerate}
    \item $1 \in \mathbb{N}$
    \item If $n \in \mathbb{N}$, then $n+1 \in \mathbb{N}$
    \item $1$ is not the successor of any number in $\mathbb{N}$
    \item If $m$ and $n$ have the same successor, then $m = n$
    \item If $x \subseteq N$ such that 
    \item \begin{enumerate}
        \item $1 \in X$
        \item if $n \in X$ and $n+1\in X$
    \end{enumerate} 
    Then $X = \mathbb{N}$. This is the \underline{Principle of Induction}. 
\end{enumerate}

\begin{definition}
    Induction. Given some property $P(n)$, then if $P(1)$ is true and $P(n) \implies P(n+1)$ for all $n \in \mathbb{N}$. Then for all $k \in \mathbb{N}$, $P(k)$. 
\end{definition}



\end{document}
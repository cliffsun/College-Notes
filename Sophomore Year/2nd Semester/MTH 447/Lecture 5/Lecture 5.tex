\documentclass{article}
\usepackage[left=2cm, right=2cm, top=2cm, bottom=2cm]{geometry}
\usepackage{graphicx}
\usepackage{amsmath}
\usepackage{amssymb}
\usepackage{amsthm}
\usepackage{fancyhdr}
\usepackage{verbatim}
\usepackage{listings}
\usepackage{xcolor}
\usepackage{pgfplots}

\lstset{
    language=C++,
    basicstyle=\ttfamily\footnotesize,
    keywordstyle=\color{blue},
    commentstyle=\color{green},
    stringstyle=\color{red},
    numbers=left,
    numberstyle=\tiny\color{gray},
    frame=single,
    breaklines=true,
    captionpos=b,
}


\title{MTH 447: Lecture 5}
\author{Cliff Sun}

\newtheorem{theorem}{Theorem}[section]
\newtheorem{lemma}[theorem]{Lemma}
\newtheorem{definition}[theorem]{Definition}
\newtheorem{conjecture}[theorem]{Conjecture}
\newtheorem{proposition}[theorem]{Proposition}
\newtheorem{corollary}[theorem]{Corollary}
\newtheorem{one minute paper}[theorem]{One Minute Paper}

\pagestyle{fancy}
\lhead{\textbf{\thepage}\ \ \nouppercase{\rightmark}}
\chead{MTH 447: Lecture 5}
\rhead{Cliff Sun}

\begin{document}

\maketitle

\begin{definition}
    A sequences with entries in $S$ is a function 
    \begin{equation}
        f: \mathbb{N} \rightarrow \mathbb{S}
    \end{equation}
    With
    \begin{equation}
        (x_1, x_2, x_3, \cdot, x_n)
    \end{equation}
\end{definition} 

\begin{definition}
    A sequence $(x_n)$ converges to $L$ or has a limit $L$ if 
    \begin{equation}
        \forall \epsilon > 0, \exists N \in \mathbb{N}
    \end{equation}
    Such that if
    \begin{equation}
        n > N, |x_n - L| < \epsilon
    \end{equation}
\end{definition}

\begin{theorem}
    If $x_n \rightarrow L$ and $x_n \rightarrow M$, then 
    \begin{equation}
        L = M
    \end{equation}
    That is limits are unique. 
\end{theorem}

\begin{proof}
    Let $\epsilon = \frac{|L - M|}{2}$. Then assume $\epsilon > 0$ and $L \neq M$. Then $x_n \rightarrow L$, $\exists N_1 \in \mathbb{N}$ such that $n > N_1$
    \begin{equation}
        |x_n - L| < \epsilon
    \end{equation}
    Since $x_n \rightarrow M$, $\exists N_2$ such that $n > N_2$ $\implies |x_n - M| < \epsilon$. Let $N = \max(N_1, N_2)$, then 
    \begin{equation}
        n > N \implies |x_n - L| < \epsilon \land |x_n - M| < \epsilon
    \end{equation}
    Then 
    \begin{equation}
        |L-M|
    \end{equation}
    \begin{equation}
        \iff |L - x_n + x_n - M|
    \end{equation}
    \begin{equation}
        \iff |(L - x_n) + (x_n - M)|
    \end{equation}
    \begin{equation}
        \leq |x_n - L| + |x_n - M|
    \end{equation}
    \begin{equation}
        < 2\epsilon
    \end{equation}
    \begin{equation}
        \iff |L - M|
    \end{equation}
    This is a contradiction, this concludes the proof. 
\end{proof}

\begin{definition}
    Let $A$ be a finite set, 
    \begin{equation}
        A^{n} = \text{ all strings of length n from A }
    \end{equation}
    \begin{equation}
        \tilde{A}^{n} = \cup_{k=1}^{n}A^{k}
    \end{equation}
    \begin{equation}
        A^{\infty} = \text{ all finite strings from A }
    \end{equation}
    \begin{equation}
        \iff \cup_{k=1}^{\infty}A_{n}
    \end{equation}
    \begin{equation}
        A^{*} = \text{ all infinite strings from A }
    \end{equation}
\end{definition}

\begin{definition}
    We say a set $S$ is countable if 
    \begin{enumerate}
        \item it is finite
        \item $S$ is infinite and its cardinality is $\aleph_0$.
    \end{enumerate}
\end{definition}

\begin{theorem}
    \begin{enumerate}
        \item $A^{n}$ and $\tilde{A}^{n}$ are finite sets
        \item $A^{\infty}$ is countable 
        \item $A^{*}$ is not countable 
    \end{enumerate}
\end{theorem}
\begin{proof}
    Suppose we have a list $(x_1,x_2,\dots)$ that contains all elements of $A^{*}$. 
\end{proof}
\end{document}
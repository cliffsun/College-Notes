\documentclass{article}
\usepackage[left=2cm, right=2cm, top=2cm, bottom=2cm]{geometry}
\usepackage{graphicx}
\usepackage{amsmath}
\usepackage{amssymb}
\usepackage{amsthm}
\usepackage{fancyhdr}
\usepackage{verbatim}
\usepackage{listings}
\usepackage{xcolor}
\usepackage{pgfplots}

\lstset{
    language=C++,
    basicstyle=\ttfamily\footnotesize,
    keywordstyle=\color{blue},
    commentstyle=\color{green},
    stringstyle=\color{red},
    numbers=left,
    numberstyle=\tiny\color{gray},
    frame=single,
    breaklines=true,
    captionpos=b,
}


\title{MTH 447: Lecture \# 16}
\author{Cliff Sun}

\newtheorem{theorem}{Theorem}[section]
\newtheorem{lemma}[theorem]{Lemma}
\newtheorem{definition}[theorem]{Definition}
\newtheorem{conjecture}[theorem]{Conjecture}
\newtheorem{proposition}[theorem]{Proposition}
\newtheorem{corollary}[theorem]{Corollary}
\newtheorem{one minute paper}[theorem]{One Minute Paper}

\pagestyle{fancy}
\lhead{\textbf{\thepage}\ \ \nouppercase{\rightmark}}
\chead{MTH 447: Lecture \# 16}
\rhead{Cliff Sun}

\begin{document}

\maketitle

\section*{Ratio and Root Tests}

Suppose $a_n$ is a sequence. Define

\begin{equation}
    r_n = \bigg|\frac{a_{n+1}}{a_n}\bigg|
\end{equation}
And 
\begin{equation}
    \rho_n = |a_n|^{\frac{1}{n}}
\end{equation}

\begin{theorem}
    Consider $\sum a_n$
    \begin{enumerate}
        \item If $\lim \sup r_n < 1$, then series absolutely converges
        \item If $\lim \inf r_n > 1$, then series diverges 
        \item If $\lim \inf r_n \leq 1 \leq \lim \sup r_n$, then we can't say anything about the series
        \item If $\lim \sup \rho_n < 1$, then series absolutely converges 
        \item If $\lim \sup \rho_n > 1$, then series diverges. 
        \item If $\lim \sup \rho_n = 1$, then we can't anything about the series.  
    \end{enumerate}
\end{theorem}

\begin{proof}
    Proof of (4). Assume $\lim \sup \rho_n < 1$. Then $\lim \sup \rho_n = \alpha$. Thus, $\forall \epsilon > 0$, $\exists N$ such that $n > N \implies$
    \begin{equation}
        \alpha  - \epsilon \leq \sup_{k>n}\rho_k \leq \alpha + \epsilon
    \end{equation} 
    That is 
    \begin{equation}
        \rho_k \leq \alpha + \epsilon
    \end{equation}
    \begin{equation}
        \iff |a_k|^{\frac{1}{k}} \leq \alpha + \epsilon
    \end{equation}
    \begin{equation}
        |a_k| \leq (\alpha + \epsilon)^k
    \end{equation}
    That is 
    \begin{equation}
        \sum_{k=N+1}^{\infty}|a_k| \leq \sum_{k=N+1}^{\infty}(\alpha + \epsilon)^k
    \end{equation}
    This absolutely converges if $|\alpha + \epsilon| < 1$. 
\end{proof}

\begin{proof}
    Proof of (5). If $\lim \sup \rho_n = \alpha > 1$. Then there exists a subsequence that converges to $\alpha > 1$. There exists infinitely many $n$ such that 
    \begin{equation}
        |a_n| > 1
    \end{equation}
    Thus this shows that $\sum a_n$ diverges. 
\end{proof}

\begin{proof}
    Proof of (1), if $\lim \sup r_n < 1$ $\iff$ $\lim \sup \rho_n < 1 \implies$ convergence. 
\end{proof}

\begin{proof}
    Same argument for (2).
\end{proof}

\subsection*{Alternating Series Test}
\begin{theorem}
    Assume $a_n$ is a non-negative sequence and $a_n \rightarrow 0$. Consider
    \begin{equation}
        \sum_{n=1}^{\infty}(-1)^{n}a_n
    \end{equation}
    or 
    \begin{equation}
        \sum_{n=1}^{\infty}(-1)^{n+1}a_n
    \end{equation}  
    Both series converge.
\end{theorem}


\end{document}
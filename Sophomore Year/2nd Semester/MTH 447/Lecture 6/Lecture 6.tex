\documentclass{article}
\usepackage[left=2cm, right=2cm, top=2cm, bottom=2cm]{geometry}
\usepackage{graphicx}
\usepackage{amsmath}
\usepackage{amssymb}
\usepackage{amsthm}
\usepackage{fancyhdr}
\usepackage{verbatim}
\usepackage{listings}
\usepackage{xcolor}
\usepackage{pgfplots}

\lstset{
    language=C++,
    basicstyle=\ttfamily\footnotesize,
    keywordstyle=\color{blue},
    commentstyle=\color{green},
    stringstyle=\color{red},
    numbers=left,
    numberstyle=\tiny\color{gray},
    frame=single,
    breaklines=true,
    captionpos=b,
}


\title{MTH 447: Lecture 6}
\author{Cliff Sun}

\newtheorem{theorem}{Theorem}[section]
\newtheorem{lemma}[theorem]{Lemma}
\newtheorem{definition}[theorem]{Definition}
\newtheorem{conjecture}[theorem]{Conjecture}
\newtheorem{proposition}[theorem]{Proposition}
\newtheorem{corollary}[theorem]{Corollary}
\newtheorem{one minute paper}[theorem]{One Minute Paper}

\pagestyle{fancy}
\lhead{\textbf{\thepage}\ \ \nouppercase{\rightmark}}
\chead{MTH 447: Lecture 6}
\rhead{Cliff Sun}

\begin{document}

\maketitle

Previously, we defined a sequence $(x_n)$ and the epsilon-delta definition of a limit. 

\subsection*{Example}
\begin{proof}
    Show 
    \begin{equation}
        \frac{1}{n^p} \rightarrow 0
    \end{equation}
    Then 
    \begin{equation}
        |\frac{1}{n^p} - 0| < \epsilon
    \end{equation}
    \begin{equation}
        \implies \frac{1}{n^p} < \epsilon
    \end{equation}
    \begin{equation}
        n > \frac{1}{\epsilon^{\frac{1}{p}}}
    \end{equation}
    Then let 
    \begin{equation}
        N = \lceil\epsilon^{-\frac{1}{p}}\rceil
    \end{equation}
\end{proof}

\subsection*{Example}
\begin{proof}
    Prove limit of
    \begin{equation}
        x_n = \frac{3n + 5}{7n - 1}
    \end{equation}
    Guess $L = \frac{3}{7}$. Note that 
    \begin{equation}
        x_n = \frac{3 + \frac{5}{n}}{7 - \frac{1}{n}}
    \end{equation}
    We can use the previous theorem to note that 
    \begin{equation}
        \lim \frac{5}{n} = \lim \frac{1}{n} = 0
    \end{equation}
    But we use the limit definition, namely 
    \begin{equation}
        \bigg| \frac{3n + 5}{7n - 1} - \frac{3}{7} \bigg| < \epsilon
    \end{equation}
    Then 
    \begin{equation}
        \bigg|\frac{21n + 35 - 21n + 3}{49n - 7}\bigg|
    \end{equation}
    \begin{equation}
        \frac{38}{49n - 7} < \epsilon
    \end{equation}
    \begin{equation}
        \frac{1}{49}\left(\frac{38}{\epsilon} + 7\right) < n
    \end{equation}
    Then 
    \begin{equation}
        N = \lceil \frac{1}{49}\left(\frac{38}{\epsilon} + 7\right) \rceil
    \end{equation}
\end{proof}

\subsection*{Example}

\begin{proof}
    Show limit of 
    \begin{equation}
        x_n = \frac{\sin(3n)}{n}
    \end{equation}
    Claim $L = 0$, then 
    \begin{equation}
        |\frac{\sin(3n)}{n} - 0| < \epsilon
    \end{equation}
    Then 
    \begin{equation}
        \frac{|sin(3n)|}{n} < \epsilon
    \end{equation}
    Then 
    \begin{equation}
        \frac{|sin(3n)|}{n} \leq \frac{1}{n}  
    \end{equation}
    Then 
    \begin{equation}
        \frac{1}{n} < \epsilon \iff n > \frac{1}{\epsilon}
    \end{equation}
    IF $n > \frac{1}{\epsilon}$, then 
    \begin{equation}
        \frac{1}{n} < \epsilon
    \end{equation}
    Then 
    \begin{equation}
        \frac{|\sin(3n)|}{n} < \epsilon
    \end{equation}
    \begin{equation}
        = |x_n - 0| < \epsilon
    \end{equation}
\end{proof}

\subsection*{Example}
\begin{proof}
    \begin{equation}
        x_n = \frac{2n + 4}{3n^4 - 2n + 7}
    \end{equation}
    \begin{equation}
        \bigg|\frac{2n + 4}{3n^4 - 2n + 7}\bigg| < \epsilon
    \end{equation}
    We replace 
    \begin{equation}
        2n + 4 \leq 3n
    \end{equation}
    \begin{equation}
        \iff 4 \leq n
    \end{equation}
    and 
    \begin{equation}
        3n^4 - 2n + 7 \geq 2n^4
    \end{equation}
    \begin{equation}
        n^4 \geq 2n - 4
    \end{equation}
    \begin{equation}
        n^4 \geq 2n
    \end{equation}
    \begin{equation}
        n^3 \geq 2
    \end{equation}
    \begin{equation}
        n \geq 2
    \end{equation}
    Then if $n > \max(2,4)$, then 
    \begin{equation}
        \frac{2n + 4}{3n^4 - 2n + 7} \leq \frac{3n}{2n^4} \leq \frac{3}{2}n^{-3}
    \end{equation}
    Then 
    \begin{equation}
        \frac{3}{2}n^{-3} < \epsilon
    \end{equation}
    Then 
    \begin{equation}
        n > \left(\frac{3}{2\epsilon}\right)^{\frac{1}{3}}
    \end{equation}
    Then 
    \begin{equation}
        N = \max(\lceil \left(\frac{3}{2\epsilon}\right)^{\frac{1}{3}}\rceil, 4)
    \end{equation}
\end{proof}

\subsection*{Example}
\begin{proof}
    \begin{equation}
        x_n = n^2
    \end{equation}
    We prove that $x_n$ has no limit. Assume it did, then 
    \begin{equation}
        |n^2 - L| < \epsilon
    \end{equation}
    Pick $\epsilon = 1$, then 
    \begin{equation}
        |n^2 - L < 1
    \end{equation}
    Then 
    \begin{equation}
        n^2 < L + 1
    \end{equation}
    Then 
    \begin{equation}
        n < \sqrt{L + 1}
    \end{equation}
    Pick a $N$ larger than that, then $n^2$ doesn't converge to this. Thus $n^2$ has no limit. 
\end{proof}

\end{document}
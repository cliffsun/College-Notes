\documentclass{article}
\usepackage[left=2cm, right=2cm, top=2cm, bottom=2cm]{geometry}
\usepackage{graphicx}
\usepackage{amsmath}
\usepackage{amssymb}
\usepackage{amsthm}
\usepackage{fancyhdr}
\usepackage{verbatim}
\usepackage{listings}
\usepackage{xcolor}
\usepackage{pgfplots}

\lstset{
    language=C++,
    basicstyle=\ttfamily\footnotesize,
    keywordstyle=\color{blue},
    commentstyle=\color{green},
    stringstyle=\color{red},
    numbers=left,
    numberstyle=\tiny\color{gray},
    frame=single,
    breaklines=true,
    captionpos=b,
}


\title{MTH 447: Lecture 17}
\author{Cliff Sun}

\newtheorem{theorem}{Theorem}[section]
\newtheorem{lemma}[theorem]{Lemma}
\newtheorem{definition}[theorem]{Definition}
\newtheorem{conjecture}[theorem]{Conjecture}
\newtheorem{proposition}[theorem]{Proposition}
\newtheorem{corollary}[theorem]{Corollary}
\newtheorem{one minute paper}[theorem]{One Minute Paper}

\pagestyle{fancy}
\lhead{\textbf{\thepage}\ \ \nouppercase{\rightmark}}
\chead{MTH 447: Lecture 17}
\rhead{Cliff Sun}

\begin{document}

\maketitle

\begin{definition}
    A repeating decimal is a sequence such that $\exists n,p$ with 
    \begin{equation}
        d_{n + kp} = d_n \;\; \forall k \in \mathbb{N}
    \end{equation}
\end{definition}

Consider two finite sequences $d = (d_1,d_2,\dots, d_n)$ and $e = (e_1,e_2,\dots,e_p)$, then 
\begin{equation}
    0.d_1d_2\dots d_ne_1\dots e_pe_1\dots e_p\dots
\end{equation}

\begin{theorem}
    Consider the fraction $a/b = k$, and $a,b \in \mathbb{Z}$. Assume we have that two non negative integers $n_1,n_2$ such that 
    \begin{equation}
        a10^{n_1}(\mod b) = a10^{n_2} (\mod b)
    \end{equation}
    Then the decimal sequence starting at $n_1$ equals the decimal sequence starting at $n_2$. 
\end{theorem}

\end{document}
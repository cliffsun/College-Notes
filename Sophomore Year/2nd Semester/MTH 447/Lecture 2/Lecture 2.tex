\documentclass{article}
\usepackage[left=2cm, right=2cm, top=2cm, bottom=2cm]{geometry}
\usepackage{graphicx}
\usepackage{amsmath}
\usepackage{amssymb}
\usepackage{amsthm}
\usepackage{fancyhdr}
\usepackage{verbatim}
\usepackage{listings}
\usepackage{xcolor}
\usepackage{pgfplots}

\lstset{
    language=C++,
    basicstyle=\ttfamily\footnotesize,
    keywordstyle=\color{blue},
    commentstyle=\color{green},
    stringstyle=\color{red},
    numbers=left,
    numberstyle=\tiny\color{gray},
    frame=single,
    breaklines=true,
    captionpos=b,
}


\title{MTH 447: Lecture 2}
\author{Cliff Sun}

\newtheorem{theorem}{Theorem}[section]
\newtheorem{lemma}[theorem]{Lemma}
\newtheorem{definition}[theorem]{Definition}
\newtheorem{conjecture}[theorem]{Conjecture}
\newtheorem{proposition}[theorem]{Proposition}
\newtheorem{corollary}[theorem]{Corollary}
\newtheorem{one minute paper}[theorem]{One Minute Paper}

\pagestyle{fancy}
\lhead{\textbf{\thepage}\ \ \nouppercase{\rightmark}}
\chead{MTH 447: Lecture 2}
\rhead{Cliff Sun}

\begin{document}

\maketitle

The Algebra of $\mathbb{Z}$:
\begin{enumerate}
    \item $(x+y) + z = x + (y+z)$
    \item $x + y = y + x$
    \item $x + 0 = x$ for some $0 \in \mathbb{Z}$
    \item $x + (-x) = 0$ for some $-x \in \mathbb{Z}$
\end{enumerate}

This is an abelian group. We then introduce the multiplication rules

\begin{enumerate}
    \item $(xy)z = x(yz)$
    \item $xy = yx$
    \item $x \cdot 1 = 1 \cdot x = x$
    \item $x(y+z) = xy + xz$
\end{enumerate}

These proporties form a communative ring, but NOT a field because there is no multiplicative inverse. Recall, equivalence relations are
\begin{enumerate}
    \item Reflexive
    \item Symmetric
    \item Transitive
\end{enumerate}

The statement $ad = bc$ is an equivalence relation. Where $d, c \neq 0$. In the rational numbers $\mathbb{Q}$, we have a new property that 
\begin{enumerate}
    \item $x x^{-1} =1$ for some $x^{-1} \in \mathbb{Q}$
\end{enumerate}

Now our set of numbers is a field. Note that 
\begin{equation}
    \mathbb{Q} = \left[\mathbb{Z} \times (\mathbb{Z} \ 0)\right] / \mathbb{\sim}
\end{equation}

Where $\sim$ is the equivalence relation as above. This creates a coordinate system with $(p,q)$ where $p$ is the top part of the fraction and etc. But $\mathbb{Q}$ has a lot of "holes", that 
we can't use analysis on it. Example: we claim that there is no rational solution for $c^2 = 2$. 

\begin{proof}
    We assume that $x = \frac{p}{q}$ and $x^2 = 2$. Then 
    \begin{equation}
        \frac{p^2}{q^2} = 2
    \end{equation}
    \begin{equation}
        p^2 = 2q^2
    \end{equation}
    Thus the right hand side is even. Therefore, $p^2$ is even. Now that $p$ is even. Thus $p = 2k$. Then 
    \begin{equation}
        4k^2 = 2q^2
    \end{equation}
    \begin{equation}
        q^2 = 2k^2
    \end{equation}
    Therefore, $q^2$ is even, which implies that $q$ is also even. Therefore, there is no rational solution to $x^2 = 2$. 
\end{proof}

\begin{theorem}
    Let $x_1 < x_2$ be in $\mathbb{Q}$, then there are infinitely many irrational numbers between $x_1$ and $x_2$. 
\end{theorem}

\begin{lemma}
    Let $x_1 < x_2$ be in $\mathbb{Q}$. Then 
    \begin{equation}
        x_1 + \frac{x_2 - x_1}{n} \in \mathbb{Q}
    \end{equation}
    Is rational. Thus, there are infinitely many rational numbers between $x_1$ and $x_2$. 
\end{lemma}
\begin{lemma}
    Given $x_1, x_2 \in \mathbb{Q}$, and $x_1 < x_2$. Then
    \begin{equation}
        x_1 + \frac{x_2 - x_1}{\sqrt{2}}
    \end{equation}
    is not rational. 
\end{lemma}

\begin{proof}
    Assume that 
    \begin{equation}
        x_1 + \frac{x_2 - x_1}{\sqrt{2}} = \frac{p}{q}
    \end{equation}
    Then we simplfy down to 
    \begin{equation}
        \sqrt{2} = \frac{x_2 - x_1}{\frac{p}{q} - x_1} \in \mathbb{Q}
    \end{equation}
    and $\frac{p}{q} \neq x_1$. This is a contradiction. 
\end{proof}
Then between 2 rational numbers, there is always at least on irrational number. 
\end{document}
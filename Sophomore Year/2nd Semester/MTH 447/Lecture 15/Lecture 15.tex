\documentclass{article}
\usepackage[left=2cm, right=2cm, top=2cm, bottom=2cm]{geometry}
\usepackage{graphicx}
\usepackage{amsmath}
\usepackage{amssymb}
\usepackage{amsthm}
\usepackage{fancyhdr}
\usepackage{verbatim}
\usepackage{listings}
\usepackage{xcolor}
\usepackage{pgfplots}

\lstset{
    language=C++,
    basicstyle=\ttfamily\footnotesize,
    keywordstyle=\color{blue},
    commentstyle=\color{green},
    stringstyle=\color{red},
    numbers=left,
    numberstyle=\tiny\color{gray},
    frame=single,
    breaklines=true,
    captionpos=b,
}


\title{MTH 447: Lecture \# 15}
\author{Cliff Sun}

\newtheorem{theorem}{Theorem}[section]
\newtheorem{lemma}[theorem]{Lemma}
\newtheorem{definition}[theorem]{Definition}
\newtheorem{conjecture}[theorem]{Conjecture}
\newtheorem{proposition}[theorem]{Proposition}
\newtheorem{corollary}[theorem]{Corollary}
\newtheorem{one minute paper}[theorem]{One Minute Paper}

\pagestyle{fancy}
\lhead{\textbf{\thepage}\ \ \nouppercase{\rightmark}}
\chead{MTH 447: Lecture \# 15}
\rhead{Cliff Sun}

\begin{document}

\maketitle

\section*{Series}

Given a sequence $(a_n)$, then 

\begin{equation}
    s_n = \sum_{k=1}^{n}a_{k}
\end{equation}

Then 

\begin{equation}
    \lim_{n\rightarrow \infty}s_n = \;?
\end{equation}

If $s_n$ is increasing 

\begin{equation}
    s_{n+1} - s_n = a_{n+1} \geq 0
\end{equation}

If this sum is bounded, then $s_n \rightarrow S$. If unbounded, then $\rightarrow \infty$. 

\begin{equation}
    \sum_{n=0}^{\infty}ar^n = \frac{a}{1 - r}
\end{equation}

\begin{definition}
    We say that the sum $\sum_{n=1}^{\infty}a_n$ satisfies the Cauchy criterion if $\forall \epsilon > 0$, there exists $N$ s.t. if $n,m > N$, then 
    \begin{equation}
        \bigg| \sum_{k=m+1}^{n}a_k \bigg| < \epsilon
    \end{equation}
\end{definition}

\begin{theorem}
    $\sum a_n$ convergs $\iff$ it satisfies the Cauchy Criterion. 
\end{theorem}

\begin{proof}
    \begin{equation}
        S_n = \sum_{k=1}^{n}a_n
    \end{equation}
    Then 
    \begin{equation}
        S_n - S_m = \bigg|\sum_{k=m+1}^{n}a_k\bigg| < \epsilon \text{ (because this sum converges)}
    \end{equation}
    This is saying that a sequence of partials sums is cauchy. 
\end{proof}

\begin{corollary}
    If $\sum a_n$ converges, then $a_n \rightarrow 0$
\end{corollary}

\begin{proof}
    This follows from the Cauchy Criterion. Choose $n = m+1$, then 
    \begin{equation}
        \big| a_{m+1} \big| < \epsilon
    \end{equation}
\end{proof}

What about the converse? This is false. 

\begin{theorem}
    If $a_n \geq 0$ and $\sum_{n=1}^{\infty}a_n$ converges, and if $|b_n| \leq a_n$, then $\sum b_n$ converges. 
\end{theorem}

\begin{theorem}
    If $a_n \leq b_n$ and $\sum a_n = \infty$, then $\sum b_n = \infty$. 
\end{theorem}

\begin{proof}
    \begin{equation}
        \bigg|\sum_{k= m+1}^{n}b_k\bigg| \leq \sum_{k=m+1}^{n}|b_n| \leq \sum_{k=m+1}^{n}a_k
    \end{equation}
    If $a_n$ satifies the Cauchy Criterion, then so does $b_n$. Thus, $b_n$ converges. 
\end{proof}

\begin{definition}
    We say that a series $\sum a_k$ is \underline{absolutely convergent} if 
    \begin{equation}
        \sum_{k=1}^{\infty}|a_n| \text{ is convergent}
    \end{equation}
\end{definition}

\begin{corollary}
    If $\sum a_k$ is absolutely convergent, then this sum converges. 
\end{corollary}

\end{document}
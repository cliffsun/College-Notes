\documentclass{article}
\usepackage[left=3cm, right=3cm, top=2cm, bottom=2.5cm]{geometry}
\usepackage{graphicx}
\usepackage{amsmath}
\usepackage{amssymb}
\usepackage{amsthm}
\usepackage{fancyhdr}

\title{Quantum Physics Lecture 1}
\author{Cliff Sun}

\newtheorem{theorem}{Theorem}[section]
\newtheorem{lemma}[theorem]{Lemma}
\newtheorem{definition}[theorem]{Definition}
\newtheorem{conjecture}[theorem]{Conjecture}
\newtheorem{proposition}[theorem]{Proposition}
\newtheorem{corollary}[theorem]{Corollary}
\newtheorem{one minute paper}[theorem]{One Minute Paper}

\pagestyle{fancy}
\lhead{\textbf{\thepage}\ \ \nouppercase{\rightmark}}
\chead{Quantum Physics Lecture 1}
\rhead{Cliff Sun}

\begin{document}

\maketitle

\begin{center}
    \begin{tabular}{|c|c|c|}
        \hline
        Classical Mechanics & Quantum Mechanics \\ \hline
        $x(t)$ & $\Psi(x,t)$ \\ \hline
        Definite position as a function of time & Probabilistic description of the particle as a function of space and time \\ \hline
    \end{tabular} \\
\end{center}

\begin{center}
    Let's dissect this Wavefunction idea further. \\   
\end{center}

\begin{center}
    \begin{tabular}{|c|c|c|}
        \hline
        Classical Mechanics & Waves \\ \hline
        $x(t)$ & $y(x,t)$ \\ \hline
        Definite position as a function of time & Used to describe B Field, E Field, etc. \\ \hline
    \end{tabular}
\end{center}

Classical Wave Equation:

\begin{equation}
    \frac{\partial^2 y}{\partial x^2} = \frac{1}{v^2}\frac{\partial^2 y}{\partial t^2}
\end{equation}

Solution:

\begin{equation}
    y(x,t) = A\cos(\frac{2\pi x}{\lambda} - \frac{2\pi t}{T})
\end{equation}

Such that 

\begin{equation}
    \frac{\lambda}{T} = v
\end{equation}



\end{document}
\documentclass{article}
\usepackage[left=2cm, right=2cm, top=1cm, bottom=1cm]{geometry}
\usepackage{graphicx}
\usepackage{amsmath}
\usepackage{amssymb}
\usepackage{amsthm}
\usepackage{fancyhdr}
\usepackage{verbatim}
\usepackage{listings}
\usepackage{xcolor}

\lstset{
    language=C++,
    basicstyle=\ttfamily\footnotesize,
    keywordstyle=\color{blue},
    commentstyle=\color{green},
    stringstyle=\color{red},
    numbers=left,
    numberstyle=\tiny\color{gray},
    frame=single,
    breaklines=true,
    captionpos=b,
}


\title{CS 225: Lecture 3}
\author{Cliff Sun}

\newtheorem{theorem}{Theorem}[section]
\newtheorem{lemma}[theorem]{Lemma}
\newtheorem{definition}[theorem]{Definition}
\newtheorem{conjecture}[theorem]{Conjecture}
\newtheorem{proposition}[theorem]{Proposition}
\newtheorem{corollary}[theorem]{Corollary}
\newtheorem{one minute paper}[theorem]{One Minute Paper}

\pagestyle{fancy}
\lhead{\textbf{\thepage}\ \ \nouppercase{\rightmark}}
\chead{CS 225: Lecture 3}
\rhead{Cliff Sun}

\begin{document}

\maketitle

\section*{
    Lecture Span
}

\begin{itemize}
    \item Templates
\end{itemize}

\section*{Templates}
A way to write generic code, whose type is determined at compliation. Compiler will not use the recipe until you explicitly use it. 

\subsection*{Example template in C++}
\begin{verbatim}
    T sum (T a, T b) {
        
    }
\end{verbatim}

\section*{List ADT}

A list is an ordered collection of items. It can either be homoegenous or hetergenous, and can be of fixed size or resizeable.

A minimal set of operations can be used to achieve all tasks
\begin{enumerate}
    \item Insert
    \item Delete
    \item isEmpty
    \item getData
    \item Create an empty list
\end{enumerate}

\subsection*{List Implementations}

\begin{enumerate}
    \item Linked List (series of objects linked together using pointers )
    \item Array List
\end{enumerate}


\begin{lstlisting}[caption=Example Linked List Single Object]
    class ListNode {
        T& data; \\ houses the current data
        next; \\ points to the next object
    }
\end{lstlisting}

Some things to note:
\begin{enumerate}
    \item data is stored as a reference because we don't care about its memory management
    \item next is a pointer because a reference cannot point to null
\end{enumerate}

\end{document}
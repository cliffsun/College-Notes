\documentclass{article}
\usepackage[left=3cm, right=3cm, top=2cm, bottom=2.5cm]{geometry}
\usepackage{graphicx}
\usepackage{amsmath}
\usepackage{amssymb}
\usepackage{amsthm}
\usepackage{fancyhdr}

\title{MTH 416: Introduction \& Vector Spaces}
\author{Cliff Sun}

\newtheorem{theorem}{Theorem}[section]
\newtheorem{lemma}[theorem]{Lemma}
\newtheorem{definition}[theorem]{Definition}
\newtheorem{conjecture}[theorem]{Conjecture}
\newtheorem{proposition}[theorem]{Proposition}
\newtheorem{corollary}[theorem]{Corollary}
\newtheorem{one minute paper}[theorem]{One Minute Paper}

\pagestyle{fancy}
\lhead{\textbf{\thepage}\ \ \nouppercase{\rightmark}}
\chead{MTH 416: Introduction \& Vector Spaces}
\rhead{Cliff Sun}

\begin{document}

\maketitle

\section*{Lecture Span}
\begin{itemize}
    \item Introduction To Abstract Linear Algebra
    \item Vector Spaces
\end{itemize}

Textbook: Linear Algebra by Friedberg, Insel, \& Spence \\

Free Online Textbook: A First Course in Linear Algebra by Robert Beezer \\

\section*{Introduction to Linear Algebra}

\subsection*{Main Objects of Study}
\begin{itemize}
    \item Vector Spaces
    \item Linear Transformation
    \item Systems of Linear Equations
    \item Matrices \& Determinants
    \item Eigenvalues \& Eigenvectors
    \item Inner Product Spaces
\end{itemize}

Given a vector $L$ such that

\begin{equation}
    L = \{tv : t \in \mathbb{R}\}
\end{equation}

Where 

\begin{equation}
    L = (t, t, t)
\end{equation}

And given a vector 

\begin{equation}
    w = (1,0,0)
\end{equation}

We have that 

\begin{equation}
    L + w = (2,1,1) \textbf{ such that } t = 1 \iff P
\end{equation}

thus
\begin{equation}
    P = \{t + u, t, t\} \quad t,u \in \mathbb{R}
\end{equation}

Remark: $P$ is a plane. 

\newpage

\begin{definition}
    A \underline{vector space} $V$ is any set which behaves algebraically like $\mathbb{R}^n$. In particular, 
    \begin{enumerate}
        \item Given any $v, w \in V$, we can add them such that $v + w \in V$ for some vector space $V$
        \item Given any $v \in V$ and $c \in \mathbb{R}$, we have that $cV \in \mathbb{V}$
        \item These operations satisfy all usual rules of arithmetic
    \end{enumerate}
\end{definition}

\subsubsection*{Remarks}
    \begin{enumerate}
        \item No cross product: this is not present in a \underline{general} vector space
        \item No dot product: this is a unique operation that goes beyond basic arithmetic \& geometric intuition.
        \item We don't know what vectors are
    \end{enumerate}

\subsection*{Why is this so abstract?}

Short answer: Abstraction allows us to generalize vectors into a more applicable theory, (Physics - Quantum Mechanics, Computer Science - ML, etc.) 

\begin{itemize}
    \item "Vectors" can be functions, equivalence classes, etc.
\end{itemize}

As long as we can prove that these mathematical objects are vector spaces, then we can apply Abstract Linear Algebra onto it. 

\begin{definition}
    \underline{Linear Transformations} is a function that transforms a vector space into another vector space. Preserves vector addition \& scalar multiplication geometrically. 
\end{definition}

Example of L.T:

\begin{equation}
    T(x,y) = (2x, 3y)
\end{equation}

Recall: In Differential Calculus, we approximate functions of 1 variable with affine linear functions (Form: $ax +b$, namely add a constant) using derivatives. In multivariable Calculus, we approximate functions of multiple
variables with linear transformations (Linear functions of $x,y,z\dots$) + constant vector. 

\begin{definition}
    A \underline{vector space over $\mathbb{R}^n$} is a set equipped with 2 operations:
    \begin{enumerate}
        \item Addition: given 2 elements $v,w \in \mathbb{R}^n$, then $v+w$ produces a unique vector
        \item Scalar multiplication: given a vector $v$ and a scalar $c \in \mathbb{R}$, $cv \in \mathbb{R}^n$ produces some unique vector. 
    \end{enumerate}
    They all satisfy the following 8 properties:
    \begin{enumerate}
        \item $\forall x,y \in V, \quad x + y \iff y + x$
        \item $\forall x,y,z \in V, \quad (x+y) + z \iff x + (y+z)$
        \item $\exists \; 0 \in V$ such that for all vectors $x \in V$, we have that $0 + x = x$
        \item $\forall x \in V, \exists y \in V$ such that $x + y = 0$. In other words, $y = -x$
        \item For each $x \in V$, $1\cdot x = x$
        \item For each $a,b \in \mathbb{R}$ and $x \in V$, $(ab)x = a(bx)$
        \item For each $a \in \mathbb{R}$ and $x,y \in V$, $a(x+y) = ax + ay$
        \item For each $a,b \in \mathbb{R}$ and $x \in V$, we have that $(a + b)x = ax + bx$
    \end{enumerate}
    \subsubsection*{Terminology}
    Elements of $V$ are called "vectors", and elements of $\mathbb{R}$ are called "scalars". 
\end{definition}

\begin{conjecture}
    For any $x \in V$, 
    \begin{equation}
        0 \cdot \vec{x} = \vec{0}
    \end{equation}
\end{conjecture}

Example: $\mathbb{R}^n$ for any $n \geq 0$
\begin{enumerate}
    \item As a set, $\mathbb{R}^n = \{(x_1,x_2,x_3,\dots)\}: x_i \in \mathbb{R}$
    \item Given $x = (x_1,x_2,\dots)$ and $y = (y_1,y_2,\dots)$, we define $x + y = (x_1 + y_1, \dots)$ 
    \item Given $x \in \mathbb{R}^n$, and $c \in \mathbb{R}$, we define $cx = (cx_1, cx_2, \dots)$
\end{enumerate}

\begin{proof}
    Property 3: the 0 vector is $(0, 0, 0, \dots)$. Thus we have that $v + 0 = v$ for all $v \in V$
\end{proof}

\begin{proof}
    Property 4: Given $x \in \mathbb{R}^n$, we choose $y = -x \iff y = (-x_1, -x_2, \dots)$ which means that $x + y = 0 \iff (0,0,0,0, \dots)$
\end{proof}

\begin{proof}
    Propety 6: Given $a,b \in \mathbb{R}$ and $x \in \mathbb{R}^n$, we first calculate the left side.
    \begin{equation}
        (ab)x \iff ((ab)x_1, (ab)x_2, \dots)
    \end{equation}
    We then manipulate equation (8) using the associate rule of scalar multiplication:
    \begin{equation}
        ((ab)x_1, (ab)x_2, \dots) \iff (a(bx_1), a(bx_2), \dots) \iff a(bx)
    \end{equation}
\end{proof}

Example 2: $m \times n matrices$
\begin{enumerate}
    \item As a set, $M_{m\times n}(\mathbb{R})$ represents a matrix of values with $m$ rows and $n$ columns. 
    \item Addition \& scalar multiplication are defined entry wise. 
\end{enumerate}

Example $m \times n$ matrix:

\begin{equation}
    \begin{pmatrix}
        a_{11} & \dots & a_{1n} \\
        \dots & \dots & \dots \\
        a_{m1} & \dots & a_{mn} \\
    \end{pmatrix}
\end{equation}

\subsection*{Notation}

Given a matrix $A \in M_{m\times n}(\mathbb{R})$, we let $A_{ij} = $ the entry in i-th row, and j-th column where $0 \leq i \leq m$ and $0 \leq j \leq n$. 

Claim: $M_{m\times n}(\mathbb{R})$ is a vector space on $\mathbb{R}$.

\begin{proof}
   $ 0_{m \times n}$ = 
   \[
    \begin{pmatrix}
        0 & \dots & 0 \\
        \dots & \dots & \dots \\
        0 & \dots & 0 \\
    \end{pmatrix}
    \]
    We calculate that $A + 0_{m,n} = A$ for all $A \in M$
\end{proof}

Example 3: Real-valued functions on a set
\begin{enumerate}
    \item Fix a set $S$, and define $F(S,\mathbb{R}) = \{ \text{Functions } f: S \rightarrow \mathbb{R}\}$
    \item Given $f,g \in F(S, \mathbb{R})$, we define $(f + g)(s) = f(s) + g(s)$
    \item Given $f \in F(S, \mathbb{R})$, and $c \in \mathbb{R}$, $(cf)(s) = c \dot f(s)$ for all $s \in S$ 
\end{enumerate}

Claim: $F(S,\mathbb{R})$ is a vector space over $\mathbb{R}$, given any set $S$

\begin{proof}
    Axiom 3: Let $f_0$ be the function $f: S \rightarrow 0$
\end{proof}

Claim: For all $g \in F(S, \mathbb{R})$, we have that 
\begin{proof}
    \begin{equation}
        g + f_0 = g
    \end{equation}
    
    For any $s \in S$
    \begin{equation}
        (g+f_0)(s) = g(s) + f_0(s) \iff g(s) + 0 \iff g(s)
    \end{equation}
\end{proof} 
\end{document}
\documentclass{article}
\usepackage[left=2cm, right=2cm, top=2cm, bottom=2cm]{geometry}
\usepackage{graphicx}
\usepackage{amsmath}
\usepackage{amssymb}
\usepackage{amsthm}
\usepackage{fancyhdr}
\usepackage{verbatim}
\usepackage{listings}
\usepackage{xcolor}
\usepackage{pgfplots}

\lstset{
    language=C++,
    basicstyle=\ttfamily\footnotesize,
    keywordstyle=\color{blue},
    commentstyle=\color{green},
    stringstyle=\color{red},
    numbers=left,
    numberstyle=\tiny\color{gray},
    frame=single,
    breaklines=true,
    captionpos=b,
}


\title{MTH 416: Lecture 23}
\author{Cliff Sun}

\newtheorem{theorem}{Theorem}[section]
\newtheorem{lemma}[theorem]{Lemma}
\newtheorem{definition}[theorem]{Definition}
\newtheorem{conjecture}[theorem]{Conjecture}
\newtheorem{proposition}[theorem]{Proposition}
\newtheorem{corollary}[theorem]{Corollary}
\newtheorem{one minute paper}[theorem]{One Minute Paper}

\pagestyle{fancy}
\lhead{\textbf{\thepage}\ \ \nouppercase{\rightmark}}
\chead{MTH 416: Lecture 23}
\rhead{Cliff Sun}

\begin{document}

\maketitle

\section*{Lecture Span}
\begin{itemize}
    \item Inner product spaces
    \item Orthonormal bases \& Gram-Schmidt
\end{itemize}

Recall: Inner product axioms
\begin{enumerate}
    \item $\langle x + z, y \rangle = \langle x,y \rangle + \langle z,y \rangle$
    \item $\langle cx,y \rangle = c\langle x,y \rangle$
    \item $\overline{\langle x,y \rangle} = \langle y,x \rangle$
    \item If $x \neq 0$, then $\langle x,x \rangle > 0$ and $||x|| = \sqrt{\langle x,x \rangle}$ 
\end{enumerate}

\begin{theorem}
    Let $V$ be an inner product space, for all $x,y \in V$, we have that 
    \begin{enumerate}
        \item (Cauchy-Schwarz) $|\langle x,y \rangle| \leq ||x||||y||$
        \item (Triangle Inequality) $||x + y|| \leq ||x|| + ||y||$
    \end{enumerate}
\end{theorem}

\begin{proof}
    We can prove $2$ assuming $1$. We first square both sides:
    \begin{equation}
        ||x + y||^2 = \langle x + y, x + y \rangle 
    \end{equation}
    \begin{equation}
        = \langle x, x+y \rangle + \langle y, x + y \rangle
    \end{equation}
    \begin{equation}
        = \langle x, x \rangle + \langle x, y \rangle + \langle y,x \rangle + \langle y,y \rangle 
    \end{equation}
    Note:
    \begin{equation}
        \langle x,y \rangle + \langle y,x \rangle = 2\mathbb{R}(\langle x,y \rangle)
    \end{equation}
    \begin{equation}
        = \langle x,x \rangle + 2\mathbb{R}(\langle x,y \rangle) + \langle y,y \rangle 
    \end{equation}
    We note that if $2\mathbb{R}(\langle x,y \rangle) = 2a$, then 
    \begin{equation}
        2|a| \leq |a + bi| + |a-bi|
    \end{equation}
    \begin{equation}
        2|a| \leq 2\sqrt{a^2 + b^2}
    \end{equation}
    Then the whole equation simplifies down to 
    \begin{equation}
        \leq \langle x,x \rangle + 2|\langle x,y \rangle| + \langle y,y \rangle 
    \end{equation}
    \begin{equation}
        \leq ||x||^2 + 2||x|| ||y|| + ||y||^2
    \end{equation}
    \begin{equation}
        = (||x|| + ||y||)^2
    \end{equation}
\end{proof}

We prove $(1)$, then 
\begin{proof}
    It is clearly true if $y = 0$, thus we assume that $y \neq 0$. For any $c \in F$, then 
    \begin{equation}
        0 \leq \langle x - cy, x - cy \rangle
    \end{equation}
    We first expand out the first value:
    \begin{equation}
        = \langle x, x-cy \rangle -c \langle y,x-cy \rangle 
    \end{equation}
    \begin{equation}
        = \overline{\langle x-cy, x \rangle} - c \overline{\langle x-cy, y \rangle}
    \end{equation}
    \begin{equation}
        = \langle x,x \rangle - \overline{c\langle y,x \rangle} - c\overline{\langle x,y \rangle} + c\overline{c\langle y,y \rangle}
    \end{equation}
    \begin{equation}
        = \langle x,x \rangle - \overline{c}\langle x,y \rangle - c \overline{\langle x,y \rangle} + c\overline{c}\langle y,y \rangle
    \end{equation}
    Choose
    \begin{equation}
        c = \frac{\langle x,y \rangle}{\langle y,y \rangle }
    \end{equation}
    Then we have 
    \begin{equation}
        \overline{c}\langle x,y \rangle = \frac{\overline{\langle x,x \rangle}}{\langle y,y \rangle}\langle x,y \rangle = \frac{\langle x,y \rangle \langle y,x \rangle}{\langle y,y \rangle}
    \end{equation}
    \begin{equation}
        c\overline{\langle x,y \rangle} =  \frac{\langle x,y \rangle \langle y,x \rangle}{\langle y,y \rangle}
    \end{equation}
    \begin{equation}
        c\overline{c} \langle y,y \rangle = \frac{\langle x,y \rangle \langle y,x \rangle}{\langle y,y \rangle}
    \end{equation}
    So 
    \begin{equation}
        0 \leq \langle x,x \rangle - \frac{\langle x,y \rangle \langle y,x \rangle}{\langle y,y \rangle}
    \end{equation}
    Thus 
    \begin{equation}
        \langle x,y \rangle \langle y,x \rangle \leq \langle x,x \rangle \langle y,y \rangle 
    \end{equation}
    \begin{equation}
        = (|\langle x,y \rangle|)^2 \leq ||x||^2||y||^2
    \end{equation}
    Then, because all numbers are positive, we have that  
    \begin{equation}
        |\langle x,y \rangle| \leq ||x|| ||y||
    \end{equation}
    This is the Cauchy-Schwarz inequality. 
\end{proof}

\begin{definition}
    Let $V$ be an inner product space with $x,y \in V$, then 
    \begin{enumerate}
        \item $x$ and $y$ are \underline{orthogonal} if $\langle x,y \rangle = 0$
        \item A subset $S \subseteq V$ is \underline{orthogonal} if all pairs of distinct $x,y \in S$ are orthogonal
        \item $x$ is a unit vector if $\langle x,x \rangle = 1 \iff ||x|| = 1 = \sqrt{\langle x,x \rangle}$
        \item A set $S \subseteq V$ is orthonormal if it is orthogonal and consists entirely of unit vectors. 
    \end{enumerate}
\end{definition}

\subsection*{Example 1}
Let $V = \mathbb{C}^2$, then 
\begin{enumerate}
    \item $\{e_1,e_2\}$ are still orthonormal. 
    \item $v = \begin{pmatrix}
        2 + i \\
        i
    \end{pmatrix}$ and $w = \begin{pmatrix}
        -1 \\
        1 + 2i
    \end{pmatrix}$ are orthogonal. Then 
    \begin{equation}
        \langle x,w\rangle = (2 + i)(-1) + i(\overline{1+2i}) = 0
    \end{equation}
\end{enumerate}

\subsection*{Example 2}

Let $V = P_1(\mathbb{R})$ with 
\begin{equation}
    \langle f,g \rangle = \int_{-1}^{1}f(x)g(x)dx
\end{equation}
Claim vectors that 
\begin{equation}
    f(x) = \frac{1}{\sqrt{2}}
\end{equation}
and 
\begin{equation}
    g(x) = \sqrt{\frac{3}{2}}x
\end{equation}
are orthonormal basis. We check:
\begin{equation}
    \langle f,g \rangle \int_{-1}^{1} \frac{1}{\sqrt{2}}\sqrt{\frac{3}{2}}xdx
\end{equation}
This is an odd function, thus this evaluates to 0.
\begin{equation}
    \langle f,f \rangle = \int_{-1}^{1}\frac{1}{2}dx \iff 2 \cdot \frac{1}{2} = 1
\end{equation}
\begin{equation}
    \langle g,g \rangle = \int_{-1}^{1}\frac{3}{2}x^2 dx \iff x^3(\frac{1}{2}) = \frac{1}{2} \cdot 2 = 1
\end{equation}

\subsection*{Goal:}
Let's try to orthonormal bases for a given inner product space. Then 
\begin{theorem}
    Every finite dimensional inner product space has an orthonormal basis. 
\end{theorem}
\begin{theorem}
    Given an orthogonal set 
    \begin{equation}
        S = \{v_1, \dots, v_k\} \text{ with } 0 \notin S
    \end{equation}
    with 
    \begin{equation}
        y = a_1v_1 + \dots + a_kv_k
    \end{equation}
    \begin{equation}
        a_i = \frac{\langle y, v_i \rangle}{\langle v_i, v_i \rangle}
    \end{equation}
\end{theorem}

\begin{proof}
    Given
    \begin{equation}
        y = a_1v_1 + \dots + a_kv_k
    \end{equation}
    Calculate
    \begin{equation}
        \langle y,v_i \rangle = a_1 \langle v_1,v_i\rangle + \dots
    \end{equation}
    Because this is an orthongal basis, we have that 
    \begin{equation}
        \langle v_j, v_i \rangle = 0 \iff j \neq i
    \end{equation}
    Then the only one standing is 
    \begin{equation}
        \langle y,v_i \rangle = a_i \langle v_i, v_i \rangle
    \end{equation}
    In other words 
    \begin{equation}
        a_i = \frac{\langle y, v_i \rangle}{\langle v_i, v_i \rangle}
    \end{equation}
\end{proof}

\begin{corollary}
    If $S \subseteq V$ is an orthogonal set in $V$, and $0 \notin S$, then $S$ is linearly independent. 
\end{corollary}

\begin{proof}
    If we have any linear dependency, that is 
    \begin{equation}
        a_1v_1 + \dots + a_kv_k = 0 = y
    \end{equation}
    Then
    \begin{equation}
        a_i = \frac{\langle y, v_i \rangle }{\langle v_i, v_i \rangle }
    \end{equation}
    This means that 
    \begin{equation}
        a_i = 0
    \end{equation}
\end{proof}

\subsection*{Fourier Series}

Let $V = C[-\pi, \pi]$ where $C$ represents continuous functions in the interval $[-\pi, \pi]$. We define 
\begin{equation}
    \langle f, g \rangle = \frac{1}{\pi}\int_{\pi}^{-\pi}f(x)g(x)dx
\end{equation}then
\subsubsection*{Fact}
The following set is orthonormal in $V$:
\begin{equation}
    S = \{\frac{1}{\sqrt{2}}\} \cup \{\cos(nx) \; n \in \mathbb{N}\} \cup \{\sin(nx) \; n \in \mathbb{N}\}
\end{equation}
We can solve for the coefficients using 
\begin{equation}
    a_i = \frac{\langle y(x), v_i \rangle}{\langle v_i, v_i \rangle}
\end{equation}
$S$ is not a basis of $V$, but most functions in practice can be expressed as Fourier Series. 
\end{document}
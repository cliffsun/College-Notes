\documentclass{article}
\usepackage[left=2cm, right=2cm, top=2cm, bottom=2cm]{geometry}
\usepackage{graphicx}
\usepackage{amsmath}
\usepackage{amssymb}
\usepackage{amsthm}
\usepackage{fancyhdr}
\usepackage{verbatim}
\usepackage{listings}
\usepackage{xcolor}
\usepackage{pgfplots}

\lstset{
    language=C++,
    basicstyle=\ttfamily\footnotesize,
    keywordstyle=\color{blue},
    commentstyle=\color{green},
    stringstyle=\color{red},
    numbers=left,
    numberstyle=\tiny\color{gray},
    frame=single,
    breaklines=true,
    captionpos=b,
}


\title{MTH 416: Lecture 15}
\author{Cliff Sun}

\newtheorem{theorem}{Theorem}[section]
\newtheorem{lemma}[theorem]{Lemma}
\newtheorem{definition}[theorem]{Definition}
\newtheorem{conjecture}[theorem]{Conjecture}
\newtheorem{proposition}[theorem]{Proposition}
\newtheorem{corollary}[theorem]{Corollary}
\newtheorem{one minute paper}[theorem]{One Minute Paper}

\pagestyle{fancy}
\lhead{\textbf{\thepage}\ \ \nouppercase{\rightmark}}
\chead{MTH 416: Lecture 15}
\rhead{Cliff Sun}

\begin{document}

\maketitle

\section*{Lecture Span}
\begin{itemize}
    \item $2 \times 2$ determinants
    \item $n \times n$ determinants
\end{itemize}

\section*{Recall}

\begin{enumerate}
    \item $\det(A) \neq 0 \iff$ A is invertible
    \item $\det(AB) = \det(A)\det(B)$
    \item $L_A$ scales volumes by factor of $|\det(A)|$
    \item $\det$ is not linear, but \underline{"multilinear"}
\end{enumerate}

Let $R = [0,t]\times [0,t]$, such that $t > 0$. Then performing $T(R)$ yields
\begin{equation}
    T(R) = \text{ parallelogram scaled by t }
\end{equation}
Thus 
\begin{equation}
    \text{area}(T(R)) = t^2|\det(A)|
\end{equation}

So in general, let $S = $ some general area in $\mathbb{R}^2$. Then we claim that 
\begin{center}
    area($T(S)$) = area(S) $\cdot |\det(A)|$
\end{center}

We can divide up the region $S$ into a bunch of little parallelograms, then applying $T$ to the paralleograms scales each parallelogram by a factor $\det(A)$. Thus,
\begin{center}
    area($T(S)$) = area(S)$\cdot |\det(A)|$
\end{center}

\begin{theorem}
    (multilinearity for $2 \times 2$ det) Let 
    \begin{equation}
        \det: M_{2 \times 2}(\mathbb{R}) \rightarrow \mathbb{R}
    \end{equation}
    Becomes a linear transformation if we view it as a function of only one row, and treat the other rows as constants. That is, if $u,v,w$ are row vectors in $\mathbb{R}^2$ and $k \in \mathbb{R}$. Then 
    \begin{equation}
        \det\begin{pmatrix}
            ku + v \\
            w
        \end{pmatrix} = k\det\begin{pmatrix}
            u \\
            w
        \end{pmatrix} + \det\begin{pmatrix}
            v \\
            w
        \end{pmatrix}
    \end{equation}
    Similarly
    \begin{equation}
        \det\begin{pmatrix}
            w\\
            ku + v
        \end{pmatrix} = k\det\begin{pmatrix}
            w \\
            u
        \end{pmatrix} + \det\begin{pmatrix}
            w \\
            v
        \end{pmatrix}
    \end{equation}
\end{theorem}

\begin{proof}
    Let $u = (a_1,a_2)$, $v = (b_1,b_2)$, and $w=(c_1,c_2)$. For the first equation, we expand it out to be 
    \begin{equation}
        \begin{pmatrix}
            ka_1 + b_1 & ka_2 + b_2 \\
            c_1 & c_2
        \end{pmatrix}
    \end{equation}
    Calculating the determinant yields
    \begin{equation}
        (ka_1 + b_1)c_2 - (ka_2 + b_2)c_1
    \end{equation}
    \begin{equation}
        k(a_1c_2 - a_2c_1) + (b_1c_2 - b_2c_1)
    \end{equation}
    \begin{equation}
        = k\det\begin{pmatrix}
            u \\\
            w
        \end{pmatrix} + \begin{pmatrix}
            v \\
            w
        \end{pmatrix}
    \end{equation}
    The second calculation is similar. 
\end{proof}

\section*{n x n determinants}

\begin{definition}
    The determinant of an $n \times n$ matrix ($n > 1$) is 
    \begin{equation}
        \det(A) = \sum_{j=1}^{n}(-1)^{j+1}A_{1j}\det(\tilde{A}_{1j})
    \end{equation}
    Where $\tilde{A}$ is the matrix given by deleting row $1$ and column $j$ from $A$. This is called \underline{cofactor expansion} on the first row. 
\end{definition}

\subsection*{Example, n=2}

\begin{equation}
    A = \begin{pmatrix}
        a & b \\
        c & d
    \end{pmatrix} = \begin{pmatrix}
        A_{11} & A_{12} \\
        A_{21} & A_{22}
    \end{pmatrix}
\end{equation}

Then
\begin{equation}
    \det(A) = \sum_{j=1}^{n}(-1)^{j+1}A_{1j}\det(\tilde{A}_{1j})
\end{equation}

\begin{equation}
    = (-1)^{1 + 1}A_{11}\det(A_{22}) + (-1)^{1 + 2}A_{12}\det(A_{21})
\end{equation}
\begin{equation}
    A_{11}A_{22} - A_{12}A_{21}
\end{equation}
\begin{equation}
    ad - bc
\end{equation}

\subsection*{Example, n=3}

\begin{equation}
    A = \begin{pmatrix}
        2 & 1 & 3 \\
        5 & 0 & 2 \\
        1 & 2 & 1
    \end{pmatrix}
\end{equation}

\begin{equation}
    \det(A) = 2\det\begin{pmatrix}
        0 & 2 \\
        2 & 1
    \end{pmatrix} - 1\det\begin{pmatrix}
        5 & 2 \\
        1 & 1 
    \end{pmatrix} + 3\det\begin{pmatrix}
        5 & 0 \\
        1 & 2
    \end{pmatrix}
\end{equation}

\begin{theorem}
    The $n \times n$ determinant is multilinear in the $n rows$. That is suppose we have matirces $A,B,C \in M_{2 \times 2}(\mathbb{R})$ which are 
    identical except in row r, where 
    \begin{equation}
        a_r = kb_r + c_r \text{ for some $k \in \mathbb{R}$}
    \end{equation}
    Then 
    \begin{equation}
        \det(A) = k\det(B) + \det(C)
    \end{equation}
\end{theorem}

\begin{proof}
    Induct on $n$. 
    \subsubsection*{Base Case: n = 1}
    This just says that the $1 \times 1$ determinant is linear, which we already know to be true. 
    \subsubsection*{Inductive Step}
    Assume that this theorem is true for all $n \times n$ matrices, then we must prove it for $(n + 1) \times (n + 1)$ matrices $A,B,C$ as well.
    Let $A,B,C$ be as above, then there are 2 cases. When $r = 1$ and $r \neq 1$. 
    \subsubsection*{r=1}
    \begin{equation}
        \det(A) = \sum_{j=1}^{n}(-1)^{j+1}A_{1j}\det(\tilde{A}_{1j})
    \end{equation}
    \begin{equation}
        = \sum_{j =1}^{n+1}(-1)^{j+1}(kB_{1j} + C_{1j})\det(\tilde{A}_{1j})
    \end{equation}
    But note that 
    \begin{equation}
        \det(\tilde{A}_{1j}) = \det(\tilde{B}_{1j}) = \det(\tilde{C}_{1j})
    \end{equation}
    \begin{equation}
        = k\sum_{j=1}^{n+1}-1^{j+1}B_{1j}\det(\tilde{B}_{1j}) + \sum_{j=1}^{n+1}-1^{j+1}C_{1j}\det(\tilde{C}_{1j})
    \end{equation}
    \begin{equation}
        = k\det(B) + \det(C)
    \end{equation}
    \subsubsection*{r>1}
    \begin{equation}
        \det(A) = \sum_{j=1}^{n+1}(-1)^{j+1}A_{1j}\det(\tilde{A}_{1j})
    \end{equation}
    The matrices $\tilde{A}_{1j}$, $\tilde{B}_{1j}$, $\tilde{C}_{1j}$ are identical except for the $r-1$ row. So 
    \begin{equation}
        \det(A) = \sum_{j=1}^{n+1}(-1)^{j+1}A_{1j}\det(\tilde{A}_{1j})
    \end{equation}
    \begin{equation}
        = \sum (-1)^{j+1}A_{1j}(k\det(\tilde{B}_{1j}) + \det(\tilde{C}_{ij}))
    \end{equation}
    \begin{equation}
        = \sum (-1)^{j+1}B_{1j}\det(\tilde{B}_{1j}) + \sum(-1)^{j+1}C_{1j}\det(\tilde{C}_{1j})
    \end{equation}
    This proves the theorem. 
\end{proof}

\begin{theorem}
    Let $A \in M_{2 \times 2}(\mathbb{R})$ and let $1 \leq r \leq n$. Then 
    \begin{equation}
        \det(A) = \sum_{j=1}^{n}(-1)^{r+j}A_{rj}\det(\tilde{A}_{rj})
    \end{equation}
\end{theorem}

\begin{lemma}
    If $A$ is a $n \times n$ matrix, with row $r = e_{j}$. Then 
    \begin{equation}
        \det(A) = (-1)^{r+j}\det(\tilde{A}_{rj})
    \end{equation}
\end{lemma}

\end{document}
\documentclass{article}
\usepackage[left=2cm, right=2cm, top=1cm, bottom=1cm]{geometry}
\usepackage{graphicx}
\usepackage{amsmath}
\usepackage{amssymb}
\usepackage{amsthm}
\usepackage{fancyhdr}
\usepackage{verbatim}
\usepackage{listings}
\usepackage{xcolor}

\lstset{
    language=C++,
    basicstyle=\ttfamily\footnotesize,
    keywordstyle=\color{blue},
    commentstyle=\color{green},
    stringstyle=\color{red},
    numbers=left,
    numberstyle=\tiny\color{gray},
    frame=single,
    breaklines=true,
    captionpos=b,
}


\title{MTH 416: Lecture 3}
\author{Cliff Sun}

\newtheorem{theorem}{Theorem}[section]
\newtheorem{lemma}[theorem]{Lemma}
\newtheorem{definition}[theorem]{Definition}
\newtheorem{conjecture}[theorem]{Conjecture}
\newtheorem{proposition}[theorem]{Proposition}
\newtheorem{corollary}[theorem]{Corollary}
\newtheorem{one minute paper}[theorem]{One Minute Paper}

\pagestyle{fancy}
\lhead{\textbf{\thepage}\ \ \nouppercase{\rightmark}}
\chead{MTH 416: Lecture 3}
\rhead{Cliff Sun}

\begin{document}

\maketitle

\section*{Lecture Span}
\begin{itemize}
    \item Linear combinations
    \item Systems of linear equations
\end{itemize}

\section*{Recall}

\begin{theorem}
    If $W$ is a subset of a vector space $V$, then 
    \begin{equation}
        0 \in W \land W \text{ is closed under $+$ and $\cdot$} \iff W \text{ is a Vector Space with the same operations as $V$}
    \end{equation}
\end{theorem}

Last time, we prove $(\implies)$, this time we want to prove $(\impliedby)$. That is, if either left or right is true, then $W$ is a \underline{subspace} of $V$. 

\begin{proof}
    Suppose $W$ is a vector space with the same operations as $V$, that is $W$ satisfies all 8 axioms. Then we must prove that
    \begin{enumerate}
        \item $0 \in W$
        \item $W$ is closed under $+$ and $\cdot$
    \end{enumerate}
    We first prove $(2)$, as it is the easiest to start with. 
    \begin{proof}
        If it weren't closed under these operations, then $+$ and $\cdot$ doesn't make sense as operations under $W$. 
    \end{proof}
    We then prove $(1)$, 
    \begin{proof}
        $W$ is a vector space, so it contains a $0_W$ vector. Such that
        \begin{equation}
            w + 0_W = w \; \text{ for all } w \in W
        \end{equation} 
        Since $V$ is a vector space, it also contains $0_V$, such that
        \begin{equation}
            v + 0_V = v \; \text{ for all } v \in V
        \end{equation}
        We claim that $0_W = 0_V$. 
        \begin{proof}
            \begin{equation}
                w + 0_W = w = w + 0_V \text{ because } w \in V
            \end{equation}
            Using the cancellatin theorem we have that 
            \begin{equation}
                0_W = 0_V
            \end{equation}
        \end{proof}
    \end{proof}
\end{proof}

\newpage

\section*{Linear combinations}

\begin{definition}
    Let $u_1, \dots, u_n$ be vectors in a vector space $V$, then 
    \begin{enumerate}
        \item A \underline{linear combination} of the vectors $u_i$ is any vector that can be written as the following form:
        \begin{equation}
            u_i = a_1u_1 + \cdots + a_n u_n
        \end{equation}
        \item A set of all linear combinations $u_i$ is called the span of $u_i$. 
    \end{enumerate}
\end{definition}

\begin{theorem}
    If $u_1, \cdots, u_n$ are vectors in a vector space $V$, then $span(u_1,\cdots, u_n)$ is always going to be a subspace of $V$. 
\end{theorem}

\begin{proof}
    We must show that this span has the $0$ vector and is closed under addition \& scalar multiplication. 
    \begin{enumerate}
        \item $0 = 0u_1 + \cdots + 0u_n$
        \item Assume we are given $v = a_1u_1 + \cdots a_n u_n$ and $w = b_1u_1 + \cdots + b_n u_n$, we have that prove that $v + w$ is also a linear combination. In other words, $v + w \in span(u_i)$. We first expand this out:
        \begin{equation}
            v + w \iff (a_1u_1 + \cdots a_n u_n) + (b_1u_1 + \cdots b_n u_n) \iff (a_1 + b_1)u_1 \cdots (a_n + b_n)u_n
        \end{equation}
        This an element of $span(u_i)$. 
        \item Suppose $v = a_1 u_1 + \cdots + a_n u_n$ and $c \in \mathbb{R}$. Then, 
        \begin{equation}
            cv = c(a_1u_1 + \cdots + a_n u_n) \iff ca_1u_1 + \cdots + ca_nu_n \in span(u_i)
        \end{equation}
    \end{enumerate}
\end{proof}

Moreover, $span(u_i)$ is the smallest subspace of $V$ containing all $u_1, \cdots u_n$. That is, there cannot be a smaller subspace that also contains all $u_1, \cdots, u_n$. 
\begin{enumerate}
    \item $u_1, \cdots, u_n \in span(u_i)$ (Choose $a_i = 1$, all other $0$)
    \item If $W$ is any subspace of V also containing all $u_i$, then $span(u_i) \in W$
\end{enumerate}

\section*{Systems of Linear Equations/Linear System}

\begin{definition}
    A system of $m$ linear equations with $n$ unknowns is a system of the form:
    \begin{equation}
        a_{11}x_ + \cdots + a_{1n}x_n = b_1
    \end{equation}
    \begin{equation}
        a_{21}x_ + \cdots + a_{2n}x_n = b_2
    \end{equation}
    \begin{equation}
        a_{m1}x_ + \cdots + a_{mn}x_n = b_m
    \end{equation}
    Where $a_{ij}$ and $b_i$ are scalars.
\end{definition}

\subsection*{Goals}
\begin{enumerate}
    \item Determine whether if there is a solution $(x_1, \cdots, x_n)$
    \item Find all solutions
\end{enumerate}

\subsection*{Notation}

We can associate to a linear system the \underline{augmented} matrix
\[
\begin{pmatrix}
    a_{11} & \cdots & a_{2n} & b_1\\
    \cdots & \cdots & \cdots & \cdots\\
    a_{m1} & \cdots & a_{mn} & b_m\\
\end{pmatrix}    
\]

Where the $m\times n$ matrix are all the coefficients and the right hand $b$ are the constraints. We denote this as 
\begin{equation}
    LS(A|b)
\end{equation}

\end{document}
\documentclass{article}
\usepackage[left=2cm, right=2cm, top=2cm, bottom=2cm]{geometry}
\usepackage{graphicx}
\usepackage{amsmath}
\usepackage{amssymb}
\usepackage{amsthm}
\usepackage{fancyhdr}
\usepackage{verbatim}
\usepackage{listings}
\usepackage{xcolor}
\usepackage{pgfplots}

\lstset{
    language=C++,
    basicstyle=\ttfamily\footnotesize,
    keywordstyle=\color{blue},
    commentstyle=\color{green},
    stringstyle=\color{red},
    numbers=left,
    numberstyle=\tiny\color{gray},
    frame=single,
    breaklines=true,
    captionpos=b,
}


\title{MTH 416: Lecture 24}
\author{Cliff Sun}

\newtheorem{theorem}{Theorem}[section]
\newtheorem{lemma}[theorem]{Lemma}
\newtheorem{definition}[theorem]{Definition}
\newtheorem{conjecture}[theorem]{Conjecture}
\newtheorem{proposition}[theorem]{Proposition}
\newtheorem{corollary}[theorem]{Corollary}
\newtheorem{one minute paper}[theorem]{One Minute Paper}

\pagestyle{fancy}
\lhead{\textbf{\thepage}\ \ \nouppercase{\rightmark}}
\chead{MTH 416: Lecture 24}
\rhead{Cliff Sun}

\begin{document}

\maketitle

\section*{Lecture Span}
\begin{itemize}
    \item Gram-Schmidt 
\end{itemize}

\textbf{Idea:} Given a basis for $V$, can we adjust it to make it orthonormal? 

Given that $\{w_1,w_2\}$ is a basis in $\mathbb{R}^2$, how to make them orthonormal?

\subsubsection*{Solution}

We let $v_1 = w_1$, then we can decompose $w_2$ to have a portion that is orthogonal to $w_1$, (call it $v_2$), and a portion that is along the lines of $w_1$, (call it $cv_1$).
How do we find $c$? We use the inner product:
\begin{equation}
    \langle w_2, v_1 \rangle = \langle v_2, v_1 \rangle + c\langle v_1,v_1 \rangle 
\end{equation}
\begin{equation}
    \implies c = \frac{\langle w_2, v_1 \rangle}{\langle v_1, v_1 \rangle} \in \mathbb{R}
\end{equation}

Given that 
\begin{equation}
    w_2 = cv_1 + v_2
\end{equation}

So 

\begin{equation}
    \{v_1 = w_1, v_2 = w_2 - \frac{\langle w_2, v_1 \rangle}{\langle v_1, v_1 \rangle}v_1\}
\end{equation}

\subsubsection*{Example in $\mathbb{R}^3$}

Given 2 orthogonal vectors $v_1, v_2$ and a 3rd non-orthogonal vector $w_3$, how do we find the 3rd orthogonal vector?

\subsubsection*{Solution}

We can write 

\begin{equation}
    w_3 = c_1v_1 + c_2v_2 + v_3
\end{equation}

WIth $v_3$ perpendicular to $v_1, v_2$. How can we find $c_1, c_2$? 

\begin{equation}
    \langle w_3, v_1 \rangle = c_1\langle v_1, v_1 \rangle + 0 + 0
\end{equation}
\begin{equation}
    c_1 = \frac{\langle w_3, v_1 \rangle}{\langle v_1, v_1 \rangle}
\end{equation}
$c_2$ can be calculated in a similar way. Therefore, 

\begin{equation}
    v_3 = w_3 - c_2v_2 - c_1v_1
\end{equation}

\begin{theorem}
    (\underline{Gram-Schmidt}): Let $V$ be any inner product space, and let $\{w_1, \dots, w_n\}$ be a linearly independent set. 
    Then define $v_1 = w_1$ and for $v_k > 1$, 
    \begin{equation}
        v_k = w_k - \sum_{j=1}^{k-1}\frac{\langle w_k, v_j \rangle }{\langle v_j, v_j \rangle}v_j 
    \end{equation}
    Then we claim
    \begin{enumerate}
        \item $span(v_1,\dots,v_n) = span(w_1, \dots, w_n)$
        \item $\{v_1,\dots,v_n\}$ is an orthogonal set 
    \end{enumerate}
\end{theorem}

\begin{proof}
    Induct on $n$, base case is $n=1$. This is trivial. For the inductive step, we consider this to be true for some $n$, and we now consider $n+1$. Given 
    $\{w_1,\dots,w_{n+1}\}$, then $\{v_1,\dots,v_n,v_{n+1}\}$. Then $v_{n+1} = w_{n+1} - c_1v_1 - \dots$. By IH, 
    \begin{equation}
        span(v_1,\dots,v_n) = span(w_1,\dots,w_n)
    \end{equation}
    But 
    \begin{equation}
        v_{n+1} \in span(w_{n+1}, v_1,\dots,v_n)
    \end{equation}
    \begin{equation}
        = span(w_1,\dots,w_{n+1})
    \end{equation}
    So 
    \begin{equation}
        span({v_i}) \subseteq span({w_i})
    \end{equation}
    To prove the opposite direction, we perform 
    \begin{equation}
        v_{n+1} = w_{n+1} - c_1v_1 - \dots
    \end{equation}
    \begin{equation}
        \implies v_{n+1} + c_1v_1 + \dots = w_{n+1}
    \end{equation}
    This implies that 
    \begin{equation}
        span(v_1,\dots,v_{n+1}) = w_{n+1}
    \end{equation}
    Since $span(v_1,\dots,v_n) = span(w_1,\dots,w_n)$, we have that 
    \begin{equation}
        span(v_1,\dots,v_{n+1}) = span(w_1,\dots,w_{n+1})
    \end{equation}
\end{proof}

We now prove 2, 

\begin{proof}
    Induct on $n$, the base case $n=1$ is vacuous (mindless, useless). For the inductive step, suppose it's true for $n$, and consider $n+1$. We must prove that 
    \begin{equation}
        v_{n+1} = w_{n+1} - \sum_{j=1}^{n}\frac{\langle w_{n+1}, v_j \rangle }{\langle v_j, v_j \rangle}v_j
    \end{equation}
    is orthogonal to $v_1, \dots, v_n$. (Note: $\{v_1,\dots,v_j\}$ are linearly independent, thus $\langle v_j, v_j \rangle \neq 0$). Fix $i \in \{1,\dots, n\}$ and calculate 
    \begin{equation}
        \langle v_{n+1}, v_i \rangle  = \langle w_{n+1}, v_i \rangle - \sum_{j=1}^{n}\frac{\langle w_{n+1}, v_j \rangle}{\langle v_j, v_j \rangle}\langle v_j, v_i \rangle 
    \end{equation}
    Considering $\langle v_j, v_i \rangle$, this evaluates to $0 \iff i \neq j$. 
    \begin{equation}
        = \langle w_{n+1}, v_j \rangle - \frac{\langle w_{n+1}, v_j\rangle }{\langle v_j, v_j \rangle}\langle v_j,v_j \rangle
    \end{equation}
    \begin{equation}
        = 0
    \end{equation}
    By induction, this proves the theorem. 
\end{proof}

\begin{corollary}
    Every finite dimensional inner product space has an orthonormal basis. 
\end{corollary}

\begin{proof}
    Let $V$ be some finite dimensional inner product space. Let $\{w_1,\dots,w_n\}$ be a basis for $V$. Applying Gram-Schmidt gives a new orthogonal set $\{v_1, \dots, v_n\}$ with the same span. 
    So $\{v_1, \dots, v_n\}$ is an orthogonal basis, and $\{u_1,\dots,u_n\} = \left\{\frac{v_1}{||v_1||}, \dots\right\}$ is an orthonormal basis. 
\end{proof}

\begin{definition}
    Let $V$ be an inner product space and $S$ some nonempty subset. Then the \underline{orthogonal compliment} of $S$ is 
    \begin{equation}
        S^{\perp} = \{x \in V, \langle x,y \rangle = 0 \text{ for all } y \in S\}
    \end{equation}
    Note:
    \begin{enumerate}
        \item $S^{\perp}$ is a subspace of $V$. 
        \item $S^{\perp}$ is $span(S)^{\perp}$
    \end{enumerate}
\end{definition}

\begin{theorem}
    Suppose $V$ is a $n-$dimensional inner product space, and $W$ is a subspace. Then 
    \begin{enumerate}
        \item Any orthonormal basis $\{v_1, \dots, v_k\}$ for $W$ can be extended to an orthonormal basis for $V$. 
        \item If we extend, the new vectors $\{v_{k+1}, \dots, v_n\}$ is a basis for $W^{\perp}$. 
        \item $\dim(W) + \dim(W^{\perp}) = \dim(V)$
    \end{enumerate}
\end{theorem}

\begin{proof}
    Extend $\{v_1, \dots, v_k\}$ to some basis $\{v_1, \dots, v_k, w_{k+1}, \dots, w_{n}\}$ of $V$. We can turn this into an orthonormal using Gram-Schmidt + normalize the vectors. 
\end{proof}

We prove $3$, 

\begin{proof}
    $\dim(W) = k$, $\dim(W^{\perp}) = n-k$, thus $\dim(W) + \dim(W^{\perp}) = n \iff \dim(V)$.  
\end{proof}

\end{document}
\documentclass{article}
\usepackage[left=2cm, right=2cm, top=2cm, bottom=2cm]{geometry}
\usepackage{graphicx}
\usepackage{amsmath}
\usepackage{amssymb}
\usepackage{amsthm}
\usepackage{fancyhdr}
\usepackage{verbatim}
\usepackage{listings}
\usepackage{xcolor}
\usepackage{pgfplots}

\lstset{
    language=C++,
    basicstyle=\ttfamily\footnotesize,
    keywordstyle=\color{blue},
    commentstyle=\color{green},
    stringstyle=\color{red},
    numbers=left,
    numberstyle=\tiny\color{gray},
    frame=single,
    breaklines=true,
    captionpos=b,
}


\title{MTH 416: Lecture 27}
\author{Cliff Sun}

\newtheorem{theorem}{Theorem}[section]
\newtheorem{lemma}[theorem]{Lemma}
\newtheorem{definition}[theorem]{Definition}
\newtheorem{conjecture}[theorem]{Conjecture}
\newtheorem{proposition}[theorem]{Proposition}
\newtheorem{corollary}[theorem]{Corollary}
\newtheorem{one minute paper}[theorem]{One Minute Paper}

\pagestyle{fancy}
\lhead{\textbf{\thepage}\ \ \nouppercase{\rightmark}}
\chead{MTH 416: Lecture 27}
\rhead{Cliff Sun}

\begin{document}

\maketitle

\section*{Lecture Span}
\begin{itemize}
    \item Jordan Canonical Form
\end{itemize}

\begin{theorem}
    If $A \in M_{n\times n}(F)$ is such that char poly(A) splits over $F$, then $A$ is similar to some $B$ in JCF. 
\end{theorem}

\begin{center}
    \textbf{Application:} Calculating the powers of a non-diagonalizable matrix $A$
\end{center}

A matrix in JCF has a specific formula that we can use to calculate its powers.

\begin{center}
    \textbf{Recall:} To diagonalize $A$, we need a basis of eigenvectors.
\end{center}

For JCF, we need \underline{generalized eigenvectors}. 

\begin{definition}
    \begin{enumerate}
        \item A vector $v\neq 0$ is a \underline{generalized eigenvector} if $A$ for an eigenvalue of $\lambda$ is 
        \begin{equation}
            (A - \lambda I)^{p}v = 0
        \end{equation}
        For some integer $p$. 
        \item Given $\lambda$, the \underline{generalized eigenspace} $K_{\lambda}$, is the set of all $v$ (including $0$) satisfying the above. 
    \end{enumerate}
\end{definition}

\begin{lemma}
    This generalized eigenspace $K_{\lambda_i}$ is a subspace of $F^{n}$, containing the normal eigenspace. 
\end{lemma}

Given a Jordan block of typical form, what is the generalized eigenvectors? 

\begin{equation}
    Ae_{n} = \lambda e_{n} + e_{n-1}
\end{equation}
\begin{equation}
    (A - \lambda I)e_{n} = e_{n-1}
\end{equation}
\begin{equation}
    (A - \lambda I)e_{n-1} = e_{n-2} 
\end{equation}
etc. 
\begin{definition}
    A set of vectors is called a \underline{cycle of generalized eigenvectors} if it has the form
    \begin{equation}
        \beta = \{v_1, \dots, v_p\}
    \end{equation}
    Where 
    \begin{equation}
        (A- \lambda I)v_p = v_{p-1}
    \end{equation}
    \begin{equation}
        (A- \lambda I)v_{p-1} = v_{p-2}
    \end{equation}
    \begin{equation}
        \cdots
    \end{equation}
    \begin{equation}
        (A - \lambda I)v_{1} = 0
    \end{equation}
\end{definition}

\begin{center}
    \textbf{Idea of proof of main theorem:}
\end{center}
\begin{enumerate}
    \item Show that $F^{k} = \oplus_{i=1}^{k}K_{\lambda_i}$ (generalized sum of eigenspaces). That is for any $v \in F$ can be written uniquely as 
    \begin{equation}
        v = v_1 + \dots + v_k    
    \end{equation}
    for $v_i \in K_{\lambda_i}$ for each $v$. 
    \item Each $K_{\lambda_i}$ has a basis consisting of a disjoint union of cycles of generalized eigenvectors. 
    \begin{center}
        \textbf{Note:} The basis vectors for each $K_{\lambda_i}$ would compose the "diagonalized" matrix. 
    \end{center}
\end{enumerate}

\end{document}
\documentclass{article}
\usepackage[left=2cm, right=2cm, top=2cm, bottom=2cm]{geometry}
\usepackage{graphicx}
\usepackage{amsmath}
\usepackage{amssymb}
\usepackage{amsthm}
\usepackage{fancyhdr}
\usepackage{verbatim}
\usepackage{listings}
\usepackage{xcolor}
\usepackage{pgfplots}

\lstset{
    language=C++,
    basicstyle=\ttfamily\footnotesize,
    keywordstyle=\color{blue},
    commentstyle=\color{green},
    stringstyle=\color{red},
    numbers=left,
    numberstyle=\tiny\color{gray},
    frame=single,
    breaklines=true,
    captionpos=b,
}


\title{MTH 416: Lecture 19}
\author{Cliff Sun}

\newtheorem{theorem}{Theorem}[section]
\newtheorem{lemma}[theorem]{Lemma}
\newtheorem{definition}[theorem]{Definition}
\newtheorem{conjecture}[theorem]{Conjecture}
\newtheorem{proposition}[theorem]{Proposition}
\newtheorem{corollary}[theorem]{Corollary}
\newtheorem{one minute paper}[theorem]{One Minute Paper}

\pagestyle{fancy}
\lhead{\textbf{\thepage}\ \ \nouppercase{\rightmark}}
\chead{MTH 416: Lecture 19}
\rhead{Cliff Sun}

\begin{document}

\maketitle

\section*{Lecture Span}
\begin{itemize}
    \item Diagonalizability
    \item Eigenvalues, Eigenvectors
\end{itemize}


\begin{theorem}
    Suppose $A \in M_{n \times n}$
    \begin{enumerate}
        \item $\lambda$ is an eigenvalue $\iff$ 
        \begin{equation}
            \det(A - \lambda I) = 0
        \end{equation}
        \item Given some $0 \neq v \in \mathbb{R}^n$, then $v$ is an eigenvector with an eigenvector $\lambda \iff$
        \begin{equation}
            v \in N(A - \lambda I)
        \end{equation}
    \end{enumerate}
\end{theorem}

\begin{proof}
    We first prove $2.$. Given $v \neq 0$, then $v$ is an eigenvector with eigenvalue $\lambda \iff$
    \begin{equation}
        Av = \lambda v
    \end{equation} 
    \begin{equation}
        \iff Av = (\lambda I) v
    \end{equation}
    \begin{equation}
        \iff (A - \lambda I)v = 0
    \end{equation}
    \begin{equation}
        v \in N(A - \lambda I)
    \end{equation}
    We now prove statement $1.$, now $\lambda$ is an eigenvalue of $A \iff$
    \begin{equation}
        \exists v \neq 0 \text{ such that } v \in N(A - \lambda I)
    \end{equation} 
    \begin{equation}
        \iff \null(A - \lambda I) > 0
    \end{equation}
    \begin{equation}
        \iff A - \lambda I \text{ is not invertible. }
    \end{equation}
    \begin{equation}
        \iff \det(A - \lambda I) = 0
    \end{equation}
\end{proof}

\subsubsection*{Note 1:}

$N(A - \lambda I)$ is called the \underline{Eigenspace} of $A$ with eigenvalue $\lambda$, denoted $E_\lambda$. 
\begin{center}
    $E_\lambda = $ \{ eigenvalues with eigenvalue $\lambda$ \} $\cup$ $\{ 0 \}$
\end{center} 

\subsubsection*{Note 2:}

The function $f(t) = \det(A - tI)$ is polynomial of degree $n$, called the \underline{characteristic polynomial} of $A$. Its leading 
coefficient is always $\pm 1$, specifically $(-1)^n$. 

\begin{corollary}
    Any $n \times n$ matrix has at most $n$ eigenvalues.
\end{corollary}

\begin{proof}
    Any polynomial of degree $n$ has at most $n$ roots. 
\end{proof}

\subsubsection*{Aside}

It also makes sense to talk about eigenstuff of an operator $T$ on an infinite dimensional vector space $V$. 

\subsubsection*{Ex:}

Let $V = $ \{ functions $f: \mathbb{R} \rightarrow \mathbb{R}$ that are infinitely differentiable. \} Then we claim that every $\lambda \in \mathbb{R}$ is an eigenvalue of $T$.

\begin{proof}
    Given $\lambda$, choose $e^{\lambda x}$. Trivial
\end{proof}

\subsection*{Example}

\begin{equation}
    A = \begin{pmatrix}
        0 & 0 & 1\\
        0 & 2 & 0 \\
        -2 & 0 & 3
    \end{pmatrix}
\end{equation}

\textbf{Goal:} Calculate all eigenstuff; try to diagonolize the $A$ if possible.

\subsubsection*{Step 1:}

\begin{equation}
    \det(A - tI)
\end{equation}

\begin{equation}
    = \det\begin{pmatrix}
        -t & 0 & 1 \\
        0 & 2-t & 0 \\
        -2 & 0 & 3-t
    \end{pmatrix}
\end{equation}

Cofactor expansion on the middle row:

\begin{equation}
    = (2-t)\det\begin{pmatrix}
        -t & 1 \\
        -2 & 3-t
    \end{pmatrix}
\end{equation}

\begin{equation}
    = (2-t)(-3t + t^2 + 2)
\end{equation}

\begin{equation}
    = (2-t)(t-1)(t-2)
\end{equation}

Thus: $\lambda = 1,2$

\subsubsection*{Step 2:}

Eigenvectors for $\lambda = 1$,

\begin{equation}
    A - I = \begin{pmatrix}
        -1 & 0 & 1 \\
        0 & 1 & 0 \\
        -2 & 0 & 2
    \end{pmatrix}
\end{equation}

So 

\begin{equation}
    E_1 = N(A - I) = \text{ solution set to LS }\left(\begin{array}{ccc|c}
        -1 & 0 & 1 & 0\\
        0 & 1 & 0 & 0 \\
        -2 & 0 & 2 & 0
    \end{array}\right)
\end{equation}

We row reduce

\begin{equation}
    \left(\begin{array}{ccc|c}
        -1 & 0 & 1 & 0\\
        0 & 1 & 0 & 0 \\
        -2 & 0 & 2 & 0
    \end{array}\right)
\end{equation}

\begin{equation}
    R_3 = R_3 - 2R_1
\end{equation}

\begin{equation}
    \left(\begin{array}{ccc|c}
        -1 & 0 & 1 & 0\\
        0 & 1 & 0 & 0 \\
        0 & 0 & 0 & 0
    \end{array}\right)
\end{equation}

Thus we have that 

\begin{equation}
    a_1 = a_3
\end{equation}

\begin{equation}
    a_2 = 0
\end{equation}

Thus, 

\begin{equation}
    E_1 = \{a_3, 0, a_3\} \; a_3 \in \mathbb{R}
\end{equation}

\begin{equation}
    = span(\{(1,0,1)\})
\end{equation}

Eigenvectors for $\lambda = 2$. 

\begin{equation}
    E_2 = N(A - 2I)
\end{equation}

Row reduce 

\begin{equation}
    \left(\begin{array}{ccc|c}
        -2 & 0 & 1 & 0\\
        0 & 0 & 0 & 0 \\
        -2 & 0 & 1 & 0
    \end{array}\right)
\end{equation}

\begin{equation}
    R_3 = R_3 - R_1
\end{equation}

\begin{equation}
    \left(\begin{array}{ccc|c}
        -2 & 0 & 1 & 0\\
        0 & 0 & 0 & 0 \\
        0 & 0 & 0 & 0
    \end{array}\right)
\end{equation}

Thus 

\begin{equation}
    a_1 = \frac{1}{2}a_3
\end{equation}

Thus our solution set is 

\begin{equation}
    E_2 = \{(\frac{1}{2}a_3, a_2, a_3)\}
\end{equation}

\begin{equation}
    = span(\{(\frac{1}{2}, 0 , 1), (0,1,0)\})
\end{equation}

\subsubsection*{Step 3:}

Recall from last time, 
\begin{center}
    $A$ is diagonolize $\iff \exists$ basis of $\mathbb{R}^3$ consisting of eigenvectors of A.  
\end{center}

\begin{center}
    Try $\beta = \begin{pmatrix}
        1 & 0 & \frac{1}{2} \\
        0 & 1 & 0 \\
        1 & 0 & 1
    \end{pmatrix}$
\end{center}

This is a basis. So 

\begin{equation}
    [L_A]_\beta = \begin{pmatrix}
        1 & 0 & 0 \\
        0 & 2 & 0 \\
        0 & 0 & 2
    \end{pmatrix}
\end{equation}

Note that 

\begin{equation}
    L_A(v_1) = v_1
\end{equation}

\begin{equation}
    L_A(v_2) = 2v_2
\end{equation}

\begin{equation}
    L_A(v_3) = 2v_3
\end{equation}

Note that 

\begin{equation}
    [L_A]_\beta = Q^{-1}AQ
\end{equation}

Such that $Q$ is the change of basis matrix

\begin{equation}
    Q = \begin{pmatrix}
        1 & 0 & \frac{1}{2}\\
        0 & 1 & 0 \\
        1 & 0 & 1
    \end{pmatrix}
\end{equation}

Yes, $A$ is diagonolizable, and 

\begin{equation}
    Q^{-1}AQ = \begin{pmatrix}
        1 & 0 & 0 \\
        0 & 2 & 0 \\
        0 & 0 & 2
    \end{pmatrix}
\end{equation}

\subsection*{Application}

For the same matrix $A$, what is $A^n$? We know that 

\begin{equation}
    Q^{-1}AQ = D
\end{equation}

Thus

\begin{equation}
    A = QDQ^{-1}
\end{equation}

Thus 

\begin{equation}
    A^n = (QDQ^{-1})^n \iff QD^nQ^{-1}
\end{equation}

In general, there are 2 things that can stop us from diagonalizing a matrix:
\begin{enumerate}
    \item Not enough eigenvalues 
    \item Not enough eigenvectors 
\end{enumerate}

\begin{definition}
    If $f(t)$ is a polynomial, with coefficients in a field $F (= \mathbb{R}, \mathbb{C}, \dots)$. Then we say that $f$ splits over the field $F$ if 
    \begin{equation}
        f(t) = c(t - a_1)(t-a_2)\dots(t-a_d)
    \end{equation} 
    Where $d$ is the degree of the polynomial. 
\end{definition}

\end{document}
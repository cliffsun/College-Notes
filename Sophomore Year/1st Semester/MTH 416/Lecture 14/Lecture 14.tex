\documentclass{article}
\usepackage[left=2cm, right=2cm, top=2cm, bottom=2cm]{geometry}
\usepackage{graphicx}
\usepackage{amsmath}
\usepackage{amssymb}
\usepackage{amsthm}
\usepackage{fancyhdr}
\usepackage{verbatim}
\usepackage{listings}
\usepackage{xcolor}
\usepackage{pgfplots}

\lstset{
    language=C++,
    basicstyle=\ttfamily\footnotesize,
    keywordstyle=\color{blue},
    commentstyle=\color{green},
    stringstyle=\color{red},
    numbers=left,
    numberstyle=\tiny\color{gray},
    frame=single,
    breaklines=true,
    captionpos=b,
}


\title{MTH 416: Lecture 14}
\author{Cliff Sun}

\newtheorem{theorem}{Theorem}[section]
\newtheorem{lemma}[theorem]{Lemma}
\newtheorem{definition}[theorem]{Definition}
\newtheorem{conjecture}[theorem]{Conjecture}
\newtheorem{proposition}[theorem]{Proposition}
\newtheorem{corollary}[theorem]{Corollary}
\newtheorem{one minute paper}[theorem]{One Minute Paper}

\pagestyle{fancy}
\lhead{\textbf{\thepage}\ \ \nouppercase{\rightmark}}
\chead{MTH 416: Lecture 14}
\rhead{Cliff Sun}

\begin{document}

\maketitle

\section*{Lecture Span}
\begin{itemize}
    \item Change of Coordinates
    \item $2 \times 2$ determinants
\end{itemize}

\section*{Change of coordinates}

To write a vector in a different coordinate system, we multiply it by the change in coordinates matrix $Q^{-1}$. Where the 
columns represent the new basis in terms of the old one. To write a lin transformation $T: V \rightarrow V$ in a new coordinate system, we use 
\begin{equation}
    [T]_\beta' = Q^{-1}[T]_\beta Q
\end{equation}

\begin{definition}
    Two matrices $A,B \in M_{n \times n}(\mathbb{R})$ are similar if $B = Q^{-1}AQ$ for some invertible matrix $Q \in M_{n\times n}(\mathbb{R})$. They describe the same linear transformation, 
    but in two different coordinate systems. 
\end{definition}

\subsection*{Example}

Let
\begin{center}
    \[A = \begin{pmatrix}
        2 & 0 \\
        0 & 3
    \end{pmatrix}, B = \begin{pmatrix}
        3 & 0 \\
        0 & 2
    \end{pmatrix}\]
\end{center}

Then we claim that $A$ and $B$ are similar. Namely choose

\begin{equation}
    Q = \begin{pmatrix}
        0 & 1 \\
        1 & 0 
    \end{pmatrix}
\end{equation}

Note, $Q = Q^{-1}$, then we have that 

\begin{equation}
    Q^{-1}AQ = \begin{pmatrix}
        0 & 1 \\
        1 & 0
    \end{pmatrix}\begin{pmatrix}
        2 & 0 \\
        0 & 3
    \end{pmatrix}\begin{pmatrix}
        0 & 1 \\
        1 & 0 
    \end{pmatrix}
\end{equation}

Then the result is 
\begin{equation}
    \begin{pmatrix}
        3 & 0 \\
        0 & 2
    \end{pmatrix} \iff B
\end{equation}

\textbf{Fact:} if $A$ and $B$ are similar, then they have the same rank. 

\textbf{Reason:} Rank is an intrinsic property of linear transformations, so it doesn't depend on the linear transformation. 

\subsection*{Application of change of coordinates}

Let 

\begin{equation}
    T: \mathbb{R}^3 \rightarrow \mathbb{R}^3
\end{equation}

Let 

\begin{equation}
    W: \{<x,y,z> \in \mathbb{R}^3 : x + y + z = 0\}
\end{equation}

Basis of $W$:
\begin{equation}
    \{v_1 = \begin{pmatrix}
        1 \\
        -1 \\
        0
    \end{pmatrix}, v_2 = \begin{pmatrix}
        0 \\
        1 \\
        -1
    \end{pmatrix}\}
\end{equation}

Define 

\begin{equation}
    T: \mathbb{R}^3 \rightarrow \mathbb{R}^3
\end{equation}

to be an orthogonal projection onto $W$, that is 

\begin{center}
    $T(v) = $ the closest vector to $v$ in $W$
\end{center}

Then what is $[T]_\beta$ with respect to $\beta = \{e_1,e_2,e_3\}$. We first write $[T]_{\beta'}$ as 
\begin{equation}
    \beta' = \{v_1,v_2,v_3\}
\end{equation}
Where $v_3 = \{1,1,1\}$. So we calculate 
\begin{equation}
    T(v_1) = v_1
\end{equation}
\begin{equation}
    T(v_2) = v_2
\end{equation}
\begin{equation}
    T(v_3) = 0
\end{equation}

So 

\begin{equation}
    [T]_{\beta'} = \begin{pmatrix}
        1 & 0 & 0 \\
        0 & 1 & 0 \\
        0 & 0 & 0 
    \end{pmatrix}
\end{equation}

Then 

\begin{equation}
    [T]_{\beta'} = Q^{-1}[T]_{\beta}Q
\end{equation}

then

\begin{equation}
    Q[T]_{\beta'}Q^{-1} = [T]_\beta
\end{equation}

Where 

\begin{equation}
    Q = \begin{pmatrix}
        1 & 0 & 1 \\
        -1 & 1 & 1 \\
        0 & -1 & 1
    \end{pmatrix}
\end{equation}

Turns out, 

\begin{equation}
    Q^{-1} = \frac{1}{3}\begin{pmatrix}
        2 & -1 & -1 \\
        1 & 1 & -2 \\
        1 & 1 & 1
    \end{pmatrix}
\end{equation}

Thus,

\begin{equation}
    [T]_\beta = \frac{1}{3}\begin{pmatrix}
        2 & -1 & -1 \\
        -1 & 2 & -1 \\
        -1 & -1 & 2
    \end{pmatrix}
\end{equation}

\newpage

\section*{Determinants}

We will construct a function 

\begin{equation}
    \det : M_{n \times n}(\mathbb{R}) \rightarrow \mathbb{R}
\end{equation}

satisfying the following properties:

\begin{enumerate}
    \item $\det(A) \neq 0 \iff A $ is invertible 
    \item $\det(AB) = \det(A)\det(B)$
    \item $\det(A)$ tells how $L_A: \mathbb{R}^n \rightarrow \mathbb{R}^n$ scales volumes. 
    \item $\det(A)$ is \underline{not} a linear transformation (except for $n=1$), but is \underline{multi-linear} (will elaborate on this later on)
\end{enumerate}


\subsection*{Start with n = 1}

\begin{equation}
    \det: M_{1 \times 1}(\mathbb{R}) \rightarrow \mathbb{R}
\end{equation}
\begin{equation}
    \det(a) = a
\end{equation}
Let's check properties:
\begin{enumerate}
    \item The $1 \times 1$ matrix is invertible $\iff$ $a \neq 0$. Then the inverse is $\frac{1}{a}$. 
    \item If $A = (a)$ and $B = (b)$, then $\det(AB) = ab$
    \item If $A = (a)$, then $L_A$ is the function 
    \begin{equation}
        L_A = \mathbb{R} \rightarrow \mathbb{R}
    \end{equation} 
    \begin{equation}
        L_A(x) = ax
    \end{equation}
    Note, $L_A$ scales "volumes" by $|a|$. If the determinant is $< 0$, then $L_A$ reverses orientation of the line. 
\end{enumerate}

\subsection*{n = 2}

\begin{equation}
    \det: M_{2 \times 2} \rightarrow \mathbb{R}
\end{equation}

\begin{equation}
    \det\begin{pmatrix}
        a & b \\
        c & d
    \end{pmatrix} = ad - bc
\end{equation}

\begin{enumerate}
    \item Claim: For any $A \in M_{2 \times 2}$, $\det(A) \neq 0 \iff A$ is invertible. 
    \begin{proof}
        $(\impliedby)$ Suppose $A$ is invertible, then 
        \begin{equation}
            \det(A)\det(A^{-1}) = \det(AA^{-1}) = 1
        \end{equation}
        Thus the determinant of $A$ is non-zero.  \\
        $(\implies)$ Suppose $A$ is non-zero, then we claim that $A$ is invertible. Claim 
        \begin{equation}
            A^{-1} = \frac{1}{\det(A)}\begin{pmatrix}
                d & -b \\
                -c & a
            \end{pmatrix}
        \end{equation}
        Multiplying this out yields the identity matrix. 
    \end{proof}  
    \item Claim: If $A = \begin{pmatrix}
        a & b \\
        c & d
    \end{pmatrix}$, then $L_A: \mathbb{R}^2 \rightarrow \mathbb{R}^2$ scales the volume by $|\det(A)| = |ad - bc|$. Namely 
    \begin{center}
        $ad - bc$ is positive $\iff$ $v,w$ are positively oriented. 
    \end{center}
    In other words, $v$ to $w$ is a counterclockwise rotation less than $180$ degrees. 
    \begin{proof}
        We will first prove the claim in 2 special cases.
        \begin{enumerate}
            \item $v$ is pointing along the positive x axis. 
            Then $v = \begin{pmatrix}
                a \\
                0
            \end{pmatrix}$, then $A = \begin{pmatrix}
                a & b \\
                0 & d
            \end{pmatrix}$ Then area is base times height, which is 
            \begin{equation}
                a|d| \iff |ad| = |ad - bc| = \det(A)
            \end{equation}
            \item $A$ is a rotation matrix. Then rotation matrices don't affect area or orientation. What is $A(\theta)?$
            \begin{equation}
                A(\theta) = \begin{pmatrix}
                    \cos\theta & -\sin\theta \\
                    \sin\theta & \cos\theta 
                \end{pmatrix}
            \end{equation} 
            Then $\det(A) = \cos^2 + \sin^2 = 1$
            \item Now let ${v,w}$ be any basis and $A = \begin{pmatrix}
                v & w
            \end{pmatrix}$ Let $B = {v', w'}$, then $A = CB$ for some rotation matirx $C$. Then 
            \begin{equation}
                \det(L_A) = \det(L_CL_B)
            \end{equation}
            \begin{equation}
                \det(L_B) \iff \det(A) \iff \det(B)
            \end{equation}
        \end{enumerate}
    \end{proof}
\end{enumerate}

\end{document}
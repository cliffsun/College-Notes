\documentclass{article}
\usepackage[left=2cm, right=2cm, top=2cm, bottom=2cm]{geometry}
\usepackage{graphicx}
\usepackage{amsmath}
\usepackage{amssymb}
\usepackage{amsthm}
\usepackage{fancyhdr}
\usepackage{verbatim}
\usepackage{listings}
\usepackage{xcolor}
\usepackage{pgfplots}

\lstset{
    language=C++,
    basicstyle=\ttfamily\footnotesize,
    keywordstyle=\color{blue},
    commentstyle=\color{green},
    stringstyle=\color{red},
    numbers=left,
    numberstyle=\tiny\color{gray},
    frame=single,
    breaklines=true,
    captionpos=b,
}


\title{MTH 416: Lecture 17}
\author{Cliff Sun}

\newtheorem{theorem}{Theorem}[section]
\newtheorem{lemma}[theorem]{Lemma}
\newtheorem{definition}[theorem]{Definition}
\newtheorem{conjecture}[theorem]{Conjecture}
\newtheorem{proposition}[theorem]{Proposition}
\newtheorem{corollary}[theorem]{Corollary}
\newtheorem{one minute paper}[theorem]{One Minute Paper}

\pagestyle{fancy}
\lhead{\textbf{\thepage}\ \ \nouppercase{\rightmark}}
\chead{MTH 416: Lecture 17}
\rhead{Cliff Sun}

\begin{document}

\maketitle

\section*{Lecture Span}
\begin{itemize}
    \item Properties of determinants
    \item Elementary matrices
\end{itemize}

\begin{theorem}
    Let $E$ be an $n \times n$ elementary matrix, given by applying some elementary row operation to the identity matrix. Then for any $A \in M_{n \times n}$, applying the same 
    row operation to A produces $EA$. 
\end{theorem}

\begin{corollary}
    Every elementary matrix $E$ is invertible, and the inverse are also elementary matrices.
\end{corollary}

\begin{proof}
    Recall that every row operation can be undone by another row operation. In particular, if $E$ is an elementary matrix, then the matrix is the product of doing some elementary operation to the identity matrix. 
    Therefore, we can build another elementary matrix $E'$ that corresponds to the inverse row operation. We calculate that 
    \begin{equation}
        E'(EA) = A = E(E'A)
    \end{equation}
    Letting $A = I_n$, we have that 
    \begin{equation}
        E'E = I_n = EE'
    \end{equation}
    This concludes the proof. 
\end{proof}

In general, an inverse matrix encodes the different types of row operations needed in order to bring $A$ down to $I_n$. 

\begin{theorem}
    An $n \times n$ matrix $A$ is invertible $\iff$ it is a product of some number of elementary matrices. 
\end{theorem}

\begin{proof}
    ($\impliedby$) Suppose $A = E_1, \dots, E_n$ where $E_n$ is some invertible matrix. Then we claim that $A^{-1}$ is the reversed order of inverses for this matrix. We calculate 
    \begin{equation}
        E_n^{-1}\dots E_1^{-1}E_1 \dots E_n = I_n
    \end{equation} 
    ($\implies$) Suppose $A$ is invertible, then it row reduces to the identity matrix. Then we construct some $A^{-1}$ composed of elementary matrices multiplied together corresponding to the row operations needed to reduce $A$ down to $I_n$. Then
    \begin{equation}
        E_n\dots E_1 A = I_n
    \end{equation}
    Then 
    \begin{equation}
        A = E_1^{-1}\dots E_n^{-1}I_n
    \end{equation}
    So $A$ is a product of elementary matrices. 
\end{proof}

\begin{theorem}
    For any $A, B \in M_{n \times n}$, then 
    \begin{equation}
        \det(AB) = \det(A)\det(B)
    \end{equation}
\end{theorem}

\begin{proof}
    First assume $A$ is an elementary matrix, then 
    \begin{equation}
        \det(A) = c\cdot \det(I_n)
    \end{equation}
    Where $c \neq 0$, that depends on the type of row operation being performed. We also have 
    \begin{equation}
        \det(AB) = c\cdot \det(B)
    \end{equation}
    Because this same row operation is being performed to $B$. Thus, 
    \begin{equation}
        \det(AB) = c \cdot \det(B) = \det(A)\det(B)
    \end{equation} 
    In general, there are 2 cases:
    \begin{enumerate}
        \item $A$ is an invertible matrix, then $A = E_1\dots E_k$ for some elementary matrices $E_i$. Then 
        \begin{equation}
            \det(AB) = \det(E_1\dots E_k B) = \det(E_1) \cdot(E_2 \dots E_k B) 
        \end{equation}
        We can perform this recursively, 
        \begin{equation}
            = \det(A)\det(B)
        \end{equation}
        \item If $A$ isn't invertible, then $\det(A) = 0$, so we want to prove that $\det(AB) = 0$. In other words, $AB$ is non invertible either. Since $A$ isn't invertible, then $L_A$ is not an isomorphism. Then $L_A$ is
        neither one-to-one or onto. Then $L_{AB}$ is $L_A \cdot L_B$. Then this isn't onto, because $L_B$ is both one-to-one and onto, thus all $\mathbb{R}^n$ vectors are mapped to $\mathbb{R}^n$ vectors. 
        But these same vectors aren't mapped onto $\mathbb{R}^n$ by $L_A$, thus this isn't one-to-one or onto due to this bottleneck. Therefore, $AB$ is not invertible. Thus its determinant is $0$. 
    \end{enumerate}
\end{proof}

\begin{theorem}
    For any $A \in M_{n \times n}$, then 
    \begin{equation}
        \det(A) = \det(A^T)
    \end{equation}
\end{theorem}

Why do we care?: Collectively, we've proved the following statements:
\begin{enumerate}
    \item Cofactor expansion on the r-th row
    \item $\det(A)$ and row-reduction
    \item $\det$ is multilinear as a function of one row
\end{enumerate}

The theorem implies that this all works with rows and columns switched.

\begin{lemma}
    If $A \in M_{m \times n}$, and $B \in M_{n \times p}$, then 
    \begin{equation}
        (AB)^T = B^TA^T
    \end{equation}
    Note, in general, you can't multiply $A^TB^T$. 
\end{lemma}

\begin{proof}
    Note that $(AB)^T$ and $B^TA^T$ are both $p \times m$ matrices. We first see that 
    \begin{equation}
        (AB)^T_{ki} = AB_{ik} \iff \sum_{j=1}^{n}A_{ij}B_{jk}
    \end{equation} 
    But
    \begin{equation}
        (B^TA^T)_{ki} = \sum_{j=1}^{n}(B^T)_{kj}A^{T}_{ji} \iff \sum_{j=1}^{n}B_{jk}A_{ij}
    \end{equation}
    Which is equal to equation $14$. Thus this concludes the proof. 
\end{proof}

We now prove that $\det(A^T) = \det(A)$. 

\begin{proof}
    We first assume that A is an elementary matrix, then $\det(A) = \det(A^T)$ by HW 8 (bruh). In general, there are 2 cases: 
    \begin{enumerate}
        \item A is invertible $\iff$ A is the product of elementary matrices. Then 
        \begin{equation}
            \det(A^T) = {\det(E_{n}^T}^{-1} \dots {E_1^{-1}}^T) \iff \det(E_n)\dots\det(E_1)
        \end{equation}
        \item Suppose $A$ is non invertible, then $\det(A)$ is 0. Then we prove that $A^{T}$ is also not invertible. We prove two cases of this 
        \begin{proof}
            If $A$ is non-invertible, then $rank(A) < n$. This means that $\dim(\text{column space}) = rank(A)$. And taking the transpose of the pivots still comes up with white space, thus $A^T$ is non-invertible.  
        \end{proof} 
        \item We can also prove this with contrapositive, that is suppose $A^T$ is invertible, then $A$ is also invertible. 
    \end{enumerate}
\end{proof}

\begin{theorem}
    For any $A \in M_{n \times n}$, the linear transformation $L_A$ scales volumes by a factor of $|\det(A)|$. And $L_A$ reverses orientation $\iff$ $\det(A) < 0$. 
\end{theorem}

What if $A$ is an elementary matrix? 

\end{document}
\documentclass{article}
\usepackage[left=2cm, right=2cm, top=1cm, bottom=2cm]{geometry}
\usepackage{graphicx}
\usepackage{amsmath}
\usepackage{amssymb}
\usepackage{amsthm}
\usepackage{fancyhdr}
\usepackage{verbatim}
\usepackage{listings}
\usepackage{xcolor}
\usepackage{pgfplots}

\lstset{
    language=C++,
    basicstyle=\ttfamily\footnotesize,
    keywordstyle=\color{blue},
    commentstyle=\color{green},
    stringstyle=\color{red},
    numbers=left,
    numberstyle=\tiny\color{gray},
    frame=single,
    breaklines=true,
    captionpos=b,
}


\title{MTH 416: Lecture 9}
\author{Cliff Sun}

\newtheorem{theorem}{Theorem}[section]
\newtheorem{lemma}[theorem]{Lemma}
\newtheorem{definition}[theorem]{Definition}
\newtheorem{conjecture}[theorem]{Conjecture}
\newtheorem{proposition}[theorem]{Proposition}
\newtheorem{corollary}[theorem]{Corollary}
\newtheorem{one minute paper}[theorem]{One Minute Paper}

\pagestyle{fancy}
\lhead{\textbf{\thepage}\ \ \nouppercase{\rightmark}}
\chead{MTH 416: Lecture 9}
\rhead{Cliff Sun}

\begin{document}

\maketitle

\section*{Lecture Span}
\begin{itemize}
    \item Midterm Announcement
    \item Linear transformations and rank-nullity theorem
\end{itemize}

\section*{Midterm}

\textbf{In class next week thursday, HW 5 is due the week after M1. And midterm will cover up to this lecture}

\section*{Linear transformations}

\subsection*{Recall}

\begin{equation}
    T: V \rightarrow W
\end{equation}

is a linear transformation if 

\begin{enumerate}
    \item $T(x+y) = T(x) + T(y)$
    \item $T(cx) = cT(x)$
\end{enumerate}

for all $x,y \in V, c \in \mathbb{R}$. For any linear transformation $T: V \rightarrow W$, we can define two important subspaces:

\begin{definition}
    Given a linear transformation: $T: V \rightarrow W$:
    \begin{enumerate}
        \item The \underline{range} (or image) of $T$ is the set
        \begin{equation}
            R(T) = \{T(v): v \in V\} \subseteq W
        \end{equation}
        \item The \underline{kernel} or (nullspace) of T is 
        \begin{equation}
            N(T) = \{v \in V: T(v) = 0\} \subseteq V
        \end{equation}
    \end{enumerate}
\end{definition}

\begin{theorem}
    \begin{enumerate}
        \item $R(T)$ is a subspace of $W$
        \item $N(T)$ is a subspace of $V$
    \end{enumerate}
\end{theorem}

\begin{proof}
    \begin{enumerate}
        \item $R(T)$ contains $0_w$ because $0_w = T(0_v)$
        if $w_1$ and $w_2$ are in $R(T)$, then we must prove that $w_1 + w_2 \in R(T)$.
        \begin{proof}
            $w_1 = T(v_1)$ and $w_2 = T(v_2)$, for some $v_1, v_2 \in V$. Then $w_1 + w_2 = T(v_1 + v_2)$ which is in $R(T)$. 
        \end{proof}
        Finally, if $w = T(v) \in R(T)$, and $c \in \mathbb{R}$, then 
        \begin{equation}
            cw = c(T(v)) = T(cv) \in R(T)
        \end{equation}

        \item $N(T)$ contains $0_v$ because $T(0_v) = 0_w$. 
        If $v_1, v_2 \in N(T)$, then 
        \begin{equation}
            T(v_1 + v_2) = T(v_1) + T(v_2) = 0 + 0 = 0
        \end{equation}
        if $v \in N(T)$ and $c \in \mathbb{R}$, then 
        \begin{equation}
            T(cv) = c(T(v)) = c(0) = 0
        \end{equation}
    \end{enumerate}
\end{proof}

Because we know that $R(T)$ and $N(T)$ are subspaces, then we can talk about their dimensions. 

\begin{definition}
    The dimension of $R(T)$ is called the \underline{rank}, and the dimension of $N(T)$ is called the \underline{nullity}. 
\end{definition}

\subsection*{Example 1.}

\begin{equation}
    T: \mathbb{R}^3 \rightarrow \mathbb{R}^4
\end{equation}
\begin{equation}
    T(a,b,c) = (a,b,c,a + b + c)
\end{equation}

Then 

\begin{equation}
    R(T) = \{(a,b,c,a+b+c): a,b,c \in \mathbb{R}\}
\end{equation}

This is a 3 dimensional subspace of $\mathbb{R}^4$. 

\begin{equation}
    N(T) = \{(a,b,c) \in \mathbb{R}^3 : (a,b,c,a+b+c) = 0\}
\end{equation}

So 

\begin{equation}
    N(T) = \{(0,0,0)\}
\end{equation}

0 dimensional subspace in $\mathbb{R}^3$. 

\subsection*{Example 2}

\begin{equation}
    T: \mathbb{R}^3 \rightarrow \mathbb{R}^2
\end{equation}

\begin{equation}
    T(a,b,c) = (a,b)
\end{equation}

\begin{equation}
    R(T) = \{(a,b) : (a,b,c) \in \mathbb{R}^3\} \iff \mathbb{R}^2
\end{equation}

In other words, $R(T)$ is surjective. 

\begin{equation}
    N(T) = \{(a,b,c) \in \mathbb{R}^3: (a,b) = 0\} \iff \{(0,0,c) : c \in \mathbb{R}\}
\end{equation}

This is the z-axis. So rank = 2, and nullity = 1. 

\subsubsection*{Note}

For $T : V \rightarrow W$ linear, 
\begin{enumerate}
    \item $0 \leq rank(T) \leq dim(W)$ and 
    \item $0 \leq nullity(T) \leq dim(V)$
\end{enumerate}

\begin{theorem}
    Suppose $T : V \rightarrow W$, and $\beta = \{u_1, \dots, u_n\}$ is a basis for $V$. Then 
    \begin{enumerate}
        \item $R(T) = span(\{T(v_1), \dots, T(v_n)\})$
        \item $T$ is completely determined by what it "does" to $\{u_1, \dots, u_n\}$
    \end{enumerate}
\end{theorem}

\begin{proof}
    For (2), let $v \in V$ be arbitrary, since $\beta$ is a basis, then $v$ can be uniquely expressed as 
    \begin{equation}
        v = a_1v_1 + \dots + a_nv_n
    \end{equation}
    Then:
    \begin{equation}
        T(v) = T(a_1v_1 + \dots + a_nv_n)
    \end{equation}
    \begin{equation}
        \iff a_1T(v_1) + \dots a_nT(v_n)
    \end{equation}
    This shows that $T(v)$ is completely determined by $T(v_i)$.
\end{proof}

\begin{proof}
    For (1), $R(T)$ is the set of all possible $T(v)$, which according to the above proof, is 
    \begin{equation}
        span(\{T(v_1), \dots, T(v_n)\})
    \end{equation}
\end{proof}

In fact, generally given any $w_1, \dots, w_n \in W$, then there is always exactly one linear transformation
\begin{equation}
    T: V \rightarrow W
\end{equation}
such that $T(v_i) = w_i$ for each $i$. 

\begin{theorem}
    Given a linear transformation $T: V \rightarrow W$, we have that 
    \begin{equation}
        N(T) = \{0\} \iff T \text{ is injective, or 1 to 1 }
    \end{equation}
\end{theorem}

\begin{proof}
    First assume $N(T) = \{0\}$, then we claim that $T$ is injective. Suppose 
    \begin{equation}
        T(v) = T(v')
    \end{equation}
    For $v, v' \in V$, then we prove that $v = v'$. Then
    \begin{equation}
        T(v - v') \iff T(v) - T(v') \iff 0
    \end{equation}
    Thus, $v - v' \in N(T)$, but 
    \begin{equation}
        N(T) = \{0\}
    \end{equation}
    then 
    \begin{equation}
        v = v'
    \end{equation}
\end{proof}

\begin{proof}
    Now, suppose $T$ is injective, that is 
    \begin{equation}
        T(v) = T(v') \implies v = v'
    \end{equation}
    Thus, there is at most one vector such that $T(w) = 0$, namely $w = 0$. So it must be the only one. 
\end{proof}

\section*{Rank nullity theorem}

\begin{theorem}
    \underline{Rank-nullity theorem} states that suppose $T$ is linear transformation from $V \rightarrow W$ where $V$ is finite dimensional. Then 
    \begin{equation}
        \dim(R(T)) + \dim(N(T)) = \dim(V)
    \end{equation} 
\end{theorem}

Nullity = number of dimensions "flattened" out, and Rank = number of dimensions left.

\subsection*{Example 1}

\begin{equation}
    T: \mathbb{R}^{a+b} \rightarrow \mathbb{R}^{a+c}
\end{equation}
for some $a,b,c \geq 0$, then 
\begin{equation}
    T(x_1, \dots x_a, \dots x_{a+b}) = (x_1, \dots, x_a, 0, \dots, 0)
\end{equation}

For this $T$, $R(T) = \{(x_1, \dots, x_a, 0, \dots, 0)\}$, then the dimension of $R(T) = a$. Then 
\begin{equation}
    N(T) = \{(x_1, \dots, x_{a+b}) \in \mathbb{R}^{a+b} : x_1 = \dots = x_a = 0\}
\end{equation} 
This is 
\begin{equation}
    N(T) = \{(0, \dots, 0, x_{a+1}, \dots, x_{a+b})\}
\end{equation}
So 
\begin{equation}
    \dim(N(T)) = nullity = b
\end{equation}
Thus 
\begin{equation}
    \dim(R(T)) + \dim(N(T)) = a + b = \dim(V)
\end{equation}

\begin{proof}
    Let $\dim(V) = n$ and let $\dim(N(T)) = k$. Choose a basis $\{v_1, \dots, v_k\}$ for $N(T)$. By corallary of the replacement theorem:
    $\{v_1, \dots, v_k\}$ can be extended to a basis $v_1, \dots, v_n$ for $V$. Then we claim that $\{T(v_{k+1}), \dots, T(v_n)\} = R(T)$.
    \subsubsection*{Note}
    If this is true, then $\dim(R(T)) = n-k$ so 
    \begin{equation}
        \dim(R(T)) + \dim(N(T)) = n-k + k = n = \dim(V)
    \end{equation}
    In other words, we claim that 
    \begin{enumerate}
        \item $span(T(v_{k+1}), \dots, T(v_n)) = R(T)$
        \item $T(v_{k+1}), \dots, T(v_n)$ are linearly independent. 
    \end{enumerate}
    \begin{proof}
        This is the proof of (1), 
        \begin{equation}
            R(T) = span(T(v_1), \dots, T(v_n))
        \end{equation}
        But 
        \begin{equation}
            T(v_1), \dots, T(v_k) = 0, \dots, 0
        \end{equation}
        Thus 
        \begin{equation}
            R(T) = span(T(v_{k+1}), \dots T(v_n))
        \end{equation}
        \begin{proof}
            This is the proof of linearly independence. For the sake of contradiction, then suppose we have a linear dependency, then 
            \begin{equation}
                a_{k_1}T(v_{k+1}) + \dots + a_nT(v_n) = 0
            \end{equation}
            We claim that 
            \begin{equation}
                a_{k+1}, \dots, a_n = 0
            \end{equation}
            Since $T$ is linear, then we have that 
            \begin{equation}
                0 = T(a_{k+1}v_{k+1}) + \dots + T(a_nv_n) 
            \end{equation}
            \begin{equation}
                \iff T(a_{k+1}v_{k+1} + \dots + a_nv_n)
            \end{equation}
            Then 
            \begin{equation}
                a_{k+1}v_{k+1} + \dots + a_nv_n \in N(T)
            \end{equation}
            Therefore, 
            \begin{equation}
                a_{k+1}v_{k+1} + \dots + a_nv_n = b_1v_1 + \dots + b_kv_k
            \end{equation}
            Then 
            \begin{equation}
                -b_1v_1 - \dots - b_kv_k + a_{k+1}v_{k+1} + \dots + a_nv_n = 0
            \end{equation}
            But because $v_1, \dots, v_n$ are a basis for $V$, then that means that 
            \begin{equation}
                b_1,\dots b_k, a_{k+1}, \dots a_n = 0
            \end{equation}
            In particular, 
            \begin{equation}
                a_{k+1}, \dots, a_n = 0
            \end{equation}
            Thus all vectors in $R(T)$ are linearly independent. 
        \end{proof}
    \end{proof}
\end{proof}

\end{document}
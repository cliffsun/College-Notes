\documentclass{article}
\usepackage[left=2cm, right=2cm, top=2cm, bottom=2cm]{geometry}
\usepackage{graphicx}
\usepackage{amsmath}
\usepackage{amssymb}
\usepackage{amsthm}
\usepackage{fancyhdr}
\usepackage{verbatim}
\usepackage{listings}
\usepackage{xcolor}
\usepackage{pgfplots}

\lstset{
    language=C++,
    basicstyle=\ttfamily\footnotesize,
    keywordstyle=\color{blue},
    commentstyle=\color{green},
    stringstyle=\color{red},
    numbers=left,
    numberstyle=\tiny\color{gray},
    frame=single,
    breaklines=true,
    captionpos=b,
}


\title{MTH 416: Lecture 20}
\author{Cliff Sun}

\newtheorem{theorem}{Theorem}[section]
\newtheorem{lemma}[theorem]{Lemma}
\newtheorem{definition}[theorem]{Definition}
\newtheorem{conjecture}[theorem]{Conjecture}
\newtheorem{proposition}[theorem]{Proposition}
\newtheorem{corollary}[theorem]{Corollary}
\newtheorem{one minute paper}[theorem]{One Minute Paper}

\pagestyle{fancy}
\lhead{\textbf{\thepage}\ \ \nouppercase{\rightmark}}
\chead{MTH 416: Lecture 20}
\rhead{Cliff Sun}

\begin{document}

\maketitle

\section*{Lecture Span}
\begin{itemize}
    \item Criteria for Diagonalizability
\end{itemize}

Recall: 2 things can prevent diagonalizability:
\begin{enumerate}
    \item Not enough eigenvalues
    \item Not enough eigenvectors
\end{enumerate}

\begin{theorem}
    \textbf{Fundamental Theorem of Algebra}: Every non-zero polynomial with complex number coefficients splits completely over the 
    Complex Numbers. In other words, 
    \begin{equation}
        f(t) = c(t-a_1)(t-a_2)\dots
    \end{equation}
    Note, this is not true of $\mathbb{R}$, e.g. $f(t) = t^2 + 1$
\end{theorem}

\begin{theorem}
    Let $A \in M_{n \times n}(F)$ where $F$ is some field, then if $A$ is diagonalizable, then the characteristic polynomial of $A$ splits completely in $F$. 
\end{theorem}

\begin{proof}
    Let $A \in M_{n \times n}$ be diagonalizable, then $A$ is similar to some diagonal matrix $D$. Then 
    \begin{center}
        char poly($A$) = char poly($D$)
    \end{center}
    \begin{equation}
        = \det(D) \iff (\lambda_1 - t)(\lambda_2 - t)\dots
    \end{equation}
    This completely splits $F$. 
\end{proof}

Is this an iff statement? NO

For example, 

\begin{equation}
    A = \begin{pmatrix}
        1 & 1 \\
        0 & 1
    \end{pmatrix}
\end{equation}

Then 

\begin{equation}
    char(A) = (t-1)^2
\end{equation}

In other to diagonalize the matrix, we need a basis of $\mathbb{R}^2$ eigenvectors, which we don't have. 

\begin{definition}
    Suppose $\lambda$ is an eigenvalue of $A$
    \begin{enumerate}
        \item The \underline{algebraic multiplicity} of $\lambda$ is the number of times $(t-\lambda)$ divides the polynomial. 
        \item The \underline{geometric multiplicity} of $\lambda$ is the dimension of the Eigenspace associated with $\lambda$. 
    \end{enumerate}
\end{definition}

Main facts about algebraic \& geometric multiplicity:

\begin{theorem}
    For any $\lambda$,
    \begin{center}
        geo multi $\leq$ alg multi
    \end{center}
\end{theorem}

\begin{theorem}
    A matrix $A \in M_{n \times n}(\mathbb{R})$ is diagonalizable iff the follow are true:
    \begin{enumerate}
        \item Its characteristic polynomial splits completely
        \item For every eigenvalue, the geometric multiplicity and algebraic multiplicity are equal. 
    \end{enumerate} 
\end{theorem}

\begin{corollary}
    If $A \in M_{n \times n}$ has $n$ distinct real eigenvalues, then it is diagonalizable.  
\end{corollary}

Note that the converse is false. 

\begin{proof}
    Let $M \in M_{n \times n}(\mathbb{R})$ and let $\lambda$ be an eigenvalue of $A$. Then, we prove that 
    \begin{center}
        geo multi $\leq$ alg multi
    \end{center}
    If $\lambda = 1$ has geo multi = $k$, then $\dim(E_\lambda) = k$. Then we choose a basis for $E_\lambda$, then we extend this 
    to a basis $\beta = \{v_1, \dots, v_n\}$ for all $\mathbb{R}^n$. Then we write $L_A$ in $\beta$ coordinates. Then we have that the right hand corner of this matrix is a 
    diagonal line of $\lambda$'s. Then 
    \begin{center}
        char poly(A) = char poly($[L]_\beta$)
    \end{center}
    Then char poly($[L]_\beta$) has at least $k$ repititions of $\lambda$, thus 
    \begin{center}
        geo multi $\leq$ alg multi
    \end{center}
\end{proof}

\begin{proof}
    We now prove theorem 0.5, then we first proceed in the $\implies$ direction. We've already proved that char poly($A$) splits completely. We now note that $A \sim D$ where $D$ is some 
    diagonal matrix. Then for any $\lambda$, alg multi $\lambda$ = \# of times $\lambda$ appears on $D$. Claim: the geo multi of $\lambda$ is the same as 
    \begin{equation}
        = \dim(E_\lambda)
    \end{equation}
    \begin{equation}
        = \dim N(A - \lambda I)
    \end{equation}
    \begin{equation}
        \dim N(D - \lambda I)
    \end{equation}
    \begin{equation}
        = \text{ \# of times $\lambda$ appears on diagonal }
    \end{equation}
    \begin{center}
        = alg multi of $\lambda$
    \end{center}
    ($\impliedby$) Suppose that char poly(A) splits and all $\lambda$ have geo multi = alg multi. 
    \begin{equation}
        char poly(A) = (-1)^n(t - \lambda_1)^{m_1}\dots
    \end{equation}
    Where $\lambda_i$ are all distinct and real numbers, and $m_i$ are the algebraic multiplicity of each $\lambda_i$. The degree of this polynomial is 
    \begin{equation}
        m_1 + \dots + m_k = n
    \end{equation}
    But since $m_i = $ geo multi of $\lambda_i$ which is the dimension of $E_{\lambda_i}$. Let 
    \begin{equation}
        \beta_i = \text{ a basis for $E_{\lambda_i}$}
    \end{equation}
    This contains $m_i$ vectors. Let 
    \begin{equation}
        \beta = \beta_1 \cup \dots \cup \beta_k
    \end{equation}
    This is a set of $n$ linearly independent vectors. But how do we know this?? To complete the proof, we need the following lemma
    \begin{lemma}
        If $\beta_i$ is a linearly independent set in $E_{\lambda_i}$ for various distinct eigenvalues $\lambda_1, \dots, \lambda_k$, then 
        \begin{equation}
            \beta = \beta_1 \cup \dots \cup \beta_k
        \end{equation} 
        Is another linearly independent set. 
    \end{lemma}
\end{proof}



\end{document}
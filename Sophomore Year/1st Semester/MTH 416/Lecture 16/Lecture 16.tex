\documentclass{article}
\usepackage[left=2cm, right=2cm, top=2cm, bottom=2cm]{geometry}
\usepackage{graphicx}
\usepackage{amsmath}
\usepackage{amssymb}
\usepackage{amsthm}
\usepackage{fancyhdr}
\usepackage{verbatim}
\usepackage{listings}
\usepackage{xcolor}
\usepackage{pgfplots}

\lstset{
    language=C++,
    basicstyle=\ttfamily\footnotesize,
    keywordstyle=\color{blue},
    commentstyle=\color{green},
    stringstyle=\color{red},
    numbers=left,
    numberstyle=\tiny\color{gray},
    frame=single,
    breaklines=true,
    captionpos=b,
}


\title{MTH 416: Lecture 16}
\author{Cliff Sun}

\newtheorem{theorem}{Theorem}[section]
\newtheorem{lemma}[theorem]{Lemma}
\newtheorem{definition}[theorem]{Definition}
\newtheorem{conjecture}[theorem]{Conjecture}
\newtheorem{proposition}[theorem]{Proposition}
\newtheorem{corollary}[theorem]{Corollary}
\newtheorem{one minute paper}[theorem]{One Minute Paper}

\pagestyle{fancy}
\lhead{\textbf{\thepage}\ \ \nouppercase{\rightmark}}
\chead{MTH 416: Lecture 16}
\rhead{Cliff Sun}

\begin{document}

\maketitle

\section*{Lecture Span}
\begin{itemize}
    \item $n \times n$ determinant
    \item Properties \& calculating with row operations
\end{itemize}

\section*{Recall}

\begin{theorem}
    For any $1 \leq r \leq n$, then 
    \begin{equation}
        \det(A) = \sum_{j=1}^{n}(-1)^{r+j}A_{rj}\det(\tilde{A_{rj}})
    \end{equation}
\end{theorem}

then 

\begin{lemma}
    If $A$ has some row that is $e_j$, then 
    \begin{equation}
        \det(A) = (-1)^{r + j}\det(\tilde{A}_{rj})
    \end{equation}
\end{lemma}

\begin{proof}
    Using the Lemma, let $A \in M_{2 \times 2}(\mathbb{R})$, then define new matrices $B_j$ such that 
    $B_j$ is identical to $A$ except that row $r$ is replaced by $e_j$. Then 
    \begin{equation}
        \det(A) = \sum A_{rj}\det(B_j)
    \end{equation}
    \begin{equation}
        \det(A) = \sum (-1)^{r + j}A_{rj}\det(\tilde{A_{rj}})
    \end{equation}
\end{proof}

To prove the lemma, we induct on $n$. 

\begin{proof}
    Base cases: $n = 1$ and $n = 2$. For $n = 1$, we define the determinant of the $0 \times 0$ matrix to be 1. 

    \subsubsection*{Induction}

    Assume true for some $n$, then let $A$ be some form $(n + 1) \times (n + 1)$. Then we use cofactor expansion on the first row:
    \begin{equation}
        \det(A) = \sum (-1)^{1 + j}A_{1j}\det(\tilde{A_{1j}})
    \end{equation}
    Each $\tilde(A_{1j})$ has two forms:
    \begin{enumerate}
        \item If we cross out column $j \neq k$ (or not where the standard basis appears), then we can use the IH to calculate the determinant. 
        \item If we do cross out column $j = k$, then we can use $\tilde{A_{1j}}$ has a row of all $0's$, then we can use the following theorem:
        \begin{theorem}
            If $A$ has a row of all $0's$, then the determinant is 0. 
        \end{theorem}
        \begin{proof}
            If row $r$ of $A$ is all 0's, then row $r = 0 \cdot $(row $r$), so multilinearity states that 
            \begin{equation}
                \det(A) = 0\cdot \det(A) = 0
            \end{equation} 
        \end{proof}
    \end{enumerate}
\end{proof}

\subsection*{Triangular matrices}

Calculate the determinant of this matrix:

\begin{equation}
    \begin{pmatrix}
        2 & 1 & 4 & 5\\
        0 & 1 & 1 & 6\\
        0 & 0 & 4 & 9\\
        0 & 0 & 0 & 3
    \end{pmatrix}
\end{equation}

Expand this out using the last row:

\begin{equation}
    \det(A) = - 0 + 0 - 0 + 3\det\begin{pmatrix}
        2 & 1 & 4 \\
        0 & 1 & 1 \\
        0 & 0 & 4
    \end{pmatrix}
\end{equation}

\begin{equation}
    \det(A) = 3 \cdot 4 \cdot \det \begin{pmatrix}
        2 & 1 \\
        0 & 1
    \end{pmatrix}
\end{equation}

\begin{equation}
    = 3 \cdot 4 \cdot 1 \cdot 2
\end{equation}

These are the diagonal entries of the matrix. 

\begin{theorem}
    If $A$ is upper or lower triangular, then 
    \begin{equation}
        \det(A) = \text{ product of the diagonal entries }
    \end{equation}
\end{theorem}

\begin{theorem}
    If $A$ has two identical rows, then the determinant of $A = 0$. 
\end{theorem}

\begin{proof}
    Induct on $n$. The base case is $n = 2$ is trivial. 
    \subsubsection*{Induction}
    Suppose the theorem is true for some $n$, then let $A$ be $(n + 1) \times (n + 1)$ with 2 identical rows. Then we calculate the determinant of $A$ using cofactor 
    expansion using any other rows other than the two rows. 
    \begin{equation}
        \det(A) = \sum_{j = 1}^{n + 1}(-1)^{r + j}A_{rj}\det(\tilde{A_{rj}})
    \end{equation}
    But $\tilde{A_{rj}}$ still has two identical rows, which means that we can apply the base case. Thus, by the IH, this determinant is $0$. 
\end{proof}

\begin{theorem} \textbf{Main Theorem}
    Suppose $A$ and $B$ are $M_{2 \times 2}(\mathbb{R})$, and $B$ is the result of applying one elementary row operation to $A$. Then,
    \begin{enumerate}
        \item Switching 2 rows, then $\det B = -\det A$
        \item Scaling a row by factor $c \neq 0$, then $\det B = c\det A$
        \item Adding $c R_{i}$ to  $R_{j}$, then $\det(B) = \det(A)$ 
    \end{enumerate} 
\end{theorem}

\begin{corollary}
    $\det(A) \neq 0 \iff$ A is invertible. 
\end{corollary}

\begin{proof}
    If $A$ is invertible, then it row-reduces to $I_n$. Since $\det(I_n) = 1$, then it follows that $\det(A) \neq 0$. If $A$ is not invertible, then it reduces to some RREF matrix with \# of pivots less than n.
    This has a row of all zeros, so $\det(B) = 0$, and the $\det(A) = 0$.  
\end{proof}

\begin{theorem}
    If $A$ has 2 identical rows, then $\det(A) = 0$.
\end{theorem}

\begin{proof}
    Use multilinearity and above theorem. 
    \subsubsection*{Property 2}
    This is just multilinearity. 
    \subsubsection*{Property 3}
    Let
    \begin{equation}
        A = \begin{pmatrix}
            \dots \\
            R_i \\
            \dots \\
            R_j \\
            \dots 
        \end{pmatrix}
    \end{equation}
    Then 
    \begin{equation}
        B = \begin{pmatrix}
            \dots \\
            R_i \\
            \dots \\
            R_j + cR_i\\
            \dots 
        \end{pmatrix}
    \end{equation}
    Taking this determinant, we notice that 
    \begin{equation}
        \det(B) = \det\begin{pmatrix}
            \dots \\
            R_i \\
            \dots \\
            R_j\\
            \dots
        \end{pmatrix} + c \det\begin{pmatrix}
            \dots \\
            R_i \\
            \dots \\
            R_i\\
            \dots
        \end{pmatrix}
    \end{equation}
    The first matrix is $A$, and the second matrix has 2 identical rows, thus its determinant is $0$. Therefore, 
    \begin{equation}
        \det(B) = \det(A)
    \end{equation}
    \subsection*{Property 1}
    Let 
    \begin{equation}
        A = \begin{pmatrix}
            \dots \\
            R_i \\
            \dots \\
            R_j\\
            \dots
        \end{pmatrix}
    \end{equation}
    Then we switch two rows 
    \begin{equation}
        B = \begin{pmatrix}
            \dots \\
            R_j \\
            \dots \\
            R_i\\
            \dots
        \end{pmatrix}
    \end{equation}
    Trick: consider 
    \begin{equation}
        C = \begin{pmatrix}
            \dots \\
            R_i + R_j \\
            \dots \\
            R_i + R_j\\
            \dots
        \end{pmatrix}
    \end{equation}
    Thus, $\det(C) = 0$. But we can also calculate the determinant of $C$ using multilinearity. 
    \begin{equation}
        \det(C) = \det(\begin{pmatrix}
            \dots \\
            R_i\\
            \dots \\
            R_i + R_j\\
            \dots
        \end{pmatrix} + \begin{pmatrix}
            \dots \\
            R_j\\
            \dots \\
            R_i + R_j\\
            \dots
        \end{pmatrix})
    \end{equation}
    \begin{equation}
        \det(C) = \det(\begin{pmatrix}
            \dots \\
            R_i\\
            \dots \\
            R_j\\
            \dots
        \end{pmatrix})  + \det\begin{pmatrix}
            \dots \\
            R_i\\
            \dots \\
            R_i\\
            \dots
        \end{pmatrix} + \det\begin{pmatrix}
            \dots \\
            R_j\\
            \dots \\
            R_j\\
            \dots
        \end{pmatrix} + \det\begin{pmatrix}
            \dots \\
            R_j\\
            \dots \\
            R_i\\
            \dots
        \end{pmatrix}
    \end{equation}
    Thus 
    \begin{equation}
        \det\begin{pmatrix}
            \dots \\
            R_j\\
            \dots \\
            R_i\\
            \dots
        \end{pmatrix} = -\det(\begin{pmatrix}
            \dots \\
            R_i\\
            \dots \\
            R_j\\
            \dots
        \end{pmatrix})  
    \end{equation}
\end{proof}

\end{document}
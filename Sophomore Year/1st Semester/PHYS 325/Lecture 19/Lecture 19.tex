\documentclass{article}
\usepackage[left=2cm, right=2cm, top=2cm, bottom=2cm]{geometry}
\usepackage{graphicx}
\usepackage{amsmath}
\usepackage{amssymb}
\usepackage{amsthm}
\usepackage{fancyhdr}
\usepackage{verbatim}
\usepackage{listings}
\usepackage{xcolor}
\usepackage{pgfplots}

\lstset{
    language=C++,
    basicstyle=\ttfamily\footnotesize,
    keywordstyle=\color{blue},
    commentstyle=\color{green},
    stringstyle=\color{red},
    numbers=left,
    numberstyle=\tiny\color{gray},
    frame=single,
    breaklines=true,
    captionpos=b,
}


\title{Generic Title}
\author{Cliff Sun}

\newtheorem{theorem}{Theorem}[section]
\newtheorem{lemma}[theorem]{Lemma}
\newtheorem{definition}[theorem]{Definition}
\newtheorem{conjecture}[theorem]{Conjecture}
\newtheorem{proposition}[theorem]{Proposition}
\newtheorem{corollary}[theorem]{Corollary}
\newtheorem{one minute paper}[theorem]{One Minute Paper}

\pagestyle{fancy}
\lhead{\textbf{\thepage}\ \ \nouppercase{\rightmark}}
\chead{Generic Title}
\rhead{Cliff Sun}

\begin{document}

\maketitle

\section*{Delta Distribution}

Denoted as $\delta(t - a)$. Note thta 

\begin{equation}
    f(a) = \int_{-\infty}^{\infty}f(t)\delta(t-a)dt
\end{equation}

For functions continuous at $t = a$. Note that $\delta(\pm\infty) = 0$. Note that 
\begin{center}
    Dirac Delta $\neq$ Kronecker Delta
\end{center}

Example:

\begin{equation}
    \int_\mathbb{R}\sin(t)\delta(t-\frac{3}{2}\pi)dt = \sin(\frac{3}{2}\pi) = -1
\end{equation}

Note: because the delta distribution is defined within the integral, any delta distribution that isn't in its standard form, that is 

\begin{equation}
    \delta(f(t)) \neq \delta(t - a)
\end{equation}

Then performing u substitution is absolutely necessary. 

\section*{Impulse Forces}

Consider force $\Delta t << \frac{1}{\omega_n}$, then 
\begin{equation}
    F(t) = I\delta(t)
\end{equation}
A response to an impulse is called a \underline{Green's Function}. Then a response to an impulse function at $t=0$, we have that 
\begin{equation}
    G(t) = \begin{cases}
        0 & t< 0 \\
        \exp(-\frac{c}{2m}t)\frac{\sin(\omega_d t)}{m\omega_d} & t > 0
    \end{cases}
\end{equation}

This is the response for 

\begin{equation}
    m\ddot{x} + c\dot{x} + kx = \delta(t-a)
\end{equation}

Heavyside ("step") function 

\begin{equation}
    H(y-y_0) = \begin{cases}
        1 & y > y_0 \\
        0 & y < y_0
    \end{cases}
\end{equation}

You can rewrite the Green's function using this Heavyside function.

In general, 

\begin{equation}
    L[G(t-a)] = \delta(t-a)
\end{equation}

\section*{Arbitrary Forcing \& Convolution}

Force with 2 impulses, $I_1(t=t_1)$, $I_2(t=t_2)$. 

\begin{equation}
    F(t) = I_1\delta(t-t_1) + I_2\delta(t-t_2)
\end{equation}

Then 

\begin{equation}
    x_p(t) = I_1G(t - t_1) + I_2G(t-t_2)
\end{equation}

Then a force with $N$ impulses @ $t_q$

\begin{equation}
    F(t) = \sum I_q\delta(t-t_q)
\end{equation}

Now 

\begin{equation}
    x_p(t) = \sum I_qH(t-t_q)G(t-t_q)
\end{equation}

In the limit that $\Delta t \rightarrow 0$, we have that 

\begin{equation}
    F(t) = \int_{-\infty}^{\infty}F(s)\delta(t-s)ds
\end{equation}

Response $x_p(t)$:

\begin{equation}
    x_p(t) = \int_{-\infty}^{\infty}F(s)G(t-s)ds
\end{equation}

This is the general solution for a damped oscillator for an arbitrary driving force!

\end{document}
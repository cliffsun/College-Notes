\documentclass{article}
\usepackage[left=2cm, right=2cm, top=2cm, bottom=2cm]{geometry}
\usepackage{graphicx}
\usepackage{amsmath}
\usepackage{amssymb}
\usepackage{amsthm}
\usepackage{fancyhdr}
\usepackage{verbatim}
\usepackage{listings}
\usepackage{xcolor}
\usepackage{pgfplots}

\lstset{
    language=C++,
    basicstyle=\ttfamily\footnotesize,
    keywordstyle=\color{blue},
    commentstyle=\color{green},
    stringstyle=\color{red},
    numbers=left,
    numberstyle=\tiny\color{gray},
    frame=single,
    breaklines=true,
    captionpos=b,
}


\title{PHYS 325: Lecture 18}
\author{Cliff Sun}

\newtheorem{theorem}{Theorem}[section]
\newtheorem{lemma}[theorem]{Lemma}
\newtheorem{definition}[theorem]{Definition}
\newtheorem{conjecture}[theorem]{Conjecture}
\newtheorem{proposition}[theorem]{Proposition}
\newtheorem{corollary}[theorem]{Corollary}
\newtheorem{one minute paper}[theorem]{One Minute Paper}

\pagestyle{fancy}
\lhead{\textbf{\thepage}\ \ \nouppercase{\rightmark}}
\chead{PHYS 325: Lecture 18}
\rhead{Cliff Sun}

\begin{document}

\maketitle

\section*{Lecture Span}
\begin{itemize}
    \item Greens functions
    \item Delta functions
    \item Fourier Stuff
\end{itemize}

\section*{Fourier Series}

Applied for periodic functions.

\begin{equation}
    F(t) = \frac{a_0}{2} + \sum_{p=1}^{\infty}a_p\cos(p\Omega t) + \sum_{p=1}^{\infty}\sin(p\Omega t)
\end{equation}

Note that 

\begin{equation}
    a_0 = \frac{2}{T}\int_{T}F(t)dt
\end{equation}

\begin{equation}
    a_p = \frac{2}{T}\int_T F(t)\cos(p\Omega t)dt
\end{equation}

and 

\begin{equation}
    b_p = \frac{2}{T}\int_TF(t)\sin(p\Omega t)dt
\end{equation}

Note: $\cos(mx)$ and $\cos(nx)$ are "orthogonal" for $m \neq n$, that is 

\begin{equation}
    <\cos(mx), \cos(nx)>_T = 0
\end{equation}

Same argument for $\sin([m,n]x)$.

\section*{Gibbs Phenomenon}

Overshooting followed by undershooting. This is prevalent in Fourier series fitting. 

\section*{Summary}

EOM:

\begin{equation}
    m\ddot{x} + c\dot{x} + kx = F(t)
\end{equation}

Then for $\Omega = \frac{2\pi}{T}$, then 

\begin{equation}
    F(t) = \frac{a_0}{2} + \sum a_p\cos(p\Omega t) + \sum b_p\sin(p\Omega t)
\end{equation}

The steady state solution $x(t)$ is the solution for $t >> 0$. 

\end{document}
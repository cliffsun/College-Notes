\documentclass{article}
\usepackage[left=2cm, right=2cm, top=1cm, bottom=1cm]{geometry}
\usepackage{graphicx}
\usepackage{amsmath}
\usepackage{amssymb}
\usepackage{amsthm}
\usepackage{fancyhdr}
\usepackage{verbatim}
\usepackage{listings}
\usepackage{xcolor}
\usepackage{pgfplots}

\lstset{
    language=C++,
    basicstyle=\ttfamily\footnotesize,
    keywordstyle=\color{blue},
    commentstyle=\color{green},
    stringstyle=\color{red},
    numbers=left,
    numberstyle=\tiny\color{gray},
    frame=single,
    breaklines=true,
    captionpos=b,
}


\title{PHYS 325: Lecture 7}
\author{Cliff Sun}

\newtheorem{theorem}{Theorem}[section]
\newtheorem{lemma}[theorem]{Lemma}
\newtheorem{definition}[theorem]{Definition}
\newtheorem{conjecture}[theorem]{Conjecture}
\newtheorem{proposition}[theorem]{Proposition}
\newtheorem{corollary}[theorem]{Corollary}
\newtheorem{one minute paper}[theorem]{One Minute Paper}

\pagestyle{fancy}
\lhead{\textbf{\thepage}\ \ \nouppercase{\rightmark}}
\chead{PHYS 325: Lecture 7}
\rhead{Cliff Sun}

\begin{document}

\maketitle

\section*{Lecture Span}
\begin{itemize}
    \item Bead on a whirling rod
\end{itemize}

\section*{Bead on a whirling rod}

\subsection*{Given}
\begin{enumerate}
    \item Rod is spinning with angular velocity $\omega$
    \item Thus $\phi(t) = \omega t$
    \item Solve for $\vec{r}(t)$
\end{enumerate}

\subsection*{Strategy}
\begin{enumerate}
    \item Draw a sketch
    \item Choose coordinates
    \item Write out positions
    \item Velocity and acceleration vectors
    \item Plug into $F = ma$
    \item Solve Diff Eq
\end{enumerate}

\subsection*{Solving}

\begin{equation}
    a(t) = \left[\ddot{r} - r\dot{\phi}^2\right]e_r + \left[r\ddot{\phi} + 2\dot{r}\dot{\phi}\right]
\end{equation}

Plug into $F = ma$

\begin{equation}
    F_n\vec{e_\phi(t)} = m\left[\ddot{r} - r\dot{\phi}^2\right]e_r + m\left[r\ddot{\phi} + 2\dot{r}\dot{\phi}\right]e_\phi
\end{equation}

Thus

\begin{equation}
    e_r : 0 =  m\left[\ddot{r} - r\dot{\phi}^2\right]
\end{equation}
\begin{equation}
    e_\phi: F_n = m\left[r\ddot{\phi} + 2\dot{r}\dot{\phi}\right]
\end{equation}

For $e_r$, 
\begin{equation}
    0 = \ddot{r} - r\ddot{\phi}
\end{equation}
\begin{equation}
    \ddot{r} = r\omega^2
\end{equation}

Let $r = e^{\lambda t}$

\begin{equation}
    \lambda^2 = \omega^2 \iff \lambda = \pm \omega
\end{equation}

\begin{equation}
    r(t) = Ae^{\lambda t} + Be^{-\lambda t}
\end{equation}
Let \begin{equation}
    r(t = 0) = r_0 \land v(t = 0) = v_0
\end{equation}
\begin{equation}
    A = \frac{1}{2}(r_0 + \frac{v_0}{\omega})
\end{equation}
\begin{equation}
    B = \frac{1}{2}(r_0 - \frac{v_0}{\omega})
\end{equation}

\begin{center}
    \begin{tikzpicture}
        \begin{axis}[
            title={$r(t)$},
            xlabel={x},
            ylabel={y}
        ]
            \addplot[color=red, domain=-5:5, samples=100]{cosh(x)};
        \end{axis}
    \end{tikzpicture}
\end{center}

\begin{equation}
    F_n = m(r\ddot{\phi} + 2\dot{r}\dot{\phi})
\end{equation}
\begin{equation}
    \frac{F_n}{2m\omega} = \dot{r}  
\end{equation}
\begin{equation}
    \frac{F_n}{2m\omega^3} = \sinh(\omega t)
\end{equation}

\begin{center}
    \begin{tikzpicture}
        \begin{axis}[
            title={Force in a spinning rod},
            xlabel={x},
            ylabel={y}
        ]
            \addplot[color=red, domain=0:5, samples=100]{sinh(x)};
        \end{axis}
    \end{tikzpicture}
\end{center}

\section*{Fixed Bead on a spinning loop}

Loop of fixed radius $R$, spinning about its vertical axis at rate $\Omega$. Bead of mass $m$, free to move in large loop, in gravitational field.
Remember, our basis vectors are time-dependent, using spherical coordinates. We have that 
\begin{equation}
    e_r = \sin\theta\cos\phi e_x + \sin\theta\sin\phi e_y + \cos\theta e_z
\end{equation}
\begin{equation}
    e_\theta = \cos\theta\cos\phi e_x + \cos\theta\sin\phi e_y - \sin\theta e_z
\end{equation}
\begin{equation}
    e_\phi = \sin\phi e_x + \cos\phi e_y
\end{equation}

\subsection*{Given}
\begin{equation}
    r(t) = R = \text{const}
\end{equation}
\begin{equation}
    \phi(t) = \Omega t
\end{equation}
\begin{equation}
    \vec{r}(t) = r(t)e_r(t)
\end{equation}
\begin{equation}
    v(t) = \dot{r}e_r + r\dot{e_r} \iff r\dot{\theta}e_\theta + r\Omega\sin\theta e_\phi
\end{equation}
\begin{equation}
    a(t) = \text{ hell nah }
\end{equation}
\end{document}
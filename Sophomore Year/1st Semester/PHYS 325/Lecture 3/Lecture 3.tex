\documentclass{article}
\usepackage[left=2cm, right=2cm, top=1cm, bottom=1cm]{geometry}
\usepackage{graphicx}
\usepackage{amsmath}
\usepackage{amssymb}
\usepackage{amsthm}
\usepackage{fancyhdr}
\usepackage{verbatim}
\usepackage{listings}
\usepackage{xcolor}

\lstset{
    language=C++,
    basicstyle=\ttfamily\footnotesize,
    keywordstyle=\color{blue},
    commentstyle=\color{green},
    stringstyle=\color{red},
    numbers=left,
    numberstyle=\tiny\color{gray},
    frame=single,
    breaklines=true,
    captionpos=b,
}


\title{PHYS 325: Lecture 3}
\author{Cliff Sun}

\newtheorem{theorem}{Theorem}[section]
\newtheorem{lemma}[theorem]{Lemma}
\newtheorem{definition}[theorem]{Definition}
\newtheorem{conjecture}[theorem]{Conjecture}
\newtheorem{proposition}[theorem]{Proposition}
\newtheorem{corollary}[theorem]{Corollary}
\newtheorem{one minute paper}[theorem]{One Minute Paper}

\pagestyle{fancy}
\lhead{\textbf{\thepage}\ \ \nouppercase{\rightmark}}
\chead{PHYS 325: Lecture 3}
\rhead{Cliff Sun}

\begin{document}

\maketitle

\section*{Lecture Span}
\begin{itemize}
    \item Equations of Motion
    \item Simple Harmonic Oscillator
\end{itemize}

\section*{Equations of Motion (EOM)}
\textbf{Goal:} Derive EOM and solve for the particle's trajectory $\vec{r}(t)$

\begin{definition}
    An \underline{equation of motion} is a differential equation for $\vec{r}(t)$. An example is 
    \begin{equation}
        \vec{F} = m\vec{a} \iff m\ddot{\vec{r}}(t)
    \end{equation}
    \begin{center}
        or 
    \end{center}
    \begin{equation}
        \vec{F} = \dot{\vec{p}}
    \end{equation}
\end{definition}

\subsection*{Strategy}
\begin{enumerate}
    \item Choose reference frame \& coordinate system
    \item Identify relevant (external) forces, that is not including any forces within the system (molecular forces, etc.)
    \item Determine equation of motion
    \begin{equation}
        \vec{F} = m\Ddot{\vec{r}}
    \end{equation}
    \item Integrate EOM for a given $\vec{F}(\vec{r}, \dot{\vec{r}}, t)$ to find $\vec{r}(t)$
    \item Fix integration constants using initial or boundary conditions
\end{enumerate}

\subsection*{Zero forces}

From Newton's 2nd law (N2L),
\begin{equation}
    0 = \vec{F} = m\vec{a} = m\dot{\vec{v}} \implies \vec{v} = \vec{v_0}
\end{equation}
Thus
\begin{equation}
    \vec{v} = \frac{d\vec{r}}{dt} \implies \vec{r}(t) = \vec{v_0}t + \vec{k_0} 
\end{equation}

\subsection*{Constant Force}

Derive EOM:
\begin{equation}
    \vec{F} = m\vec{a} = c
\end{equation}
\begin{equation}
    \vec{a} = \frac{\vec{F}}{m} = \ddot{\vec{r}}
\end{equation}
\begin{equation}
    \vec{v}(t) = \frac{\vec{F}}{m}t + \vec{v_0}
\end{equation}
\begin{equation}
    \vec{r}(t) = \frac{\vec{F}}{2m}t^2 + \vec{v_0}t + \vec{r_0}
\end{equation}
Where
\begin{equation}
    \vec{a} = \frac{\vec{F}}{m}
\end{equation}

\newpage

\section*{Time dependent force}
\begin{enumerate}
    \item Derive EOM from N2L
    \begin{equation}
        \vec{a} = \frac{\vec{F}(t)}{m}
    \end{equation}
    \item Integrate twice to get $\vec{r}(t)$ with the appropriate constants of integration
\end{enumerate}

\subsection*{Example: Forced Harmonic Oscillator}

\textbf{Setup: } Particle of mass $m$ that moves along the x-axis $-\infty < x < \infty$ and is subject to a force $\vec{F} = F_0\cos(\alpha t)$. Find the equation of motion. The particle starts at $t = 0, x = 0, v = 0$

\begin{equation}
    \vec{a} = \frac{F_0}{m}\cos(\alpha t)
\end{equation}
\begin{equation}
    \vec{v} = \frac{F_0}{\alpha m}\sin(\alpha t)
\end{equation}
\begin{equation}
    \vec{r} = -\frac{F_0}{\alpha^2 m}\cos(\alpha t) + r_0
\end{equation}
but 
\begin{equation}
    \vec{r}(0) = -\frac{F_0}{\alpha^2 m} + r_0 = 0
\end{equation}
Thus
\begin{equation}
    r_0 = \frac{F_0}{\alpha^2 m}
\end{equation}
So 
\begin{equation}
    \vec{r} = \frac{F_0}{\alpha^2 m}(1-\cos(\alpha t))
\end{equation}

\section*{Position Dependent Force}
\begin{equation}
    ma = F(x) = m\dot{v}
\end{equation}
\begin{equation}
    \frac{dv}{dt} = \frac{dv}{dx} \cdot \frac{dx}{dt} \iff v \cdot \frac{dv}{dx}
\end{equation}
\begin{equation}
    mv\frac{dv}{dx} = F(x)
\end{equation}
\begin{equation}
    \frac{1}{2}m(v^2 - v_0^2) = \int_{x_0}^{x}F(x)dx
\end{equation}
If Force is conservative, then it could be written as a potential gradient.
\begin{itemize}
    \item Path independent
    \item $\vec{F} = -\vec{\nabla} U$
    \item Conservative means there musn't be friction, or be time dependent(?)
\end{itemize}

Assuming that Force is conservative, then we call $F = -\frac{dU}{dx}$, then 
\begin{equation}
    \int_{x_0}^{x}-\frac{dU}{dx} = -[U(x) - U(x_0)]
\end{equation}
Then,
\begin{equation}
    E = KE + U(x) = KE_0 + U(x_0)
\end{equation}
Conservation of mechanical energy!!
Then
\begin{equation}
    v(x) = \pm \sqrt{\frac{2}{m}(E - U(x))}
\end{equation}

\end{document}
\documentclass{article}
\usepackage[left=2cm, right=2cm, top=1cm, bottom=1cm]{geometry}
\usepackage{graphicx}
\usepackage{amsmath}
\usepackage{amssymb}
\usepackage{amsthm}
\usepackage{fancyhdr}
\usepackage{verbatim}
\usepackage{listings}
\usepackage{xcolor}
\usepackage{pgfplots}

\lstset{
    language=C++,
    basicstyle=\ttfamily\footnotesize,
    keywordstyle=\color{blue},
    commentstyle=\color{green},
    stringstyle=\color{red},
    numbers=left,
    numberstyle=\tiny\color{gray},
    frame=single,
    breaklines=true,
    captionpos=b,
}


\title{PHYS 325: Lecture 12}
\author{Cliff Sun}

\newtheorem{theorem}{Theorem}[section]
\newtheorem{lemma}[theorem]{Lemma}
\newtheorem{definition}[theorem]{Definition}
\newtheorem{conjecture}[theorem]{Conjecture}
\newtheorem{proposition}[theorem]{Proposition}
\newtheorem{corollary}[theorem]{Corollary}
\newtheorem{one minute paper}[theorem]{One Minute Paper}

\pagestyle{fancy}
\lhead{\textbf{\thepage}\ \ \nouppercase{\rightmark}}
\chead{PHYS 325: Lecture 12}
\rhead{Cliff Sun}

\begin{document}

\maketitle

\section*{Lecture Span}
\begin{itemize}
    \item Two body problem (in Newtonian Gravity)
\end{itemize}

\section*{Two body problem}

Note that there are a total of 6 differential equations. 

\subsection*{Simplifications}
\begin{enumerate}
    \item Work in the center of mass frame. Introduce the total mass $M = m_1 + m_2$ and $\mu = \frac{m_1m_2}{M}$ is reduced mass. And 
    \begin{equation}
        \vec{r} = \vec{r}_2 - \vec{r}_1
    \end{equation}
    Center of mass = 
    \begin{equation}
        \vec{C} = \frac{m_1\vec{r}_1 + m_2\vec{r}_2}{M}
    \end{equation}
    Thus 
    \begin{equation}
        m_1\ddot{r_1} + m_2\ddot{r_2} = -\frac{Gm_1m_2}{|r_2-r_1|^2} (\frac{(\vec{r_1}-\vec{r_2} + \vec{r_2} - \vec{r_1})}{|\vec{r_2} - \vec{r_1}}) \iff 0
    \end{equation}
    \begin{equation}
        \iff (m_1 + m_2)\ddot{\vec{C}} = 0
    \end{equation}
    Thus
    \begin{equation}
        \ddot{\vec{C}} = 0
    \end{equation}
    So the EOM of $\vec{r}$ is 
    \begin{equation}
        \ddot{r} = -\frac{G\mu}{|\vec{r}|^2}\frac{\vec{r}}{|\vec{r}|}
    \end{equation}
    Thus the relative position of $\vec{r}$ is governed by the same EOM as a test mass $\mu$ in the 
    gravitational field of $M$. 
\end{enumerate}

\subsection*{Energy}

\begin{equation}
    E = T + U = \frac{1}{2}m_1\dot{r_1}^2 + \frac{1}{2}m_2\dot{r_2}^2 - \frac{Gm_1m_2}{(r_2-r_1)^2}
\end{equation}

\begin{equation}
    E = \frac{1}{2}\mu\dot{r}^2 - \frac{GM_\mu}{r} + \frac{1}{2}M\dot{C}^2
\end{equation}

\end{document}
\documentclass{article}
\usepackage[left=2cm, right=2cm, top=2cm, bottom=2cm]{geometry}
\usepackage{graphicx}
\usepackage{amsmath}
\usepackage{amssymb}
\usepackage{amsthm}
\usepackage{fancyhdr}
\usepackage{verbatim}
\usepackage{listings}
\usepackage{xcolor}
\usepackage{pgfplots}

\lstset{
    language=C++,
    basicstyle=\ttfamily\footnotesize,
    keywordstyle=\color{blue},
    commentstyle=\color{green},
    stringstyle=\color{red},
    numbers=left,
    numberstyle=\tiny\color{gray},
    frame=single,
    breaklines=true,
    captionpos=b,
}


\title{PHYS 325: Lecture 14}
\author{Cliff Sun}

\newtheorem{theorem}{Theorem}[section]
\newtheorem{lemma}[theorem]{Lemma}
\newtheorem{definition}[theorem]{Definition}
\newtheorem{conjecture}[theorem]{Conjecture}
\newtheorem{proposition}[theorem]{Proposition}
\newtheorem{corollary}[theorem]{Corollary}
\newtheorem{one minute paper}[theorem]{One Minute Paper}

\pagestyle{fancy}
\lhead{\textbf{\thepage}\ \ \nouppercase{\rightmark}}
\chead{PHYS 325: Lecture 14}
\rhead{Cliff Sun}

\begin{document}

\maketitle

\section*{Lecture Span}
\begin{itemize}
    \item Harmonic \& Damped Oscillators
\end{itemize}

\section*{Harmonic \& Damped Oscillators}

\subsection*{Simple harmonic motion}

\begin{equation}
    m_{eff}\ddot{x} + k_{eff}x = 0
\end{equation}

\subsection*{Physical Pendulum}

Given a distance $L$ from the COM, then the set-up would be 

\begin{equation}
    I\vec{\alpha} = \vec{\tau}
\end{equation}

Note:

\begin{equation}
    T = \frac{1}{2}I_p\dot{\theta}^2
\end{equation}

\section*{Damped Harmonic Oscillator}

Oscillations are damped due to friction or some other opposing force. Thus 
\begin{equation}
    \dot{E} \neq 0 
\end{equation}

Thus we have that 

\begin{equation}
    \dot{E} = P_{\text{diss}}
\end{equation}

\begin{equation}
     = F \dot{x} = F \cdot v
\end{equation}

\end{document}
\documentclass{article}
\usepackage[left=2cm, right=2cm, top=1cm, bottom=1cm]{geometry}
\usepackage{graphicx}
\usepackage{amsmath}
\usepackage{amssymb}
\usepackage{amsthm}
\usepackage{fancyhdr}
\usepackage{verbatim}
\usepackage{listings}
\usepackage{xcolor}
\usepackage{pgfplots}

\lstset{
    language=C++,
    basicstyle=\ttfamily\footnotesize,
    keywordstyle=\color{blue},
    commentstyle=\color{green},
    stringstyle=\color{red},
    numbers=left,
    numberstyle=\tiny\color{gray},
    frame=single,
    breaklines=true,
    captionpos=b,
}


\title{PHYS 325: Lecture 5}
\author{Cliff Sun}

\newtheorem{theorem}{Theorem}[section]
\newtheorem{lemma}[theorem]{Lemma}
\newtheorem{definition}[theorem]{Definition}
\newtheorem{conjecture}[theorem]{Conjecture}
\newtheorem{proposition}[theorem]{Proposition}
\newtheorem{corollary}[theorem]{Corollary}
\newtheorem{one minute paper}[theorem]{One Minute Paper}

\pagestyle{fancy}
\lhead{\textbf{\thepage}\ \ \nouppercase{\rightmark}}
\chead{PHYS 325: Lecture 5}
\rhead{Cliff Sun}

\begin{document}

\maketitle

\section*{Lecture Span}
\begin{itemize}
    \item Non-linear Drag
    \item Time varying mass
\end{itemize}

\section*{Non-linear Drag}

\begin{equation}
    F(v) = -mg + cv^2 \iff -mg + \frac{c}{m}mv^2 \land \sigma = \frac{c}{m}
\end{equation}
Thus, this simplifies to
\begin{equation}
    F(v) = m(g - \sigma)v^2 \iff m\dot{v} = m(g - \sigma v^2) \iff \boxed{\dot{v} = g-\sigma v^2}
\end{equation}
\begin{equation}
    \frac{\dot{v}}{g - \sigma v^2} = 1
\end{equation}
\begin{equation}
    \int_{v_0}^{v} \frac{\dot{v}}{g - \sigma v^2} dt = \int_{t_0}^{t} dt
\end{equation}
\begin{equation}
    \frac{-1}{\sqrt{g\sigma}}\tanh^{-1}\left(v\sqrt{\frac{\sigma}{g}}\right) = t
\end{equation}
\begin{equation}
    v(t) = -\sqrt{\frac{g}{\sigma}}\tanh\left(\sqrt{g \sigma} t\right)
\end{equation}
\begin{equation}
    t \rightarrow \infty \implies v(t) \rightarrow -\sqrt{\frac{g}{\sigma}}
\end{equation}
In general, a special case of ODEs would be if
\begin{equation}
    F(v) = f(v)g(t)
\end{equation}
Then you can split the terms and achieve the following:
\begin{equation}
    m\frac{dv}{dt}\frac{1}{f(v)} = g(t)
\end{equation}
Then you're able to integrate both sides. Then what about
\begin{equation}
    F = f(v)h(x)
\end{equation}
Then we say 
\begin{equation}
    m\frac{dv}{dt} = m\frac{dv}{dx}\frac{dx}{dt}\frac{1}{f(v)} \iff m\frac{dv}{dx}v \frac{1}{f(v)} = g(x)
\end{equation}
\begin{equation}
    \int m \frac{v}{f(v)} dv = \int g(x)
\end{equation}

\newpage

\section*{Time Varying Mass, $M(t)$}

Given this rocket with mass = $M(t)$ and velocity = $v(t)$ at time $t$, then say a mass $dm$ is thrown out of the rocket at $t + \Delta t$ at some speed $u$ relative to the rocket. We say that 
$v_f = v_i + dv$ and $M_f = M_i - dm$. Then
\begin{equation}
    P_f = dm(v-u) + M_f v_f \iff dm(v-u) + (M_i + dm)(v_i + dv)
\end{equation}
So expanding this out yields
\begin{equation}
    P_f = dm(-u) + M_iv_i -dmdv + Mdv = M_iv_i
\end{equation}

We can assume that 
\begin{equation}
    -dmdv \sim 0 \text{ due to the 2nd order pertubation }
\end{equation}

So we get 

\begin{equation}
    Mdv = udm
\end{equation}
We know that 
\begin{equation}
    dm = -dM \text{ change of the mass of the ejected propellant = change of the mass of the rocket}
\end{equation}
\begin{equation}
    Mdv = -udM
\end{equation}
\begin{equation}
    dv = -u \frac{dM}{M}
\end{equation}
\begin{equation}
    v = v_0 - u\ln(\frac{M}{M_0})
\end{equation}
Where $u$ is the speed of the propellant, we assume that this is constant. Then assume $M = 0.1M_0$, at some time $t$ with $v_0 = 0$, then we get that 
\begin{equation}
    v_f = -u\ln(0.1) \sim 2.3u
\end{equation}
Which means that about $90\%$ of the rocket must be mass in order to achieve $2.3u$, which isn't alot of velocity!

\subsection*{Time varying mass in gravity}
Now we can't assume that 
\begin{equation}
    P_i = P_f
\end{equation}
But rather
\begin{equation}
    P_f - P_i = dp = -Mgdt
\end{equation}

\begin{equation}
    Mdv + udM = -Mgdt
\end{equation}
\begin{equation}
    \frac{dv}{dt}M +  u \frac{dM}{dt} = -Mg
\end{equation}
\begin{equation}
    \frac{dv}{dt} = -\frac{u}{M}\frac{dM}{dt} - g
\end{equation}
We have to specify $\frac{dM}{dt}$ because it is a degree of freedom, and can vary from rocket to rocket. For our cases, assume that
\begin{equation}
    \frac{dM}{dt} = c \implies \dot{M} = \alpha
\end{equation}
Then 
\begin{equation}
    M = M_0 - \alpha t
\end{equation}
\begin{equation}
    \frac{dv}{dt} = -g - u\frac{\alpha}{M_0 - \alpha t}
\end{equation}
\begin{equation}
    v(t) = v_0 - gt - u\ln\left(\frac{M_0 - \alpha t}{M_0}\right)
\end{equation}

\newpage 

In general,

\begin{equation}
    \vec{F}(\vec{r}, \dot{\vec{r}}, t) = m\vec{a} = m \ddot{\vec{r}}
\end{equation}

In cartesian, we obtain a system of ODEs with respect to $m\ddot{x}$, $m\ddot{y}$, and $m\ddot{z}$. For example, assume 
that a particle lives in the B field, with a magnetic field that goes in the z direction. We have that 
\begin{equation}
    \vec{F} = q\vec{v} \times \vec{B}
\end{equation}
The particle then spirals in a helix shape, with a radius dependent on its velocity. 

\begin{equation}
    = q\left[\vec{v} \times \vec{B}\right]
\end{equation}

We take the determinant of the following matrix:

\[\left[\begin{array}{ccc}
    \vec{e_x} & \vec{e_y} & \vec{e_z} \\
    v_x & v_y & v_z \\
    B_x & B_y & B_z
\end{array}\right]\]

This simplifies down to 

\begin{equation}
    qB(v_y\vec{e_x} - v_x\vec{e_y}) = \vec{F}
\end{equation}

\begin{equation}
    F_x = qBv_y = m\ddot{x}
\end{equation}
\begin{equation}
    F_y = -qBv_x = m\ddot{y}
\end{equation}
\begin{equation}
    F_z = 0 = m\ddot{z} \implies \dot{z} = c \implies z = cx + b
\end{equation}

\begin{equation}
    qB\dot{y} = m\ddot{x}
\end{equation}
\begin{equation}
    -qB\dot{x} = m\ddot{y}
\end{equation}
\begin{equation}
    qB\ddot{y} = m\dddot{x}
\end{equation}
\begin{equation}
    qB(-\frac{qB}{m})\dot{x} = m\dddot{x}
\end{equation}
\begin{equation}
    -(\frac{qB}{m})^2 v_x = \ddot{v_x}
\end{equation}
Then, 
\begin{equation}
    \ddot{v_x} = -\omega^2v_x \land \omega^2 = (\frac{qB}{m})^2
\end{equation}
Then the solution 
\begin{equation}
    v_x = A\sin(\omega t + \phi)
\end{equation}
The velocity is going in a circle! Now let's try using complex numbers!

\begin{equation}
    v_x + iv_y = \eta 
\end{equation}
Then 
\begin{equation}
    Re(\eta) = v_x
\end{equation}
\begin{equation}
    Im(\eta) = v_y
\end{equation}
Then we have 
\begin{equation}
    \ddot{v_x} = -\omega^2 v_x
\end{equation}
\begin{equation}
    \ddot{v_y} = -\omega^2 v_y
\end{equation}
\begin{equation}
    \dot{\eta} = \dot{v_x} + i\dot{v_y}
\end{equation}
\begin{equation}
     = -i\omega(iv_y + v_x)
\end{equation}
\begin{equation}
    \dot{\eta} = -i\omega\eta
\end{equation}
\begin{equation}
    \eta = e^{-i\omega t}
\end{equation}
\end{document}
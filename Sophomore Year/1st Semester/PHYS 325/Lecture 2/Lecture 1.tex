\documentclass{article}
\usepackage[left=2cm, right=2cm, top=1cm, bottom=1cm]{geometry}
\usepackage{graphicx}
\usepackage{amsmath}
\usepackage{amssymb}
\usepackage{amsthm}
\usepackage{fancyhdr}

\title{PHYS 325: Lecture 2}
\author{Cliff Sun}

\newtheorem{theorem}{Theorem}[section]
\newtheorem{lemma}[theorem]{Lemma}
\newtheorem{definition}[theorem]{Definition}
\newtheorem{conjecture}[theorem]{Conjecture}
\newtheorem{proposition}[theorem]{Proposition}
\newtheorem{corollary}[theorem]{Corollary}
\newtheorem{one minute paper}[theorem]{One Minute Paper}

\pagestyle{fancy}
\lhead{\textbf{\thepage}\ \ \nouppercase{\rightmark}}
\chead{PHYS 325: Lecture 2}
\rhead{Cliff Sun}

\begin{document}

\maketitle

\section*{Lecture Span}
\begin{itemize}
    \item Vector Recap
    \item Newton's 3 Laws
    \item 1 Dimensional Particle Dynamics
\end{itemize}

\section*{Vector Recap}
\subsection*{Vector Differentiation}
Given a vector space in $\mathbb{R}^3$ with a set of basis: $(\vec{e_x}, \vec{e_y}, \vec{e_z})$
\begin{equation}
    \vec{r} = x\vec{e_x} + y\vec{e_y} + z\vec{e_z}
\end{equation}
\begin{equation}
    \dot{\vec{r}} = \dot{x}\vec{e_x} + \dot{y}\vec{e_y} + \dot{z}\vec{e_z}
\end{equation}
Which is equivalent to 
\begin{equation}
    \vec{v} = v_x\vec{e_x} + v_y\vec{e_y} + v_z\vec{e_z}
\end{equation}
Similarly
\begin{equation}
    \vec{a} = \frac{d\vec{v}}{dt} \iff \frac{d^2\vec{r}}{dt^2}
\end{equation}

\subsection*{Leibniz/Product Rule}
\begin{equation}
    \frac{d}{dt}(f(t)g(t)) = \dot{f}g + f\dot{g}
\end{equation}

\subsection*{Dot product}

\begin{equation}
    \frac{d}{dt}(\vec{v} \cdot \vec{w}) = \vec{v} \cdot \dot{\vec{w}} + \dot{\vec{v}} \cdot \vec{w}
\end{equation}

\subsection*{Cross Produc}

\begin{equation}
    \frac{d}{dt}(\vec{r} \times \vec{w}) = \vec{r} \times \dot{\vec{w}} + \dot{\vec{r}} \times \vec{w}
\end{equation}

\section*{Newton's 3 Laws}
\begin{enumerate}
    \item 1st law: object in motion stays in motion
    \item 2nd law: $\vec{F} = m\vec{a} \iff m\ddot{\vec{r}}$ (an ordinary differential equation!) 
    \item 3rd law: Equal and opposite reactions
\end{enumerate}

Consider a box of mass $m$ being pushed on by a constant force $F_0$ at $x_0$ with an initial velocity $v_0$. Find the box's displacement. Answer:
\begin{equation}
    x(t) = \frac{F_0}{2m}t^2 + v_0t + x_0
\end{equation}

\subsection*{Linear Momentum}
\begin{equation}
    \vec{p} = m\vec{v}
\end{equation}
\begin{equation}
    \dot{\vec{p}} = m\dot{\vec{v}} \iff m\vec{a} \iff \vec{F} = \dot{\vec{p}}
\end{equation}

\subsection*{Inertial Frames}
One system in 2 reference frames
\begin{enumerate}
    \item Cartesian coordinate with origin at rest. Particle at rest $\vec{a} = 0$
    \item Rotating reference frame, particle in rest has a non-zero acceleration. Newton's Laws don't hold in non-zero acceleration frames.  
\end{enumerate}

\begin{definition}
    \underline{Inertial Frames} is a reference frame in which Newton's Laws can predict its motion. In other words, particles move with a constant velocity if there is no force acting on them. That is
    \begin{equation}
        \ddot{\vec{r}} = 0 \implies \vec{F} = 0
    \end{equation} 
    Inertial Frames are not unique, i.e there exists more than one inertia frames.
\end{definition}

Take inertial frame $S$ with $\vec{r}$, then we can construct another inertial frame $S'$ with another $\vec{r}'$. 
They include
\begin{itemize}
    \item Boosts
    \item Translations
    \item Rotations
\end{itemize}

\textbf{Translations}
\begin{equation}
    \vec{r'} = \vec{r} + \vec{k}
\end{equation}

\textbf{Boost}
\begin{equation}
    \vec{v'} = \vec{v} + \vec{k}
\end{equation}

\textbf{Rotation}
\begin{equation}
    \vec{r'} = A\vec{r}
\end{equation}
\begin{proof}
    Suppose that $\ddot{\vec{r}} = 0$, then $\ddot{\vec{r'}} = A\ddot{\vec{r}} \iff A0 \iff 0$. Because $A$ is not time dependent.  
\end{proof}

\end{document}
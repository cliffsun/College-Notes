\documentclass{article}
\usepackage[left=2cm, right=2cm, top=1cm, bottom=1cm]{geometry}
\usepackage{graphicx}
\usepackage{amsmath}
\usepackage{amssymb}
\usepackage{amsthm}
\usepackage{fancyhdr}
\usepackage{verbatim}
\usepackage{listings}
\usepackage{xcolor}
\usepackage{pgfplots}

\lstset{
    language=C++,
    basicstyle=\ttfamily\footnotesize,
    keywordstyle=\color{blue},
    commentstyle=\color{green},
    stringstyle=\color{red},
    numbers=left,
    numberstyle=\tiny\color{gray},
    frame=single,
    breaklines=true,
    captionpos=b,
}


\title{PHYS 325: Lecture 8}
\author{Cliff Sun}

\newtheorem{theorem}{Theorem}[section]
\newtheorem{lemma}[theorem]{Lemma}
\newtheorem{definition}[theorem]{Definition}
\newtheorem{conjecture}[theorem]{Conjecture}
\newtheorem{proposition}[theorem]{Proposition}
\newtheorem{corollary}[theorem]{Corollary}
\newtheorem{one minute paper}[theorem]{One Minute Paper}

\pagestyle{fancy}
\lhead{\textbf{\thepage}\ \ \nouppercase{\rightmark}}
\chead{PHYS 325: Lecture 8}
\rhead{Cliff Sun}

\begin{document}

\maketitle

\section*{Lecture Span}
\begin{itemize}
    \item Conservative Forces
\end{itemize}

\section*{Conservative Forces}

\begin{definition}
    A force $\vec{F}(r)$ is conservative if it can be written as a gradient of a potential $U(r)$, that is 
    \begin{equation}
        \vec{F}(r) = -\vec{\nabla}U(r)
    \end{equation}
\end{definition}

\subsubsection*{Consequence 1}

Curl of conservative force vanishes, but note, not every force has
\begin{equation}
    \nabla \times F = 0
\end{equation}

In general, every force can be expressed as the following:

\begin{equation}
    \vec{F} = \vec{F}_{\text{cons.}} + \vec{F}_{\text{diss}}
\end{equation}

Thus

\begin{equation}
    \nabla \times \vec{F} = \nabla \times \vec{F}_{\text{cons.}} + \nabla \times \vec{F}_{\text{diss}} \neq 0
\end{equation}

\subsection*{Consequence 2}

Let's consider the work along a path, from $p_1 \rightarrow p_2$. To compute the work along this path:

\begin{equation}
    W = \int_{p_1}^{p_2}\vec{F} \cdot d\vec{r}
\end{equation}

If $\vec{F}$ is conservative, then we get the following relationship

\begin{equation}
    W = -\int_{p_1}^{p_2}\nabla U \cdot d\vec{r} \iff U(p_1) - U(p_2)
\end{equation}

If path is closed, that is $(p_1 \rightarrow p_1)$, then 

\begin{equation}
    W = -(U(p_1) - U(p_1)) = 0
\end{equation}

Conservative forces do NO work along a closed path.

\section*{Energy Conservation}

If force is conservative, that is $\vec{F}(r) = -\nabla U$, then E = KE + PE is constant. We call this constant of the motion. But how do we go about this?
An idea would be to take the time derivative of $E$, and see what happens. 

We are trying to prove:
\begin{equation}
    D_t(E) = D_t(T + U) = 0
\end{equation}

Thus,

\begin{equation}
    D_t T = \frac{1}{2}m|v|^2 = mv\cdot \frac{dv}{dt} \iff m\vec{v} \cdot \vec{a} \iff \vec{v} \cdot \vec{F}
\end{equation}

Note, $\vec{v} \cdot \vec{F}$ can be non-zero.

\begin{equation}
    D_t(U(t,\vec{r}(t))) = \partial_t U + (\partial_x U \partial_t x + \partial_y U \partial_t y + \partial_z U \partial_t z)
\end{equation}

\begin{equation}
    D_t(U(t,\vec{r}(t))) = \partial_t U + \partial_t \vec{r} \cdot \nabla U \iff \partial_t U + -\vec{v} \cdot \vec{F}
\end{equation}

\begin{equation}
    D_t(E) = \vec{v} \cdot \vec{F} + -\vec{v} \cdot \vec{F} + \partial_t U \iff \partial_t U
\end{equation}

If $U$ has no explicit time dependence, then $D_t E = 0$, thus 

\begin{equation}
    E = c \; \text{ for some fixed } c \in \mathbb{R}
\end{equation}

\section*{How to find the potential given the force}

\begin{enumerate}
    \item Indefinitely integrate $\vec{F} = -\nabla U$
    \item Definitely integrate $\vec{F} = -\nabla U \iff \int_{p_1}^{p_2}dU \iff -\int_{p_1}^{p_2}\vec{F} \cdot d\vec{r}$
\end{enumerate}

\subsection*{Example}

\begin{equation}
    \text{Force } = (2xy + 1)\hat{i} + (x^2+2)\hat{j}
\end{equation}

Trick: choose a convenient path $\Gamma$

\begin{equation}
    -U = \int_{\Gamma} \vec{F} \cdot d\vec{r} = \int_{\Gamma} (F_xdx + F_ydy + F_zdz)
\end{equation}

We decompose our arbitrary path into its counterparts:

\begin{equation}
    \int_{0}^{x} F_{x'}(x',0,0)dx' + \int_{0}^{y}F_{y'}(x,y',0)dy' + \int_{0}^{z}F_z(x,y,z')dz'
\end{equation}

\begin{equation}
    = \int_{0}^{x}dx' + \int_{0}^{y}(x^2 + 2)dy' + \int_{0}^{z}0 dz'
\end{equation}

\begin{equation}
    = x + (x^2y+2y)
\end{equation}

\begin{equation}
    U = -x - x^2y - 2y
\end{equation}

\section*{Central Forces}

\begin{definition}
    A \underline{Central Force} that is directed away from a center point. Could be centripetal, etc. We depict this as 
    \begin{equation}
        \vec{F} = f(r,\theta,\phi)\vec{e_r}
    \end{equation}
\end{definition}

\subsection*{Conservation of Angular Momentum}

\begin{definition}
    \underline{Angular Momentum}, or $\vec{L}$, is defined with respect to a refernece point. (this means that this is a pseudovector!) Then we define 
    \begin{equation}
        \vec{L} = \vec{r} \times m\vec{v}
    \end{equation}
    Thus the magnitude of the angular momentum is 
    \begin{equation}
        |\vec{L}| = m|\vec{r}||\vec{v}|\sin\alpha
    \end{equation}
    This vector is orthogonal to the $\vec{r}-\vec{v}$ plane. 
\end{definition}

\subsubsection*{Notes}

\begin{enumerate}
    \item If reference point is at the center of a central force field, then
    \begin{equation}
        \vec{L} = c \; c \in \mathbb{Z}
    \end{equation}
    Or would be constant with respect to motion.
\end{enumerate}

\end{document}
\documentclass{article}
\usepackage[left=2cm, right=2cm, top=2cm, bottom=2cm]{geometry}
\usepackage{graphicx}
\usepackage{amsmath}
\usepackage{amssymb}
\usepackage{amsthm}
\usepackage{fancyhdr}
\usepackage{verbatim}
\usepackage{listings}
\usepackage{xcolor}
\usepackage{pgfplots}

\lstset{
    language=C++,
    basicstyle=\ttfamily\footnotesize,
    keywordstyle=\color{blue},
    commentstyle=\color{green},
    stringstyle=\color{red},
    numbers=left,
    numberstyle=\tiny\color{gray},
    frame=single,
    breaklines=true,
    captionpos=b,
}


\title{PHYS 325: Lecture 23}
\author{Cliff Sun}

\newtheorem{theorem}{Theorem}[section]
\newtheorem{lemma}[theorem]{Lemma}
\newtheorem{definition}[theorem]{Definition}
\newtheorem{conjecture}[theorem]{Conjecture}
\newtheorem{proposition}[theorem]{Proposition}
\newtheorem{corollary}[theorem]{Corollary}
\newtheorem{one minute paper}[theorem]{One Minute Paper}

\pagestyle{fancy}
\lhead{\textbf{\thepage}\ \ \nouppercase{\rightmark}}
\chead{PHYS 325: Lecture 23}
\rhead{Cliff Sun}

\begin{document}

\maketitle

\section*{Dynamics in rotating frames}

\begin{equation}
    F_{\text{true}} = ma_{oa} - F_{\text{Fict}}
\end{equation}

\begin{equation}
    \iff ma_{\text{app}} = F_{\text{true}} + F_{\text{Fict}}
\end{equation}

\begin{equation}
    \iff ma = F_{\text{true}} - ma_{OO'} - m\omega \times (\omega \times r_{o'a}) - 2m\omega \times v_{\text{app}} - m\dot{\omega} \times r_{ra'}
\end{equation}

\begin{equation}
    ma_{oo'} = \text{ elevator force }
\end{equation}

\begin{equation}
    2m\omega \times v_{\text{app}} = \text{ coriolis force }
\end{equation}

\begin{equation}
    m\dot{\omega} \times r_{o'a} = \text{euler force }
\end{equation}

\subsection*{Example: Puck on an icy disk}

\begin{itemize}
    \item Puck launched from perimeter towards $B$
    \item $x(t=0) = 0$, $\dot{x}(t=0) = 0$. $y(t=0) = -R$, $\dot{y}(t=0) = u$. 
    \item Find the trajectory of $r$ in the rotating frame. 
\end{itemize}

\subsubsection*{Calculation}

\begin{enumerate}
    \item EOM 
    \begin{itemize}
        \item $F_{\text{true}} = 0$
        \item $a_{oo'} = 0$ (doesn't move)
        \item $\dot{\omega} \times r_{o'a} = 0$, constant $\omega$
    \end{itemize}
    \item centrifugal force: 
    \begin{equation}
        \omega \times (\omega \times r_{o'a})
    \end{equation}
    \begin{equation}
        \omega = \omega e_z
    \end{equation}
    \begin{equation}
        r_{o'a} = xe_{x} + ye_y
    \end{equation}
    \begin{equation}
        = -\omega^2(xe_x + ye_y)
    \end{equation}
    \item Coriolis force 
    \begin{equation}
        \omega \times v_{app} = \omega \times (\dot{x}e_{x} + \dot{y}e_y)
    \end{equation}
    \begin{equation}
        = \omega\dot{x}e_y - \omega\dot{y}e_x
    \end{equation}
    \item EOM is 
    \begin{equation}
        m(\ddot{x}e_x + \ddot{y}e_y) = ma_{app} = m\omega^2(xe_x + ye_y) - 2m\omega(\dot{x}e_y + \dot{y}e_x)
    \end{equation}
    Then 
    \begin{equation}
        e_x: \ddot{x} = \omega^2 x + 2\omega \dot{y}
    \end{equation}
    \begin{equation}
        e_y: \ddot{y} = \omega^2y - 2\omega \dot{x}
    \end{equation}
    Trick: use complex function $\Psi = x(t) + iy(t) \in \mathbb{C}$. Inserting it into the equation yields
    \begin{equation}
        \ddot{\Psi} + 2i\omega \dot{\Psi} - \omega^2 \Psi = 0
    \end{equation}
    Let ansatz: $\Psi = e^{\lambda t}$, then 
    \begin{equation}
        \lambda^2 + 2i\omega\lambda - \omega^2 = 0
    \end{equation}
    then 
    \begin{equation}
        \lambda = -\frac{2i \omega}{2} \pm \sqrt{-\omega^2 + \omega^2} = -i\omega
    \end{equation}
    Thus 
    \begin{equation}
        \Psi(t) = Ae^{-\omega t} + Bte^{-i\omega t}
    \end{equation}
    Using the initial conditions 
    \begin{equation}
        \Psi(t=0) = x(t=0) + iy(t=0) = -iR = A
    \end{equation}
    \begin{equation}
        \dot{\Psi}(t=0) = \dot{x}(t=0) + i\dot{y}(t=0) = iu = -i\omega A + B
    \end{equation}
    \begin{equation}
        B = iu + R\omega 
    \end{equation}
    \begin{equation}
        \Psi(t) = -iRe^{-i\omega t} + (iu + R\omega)te^{-i\omega t}
    \end{equation}
    Then 
    \begin{equation}
        x(t) = Re(\Psi(t))
    \end{equation}
    \begin{equation}
        y(t) = Im(\Psi(t))
    \end{equation}
\end{enumerate}

\section*{Motion on rotating earth}

We define $O$ to be in the middle of Earth, and $O'$ to be on the surface of Earth. Then we also use spherical coordinates. Then 
\begin{equation}
    R_{e} = 6400 km
\end{equation}
\begin{equation}
    \omega = \frac{2\pi}{\text{day}}
\end{equation}
\begin{equation}
    v_{lab} = 1600 km/h
\end{equation}
\begin{equation}
    \dot{\omega} = 17m/s \text{ per century } \approx 0
\end{equation}

\begin{equation}
    a_{oo'} = \omega \times (\omega \times R)
\end{equation}
\begin{equation}
    = \omega^2 R \cos\theta 
\end{equation}
For $\theta = 0$, we have that 
\begin{equation}
    \omega^2 R \approx 0.034 << |g|
\end{equation}
\begin{equation}
    \omega \times (\omega \times r_{o'a}) \approx 0
\end{equation}
Then 
\begin{equation}
    ma_{\text{app}} = F_{\text{true}} - 2m\omega \times v_{\text{app}}
\end{equation}

\section*{Coriolis Force on Earth}

We define 
\begin{enumerate}
    \item $e_x = e = $ east
    \item $e_y = n = $ north 
    \item $e_z = u = $ up 
    \item $\theta = $ latitude
    \item $p = ncos\theta + u\sin\theta$
    \item $\omega = Tp$
\end{enumerate}

Then 

\begin{equation}
    m\ddot{r} = F_{\text{true}} - 2m\omega \times r
\end{equation}b

\begin{equation}
    F_{\text{true}} = -mgu 
\end{equation}

\begin{equation}
    = -mgu - 2m|\omega|p \times (-nsin\theta + ucos\theta)
\end{equation}

\end{document}
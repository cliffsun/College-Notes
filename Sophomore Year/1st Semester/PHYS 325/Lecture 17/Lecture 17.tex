\documentclass{article}
\usepackage[left=2cm, right=2cm, top=2cm, bottom=2cm]{geometry}
\usepackage{graphicx}
\usepackage{amsmath}
\usepackage{amssymb}
\usepackage{amsthm}
\usepackage{fancyhdr}
\usepackage{verbatim}
\usepackage{listings}
\usepackage{xcolor}
\usepackage{pgfplots}

\lstset{
    language=C++,
    basicstyle=\ttfamily\footnotesize,
    keywordstyle=\color{blue},
    commentstyle=\color{green},
    stringstyle=\color{red},
    numbers=left,
    numberstyle=\tiny\color{gray},
    frame=single,
    breaklines=true,
    captionpos=b,
}


\title{PHYS 325: Lecture 17}
\author{Cliff Sun}

\newtheorem{theorem}{Theorem}[section]
\newtheorem{lemma}[theorem]{Lemma}
\newtheorem{definition}[theorem]{Definition}
\newtheorem{conjecture}[theorem]{Conjecture}
\newtheorem{proposition}[theorem]{Proposition}
\newtheorem{corollary}[theorem]{Corollary}
\newtheorem{one minute paper}[theorem]{One Minute Paper}

\pagestyle{fancy}
\lhead{\textbf{\thepage}\ \ \nouppercase{\rightmark}}
\chead{PHYS 325: Lecture 17}
\rhead{Cliff Sun}

\begin{document}

\maketitle

\section*{Harmonic Force}

EOM:
\begin{equation}
    m\ddot{x} + c\dot{x} + kx = F_0\cos(\omega t - \theta)
\end{equation}

Solution:

\begin{equation}
    x(t) = x_h(t) + x_p(t)
\end{equation}

\begin{equation}
    x_p(t) = F_0G(\omega)\cos(\omega t - \theta - \phi(\omega))
\end{equation}

Note that 

\begin{equation}
    G(\omega) = \frac{1}{k}\left[(1-(\frac{\omega}{\omega_n}^2)^2 + (2\zeta\frac{\omega}{\omega_n})^2)\right]^{-\frac{1}{2}}
\end{equation}

Phase is 

\begin{equation}
    \phi(\omega) = \arctan(\frac{2\zeta\omega\omega_n}{\omega_n^2 - \omega^2})
\end{equation}

\section*{Analyze $x_p(t)$}

$x_p(t)$ has a resonance frequency, such that the amplitude of the particular solution is maximized. Thus, 

\begin{equation}
    D_w (G(\omega_0)) = 0
\end{equation}

Thus, 

\begin{equation}
    w_0 = \pm\omega_n \sqrt{1 - 2\zeta^2}
\end{equation}

\section*{Periodic Force \& Fourier Series}

Periodic Force:

\begin{equation}
    F(t + T) = F(t)
\end{equation}

with period $T = \frac{2\pi}{\Omega}$. We can describe the solution as a sum of harmonic functions. 

\section*{Fourier Series}

\begin{equation}
    f(t) = \frac{a_0}{2} + \sum_{n=1}^{\infty}a_n\cos(n\Omega t) = \sum_{p = 1}^{\infty}b_p\sin(p\Omega t)
\end{equation}

Note
\begin{itemize}
    \item $n, p \in \mathbb{N}$
\end{itemize}

with 

\begin{equation}
    a_0 = \frac{2}{T}\int_{0}^{T}F(t)dt
\end{equation}

\begin{equation}
    a_n = \frac{2}{T}\int_{0}^{T}F(t)\cos(n\Omega t)dt
\end{equation}

\begin{equation}
    b_n = \frac{2}{T}\int_{0}^{T}F(t)\sin(n\Omega t)dt
\end{equation}

\end{document}
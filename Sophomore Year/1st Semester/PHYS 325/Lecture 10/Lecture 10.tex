\documentclass{article}
\usepackage[left=2cm, right=2cm, top=1cm, bottom=1cm]{geometry}
\usepackage{graphicx}
\usepackage{amsmath}
\usepackage{amssymb}
\usepackage{amsthm}
\usepackage{fancyhdr}
\usepackage{verbatim}
\usepackage{listings}
\usepackage{xcolor}
\usepackage{pgfplots}

\lstset{
    language=C++,
    basicstyle=\ttfamily\footnotesize,
    keywordstyle=\color{blue},
    commentstyle=\color{green},
    stringstyle=\color{red},
    numbers=left,
    numberstyle=\tiny\color{gray},
    frame=single,
    breaklines=true,
    captionpos=b,
}


\title{PHYS 325: Lecture 10}
\author{Cliff Sun}

\newtheorem{theorem}{Theorem}[section]
\newtheorem{lemma}[theorem]{Lemma}
\newtheorem{definition}[theorem]{Definition}
\newtheorem{conjecture}[theorem]{Conjecture}
\newtheorem{proposition}[theorem]{Proposition}
\newtheorem{corollary}[theorem]{Corollary}
\newtheorem{one minute paper}[theorem]{One Minute Paper}

\pagestyle{fancy}
\lhead{\textbf{\thepage}\ \ \nouppercase{\rightmark}}
\chead{PHYS 325: Lecture 10}
\rhead{Cliff Sun}

\begin{document}

\maketitle

\section*{Lecture Span}
\begin{itemize}
    \item Midterm
    \item Gravitation
\end{itemize}

\section*{Midterm}

Will cover up to the gravitational forces. On \textbf{\underline{OCTOBER 10th}} or in 2 weeks! There will be a practice midterm, \& we will be allowed
our own formula sheet. 

\section*{Gravitation}

Given some mass with density $\rho(\vec{r})$, we have that

\begin{equation}
    \vec{F} = -Gm_2 \int \frac{\rho(r_n)}{|r_2-r_1|^3}(r_2-r_1)dr_1
\end{equation}

\subsubsection*{Gravitational field} 

\begin{equation}
    \vec{g} = \frac{\vec{F}}{m_2}
\end{equation}

\subsubsection*{Gravitational potential $\Phi$}

\begin{equation}
    \vec{g} = -\nabla\Phi
\end{equation}

In general, for a point mass $m_1$, we have that the \underline{potential energy} is 
\begin{equation}
    u = \frac{-Gm_1m_2}{r}
\end{equation}
and the \underline{potential} is 
\begin{equation}
    \Phi = \frac{-Gm_1}{r}
\end{equation}

For an \underline{extended mass}, we have that 

\begin{equation}
    \Phi = -G\frac{\int \rho(r_1)}{|r_2-r_1|}d^3r_1
\end{equation}

\subsection*{Example 1: Finding $\Phi$ of a spherical shell}

\subsubsection*{Set-up}
\begin{enumerate}
    \item Uniform Spherical shell
    \item Mass $m$, radius $r$, thickness $h$, $\rho = \frac{M}{4\pi R^2h}$ (Assume thickness is really small)
\end{enumerate}

Let $s(\theta)$ be the line pointing from a point on the shell to the point mass $m_2$. 

\subsubsection*{Strategy}

\begin{equation}
    \Phi = \int d\Phi \implies d\Phi - \frac{-G}{|r-r_1|}dm
\end{equation}

\begin{enumerate}
    \item Mass element of a shell
    \begin{equation}
    dm(r_1) = \rho(r_1)d^3r_1 = \rho r^2dr \sin\theta d\theta d\phi
    \end{equation}
    \item $d\Phi = -\frac{Gdm}{|r-r_1|}$ 
    \begin{equation}
        \iff \frac{-G}{s(\theta)}dm
    \end{equation}
    \item Now we integrate!
    \begin{equation}
        \Phi = \int d\Phi = -G\int \frac{dm}{s(\theta)}
    \end{equation}
    \begin{equation}
        -G\rho h R^2 \int_{0}^{2\pi}\int_{0}^{\pi}\frac{\sin\theta}{s(\theta)}d\theta d\phi
    \end{equation}
    \begin{equation}
        \iff -2\pi G\rho hR^2 \int_{0}^{\pi}\frac{\sin\theta}{s(\theta)}d\theta
    \end{equation}
    We find $s(\theta)$
    \begin{equation}
        s^2(\theta) = |r-r_1|^2 \iff r^2 - 2r \cdot r_1 + r_1^2
    \end{equation}
    \begin{equation}
        x^2 + R^2 - 2Rx\cos\theta 
    \end{equation}
    We note that 
    \begin{equation}
        \frac{d s^2}{d\theta} \iff 2s \frac{ds}{d\theta} \iff 2s (Rx\sin\theta)
    \end{equation}
    \begin{equation}
        \frac{\sin\theta}{s}d\theta = \frac{1}{Rx}ds
    \end{equation}
    \begin{equation}
        \int_{0}^{\pi}\frac{\sin\theta}{s}d\theta = \int_{s_{min}}^{s_{max}}\frac{1}{Rx}ds
    \end{equation}
    Consider 2 sections, $s_{min} = \pm(R-x) > 0$ and $s_{max} = +(R+x) > 0$
    \begin{equation}
        \Phi = -2\pi G \frac{\rho h R}{x}(s_{max} - s_{min})
    \end{equation}
    \subsubsection*{Case 1}
    $m_2$ is outside of the shell. That is $x > R$, then $s_{min} = x - R > 0$ and $s_{max} = (R + x)$ Then 
    \begin{equation}
        \Phi = 4\pi G \frac{\rho h R^2}{x} \iff \frac{-Gm}{x}
    \end{equation}

    \subsubsection*{Case 2:}
    Then $s_{max} = R - x > 0$ and $s_{min} = R + x$, then 
    \begin{equation}
        -4\pi G\rho hR = -\frac{GM}{R} = \text{ constant! }
    \end{equation}
    \subsubsection*{Results}
    For a mass that's inside a shell, the potential is constant. But for a mass outside the shell, the potential is the well known $-\frac{GM}{x}$
\end{enumerate}

\subsection*{Example 2: Finding $\Phi$ of a uniform sphere}

\subsubsection*{Set-up}
\begin{enumerate}
    \item Solid sphere with radius $R_1$
    \item Mass $M_1$, constant density inside the sphere, and 0 everywhere else. 
\end{enumerate}

\subsubsection*{Goal}
Find $\Phi = \Phi(x)$

\subsubsection*{Work}

Let's use previous results, that is relabel $R = r$ and $h = dr$

\begin{equation}
    d\Phi = \begin{cases}
        -4\pi G\frac{\rho r^2}{x}dr : x > r \\
        -4\pi G\rho r dr : x < r 
    \end{cases}
\end{equation}

\subsubsection*{Case 1 : $x > R$}

\begin{equation}
    \Phi = \int -4\pi G \int_{0}^{R} \frac{\rho r^2}{x}dr = \frac{-GM}{r}
\end{equation}

\subsubsection*{Case 2: $x < R$}

\begin{equation}
    \Phi = \int_{0}^{x}d\Phi_{inner} + \int_{x}^{R}d\Phi_{outer}
\end{equation}

\begin{equation}
    \frac{1}{2}GM \frac{3R^2 - x^2}{R^3}
\end{equation}

\end{document}
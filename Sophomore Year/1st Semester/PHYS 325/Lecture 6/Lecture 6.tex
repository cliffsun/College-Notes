\documentclass{article}
\usepackage[left=2cm, right=2cm, top=1cm, bottom=1cm]{geometry}
\usepackage{graphicx}
\usepackage{amsmath}
\usepackage{amssymb}
\usepackage{amsthm}
\usepackage{fancyhdr}
\usepackage{verbatim}
\usepackage{listings}
\usepackage{xcolor}
\usepackage{pgfplots}

\lstset{
    language=C++,
    basicstyle=\ttfamily\footnotesize,
    keywordstyle=\color{blue},
    commentstyle=\color{green},
    stringstyle=\color{red},
    numbers=left,
    numberstyle=\tiny\color{gray},
    frame=single,
    breaklines=true,
    captionpos=b,
}


\title{PHYS 325: Lecture 6}
\author{Cliff Sun}

\newtheorem{theorem}{Theorem}[section]
\newtheorem{lemma}[theorem]{Lemma}
\newtheorem{definition}[theorem]{Definition}
\newtheorem{conjecture}[theorem]{Conjecture}
\newtheorem{proposition}[theorem]{Proposition}
\newtheorem{corollary}[theorem]{Corollary}
\newtheorem{one minute paper}[theorem]{One Minute Paper}

\pagestyle{fancy}
\lhead{\textbf{\thepage}\ \ \nouppercase{\rightmark}}
\chead{PHYS 325: Lecture 6}
\rhead{Cliff Sun}

\begin{document}

\maketitle

\section*{Lecture Span}
\begin{itemize}
    \item Previous lecture
    \item Curvilinear Coordinates (non-Cartesian Coordinates)
\end{itemize}

\section*{Notes}
\begin{enumerate}
    \item Midterm 1 is in \textbf{1 month}, on October 10th, 12:00-13:30
    \item Midterm 1 covers Lecture 1 to Lecture 13
\end{enumerate}

\section*{Previous lecture}

Building from previous lecture, we considered a charged particle in a homogenous magnetic field. We found that 
\begin{equation}
    \dot{v_x} = \omega v_Y
\end{equation}
\begin{equation}
    \dot{v_y} = -\omega v_x
\end{equation}
Such that 
\begin{equation}
    \omega = \frac{qB}{m}
\end{equation}
We would also "complexify" the velocity, where we introduce a new complex variable
\begin{equation}
    \eta = v_x + iv_y
\end{equation}
Thus
\begin{equation}
    \dot{\eta} = \dot{v_x} + i\dot{v_y}
\end{equation}
Thus we insert equations 1 \& 2
\begin{equation}
    \dot{\eta} = (\omega v_y - i\omega v_x) \iff -i\omega (v_x  + iv_y) \iff i\omega \eta
\end{equation}
Then we solve this using separation of variables:
\begin{equation}
    \frac{1}{\eta}d\eta = -i\omega dt
\end{equation}
\begin{equation}
    ln(\eta) = -i\omega t + C
\end{equation}
\begin{equation}
    \eta = Ae^{-i \omega t}
\end{equation}
Such that 
\begin{equation}
    A = Ce^{-i\delta}
\end{equation}
Where our final solution condenses into 
\begin{equation}
    \eta = Ce^{-i(\omega t + \delta)}
\end{equation}
Where $\delta$ is phase. We can show that the amplitude of the velocity stays constant:
\begin{equation}
    A = \sqrt{v_x^2 + v_y^2}
\end{equation}
By taking the time derivative, that is 
\begin{equation}
    \frac{d}{dt}T = \frac{1}{2}m\frac{d}{dt}(v^2) = 0 \implies v^2 = const
\end{equation}
We take the time derivative:
\begin{equation}
    \frac{1}{2}m 2v \cdot \dot{v} \implies v \cdot (F) \implies \vec{v} \cdot (q(v \times B)) = 0
\end{equation}
Thus, the amplitude of the velocity is constant.

\subsection*{Trajectory}

We can now get the trajectory from the velocity through 
\begin{equation}
    x(t) = \int v_x dt = \int A \cos(\omega t - \delta) = \frac{A}{\omega}\sin(\omega t - \delta) + C_x
\end{equation}
\begin{equation}
    y(t) = \int v_y dt = \int A -\sin(\omega t - \delta) = \frac{A}{\omega}\cos(\omega t - \delta) + C_y
\end{equation}
\begin{equation}
    z(t) = \int v_z dt = v_{z,0}t + z_0
\end{equation}
Such that
\begin{equation}
    A = \sqrt{v_x^2 + v_y^2}
\end{equation}

\section*{Curvilinear Coordinates}

\subsection*{2-D Polar Coordinates $(r, \phi)$}

\begin{equation}
    x = r\cos(\phi(t))
\end{equation}
\begin{equation}
    y = r\sin(\phi(t))
\end{equation}
\begin{equation}
    r = x^2 + y^2
\end{equation}
\begin{equation}
    \phi(t) = \arctan(\frac{x}{y})
\end{equation}

\subsection*{3D Cylindrical Coordinates  $(r, \phi, z)$}

\begin{equation}
    x = r\cos(\phi(t))
\end{equation}
\begin{equation}
    y = r\sin(\phi(t))
\end{equation}
\begin{equation}
    z = z
\end{equation}

\subsection*{3D Spherical Coordinates $(r, \theta, \phi)$}

\begin{equation}
    \theta \in (\frac{\pi}{2}, -\frac{\pi}{2})
\end{equation}
\begin{equation}
    \phi \in (0, 2\pi]
\end{equation}
\begin{equation}
    x = r\cos\phi \sin\theta
\end{equation}
\begin{equation}
    y = r\sin\phi\sin\theta
\end{equation}
\begin{equation}
    z = r\cos\theta
\end{equation}

\subsection*{Bead on whirling stick}

\subsubsection*{Set-up}
\begin{itemize}
    \item A rigid stick whirling with frequency $\omega$
    \item a Bead sliding along stick with no friction
\end{itemize}

We choose 2D Polar Coordinates to do this problem:

\end{document}
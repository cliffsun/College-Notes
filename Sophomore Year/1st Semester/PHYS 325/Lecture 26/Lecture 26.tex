\documentclass{article}
\usepackage[left=2cm, right=2cm, top=2cm, bottom=2cm]{geometry}
\usepackage{graphicx}
\usepackage{amsmath}
\usepackage{amssymb}
\usepackage{amsthm}
\usepackage{fancyhdr}
\usepackage{verbatim}
\usepackage{listings}
\usepackage{xcolor}
\usepackage{pgfplots}

\lstset{
    language=C++,
    basicstyle=\ttfamily\footnotesize,
    keywordstyle=\color{blue},
    commentstyle=\color{green},
    stringstyle=\color{red},
    numbers=left,
    numberstyle=\tiny\color{gray},
    frame=single,
    breaklines=true,
    captionpos=b,
}


\title{PHYS 325: Lecture 26}
\author{Cliff Sun}

\newtheorem{theorem}{Theorem}[section]
\newtheorem{lemma}[theorem]{Lemma}
\newtheorem{definition}[theorem]{Definition}
\newtheorem{conjecture}[theorem]{Conjecture}
\newtheorem{proposition}[theorem]{Proposition}
\newtheorem{corollary}[theorem]{Corollary}
\newtheorem{one minute paper}[theorem]{One Minute Paper}

\pagestyle{fancy}
\lhead{\textbf{\thepage}\ \ \nouppercase{\rightmark}}
\chead{PHYS 325: Lecture 26}
\rhead{Cliff Sun}

\begin{document}

\maketitle

\begin{definition}
    A \underline{Phase space} is a space of all possible behaviors for the system. 
\end{definition}

Recap: a functional is 

\begin{equation}
    S[\vec{q}_i(t)] = \int_{t_1}^{t_2}dt L(q, \dot{q}; t)
\end{equation}

We note some things:

\begin{enumerate}
    \item If we didn't specify 2 fixed end points, maybe $q(t_1)$ but not $q(t_2)$, then 
    \begin{equation}
    \frac{\partial L}{\partial \dot{q}} \delta q \bigg|_{t_1}^{t_2} \neq 0
    \end{equation}
    So then, to "fix" an endpoint, we impose the restriction that 
    \begin{equation}
        \frac{\partial L}{\partial \dot{q}}\bigg|_{t_2} = 0
    \end{equation}
\end{enumerate}

\section*{Brachistochrone Problem}

We note that the total time $T$ is 
\begin{equation}
    T = \int_{\gamma}dt
\end{equation}
Where $\gamma$ is the curve that the particle takes. We have that 
\begin{equation}
    dt = \frac{dl}{v}
\end{equation}
Where $dl$ is a small step in the curve and $v$ is the velocity of the particle. We can derive the velocity of the particle using conservation of energy:
\begin{equation}
    \frac{1}{2}mv^2 + mgy(t) = mgy(t_0)
\end{equation}
Then 
\begin{equation}
    v^2 = 2g(y(t_0) - y(t))
\end{equation}
Thus 
\begin{equation}
    v = \sqrt{2g(y(t_0) - y(t))}
\end{equation}
We define $y(t_0) = 0$, thus 
\begin{equation}
    v = \sqrt{2g(y(t))}
\end{equation}
Where $y(t)$ is defined as positive from below the x,y axis. 
\begin{equation}
    T = \int_{t_1}^{t_2}\frac{dl}{v} \iff \int_{t_1}^{t_2}\frac{\sqrt{dx^2 + dy^2}}{\sqrt{2g(y)}}
\end{equation}
We choose $t = y$, $q = x(y)$, and $\dot{q} = x'(y)$, thus 
\begin{equation}
    \int_{t_1}^{t_2}\frac{\sqrt{x'^2 + 1}}{\sqrt{2gy}}
\end{equation}
We use the euler lagrangian equations to get 
\begin{equation}
    \frac{1}{\sqrt{2gy}}\frac{x'}{\sqrt{x'^2+1}} = c
\end{equation}
This implies that 
\begin{equation}
    x' = \pm \frac{\sqrt{2gy}c}{\sqrt{1-2gyc^2}}
\end{equation}
Thus 
\begin{equation}
    x = \frac{1}{2c}\sqrt{\frac{2}{gy}-4c^2} + \frac{1}{2gc^2}\arctan\left[\sqrt{\frac{1}{2gyc^2} - 1}\right]
\end{equation}
\end{document}
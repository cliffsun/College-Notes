\documentclass{article}
\usepackage[left=2cm, right=2cm, top=1cm, bottom=1cm]{geometry}
\usepackage{graphicx}
\usepackage{amsmath}
\usepackage{amssymb}
\usepackage{amsthm}
\usepackage{fancyhdr}

\title{PHYS 325: Lecture 1}
\author{Cliff Sun}

\newtheorem{theorem}{Theorem}[section]
\newtheorem{lemma}[theorem]{Lemma}
\newtheorem{definition}[theorem]{Definition}
\newtheorem{conjecture}[theorem]{Conjecture}
\newtheorem{proposition}[theorem]{Proposition}
\newtheorem{corollary}[theorem]{Corollary}
\newtheorem{one minute paper}[theorem]{One Minute Paper}

\pagestyle{fancy}
\lhead{\textbf{\thepage}\ \ \nouppercase{\rightmark}}
\chead{PHYS 325: Lecture 1}
\rhead{Cliff Sun}

\begin{document}

\maketitle

\section*{Lecture Span}
\begin{enumerate}
    \item Course Overview
    \item Newton's Laws
\end{enumerate}

\section*{Course Overview}

\subsection*{Logistics}
\begin{itemize}
    \item Lecture attendance via gradescope: answer $>50\%$ of questions. (NOT GRADED ON CORRECTNESS)
    \item Bonus points: Must ask question during Office Hours (2pts per week up to 30pts or $3\%$)
\end{itemize}

\subsection*{Overview of Course Topics}
\begin{enumerate}
    \item Newtonian Dynamics
    \item Conservation Laws
    \item Damped \& Driven Oscillations
    \item Motion in rotating reference frames
    \item Lagrangian Mechanics
\end{enumerate}

\section*{Ch 1: Newton's Laws}

\subsection*{1.1 General Concepts}

\subsubsection*{1.1.1 Space/Position}

\textbf{Choosing a Reference Frame}

    Note: In general, it's important to choose a coordinate system and use linearly-independent, 
unit vectors that span all of your chosen vector space. Example includes:
\begin{equation}
    \vec{i}, \vec{j}, \vec{k}
\end{equation}

In this class, we will use mostly Cartesian Coordinates and thus Cartesian Basis Vectors as seen in eq (1). 
Later, we will use spherical \& cylindrical coordinates to describe rotational motion. \\

\textbf{Position Vector}

We can choose a $\vec{r}$ that goes from $0 \rightarrow \rho$. This position vector will be a linear combination of the basis vectors
such as

\begin{equation}
    \vec{r} = a\vec{e_x} + b\vec{e_y} + c\vec{e_z} \leftarrow \text{Basis vector notation}
\end{equation}

Another form of notation of listing this vector is 

\begin{equation}
    \vec{r} = (a,b,c) \leftarrow \text{Vector notation}
\end{equation}

Important thing to note: Physical Characteristics remain the same regardless of the chosen coordinate frames. 

\newpage

\subsubsection*{1.1.2 Time}

Typically, Time is depicted with the variable $t$ or $T$. In Classical Mechanics, time is \underline{\textbf{absolute}} unlike Special Relativity. This means that
time is measured \textbf{\underline{equally}} by all obsevers. \\

\textbf{Consequences}
\begin{itemize}
    \item Information travels at an infinite speed, i.e physical changes are noticed \textbf{\underline{immediately}}
    \item People can communicate instantly
\end{itemize}

Thus, Newtonian Mechanics are valid for $v << c$ and $m >> e$ where $e = \text{mass of electron }$ and $c = \text{speed of light }$. 

\subsubsection*{1.1.3 Mass}

Mass is typically depicted using variables $m, M$ and the units of mass is [kg]. This is an instrinsic property of an object. It characertizes the object's inertia; i.e resistance to acceleration.

\subsubsection*{1.1.4 Point/Particle}

It is an object of mass $m$ but neglected size. \\

\textbf{Consequences}
\begin{itemize}
    \item Kinetic Energy is purely translational (no rotational kinetic energy)
    \item Size of object $<<$ distance to observer
\end{itemize}

\subsubsection*{1.1.5 Force}

Force is depicted by the symbol $\vec{F}$, and is the cause of an object's motion.

\textbf{Strategy for describing the motion of an object}
\begin{itemize}
    \item Identify all forces and their directions
    \item Use vector addition to find the net force
\end{itemize}

\section*{Recap on Vector Calculus}

Consider two vectors $\vec{r} = (r_1, r_2, r_3)$ and $\vec{s} = (s_1, s_2, s_3)$. 

\textbf{Properties}
\begin{itemize}
    \item Addition or $\vec{r} + \vec{s}$
    \item Scalar Multiplication for some $a \in \mathbb{R}$
    \item Magnitude (norm), basically $r = ||\vec{r}||^2 \iff \sqrt{\vec{r} \cdot \vec{r}} \iff \sqrt{\sum_i(r_i^2)}$.
    \item Dot Product: $f : \mathbb{R}^n \rightarrow \mathbb{R}$. This operation commutes, such that $\vec{s} \cdot \vec{r} \iff \vec{r} \cdot \vec{s}$
    \item Cross product: $f : \mathbb{R}^3 \rightarrow \mathbb{R}^3$. Denoted as $\vec{r} \times \vec{s} = \vec{p}$. Such that $\vec{p} \cdot \vec{r} = \vec{p} \cdot \vec{s} = 0$. In other words, $\vec{p}$ is orthogonal to both $\vec{r}$ and $\vec{s}$. 
\end{itemize}

\end{document}
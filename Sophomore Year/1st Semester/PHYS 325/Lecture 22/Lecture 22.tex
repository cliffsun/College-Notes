\documentclass{article}
\usepackage[left=2cm, right=2cm, top=2cm, bottom=2cm]{geometry}
\usepackage{graphicx}
\usepackage{amsmath}
\usepackage{amssymb}
\usepackage{amsthm}
\usepackage{fancyhdr}
\usepackage{verbatim}
\usepackage{listings}
\usepackage{xcolor}
\usepackage{pgfplots}

\lstset{
    language=C++,
    basicstyle=\ttfamily\footnotesize,
    keywordstyle=\color{blue},
    commentstyle=\color{green},
    stringstyle=\color{red},
    numbers=left,
    numberstyle=\tiny\color{gray},
    frame=single,
    breaklines=true,
    captionpos=b,
}


\title{PHYS 325: Lecture 22}
\author{Cliff Sun}

\newtheorem{theorem}{Theorem}[section]
\newtheorem{lemma}[theorem]{Lemma}
\newtheorem{definition}[theorem]{Definition}
\newtheorem{conjecture}[theorem]{Conjecture}
\newtheorem{proposition}[theorem]{Proposition}
\newtheorem{corollary}[theorem]{Corollary}
\newtheorem{one minute paper}[theorem]{One Minute Paper}

\pagestyle{fancy}
\lhead{\textbf{\thepage}\ \ \nouppercase{\rightmark}}
\chead{PHYS 325: Lecture 22}
\rhead{Cliff Sun}

\begin{document}

\maketitle

\section*{Non-rotating reference frames}

Consider
\begin{enumerate}
    \item 2 Frames accelerating relative to each other
    \item $O$: rest frame with fixed unit vectors
    \item $O'$: accelerated frame with unfixed basis vectors  
\end{enumerate}

Define
\begin{enumerate}
    \item $\vec{r}_{OA}$ from $O$ to point $A$.
    \item $\vec{r}_{OO'}$ from $O$ to accelerated frame $O'$.
    \item $\vec{r}_{O'A}$ from $O'$ to point $A$. 
\end{enumerate}

Thus 

\begin{equation}
    \vec{r}_{OA} = \vec{r}_{OO'} + \vec{r}_{O'A}
\end{equation}

Velocity is 

\begin{equation}
    \vec{v}_{OA} = \vec{v}_{OO'} + \vec{v}_{O'A}
\end{equation}

Acceleration is also defined similarly. So 

\begin{equation}
    F = ma_{OA}
\end{equation}
Thus 
\begin{equation}
    F = m(a_{OO'} + a_{O'A})
\end{equation}

Thus 

\begin{equation}
    F - ma_{OO'} = ma_{O'A}
\end{equation}

We define 

\begin{equation}
    F_{\text{fictious}} = ma_{OO'}
\end{equation}

And 

\begin{equation}
    F_{\text{true}} = ma_{O'A}
\end{equation}

Thus 

\begin{equation}
    F_{\text{tot}} = F_{\text{fictious}} + F_{\text{true}} = ma_{O'A}
\end{equation}

This is newton's 2nd law in an accelerated frame. 

\section*{Rotating Frames}

Rest frame $O$ is fixed. Consider a rotating reference frame $O'$ with $\vec{\omega}$. Then let 
\begin{enumerate}
    \item $\vec{L}$ is fixed on $O'$
    \item Then $\dot{\vec{L}} = \vec{\omega}\times \vec{L}$
\end{enumerate}

Then we define the vectors as the following:
\begin{enumerate}
    \item $\vec{r}_{OA} = \vec{r}_{OO'} + \vec{r}_{O'A}$ 
    \item $\vec{v}_{OA} = \vec{v}_{OO'} + \vec{v}_{O'A}$
    We define this further:
    \begin{equation}
        \vec{v}_{OO'} = \frac{d}{dt}\vec{r_{OO'}} = \frac{d}{dt}(\vec{r}_{OO'}\vec{e}_x + \dots)
    \end{equation} 
    Note that 
    \begin{equation}
        \dot{\vec{e}_x} = 0, \dots
    \end{equation}
    Thus 
    \begin{equation}
        \vec{v}_{OO'}\vec{e}_x + \dots
    \end{equation}
    Note 
    \begin{equation}
        \vec{v}_{O'A} = \frac{d}{dt}(x'\vec{e}_{x'} + \dots)
    \end{equation}
    These basis vectors do change in time, so 
    \begin{equation}
        = \dot{x}\vec{e}_{x'} + \dots + x\dot{\vec{e}}_{x'} + \dots 
    \end{equation}
    Note that 
    \begin{equation}
        \dot{\vec{e}}_{x'} = \omega times \vec{e}_{x'}
    \end{equation}
    \begin{equation}
         = v_{\text{apparant}} + \omega \times \vec{r}_{O'A}
    \end{equation}
\end{enumerate}

Thus 

\begin{equation}
    \vec{v}_{\text{rest frame}} = \vec{v}_{OO'} + \vec{v}_{\text{apparant}} + \omega \times \vec{r}_{O'A}
\end{equation}

\subsection*{Acceleration}

\begin{equation}
    \vec{a}_{OA} = \frac{d}{dt}\vec{v}_{OO'} + \frac{d}{dt}v_{\text{app}} + \frac{d}{dt}(\omega \times r_{O'A})
\end{equation}

We note that 

\begin{equation}
    \frac{d}{dt}v_{\text{app}} = \frac{d}{dt}(v_{\text{app}, x'}\vec{e}_{x'} + \dots)
\end{equation}

\begin{equation}
    \vec{a}_{\text{app}} + \omega \times \vec{v}_{\text{app}}
\end{equation}

And 

\begin{equation}
    \frac{d}{dt}(\omega \times r_{O'A}) = \omega \times r_{O'A} + \omega \times (v_{app} + \omega \times r_{O'A})
\end{equation}

\begin{equation}
    a_{OA} = a_{OO'} + a_{app} + \omega \times (\omega \times r_{O'A}) + 2\omega \times v_{app} + \omega \times r_{O'A}
\end{equation}

We define the following:
\begin{enumerate}
    \item $a_{OO'}$: relative acceleration between frames $O$ and $O'$.
    \item $a_{\text{app}}$: apparant acceleration of $A$ in $O'$. 
    \item $\omega \times (\omega \times r_{O'A})$: generalized centripetal acceleration
    \item $2\omega \times v_{app}$ is the coriolis "force". 
    \item $\dot{\omega} \times r_{O'A}$ is the Euler term
\end{enumerate}

\section*{Dynamics in non-inertial reference frames}

Write N2L in non-inertial reference frame, 
\begin{enumerate}
    \item account for fictious forces
\end{enumerate}


\end{document}
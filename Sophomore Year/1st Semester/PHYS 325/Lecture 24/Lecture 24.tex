\documentclass{article}
\usepackage[left=2cm, right=2cm, top=2cm, bottom=2cm]{geometry}
\usepackage{graphicx}
\usepackage{amsmath}
\usepackage{amssymb}
\usepackage{amsthm}
\usepackage{fancyhdr}
\usepackage{verbatim}
\usepackage{listings}
\usepackage{xcolor}
\usepackage{pgfplots}

\lstset{
    language=C++,
    basicstyle=\ttfamily\footnotesize,
    keywordstyle=\color{blue},
    commentstyle=\color{green},
    stringstyle=\color{red},
    numbers=left,
    numberstyle=\tiny\color{gray},
    frame=single,
    breaklines=true,
    captionpos=b,
}


\title{PHYS 325: Lecture 24}
\author{Cliff Sun}

\newtheorem{theorem}{Theorem}[section]
\newtheorem{lemma}[theorem]{Lemma}
\newtheorem{definition}[theorem]{Definition}
\newtheorem{conjecture}[theorem]{Conjecture}
\newtheorem{proposition}[theorem]{Proposition}
\newtheorem{corollary}[theorem]{Corollary}
\newtheorem{one minute paper}[theorem]{One Minute Paper}

\pagestyle{fancy}
\lhead{\textbf{\thepage}\ \ \nouppercase{\rightmark}}
\chead{PHYS 325: Lecture 24}
\rhead{Cliff Sun}

\begin{document}

\maketitle

\section*{Foucault pendulum}

We have that 

\begin{equation}
    ma_{\text{app}} = \sum F_{\text{true}} + F_{\text{fict}}
\end{equation}
We have that 
\begin{equation}
    F_{\text{true}} = -\nabla U
\end{equation}
We have that 
\begin{equation}
    U = -mgh \iff -mgL\cos\beta
\end{equation}
For small $\beta$, we have that 
\begin{equation}
    c + \frac{mg}{2L}(x^2 + y^2)
\end{equation}
Thus 
\begin{equation}
    F_{\text{net}} = -\nabla U = -\frac{mg}{L}(x\hat{e} + y\hat{n})
\end{equation}
\subsubsection*{Coriolis Force}
\begin{equation}
    F_{\text{cor}} = -2m\vec{\omega} \times \vec{v}_{\text{app}}
\end{equation}
\begin{equation}
    \ddot{x} = -\frac{g}{L}x + 2\omega \dot{y}\sin\theta
\end{equation}
\begin{equation}
    \ddot{y} = -\frac{g}{L}y - 2\omega\dot{x}\sin\theta
\end{equation}
\begin{equation}
    \ddot{z} = 0 + 2\omega \dot{x}\cos\theta 
\end{equation}
Let $\omega_z = \omega\sin\theta$
\begin{equation}
    \ddot{x} = -\frac{g}{L}x + 2\omega_z \dot{y}
\end{equation}
\begin{equation}
    \ddot{y} = -\frac{g}{L}y - 2\omega_z \dot{x}
\end{equation}
If $\theta = 0$, then this is a simple harmonic oscillator.     
\end{document}
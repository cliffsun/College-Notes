\documentclass{article}
\usepackage[left=2cm, right=2cm, top=2cm, bottom=2cm]{geometry}
\usepackage{graphicx}
\usepackage{amsmath}
\usepackage{amssymb}
\usepackage{amsthm}
\usepackage{fancyhdr}
\usepackage{verbatim}
\usepackage{listings}
\usepackage{xcolor}
\usepackage{pgfplots}

\lstset{
    language=C++,
    basicstyle=\ttfamily\footnotesize,
    keywordstyle=\color{blue},
    commentstyle=\color{green},
    stringstyle=\color{red},
    numbers=left,
    numberstyle=\tiny\color{gray},
    frame=single,
    breaklines=true,
    captionpos=b,
}


\title{PHYS 325: Lecture 27}
\author{Cliff Sun}

\newtheorem{theorem}{Theorem}[section]
\newtheorem{lemma}[theorem]{Lemma}
\newtheorem{definition}[theorem]{Definition}
\newtheorem{conjecture}[theorem]{Conjecture}
\newtheorem{proposition}[theorem]{Proposition}
\newtheorem{corollary}[theorem]{Corollary}
\newtheorem{one minute paper}[theorem]{One Minute Paper}

\pagestyle{fancy}
\lhead{\textbf{\thepage}\ \ \nouppercase{\rightmark}}
\chead{PHYS 325: Lecture 27}
\rhead{Cliff Sun}

\begin{document}

\maketitle

\section*{Lagrange Multipliers}

Extremize a functional $S_1$ subject to a global constraint $S_2$. Concrete example:
\begin{center}
    \textit{Minimize Area under curve given a fixed length of the curve}
\end{center}

Given 

\begin{equation}
    C = S_1 - \lambda S_2
\end{equation}
Where $\lambda$ is the Lagrange multiplier, we have that 
\begin{equation}
    \delta C = \delta S_1 - \lambda \delta S_2
\end{equation}
We have that 
\begin{equation}
    \delta S_1 = 0
\end{equation}
Because we want to extremize $S_1$, similarly, we have that 
\begin{equation}
    \delta S_2 = 0
\end{equation}
Because $S_2$ is constant. Thus 
\begin{equation}
    \delta S_1 - \lambda \delta S_2 = 0
\end{equation}
\subsection*{Hanging Chain}
Given a chain of mass density $\rho = \frac{dm}{dl}$, we have that 
\begin{equation}
    m = \int \rho dl \iff \int \rho \sqrt{x'^2 + 1}dy
\end{equation}
Thus
\begin{equation}
    PE = -mgh \iff -\int \rho gy\sqrt{x'^2 + 1}dy = S_1
\end{equation}
This is our first functional. Our second functional fixes the length of the chain to be a length $L$, that is 
\begin{equation}
    L = \int \sqrt{x'^2 + 1}dy = S_2
\end{equation}
We choose $\tilde{\lambda} = g\rho\lambda$, we get 
\begin{equation}
    C = PE - \tilde{\lambda}L = -g\rho \int dy (y - \lambda)\sqrt{x'^2 + 1}
\end{equation}

\section*{Local Constraints}
Consider functionals of 2 or more functions, that is 
\begin{equation}
    F(x(t),y(t))  = \int f(x,x',y,y';z)dz
\end{equation}
with conditions 
\begin{equation}
    g(x,y;z) = 0
\end{equation}
Similarly, 
\begin{equation}
    C = f - \lambda g
\end{equation}
Extremize 
\begin{equation}
    C = \int (f-\lambda g)dz
\end{equation}

\section*{Lagrangian Mechanics}

\begin{equation}
    L = T - UN
\end{equation}

\end{document}
\documentclass{article}
\usepackage[left=2cm, right=2cm, top=2cm, bottom=2cm]{geometry}
\usepackage{graphicx}
\usepackage{amsmath}
\usepackage{amssymb}
\usepackage{amsthm}
\usepackage{fancyhdr}
\usepackage{verbatim}
\usepackage{listings}
\usepackage{xcolor}
\usepackage{pgfplots}

\lstset{
    language=C++,
    basicstyle=\ttfamily\footnotesize,
    keywordstyle=\color{blue},
    commentstyle=\color{green},
    stringstyle=\color{red},
    numbers=left,
    numberstyle=\tiny\color{gray},
    frame=single,
    breaklines=true,
    captionpos=b,
}


\title{PHYS 325: Lecture 15}
\author{Cliff Sun}

\newtheorem{theorem}{Theorem}[section]
\newtheorem{lemma}[theorem]{Lemma}
\newtheorem{definition}[theorem]{Definition}
\newtheorem{conjecture}[theorem]{Conjecture}
\newtheorem{proposition}[theorem]{Proposition}
\newtheorem{corollary}[theorem]{Corollary}
\newtheorem{one minute paper}[theorem]{One Minute Paper}

\pagestyle{fancy}
\lhead{\textbf{\thepage}\ \ \nouppercase{\rightmark}}
\chead{PHYS 325: Lecture 15}
\rhead{Cliff Sun}

\begin{document}

\maketitle

\section*{Lecture Span}
\begin{itemize}
    \item Damped harmonic oscillator
\end{itemize}

The EOM for damped harmonic oscillators is 

\begin{equation}
    m\ddot{y} + c\dot{y} + ky = 0
\end{equation}

The "normal" form:

\begin{equation}
    \ddot{y} + 2\zeta\omega_n\dot{y} + \omega_n^2y = 0
\end{equation}

Then the eigen frequency, or the normal frequency is 

\begin{equation}
    \omega_n = \sqrt{\frac{k_{eff}}{m_{eff}}}
\end{equation}

and 

\begin{equation}
    \zeta = \frac{c}{2m\omega_n}
\end{equation}

Guess 

\begin{equation}
    y \sim e^{\lambda t}
\end{equation}

insert in EOM

\begin{equation}
    \lambda^2 + 2\zeta\omega_n\lambda + \omega_n^2 = 0
\end{equation}

Then 

\begin{equation}
    \lambda_\pm = -\omega_n(\zeta \pm \sqrt{\zeta^2 -1})
\end{equation}

\subsection*{Undamped}

Let $\zeta = 0$, then 

\begin{equation}
    \lambda_\pm = \pm i\omega_n
\end{equation}

\subsection*{Overdamped}

Let $\zeta > 1$

\begin{equation}
    \lambda_\pm = -\omega_n(\zeta \pm \sqrt{\zeta^2 - 1})
\end{equation}

Note that

\begin{equation}
    \lambda_\pm < 0
\end{equation}

\subsection*{Critical damping}

Let $\zeta = 1$, then 

\begin{equation}
    \lambda_\pm = -omega_n
\end{equation}

Then 

\begin{equation}
    y(t) = A_1e^{\omega_nt} + A_2te^{-\omega_nt}
\end{equation}

\subsection*{Underdamping}

Let $0 < \zeta < 1$, then 

\begin{equation}
    \lambda_pm = -\omega_n(\zeta \pm \sqrt{\zeta^2 - 1})
\end{equation}

Then 

\begin{equation}
    \lambda_pm = \omega_n\zeta \pm i\sqrt{1 - \zeta^2}
\end{equation}

Versions we can work with:

\begin{equation}
    y(t) = e^{-\zeta\omega t}(A\cos(\omega t) + B\sin(\omega t))
\end{equation}

What happens if $\zeta < 0$? Then double yo work dumbass. 

\section*{Forced oscillator}
Oscillator with external driving force. Note, energy is \underline{not} conserved. Then 

\begin{equation}
    \frac{dE}{dt} + P_{\text{diss}} = \frac{dW}{dt}_{\text{ext}} \iff F_{\text{ext}} \cdot v
\end{equation}

\begin{equation}
    m_{\text{eff}}\frac{d^2\vec{x}}{dt^2} + c_{\text{eff}}\frac{d\vec{x}}{dt} + k_{\text{eff}}\vec{x} = \vec{F_{\text{ext}}}(t)
\end{equation}

No damping means that 

\begin{equation}
    P_{\text{diss}} = 0
\end{equation}

We also have 

\begin{equation}
    T = \frac{1}{2}mv^2 \iff \frac{1}{2}mL^2\dot{\theta}^2
\end{equation}

\begin{equation}
    U = mgh \iff mgL(1-\cos\theta)
\end{equation}

Note we have that 

\begin{equation}
    F \cdot v \iff L\dot{\theta}F\cdot e_{\theta} \iff L\dot{\theta}F\cos\theta
\end{equation}

Thus we have that 

\begin{equation}
    \frac{1}{2}mL^2 2\dot{\theta}\ddot{\theta} + mgL\dot{\theta}\sin\theta = L\dot{\theta}F\cos\theta
\end{equation}

\begin{equation}
    mL\ddot{\theta} + mg\sin\theta = F\cos\theta
\end{equation}

For small angles:

\begin{equation}
    mL\ddot{\theta} + mg\theta = F(t)
\end{equation}

\section*{Cart on a cart}

\begin{itemize}
    \item Big cart as acceleration $\ddot{y}$
    \item Displacement of big truck = $y$, displacement of everything is $y + x$
    \item also another force $-kx$
\end{itemize}

Then 

\begin{equation}
    m(\ddot{x} + \ddot{y}) = -kx
\end{equation}

then 

\begin{equation}
    m\ddot{x} + kx = -m\ddot{y}
\end{equation}

\end{document}
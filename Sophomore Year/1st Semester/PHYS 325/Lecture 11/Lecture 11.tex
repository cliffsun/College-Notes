\documentclass{article}
\usepackage[left=2cm, right=2cm, top=1cm, bottom=1cm]{geometry}
\usepackage{graphicx}
\usepackage{amsmath}
\usepackage{amssymb}
\usepackage{amsthm}
\usepackage{fancyhdr}
\usepackage{verbatim}
\usepackage{listings}
\usepackage{xcolor}
\usepackage{pgfplots}

\lstset{
    language=C++,
    basicstyle=\ttfamily\footnotesize,
    keywordstyle=\color{blue},
    commentstyle=\color{green},
    stringstyle=\color{red},
    numbers=left,
    numberstyle=\tiny\color{gray},
    frame=single,
    breaklines=true,
    captionpos=b,
}


\title{PHYS 325: Lecture 11}
\author{Cliff Sun}

\newtheorem{theorem}{Theorem}[section]
\newtheorem{lemma}[theorem]{Lemma}
\newtheorem{definition}[theorem]{Definition}
\newtheorem{conjecture}[theorem]{Conjecture}
\newtheorem{proposition}[theorem]{Proposition}
\newtheorem{corollary}[theorem]{Corollary}
\newtheorem{one minute paper}[theorem]{One Minute Paper}

\pagestyle{fancy}
\lhead{\textbf{\thepage}\ \ \nouppercase{\rightmark}}
\chead{PHYS 325: Lecture 11}
\rhead{Cliff Sun}

\begin{document}

\maketitle

\section*{Lecture Span}
\begin{itemize}
    \item 2 Body problem
\end{itemize}

\section*{Midterm}
\begin{enumerate}
    \item Force in 3 dimensions
    \item Curvilinear coordinates
    \item Central force
    \item Check out the email she sent
\end{enumerate}

\section*{2 Body problem}

\subsection*{Kepler's Laws}
\begin{enumerate}
    \item Orbits around the sun are planar ellipses
    \item The line between the sun and planet produces the same area in equal time. That is given 2 points in the orbit separated by a time $dt$, then we have that
    the area between these two points (should be like a triangle) is the same everywhere given the same dt. 
    \item Orbital period $P$ around the sun is proportional to $\frac{3}{2}$ power of the semi-major axis. Ellipses have 2 axis, the semi-minor and semi-major. The semi-major is the longer radius in 
    an ellipse and the semi-minor is the smaller radius in an ellipse.  
\end{enumerate}

\subsection*{Characterize Orbits}

\begin{enumerate}
    \item We first introduce \textbf{angular momentum per unit mass} 
    \begin{equation}
        L = mr^2 \dot{\phi} \implies l = \frac{L}{m} \iff r^2\dot{\phi}
    \end{equation}
    Thus 
    \begin{equation}
        \dot{\phi} = \frac{l}{r^2}
    \end{equation}
    \textbf{Note:} The angle between $\vec{r}$ and $\vec{v}$ is NO LONGER 90 DEGREES!
    \item \textbf{Energy per unit mass} conservation
    \begin{equation}
        \epsilon = \frac{E}{m} \iff \frac{1}{m}(T + U_{grav})
    \end{equation}
    \begin{equation}
        \iff \frac{1}{2}|\vec{v}|^2 - \frac{GM}{r}
    \end{equation}
    \begin{equation}
        \iff \frac{1}{2}(\dot{r}^2 + r^2\dot{\phi}^2) - \frac{GM}{r}
    \end{equation}
    \begin{equation}
        \epsilon = \frac{1}{2}\dot{r}^2 + \frac{l^2}{2r^2} - \frac{GM}{r}
    \end{equation}
    This is a one-dimesional problem with an effective potential.
    \item Perigee [$r_p$] (minimal distance to focal point (M)) and Apogee [$r_a$] (maximal distance to focal point (M))
    \begin{itemize}
        \item Velocity vectors $\vec{v}_p$ and $\vec{v}_a$ are perpendicular to $\vec{r}_p$ and $\vec{r}_a$. Thus
        \begin{equation}
            \vec{r_p} \cdot \vec{v_p} = 0
        \end{equation}
        \item Angular momentum $\vec{l} = \vec{r} \times \vec{v}$. At the Perigee and Apogee
        \begin{equation}
            \vec{l} = \vec{r} \times \vec{v} = |r_p||v_p| = |r_a||v_a|
        \end{equation}
    \end{itemize}
    \item Energy per unit mass in $r_p$ and $r_a$ in Apogee and Perigee, we have that $\dot{r} = 0$, thus 
    \begin{equation}
        \epsilon_{p,a} = \frac{l^2}{2r_{p,a}^2} - \frac{GM}{r_{p,a}}
    \end{equation}
    \begin{equation}
        \epsilon_{p,a} = \frac{1}{2}v_{p,a}^2 - \frac{GM}{r_{p,a}}
    \end{equation}
    \begin{equation}
        \epsilon_{p,a} = \frac{1}{2}v_{p,a}^2 - \frac{GM}{l}v_{p,a}
    \end{equation}
    Thus 
    \begin{equation}
        v_{p,a} = \frac{GM}{l} \pm \sqrt{(\frac{GM}{l})^2 + 2\epsilon}
    \end{equation}
    Note, plus is for perigee and minus is for the apogee. 
    \begin{equation}
        r_{p,a} = \frac{l}{v_{p,a}} = l\left[\frac{GM}{l} \pm \sqrt{(\frac{GM}{l})^2 + 2\epsilon}\right]^{-1}
    \end{equation}
\end{enumerate}

\section*{Effective Potential and Orbits}

\begin{equation}
    \epsilon = \frac{1}{2}\dot{r}^2 + U_{eff}
\end{equation}
\begin{equation}
    U_{eff} = \frac{l^2}{2r^2} - \frac{GM}{r}
\end{equation}

\begin{enumerate}
    \item Circular Orbit with radius $r_c$
    \begin{itemize}
        \item Minimum in effective potential
        \begin{equation}
            U_{eff}'|_{r=r_c} = 0
        \end{equation}
        In this scenario, we have that 
    \begin{equation}
        \frac{mv^2}{r} = -\frac{GMm}{r^2}
    \end{equation}
    \begin{equation}
        v = \sqrt{\frac{GM}{r}}
    \end{equation}
    Thus, 
    \begin{equation}
        P = \frac{2\pi}{\omega} \implies \omega = \frac{v}{r} \implies 2\pi \sqrt{\frac{r^3}{GM}}
    \end{equation}
    We have that 
    \begin{equation}
        l = rv \implies v = \frac{GM}{l} \iff r = \frac{l^2}{GM}
    \end{equation}
    Energy
    \begin{equation}
        \epsilon = \frac{1}{2}\dot{r}^2 - \frac{GM}{r}
    \end{equation}
    \begin{equation}
        \iff -\frac{1}{2}v^2 = -\frac{1}{2}(\frac{GM}{l})^2
    \end{equation}
    \end{itemize}
    \item Elliptical orbits $\epsilon_{central} < \epsilon_{elliptical} < 0$
    \begin{itemize}
        \item Bound orbit with 
        \begin{equation}
            -\frac{1}{2}(\frac{GM}{l})^2 = \epsilon_c < \epsilon_e < 0
        \end{equation}
    \end{itemize}
    \item Parabolic "orbit" $\epsilon = 0$
    \begin{itemize}
        \item The orbiting mass gets "flung" by the larger mass $M$ (like the sun).
        \item Perigee:
        \begin{equation}
            V_p = \frac{2GM}{l}
        \end{equation}
        \begin{equation}
            r_p = \frac{l^2}{2GM}
        \end{equation}
        \item Apogee, $v_a \rightarrow 0$ and $r_a = \frac{l}{r_a} \rightarrow \infty$
    \end{itemize}
    \item Hyperbola $\epsilon > 0$
    \begin{itemize}
        \item \underline{Unbound orbit} ("scattering hyperbola")
        \begin{equation}
            v_{p,a} = \frac{GM}{l} \pm \sqrt{(\frac{GM}{l})^2 + 2\epsilon}
        \end{equation}
        For $v,r>0$, hyperbola, $v<0$ unphysical
    \end{itemize}
\end{enumerate}

\section*{Two body problem}



\end{document}
\documentclass{article}
\usepackage[left=2cm, right=2cm, top=2cm, bottom=2cm]{geometry}
\usepackage{graphicx}
\usepackage{amsmath}
\usepackage{amssymb}
\usepackage{amsthm}
\usepackage{fancyhdr}
\usepackage{verbatim}
\usepackage{listings}
\usepackage{xcolor}
\usepackage{pgfplots}

\lstset{
    language=C++,
    basicstyle=\ttfamily\footnotesize,
    keywordstyle=\color{blue},
    commentstyle=\color{green},
    stringstyle=\color{red},
    numbers=left,
    numberstyle=\tiny\color{gray},
    frame=single,
    breaklines=true,
    captionpos=b,
}


\title{PHYS 325: Lecture 20}
\author{Cliff Sun}

\newtheorem{theorem}{Theorem}[section]
\newtheorem{lemma}[theorem]{Lemma}
\newtheorem{definition}[theorem]{Definition}
\newtheorem{conjecture}[theorem]{Conjecture}
\newtheorem{proposition}[theorem]{Proposition}
\newtheorem{corollary}[theorem]{Corollary}
\newtheorem{one minute paper}[theorem]{One Minute Paper}

\pagestyle{fancy}
\lhead{\textbf{\thepage}\ \ \nouppercase{\rightmark}}
\chead{PHYS 325: Lecture 20}
\rhead{Cliff Sun}

\begin{document}

\maketitle

Given some Green's function $G(t)$ and some forcing $F(t)$, then our solution is 

\begin{equation}
    x_p(t) = \int_{\mathbb{R}}F(t')G(t-t')dt'
\end{equation}

Such that 

\begin{equation}
    G(t-t') = H(t-t')\exp\left(\frac{-c(t-t')}{2m}\right)\frac{\sin(\omega_d(t-t'))}{m\omega_d}
\end{equation}

Such that 

\begin{equation}
    H(t) = \begin{cases}
        1 & t > t' \\
        0 & t < t' 
    \end{cases}
\end{equation}

\subsection*{Example}

In the case of $F_0$ is a constant driving force, then we integrate from $0$ to $t$. 

\section*{Fourier Transformation}

\begin{equation}
    \tilde{f}(\omega) = \int_\mathbb{R}f(t)e^{-i\omega t}dt
\end{equation}

Inverse FT:

\begin{equation}
    f(t) = \frac{1}{2\pi}\int_\mathbb{R}\tilde{f}(\omega)e^{i\omega t}d\omega
\end{equation}

\end{document}
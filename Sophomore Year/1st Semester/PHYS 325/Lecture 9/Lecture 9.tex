\documentclass{article}
\usepackage[left=2cm, right=2cm, top=1cm, bottom=1cm]{geometry}
\usepackage{graphicx}
\usepackage{amsmath}
\usepackage{amssymb}
\usepackage{amsthm}
\usepackage{fancyhdr}
\usepackage{verbatim}
\usepackage{listings}
\usepackage{xcolor}
\usepackage{pgfplots}

\lstset{
    language=C++,
    basicstyle=\ttfamily\footnotesize,
    keywordstyle=\color{blue},
    commentstyle=\color{green},
    stringstyle=\color{red},
    numbers=left,
    numberstyle=\tiny\color{gray},
    frame=single,
    breaklines=true,
    captionpos=b,
}


\title{PHYS 325: Lecture 9}
\author{Cliff Sun}

\newtheorem{theorem}{Theorem}[section]
\newtheorem{lemma}[theorem]{Lemma}
\newtheorem{definition}[theorem]{Definition}
\newtheorem{conjecture}[theorem]{Conjecture}
\newtheorem{proposition}[theorem]{Proposition}
\newtheorem{corollary}[theorem]{Corollary}
\newtheorem{one minute paper}[theorem]{One Minute Paper}

\pagestyle{fancy}
\lhead{\textbf{\thepage}\ \ \nouppercase{\rightmark}}
\chead{PHYS 325: Lecture 9}
\rhead{Cliff Sun}

\begin{document}

\maketitle

\section*{Lecture Span}
\begin{itemize}
    \item Central forces
\end{itemize}

\section*{Central Forces}

Let 
\begin{equation}
    \vec{L} = \vec{r} \times m\vec{v}
\end{equation}
We try to show that $\vec{L}$ is constant. Or 

\begin{equation}
    D_t (L) = 0
\end{equation}

We expand this out

\begin{equation}
    D_t(L) = D_t(\vec{r} \times m\vec{v})
\end{equation}
\begin{equation}
    \iff \dot{\vec{r}} \times m\vec{v} + \vec{r} \times m\vec{a}
\end{equation}
\begin{equation}
    \iff \vec{r} \times \vec{F}
\end{equation}
But since $\vec{F}$ is in the direction of the $\hat{r}$ component, we have that 

\begin{equation}
    \vec{r} \times F\hat{r} \iff 0
\end{equation}

Thus 

\begin{equation}
    D_t(L) = 0 \iff L = c 
\end{equation}

Because $\vec{L}$ is constant, then we have that 
\begin{enumerate}
    \item motion remains in the $\vec{v} - \vec{r}$ plane
    \item Then we can rotate the coordinate plane such that $\theta = \frac{\pi}{2} \iff \dot{\theta} = 0$
    \item $\vec{r} = r(t)\hat{r}$
    $\vec{v} = \dot{r}\hat{r} + r\dot{\theta}\hat{\theta} + r\dot{\phi}\sin\theta\hat{\phi}$ which simplifies to 
    \begin{equation}
        \vec{v} = \dot{r}\hat{r} + r\dot{\phi}\hat{\phi}
    \end{equation}
    \item Thus $\vec{L} = \vec{r} \times m\vec{v}$
     \begin{equation}
        = mr\hat{r} \times m(\dot{r}\hat{r} + r\dot{\phi}\hat{\phi})
     \end{equation}
     \begin{equation}
        = -mr^2\dot{\phi}\hat{\theta}
     \end{equation}
     \begin{equation}
        = mr^2\dot{\phi}\hat{z}
     \end{equation}
\end{enumerate}

\section*{Conservative Central Force}

Given 
\begin{equation}
    \vec{F} = f\hat{r} \iff -\vec{\nabla}u
\end{equation}
Because there is no dependence of $\theta, \phi$, we have that 
\begin{equation}
    U = \int \vec{f} \cdot \vec{dr}
\end{equation}

We write 

\begin{equation}
    L = mr^2\dot{\phi}
\end{equation}

We write this as 

\begin{equation}
    \dot{\phi} = \frac{L}{mr^2}
\end{equation}

\begin{equation}
    E = T + U
\end{equation}

\begin{equation}
    E = \frac{1}{2}m(\dot{r}^2 + r^2\dot{\phi}^2) + U
\end{equation}
\begin{equation}
    E = \frac{1}{2}m\dot{r}^2 + \frac{L^2}{2mr^2} + U
\end{equation}

Given 

\begin{equation}
    U = \frac{L^2}{2mr^2} - \frac{k}{r}
\end{equation}

\subsection*{Case 1}

\begin{equation}
    E > 0
\end{equation}

\begin{equation}
    \frac{L^2}{2mr^2}
\end{equation}

dominates. 

\subsection*{Case 2}

\begin{equation}
    E < 0
\end{equation}

\begin{equation}
    \frac{k}{r}
\end{equation}

dominates. 

\section*{Gravitational Force}

Consider 2 masses of $m_1, m_2$, gravitational force between the 2 masses would be 

\begin{equation}
    -\frac{Gm_1m_2}{r^2}\hat{r}
\end{equation}

Consider an extended object with mass $m_1$, then 

\begin{equation}
    m_1 = \int_{V}\rho(r)dr
\end{equation}

Then, 

\begin{equation}
    F = -Gm_1\frac{\int_{V}\rho(r)}{(r_2-r_1)^3}(r_2-r_1)dr_1
\end{equation}

\end{document}
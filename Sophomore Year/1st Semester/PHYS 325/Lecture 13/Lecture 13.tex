\documentclass{article}
\usepackage[left=2cm, right=2cm, top=1cm, bottom=1cm]{geometry}
\usepackage{graphicx}
\usepackage{amsmath}
\usepackage{amssymb}
\usepackage{amsthm}
\usepackage{fancyhdr}
\usepackage{verbatim}
\usepackage{listings}
\usepackage{xcolor}
\usepackage{pgfplots}

\lstset{
    language=C++,
    basicstyle=\ttfamily\footnotesize,
    keywordstyle=\color{blue},
    commentstyle=\color{green},
    stringstyle=\color{red},
    numbers=left,
    numberstyle=\tiny\color{gray},
    frame=single,
    breaklines=true,
    captionpos=b,
}


\title{PHYS 325: Lecture 14}
\author{Cliff Sun}

\newtheorem{theorem}{Theorem}[section]
\newtheorem{lemma}[theorem]{Lemma}
\newtheorem{definition}[theorem]{Definition}
\newtheorem{conjecture}[theorem]{Conjecture}
\newtheorem{proposition}[theorem]{Proposition}
\newtheorem{corollary}[theorem]{Corollary}
\newtheorem{one minute paper}[theorem]{One Minute Paper}

\pagestyle{fancy}
\lhead{\textbf{\thepage}\ \ \nouppercase{\rightmark}}
\chead{PHYS 325: Lecture 14}
\rhead{Cliff Sun}

\begin{document}

\maketitle

\section*{Harmonic \& damped motion}

\subsection*{Simple harmonic oscillator}

\subsubsection*{Using f = ma}

Let $x(t)$ be time dependent location. And $x_0 = L$ be the equilibrium point. Then 

\begin{equation}
    m\ddot{x} = -k(x-L)
\end{equation}
Letting $x' = x - L$, then 
\begin{equation}
    m\ddot{x'} = -kx'
\end{equation}

The solution is 

\begin{equation}
    x(t) = A\cos(\omega t + \phi_0)
\end{equation}

\subsubsection*{Using energy conservation}

\begin{equation}
    E = \frac{1}{2}m\dot{x}^2 + \frac{1}{2}k(x-L)^2 = c
\end{equation}
\begin{equation}
    \partial_t E = m\ddot{x}\dot{x} + k(x-L)\dot{x} = 0
\end{equation}
Thus 
\begin{equation}
    m\ddot{x} = -k(x-L)
\end{equation}

\subsection*{Energy of Oscillations}

\begin{equation}
    U = \frac{1}{2}ky^2 \iff \frac{1}{2}kC^2\cos^2(\omega t - \phi)
\end{equation}
Non negative.

\begin{equation}
    T = \frac{m}{2}C^2\omega^2\sin^2(\omega t - \phi)
\end{equation}

Thus total energy is 

\begin{equation}
    E = \frac{kC^2}{2}
\end{equation}

\end{document}
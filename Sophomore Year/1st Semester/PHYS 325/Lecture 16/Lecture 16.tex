\documentclass{article}
\usepackage[left=2cm, right=2cm, top=2cm, bottom=2cm]{geometry}
\usepackage{graphicx}
\usepackage{amsmath}
\usepackage{amssymb}
\usepackage{amsthm}
\usepackage{fancyhdr}
\usepackage{verbatim}
\usepackage{listings}
\usepackage{xcolor}
\usepackage{pgfplots}

\lstset{
    language=C++,
    basicstyle=\ttfamily\footnotesize,
    keywordstyle=\color{blue},
    commentstyle=\color{green},
    stringstyle=\color{red},
    numbers=left,
    numberstyle=\tiny\color{gray},
    frame=single,
    breaklines=true,
    captionpos=b,
}


\title{PHYS 325: Lecture 16}
\author{Cliff Sun}

\newtheorem{theorem}{Theorem}[section]
\newtheorem{lemma}[theorem]{Lemma}
\newtheorem{definition}[theorem]{Definition}
\newtheorem{conjecture}[theorem]{Conjecture}
\newtheorem{proposition}[theorem]{Proposition}
\newtheorem{corollary}[theorem]{Corollary}
\newtheorem{one minute paper}[theorem]{One Minute Paper}

\pagestyle{fancy}
\lhead{\textbf{\thepage}\ \ \nouppercase{\rightmark}}
\chead{PHYS 325: Lecture 16}
\rhead{Cliff Sun}

\begin{document}

\maketitle

\section*{Forced oscillators}

EOM:

\begin{equation}
    m_{eff}\ddot{x} + c_{eff}\dot{x} + kx = F(t)
\end{equation}

\subsection*{General Solution}

Introduce linear operator:

\begin{equation}
    \mathbb{L}[x] = F
\end{equation}

Linear operator is a map of a function to another function. Note that linear operators obey this condition:
\begin{equation}
    L(af + bg) = aL(f) + bL(g)
\end{equation}

\section*{Force as power series}

\begin{equation}
    F_{ext} = a + bt + ct^2 + \cdots
\end{equation}

\begin{equation}
    L(x) = F_{ext}
\end{equation}

Ansatz for $x_p(t)$

\begin{equation}
    x_p(t) \sim \alpha + \beta y + \gamma t^2 + \cdots 
\end{equation}

Given some EOM with 

\begin{equation}
    F = F_0\cos(\omega t)
\end{equation}
Then the particular solution is 

\begin{equation}
    x = \frac{F_0}{k}e^{-i\theta}\left[1 + 2i\zeta\frac{\omega}{\omega_n} - \left(\frac{\omega}{\omega_n}\right)^2\right]^{-1}
\end{equation}

\begin{equation}
    \iff \tilde{G(\omega)}F_0e^{-i\theta}
\end{equation}

Note 

\begin{equation}
    \tilde{G(\omega)} = G(\omega)e^{-i\phi(\omega)}
\end{equation}

\begin{equation}
    G(\omega) = |\tilde{G}(\omega)| = \frac{1}{k}\left[(1 - \frac{\omega^2}{\omega_n^2} + (2\zeta \frac{\omega}{\omega_n})^2)\right]^{-\frac{1}{2}}
\end{equation}

This is called the \underline{response function}. 

\end{document}
\documentclass{article}
\usepackage[left=3cm, right=3cm, top=2cm, bottom=2.5cm]{geometry}
\usepackage{graphicx}
\usepackage{amsmath}
\usepackage{amssymb}
\usepackage{amsthm}
\usepackage{fancyhdr}

\title{7.2 - 7.3}
\author{Cliff Sun}

\newtheorem{theorem}{Theorem}[section]
\newtheorem{lemma}[theorem]{Lemma}
\newtheorem{definition}[theorem]{Definition}

\pagestyle{fancy}
\lhead{\textbf{\thepage}\ \ \nouppercase{\rightmark}}
\chead{7.2 - 7.3}
\rhead{Cliff Sun}

\begin{document}

\maketitle

\begin{theorem}
    A \underline{relation} from A to B is a subset 
    \begin{equation}
        R \subseteq A \times B
    \end{equation}
    In other words
    \begin{equation}
        a R b \iff (a,b) \in R
    \end{equation}
\end{theorem}

\begin{theorem}
    A \underline{function} is a relation that satisfies the vertical line test. 
    \begin{equation}
        f: A \rightarrow B
    \end{equation}
    is the statement that for every a in the domain of f, there exists some value b such that $(a,b)$ passes the vertical line test.
\end{theorem}

The image is a function from P(A) to P(B). Similarly, the preimage/inverse image is from P(B) to P(A).

In notion it would be 

\begin{equation}
    f(U) = \{f(u): u \in U\}
\end{equation}

\begin{equation}
    f^{-1}(V) = \{a \in A: f(a) \in V\}
\end{equation}

respectively. Such that U is a set of A and V is a set of B. Notice how these functions spit out a set. 

\subsubsection*{What does it mean for a function to be equal to each other?}

\begin{enumerate}
    \item Same domain, same codomain, and graph
    \item Same domain and same graph, different codomains
\end{enumerate}

\section*{7.3: Equivalence Relations}

Assume 
\begin{equation}
    R = \{(a,b): a \equiv b \mod n\}
\end{equation}

Then $a \equiv a \mod n$, $a \equiv b \implies b \equiv a$, $a \equiv b \equiv c \implies a \equiv c$.

\begin{enumerate}
    \item R is reflexive if a R a for a in A
    \item R is symmetric if a R b $\implies$ b R a
    \item R is transitive if a R b $\land$ b R c $\implies$ a R c
\end{enumerate}

An equivalence relation satisfies all three requirements. They tell us that a \& b have something in common. 
\end{document}
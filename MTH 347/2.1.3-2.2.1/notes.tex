\documentclass{article}
\usepackage[left=3cm, right=3cm, top=2cm, bottom=2.5cm]{geometry}
\usepackage{graphicx}
\usepackage{amsmath}
\usepackage{amssymb}
\usepackage{amsthm}
\usepackage{fancyhdr}

\title{Subsequences and the Squeeze Theorem}
\author{Cliff Sun}

\newtheorem{theorem}{Theorem}[section]
\newtheorem{lemma}[theorem]{Lemma}
\newtheorem{definition}[theorem]{Definition}
\newtheorem{conjecture}[theorem]{Conjecture}
\newtheorem{proposition}[theorem]{Proposition}
\newtheorem{corollary}[theorem]{Corollary}
\newtheorem{one minute paper}[theorem]{One Minute Paper}

\pagestyle{fancy}
\lhead{\textbf{\thepage}\ \ \nouppercase{\rightmark}}
\chead{Subsequences and the Squeeze Theorem}
\rhead{Cliff Sun}

\begin{document}

\maketitle

Last time, we defined some function $(x_n)$ such that 
\begin{equation}
    x_1 = 1
\end{equation} 
\begin{equation}
    x_{n+1} = \sin(x_n) \quad \forall n \geq 1
\end{equation}
Previously, we showed that this function is strictly monotone decreasing and is bounded by 0. As well, by the monotone convergence theorem, 
this implies that
\begin{equation}
    \lim_{n\rightarrow\infty}x_n \textrm{ exists }
\end{equation}
We introduce some facts:
\begin{enumerate}
    \item If $f$ is a continuous function and $(x_n)$ is a sequence which converges to some limit $x$. Then the sequence $(f(x_n))$ converges to $f(x)$. 
    \item $\sin(x)$ is continuous.  
\end{enumerate}

Claim:
\begin{equation}
    \lim_{n\rightarrow\infty}x_n = 0
\end{equation}

\begin{proof}
    Let $x = \lim_{n\rightarrow\infty}x_n$ since we know that it exists. By facts 1 and 2, 
    \begin{equation}
        \lim_{n\rightarrow\infty}\sin(x_n) = \sin(x)
    \end{equation}
    But
    \begin{equation}
        \sin(x_n) = x_{n+1}
    \end{equation}
    and
    \begin{equation}
        \sin(x) = \lim_{n\rightarrow\infty}x_{n+1} = \lim_{n\rightarrow\infty}x_{n} = x
    \end{equation}
    Thus $x = 0$. 
\end{proof}

\section*{Subsequences}

Recall, last week we found that 
\begin{equation}
    \lim_{n\rightarrow\infty}\frac{\sin(n)}{n} = 0
\end{equation}
But what about
\begin{equation}
    \lim_{n\rightarrow\infty}\frac{\sin(2n)}{2n} \textrm{ ?}
\end{equation}
Or
\begin{equation}
    \lim_{n\rightarrow\infty}\frac{\cos(n)}{n} \textrm{ ?}
\end{equation}
Or
\begin{equation}
    \lim_{n\rightarrow\infty}\frac{\sin(n)}{n} + \frac{1}{2^n} \textrm{ ?}
\end{equation}

\begin{definition}
    Let $(x_n)$ be a sequence. Let $(n_i)_{i=1}^{\infty}$ be a strictly increasing sequence of natural numbers. Then we call the sequence
    \begin{equation}
        (x_{n_i})_{i=1}^{\infty}
    \end{equation}
    a \underline{subsequence} of $(x_n)$. 
\end{definition}

\begin{theorem}
    If $(x_n)$ is a sequence that converges to some number $x$, then every subsequence must converge to the same number. 
\end{theorem}

Note, if $(x_n)$ diverges, then its subsequences may or may not diverge. 

\begin{proof}
    We prove the subsequence theorem. Let $(x_n)$ be a sequence such that
    \begin{equation}
        x = \lim_{n\rightarrow\infty}x_n \textrm{ exists }
    \end{equation}
    In particular, for all $\epsilon > 0$, there exists $M \in \mathbb{N}$ such that for all $n \geq M$, we have that 
    \begin{equation}
        |x_n - x| < \epsilon
    \end{equation}
    Let $(n_i)$ be a sequence of indicies that are strictly increasing. In particular, we can prove that $n_i \geq i$. We claim that 
    \begin{equation}
        \lim_{i\rightarrow\infty} x_i = x
    \end{equation}
    Let $\epsilon > 0$, choose $M$ to be the same integer as stated in equation 13 with our $\epsilon$. In particular, for all $n \geq M$, we have that 
    \begin{equation}
        |x_n - x| < \epsilon
    \end{equation}  
    Let $i \geq M$ be arbitrary, since $n_i \geq i$, we have that
    \begin{equation}
        |x_{n_i} - x| < \epsilon
    \end{equation}
    as claimed. 
\end{proof}

\section*{Squeeze Theorem}

\begin{theorem}
    Suppose $(a_n)$, $(b_n)$, and $(x_n)$ are sequences such that 
    \begin{enumerate}
        \item $a_n \leq x_n \leq b_n$
        \item $\lim a_n = x = \lim b_n$
    \end{enumerate}
    Then
    \begin{equation}
        \lim x_n = x
    \end{equation}
\end{theorem}

\begin{proof}
    Suppose $(a_n)$, $(b_n)$, $(x_n)$ are stated. Then for all $\epsilon > 0$, there exists 2 numbers as follows. 
    \begin{enumerate}
        \item $M_1 \in \mathbb{N}$ such that for all $n \geq M_1$, we have that 
        \begin{equation}
            |a_n - x| < \epsilon
        \end{equation}
        \item $M_2 \in \mathbb{N}$ such that for all $n \geq M_2$, we have that 
        \begin{equation}
            |b_n - x| < \epsilon
        \end{equation}
    \end{enumerate}
    We claim that 
    \begin{equation}
        \lim x_n = x
    \end{equation}
    that is for all $\epsilon > 0$, there exists $M \in \mathbb{N}$ such that for all $n \geq M$, we have that 
    \begin{equation}
        |x_n - x| < \epsilon
    \end{equation}
    Let $\epsilon > 0$, plugging $\epsilon$ into equations 19 and 20, we get $M_1, M_2 \in \mbox{N}$. Then we choose $M = \max(M_1, M_2)$. 
    Let $n \geq M$, then $n \geq M_1 \land n \geq M_2$. Then in particular
    \begin{equation}
        x - \epsilon < a_n < x + \epsilon
    \end{equation}
    \begin{equation}
        x - \epsilon < b_n < x + \epsilon
    \end{equation}
    But we have that 
    \begin{equation}
        a_n \leq x_n \leq b_n
    \end{equation}
    Then we have that 
    \begin{equation}
        x - \epsilon < a_n \leq x_n \leq b_n < x + \epsilon
    \end{equation}
    Thus, 
    \begin{equation}
        x - \epsilon < x_n < x + \epsilon
    \end{equation}
    Thus
    \begin{equation}
        |x_n - x| < \epsilon
    \end{equation}
\end{proof}

\end{document}
\documentclass{article}
\usepackage[left=3cm, right=3cm, top=2cm, bottom=2.5cm]{geometry}
\usepackage{graphicx}
\usepackage{amsmath}
\usepackage{amssymb}
\usepackage{amsthm}
\usepackage{fancyhdr}

\title{8.1 - Cardinalities of Infinite Sets}
\author{Cliff Sun}

\newtheorem{theorem}{Theorem}[section]
\newtheorem{lemma}[theorem]{Lemma}
\newtheorem{definition}[theorem]{Definition}
\newtheorem{conjecture}[theorem]{Conjecture}

\pagestyle{fancy}
\lhead{\textbf{\thepage}\ \ \nouppercase{\rightmark}}
\chead{Cardinalities of Infinite Sets}
\rhead{Cliff Sun}

\begin{document}

\maketitle

Recall:

\begin{theorem}
    Let $A$ and $B$ be finite sets, then 
    \begin{equation}
        |A| = |B| \iff \exists F: A \rightarrow B \quad \textrm{Bijective}
    \end{equation}
    In addition:
    \begin{equation}
        |A| <= |B| \iff \exists f: A \rightarrow B \quad \textrm{Injective}
    \end{equation}
    Similarly:
    \begin{equation}
        |A| >= |B| \iff \exists f: A \rightarrow B \quad \textrm{Surjective}
    \end{equation}
\end{theorem}

\begin{definition}
    Let $A$ and $B$ be arbitrary sets, perhaps infinite. Then the following is true:
    \begin{enumerate}
        \item $|A| = |B|$ if there exists a bijection from A to B
        \item $|A| <= |B|$ if there exists an injective function from A to B
        \item $|A| < |B|$ if there exists only an injective function. 
    \end{enumerate}
\end{definition}

But $|A|$ hasn't been explicitly stated yet. And it's not a number or $\infty$. Then we claim that 
\begin{equation}
    |\mathbb{N}| = |2\mathbb{N}|
\end{equation}
Or that the cardinality of the natural numbers is the same as that of the even natural numbers. 

\begin{proof}
    To prove this claim, we must give a bijective function from $\mathbb{N} \rightarrow 2\mathbb{N}$. Then choose a function as the following:
    \begin{equation}
        f: \{\mathbb{N} \rightarrow 2\mathbb{N} : f(x) = 2x \}
    \end{equation} 
    This is bijective, thus we have proved the claim. 
\end{proof}

Claim: $|\mathbb{Z}| <= |\mathbb{R}|$

\begin{proof}
    THe function $f: \mathbb{Z} \rightarrow \mathbb{R}$ such that $f(n) = n$ is injective.
\end{proof}

\begin{conjecture} 
On any collection of sets, the relation 
\begin{equation}
    A \sim B \iff |A| = |B|
\end{equation}
is an equivalence relation. That is that this relation is reflexive, symmetric, and transitive.
\end{conjecture}

\begin{conjecture}
    $|A| = |B| \implies |A| <= |B|$
\end{conjecture}

\begin{conjecture}
    $|A| <= |B| <= |C| \implies |A| <= |C|$
\end{conjecture}

\begin{conjecture}
    $|A| <= |B| <= |A| \implies |A| = |B|$
\end{conjecture}

We first prove the first conjecture.

\begin{proof}
    To prove that this relation is in-fact an equivalence relation, we must first prove its reflexivity. That it
    for any set A, there exists a bijection from A to A. To choose such a function, we choose the identity funtion, that it 
    \begin{equation}
        f: A \rightarrow A \textrm{ such that } f(n) = n
    \end{equation}
    To prove symmetry, if $f: A \rightarrow B$ is a bijection, then we choose $f^{-1}$ as the bijective function from B to A. Thus this relation is symmetric. Next to prove 
    that this relation is transitive. First assume that 
    \begin{equation}
        f: A \rightarrow B \textrm{ is bijective }
    \end{equation}
    and that $|B| = |C|$, that is 
    \begin{equation}
        g: B \rightarrow C \textrm{ is bijective, }
    \end{equation} 
    Then we choose a function $g \circ f$ is bijective. 
\end{proof}

\begin{definition}
    If $A$ is a set, then its \underline{cardinality} $|A|$ is defined to be its equivalence class under the previous relation. 
\end{definition}

\section*{Countably Infinite Sets}

\begin{definition}
    \begin{enumerate}
    \item $|\mathbb{N}$ is called $\aleph_0$
    \item We call a set $A$ \underline{countably infinite} (denumerable) $\iff$ $|A| = \aleph_0$.
    \item We call a set countable if $|A| <= \aleph_0$
    \end{enumerate}
\end{definition}



\end{document}
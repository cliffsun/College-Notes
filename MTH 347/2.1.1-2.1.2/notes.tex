\documentclass{article}
\usepackage[left=3cm, right=3cm, top=2cm, bottom=2.5cm]{geometry}
\usepackage{graphicx}
\usepackage{amsmath}
\usepackage{amssymb}
\usepackage{amsthm}
\usepackage{fancyhdr}

\title{Monotone Sequences and Tails of Sequences}
\author{Cliff Sun}

\newtheorem{theorem}{Theorem}[section]
\newtheorem{lemma}[theorem]{Lemma}
\newtheorem{definition}[theorem]{Definition}
\newtheorem{conjecture}[theorem]{Conjecture}
\newtheorem{proposition}[theorem]{Proposition}
\newtheorem{corollary}[theorem]{Corollary}
\newtheorem{one minute paper}[theorem]{One Minute Paper}

\pagestyle{fancy}
\lhead{\textbf{\thepage}\ \ \nouppercase{\rightmark}}
\chead{Monotone Sequences and Tails of Sequences}
\rhead{Cliff Sun}

\begin{document}

\maketitle

\begin{definition}
    Let $(x_n)$ be a sequence:
    \begin{enumerate}
        \item $(x_n)$ is (monotone) increasing if $x_{n+1} \geq x_n$ for all $n$
        \item $(x_n)$ is (monotone) decreasing if $x_{n+1} \leq x_n$ for all $n$
        \item $(x_n)$ is monotone if $(x_n)$ is either monotone increasing or monotone decreasing
    \end{enumerate}
\end{definition}

\begin{theorem}
    Let $(x_n)$ be a montone sequence. 
    \begin{enumerate}
        \item $(x_n)$ converges if and only if it is bounded.
        \item If $(x_n)$ is increasing and bounded, then 
        \begin{equation}
            \lim_{n \rightarrow \infty}x_n = \sup\{x_n : n \in \mathbb{N}\}
        \end{equation}
        \item If $(x_n)$ is decreasing and bounded, then 
        \begin{equation}
            \lim_{n\rightarrow\infty}x_n = \inf(x_n : n \in \mathbb{N})
        \end{equation}
    \end{enumerate}
\end{theorem}

Note:
\begin{enumerate}
    \item If $(x_n)$ is increasing, then $(x_n)$ is bounded below by $x_1$.
    \item If $(x_n)$ is decreasing, then $(x_n)$ is bounded above $x_1$
    \item The only interesting question is if $(x_n)$ is bounded above/below if it's increasing/decreasing. 
\end{enumerate}

\begin{corollary}
    The comparison test for infinite series says that if $\Sigma a_n$ and $\Sigma b_n$ are infinite series and $a_n, b_n \geq 0$, and $a_n \leq b_n$ for all n.
    Then if $\Sigma b_n$ converges, then $\Sigma a_n$ converges. 
\end{corollary}

\begin{definition}
    The sum of some series $a_n$ is the limit as $n \rightarrow \infty$ of the sequence $s_n$ where 
    \begin{equation}
        s_n = a_1 + a_2 + \cdots + a_n
    \end{equation}
\end{definition}

\begin{proof}
    We already know the "$\implies$" of $(1)$, we will prove $(2)$. By replacing $x_n = -x_n$, we will prove $(3)$. And combining $(2)$ and $(3)$, we will have proved "$\impliedby$" of $(1)$. So suppose that $(x_n)$ is increasing and bounded above. Since $x_n$ is nonempty and bounded above, we have that it must have a least upper bound, in other words it has a supremum. So we must prove that
    \begin{center}
        For all $\epsilon > 0$, there exists some $M \in \mathbb{N}$ such that for all $n \geq M$, $|x_n - x| < \epsilon$ where $x = \sup(x_n)$.
    \end{center}
    Recall from homework 11, for any nonempty, bounded above set $S$, and for any $\epsilon > 0$, there exists an $x \in S$ such that 
    \begin{equation}
        \sup(S) - \epsilon < x \leq \sup(S)
    \end{equation}
    In particular, given any $\epsilon > 0$, there exists some $M \in \mathbb{N}$ such that $\sup(x_n) - \epsilon < x_M \leq \sup(x_n)$. Choose that $M$, then we have that 
    \begin{equation}
        x - \epsilon < x_M \leq x_n \leq x < x + \epsilon
    \end{equation}
    But in particular
    \begin{equation}
        |x_n - x| < \epsilon
    \end{equation}
    This concludes the proof. 
\end{proof}

\section*{Tails of a sequence}

\begin{definition}
    Let $(x_n)$ be a sequence, and let $k \in \mathbb{N}$. Then the k-Tail of $(x_n)$ is the sequence 
    \begin{equation}
        (x_{n+k})_{n=1}^{\infty} = (x_{k+1}, x_{k+2}, \cdots)
    \end{equation}
    In other words, you drop the first $k$ terms. 
\end{definition}

\begin{theorem}
    If $(x_n)$ is any sequence, $x \in \mathbb{R}$, and $k \in \mathbb{N}$, then 
    \begin{equation}
        \lim_{n\rightarrow\infty}(x_n) = x \iff \lim_{n\rightarrow\infty}(x_{n+k}) = x
    \end{equation}
\end{theorem}

In particular
\begin{enumerate}
    \item If $(x_n)$ converges to x, then all of its k-tails must also converge to the same number $x$. 
    \item If $(x_n)$ diverges, then all of its k-tails also diverge. 
\end{enumerate}

\end{document}
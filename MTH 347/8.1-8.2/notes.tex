\documentclass{article}
\usepackage[left=3cm, right=3cm, top=2cm, bottom=2.5cm]{geometry}
\usepackage{graphicx}
\usepackage{amsmath}
\usepackage{amssymb}
\usepackage{amsthm}
\usepackage{fancyhdr}

\title{8.1-8.2}
\author{Cliff Sun}

\newtheorem{theorem}{Theorem}[section]
\newtheorem{lemma}[theorem]{Lemma}
\newtheorem{definition}[theorem]{Definition}
\newtheorem{conjecture}[theorem]{Conjecture}

\pagestyle{fancy}
\lhead{\textbf{\thepage}\ \ \nouppercase{\rightmark}}
\chead{8.1-8.2}
\rhead{Cliff Sun}

\begin{document}

\maketitle

\begin{theorem}
    $\mathbb{Q}$ is countably infinite. 
\end{theorem}

First, we prove that $\mathbb{Q}^+$ is countably infinite.

\begin{proof}
    We can list all the positive rational numbers in a 2-d grid, that is

    \begin{center}
        \begin{tabular}{c c c}
            $1/1$ & $2/1$ & $\dots$ \\
            $1/2$ & $2/2$ & $\dots$ \\
            $1/3$ & $2/3$ & $\dots$
         \end{tabular}
    \end{center}

     We can cross out all the repeat values, then this grid reduces to the following list:

     \begin{center}
        \begin{tabular}{ |c|c|c|c| } 
            n & 1 & 2 & 3\\
            f(n) & $\frac{1}{1}$ & $\frac{2}{1}$ & $\frac{1}{3}$
        \end{tabular}
     \end{center}
     This list would go on forever, and encapsulate all the possible rational numbers. It follows that this function $f(n)$ is a bijection from $\mathbb{N} \Rightarrow \mathbb{Q}^+$. Thus 
     \begin{equation}
        |\mathbb{Q}^+| = |\mathbb{N}| = \aleph_0
     \end{equation}
     Thus, this concludes this proof. 
\end{proof}

Similarly, we can also use this proof when constructing a function that maps from $\mathbb{Z} \rightarrow \mathbb{Q}$, so that $|\mathbb{Z}| = |\mathbb{Q}|$. But it follows that $|\mathbb{Z}| = \aleph_0$. Thus it follows that the rational numbers is in fact countably infinite. 

\begin{theorem}
    In regards to countable sets, we have the following theorems:
    \begin{enumerate}
        \item If $A \subseteq B$, then $|A| \leq |B|$.
        \item The union of countably many countable sets is countable.
        \item The cartesian product of finitely many countable sets is always countable.
    \end{enumerate}
\end{theorem}

Applications:

Let $\mathbb{Z}[x]$ denote the set of polynomials in x with integer coefficients. Then this set is countable. 

\begin{proof}
    For each $n \in \mathbb{N}$, let 
    \begin{equation}
        A_n = \{f \in \mathbb{Z}[x] : deg(f) \leq n\}
    \end{equation}
    Then we can prove that each $A_n$ is countable using a bijection:
    \begin{equation}
        g: \mathbb{Z}^{n+1} \rightarrow A_n
    \end{equation}
    (Think of it like a dot product)
    Then since $\mathbb{Z}^{n+1}$ is countable, and since $\mathbb{Z}[x]$ is the union of all $A_n$, it follows that $\mathbb{Z}[x]$ is countable. 
\end{proof}

\begin{lemma}
    There are only countably many algebraic numbers. 
\end{lemma}

Recall: a number if algebraic if it's a root of a non-zero polynomial in $\mathbb{Z}[x]$.

\begin{proof}
    For each nonzero $f \in \mathbb{Z}[x]$, let 
    \begin{equation}
        B_f = \{\textrm{Zeros of F}\}
    \end{equation}
    Each $B_f$ is finite, thus the union of all $B_f$ is countable. Thus, we conclude the proof. 
\end{proof}

\begin{theorem}
    $[0,1]$ is uncountable. 
\end{theorem}

\begin{proof}
    We must do 2 things
    \begin{enumerate}
        \item Prove that $\exists f : \mathbb{N} \rightarrow [0,1]$ is injective.
        \item Prove that $\nexists g : \mathbb{N} \nrightarrow [0,1]$ is surjective. 
    \end{enumerate}
    For (1), define a function $f = \frac{1}{n}$

    For (2), assume that such a function does exist for the sake of contradiction. 
\end{proof}

\end{document}
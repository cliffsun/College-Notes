\documentclass{article}
\usepackage[left=2cm, right=2cm, top=2cm, bottom=2.5cm]{geometry}
\usepackage{graphicx}
\usepackage{amsmath}
\usepackage{amssymb}
\usepackage{amsthm}
\usepackage{fancyhdr}

\title{6.2/6.3 - Power Sets and Indexed Collections of Sets}
\author{Cliff Sun}

\newtheorem{theorem}{Theorem}[section]
\newtheorem{lemma}[theorem]{Lemma}
\newtheorem{definition}[theorem]{Definition}

\pagestyle{fancy}
\lhead{\textbf{\thepage}\ \ \nouppercase{\rightmark}}
\chead{6.2/6.3 - Power Sets and Indexed Collections of Sets}
\rhead{Cliff Sun}

\begin{document}

\maketitle

\begin{theorem}
    If $A \subseteq B$, then $P(A) \subseteq P(B)$
\end{theorem}

\begin{proof}
    Suppose that $A \subseteq B$, then to prove that $P(A) \subseteq P(B)$,
    choose c to be an element of $P(A)$. That is $c \in A$. Thus, $c \in A \in B$, 
    thus it follows that $c \in P(B)$. This proves the theorem that $P(A) \in P(B)$. 
\end{proof}

How do power sets relate to Cartesian Products? That is Is $P(A \times B) = P(A) \times P(B)$? \\

Suppose that that $|A| = a$ and $|B| = b$, then $P(A \times B) = 2^{ab}$ and $P(A) \times P(B) = 2^{a + b}$.\\

However, suppose that A and B are disjoint, that is they don't have any elements in common. Then the cardinality of 
$P(A) \times P(B)$ and $P(A \cup B)$ have the same cardinality. Therefore, there must be a bijection between them. 

\section*{Indexed Collections of Sets}

Let I be any set, call it an "Indexed Set". Then, for $n \in I$, suppose that we have some set called $A_n$. 
Generally, $I$ could be the natural numbers, the real numbers, etc. Given this set-up, we can write down some definitions:

\begin{theorem}
    The collection of all of these sets is called an "Index Collection Of Sets", written in mathematical notation is the following:

    \begin{equation}
        \{A_n: n \in I\}
    \end{equation}
\end{theorem}


\begin{theorem}
    $\cup_{n \in I} A_n = \{x: x \in A_n \textrm{for some} n \in I\}$ \\
    $\cap_{n \in I} A_n = \{x: x \in A_n \forall n \in I\}$ \\
    The collection $\{A_n: n \in I\}$ is pairwise disjoint if we have that $A_n \cap A_m$ for $m \neq n$ in I. 
\end{theorem}

Simply speaking, this $\cup$ represents the union of all sets, where $\cap$ is the intersection of all sets. \\

General fact: 

\begin{enumerate}
    \item For any $m \in I$, $A_m$ is always a subset of the union of the Sets
    \item For any $n \in I$, the intersection of all of the sets will be a subset of $A_n$
\end{enumerate}


\end{document}
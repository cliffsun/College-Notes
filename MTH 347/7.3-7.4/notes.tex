\documentclass{article}
\usepackage[left=3cm, right=3cm, top=2cm, bottom=2.5cm]{geometry}
\usepackage{graphicx}
\usepackage{amsmath}
\usepackage{amssymb}
\usepackage{amsthm}
\usepackage{fancyhdr}

\title{7.2-7.3; Equivalence Relations and Partitions}
\author{Cliff Sun}

\newtheorem{theorem}{Theorem}[section]
\newtheorem{lemma}[theorem]{Lemma}
\newtheorem{definition}[theorem]{Definition}

\pagestyle{fancy}
\lhead{\textbf{\thepage}\ \ \nouppercase{\rightmark}}
\chead{7.2-7.3}
\rhead{Cliff Sun}

\begin{document}

\maketitle

To recap, if R is an Equivalence relation, then 

\begin{equation}
    x R y \iff \textrm{x and y have something in common}
\end{equation}

Then abstractly, each piece in $A$ has something in commmon with each other. 

\begin{definition}
    Let $\sim$ be an Equivalence relation on a set $A$. 
    \begin{enumerate}
        \item For an $x \in A$, the equivalence class of x is $[x] = \{z \in A: z \sim x\}$
        \item The quotient of A by $\sim$ is the set of all equivalence classes for each element $x \in A$
    \end{enumerate}
\end{definition}

\section*{Partitions}

\begin{definition}
    Let X be a set, and let $\{A_n: n \in I\}$ be a collection of non-empty subsets of X.
    We call this a paritition if 2 things are true
    \begin{enumerate}
        \item The Union of all $A_n$'s is X
        \item They are all pairwise disjoint, in other words. $\forall m,n \in I$, either $A_m = A_n$ or $A_m \cap A_n = \emptyset$
    \end{enumerate}
\end{definition}

Given this defintion, we can relate it to Equivalence relations by stating 
\begin{equation}
    m \sim n \iff \textrm{m and n are in the same piece of the partition}
\end{equation}

\begin{lemma}
    Let $\sim$ be an Equivalence relation on a set A, Then the following are Equivalent:
    \begin{equation}
        \forall x,y \in A, x \sim y \iff [x] = [y]
    \end{equation}
\end{lemma}

\end{document}
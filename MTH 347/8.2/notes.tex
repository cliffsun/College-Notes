\documentclass{article}
\usepackage[left=3cm, right=3cm, top=2cm, bottom=2.5cm]{geometry}
\usepackage{graphicx}
\usepackage{amsmath}
\usepackage{amssymb}
\usepackage{amsthm}
\usepackage{fancyhdr}

\title{Generic Homework}
\author{Cliff Sun}

\newtheorem{theorem}{Theorem}[section]
\newtheorem{lemma}[theorem]{Lemma}
\newtheorem{definition}[theorem]{Definition}
\newtheorem{conjecture}[theorem]{Conjecture}

\pagestyle{fancy}
\lhead{\textbf{\thepage}\ \ \nouppercase{\rightmark}}
\chead{Generic Homework}
\rhead{Cliff Sun}

\begin{document}

\maketitle

Recall:

\begin{enumerate}
    \item $\mathbb{Q}$ is countable.
    \item $\bar{\mathbb{Q}} = \{\textrm{Algebraic numbers}\}$ is countable.
    \item $[0,1]$ is uncountable. This implies that $\mathbb{R}$ is also uncountable. 
\end{enumerate}

\begin{corollary}
    $\mathbb{R}/\mathbb{Q}$ and $\mathbb{R}/\bar{\mathbb{Q}}$ are uncountable. 
\end{corollary}

\begin{proof}
    Suppose for contradiction that $\mathbb{R}/\mathbb{Q}$ is countable. Then $\mathbb{R}$ is the union of both of these countable sets. Thus $\mathbb{R}$ is also countable. This is a contradiction, thus $\mathbb{R}/\mathbb{Q}$ is uncountable. 
\end{proof}

\begin{theorem}
    Cantor's theorem. For any set $A$, $|A| < |P(A)|$. 
\end{theorem}

\begin{proof}
    We first must show that there exists an injective function from A to its powerset. Secondly, there cannot exist a bijective function from A to its powerset. We first choose
    \begin{equation}
        f(a) = \{a\}
    \end{equation}
    This is injective if $\{a\} = \{b\}$, then $a = b$. Secondly, suppose for the sake of contradiction that g is a bijection from A to its powerset. In particular, g is surjective. However, we state that there exists some set
    $X$ such that $X$ is not an element of $g(a_i)$ for $a_i \in A$. We choose $X$ to be:
    \begin{equation}
        X = \{a \in A : a \notin g(a)\}
    \end{equation}
    Then we claim that X cannot be in $g(a_i)$. 
    \begin{proof}
        By definition, b is in X iff
        \begin{equation}
            b \in X \iff b \notin g(b)
        \end{equation}
        But that means that $X \neq g(b)$ since one contains b and the other does not. 
    \end{proof}
\end{proof}

\end{document}
\documentclass{article}
\usepackage[left=3cm, right=3cm, top=2cm, bottom=2.5cm]{geometry}
\usepackage{graphicx}
\usepackage{amsmath}
\usepackage{amssymb}
\usepackage{amsthm}
\usepackage{fancyhdr}

\title{Fourier Series}
\author{Cliff Sun}

\newtheorem{theorem}{Theorem}[section]
\newtheorem{lemma}[theorem]{Lemma}
\newtheorem{definition}[theorem]{Definition}

\pagestyle{fancy}
\lhead{\textbf{\thepage}\ \ \nouppercase{\rightmark}}
\chead{Fourier Series}
\rhead{Cliff Sun}

\begin{document}

\maketitle

Recap: previously we talked about the Fourier Sine and Cosine Series. Such that
we can express functions as a sum of sine functions. Same concept for the cosine series. 


\begin{definition}
    The Fourier Sine Series is defined as the following:
\begin{equation}
    \phi(x) = \sum_{n=1}^{\infty}A_n\sin(\frac{n\pi x}{l})
\end{equation}

such that 

\begin{equation}
    A_n = \frac{2}{l}\int_{0}^{l}\phi(x)\sin(\frac{n\pi x}{l})dx
\end{equation}
\end{definition}

Similarly:

\begin{definition}
    The Fourier cosine Series is defined as the following:
\begin{equation}
    \phi(x) = \frac{A_0}{2} + \sum_{n=1}^{\infty}A_n\cos(\frac{n\pi x}{l})
\end{equation}

such that 

\begin{equation}
    A_n = \frac{2}{l}\int_{0}^{l}\phi(x)\cos(\frac{n\pi x}{l})dx
\end{equation}
\end{definition}

Then, we arrive at the full Fourier Series:

\begin{definition}
    Let $\psi(x)$ be a function with a period of 2l. Then $\psi(x)$ is equal to 
    \begin{equation}
        \psi(x) = \frac{A_0}{2} + \sum_{n=1}^{\infty}\{A_n\cos(\frac{n\pi x}{l}) + B_n\sin(\frac{n\pi x}{l})\}
    \end{equation}
    such that 
    \begin{equation}
        A_n = \frac{}{l}\int_{0}^{l}\psi(x)\cos(\frac{n\pi x}{l})dx
    \end{equation}
    and
    \begin{equation}
        B_n = \frac{}{l}\int_{0}^{l}\psi(x)\sin(\frac{n\pi x}{l})dx
    \end{equation}
\end{definition}

To find such a $\psi(x)$, we choose some function $\phi(x)$ such that $\phi(x)$ is odd and has a period of 2l. This will mean that
there will exist discontinuities throughout the function, but is necessary for the Full Fourier Series. 


As well, we notice that the sine series appears to be approximating the odd extension of the function, in other words, does better in approximating odd functions. 
And vice versa for the cosine series. In other words, we see that every function can be broken down as the following:
\begin{equation}
    f(x) = f_{odd}(x) + f_{even}(x)
\end{equation}

Such that the odd extension can be approximated with the sine series and the latter being approximated with the cosine series. 

In general, taking the sine fourier series of a function is equivalent to taking the full fourier series of the odd extension. 

\begin{proof}
    Suppose $\phi(x)$ can be written as a sum of sines. And $A_n$ is defined appropriately. Then we must show that 
    \begin{equation}
        \phi_{odd}(x) = \frac{\bar{A_0}}{2} + \sum_{n=1}^{\infty}\{\bar{A_n}\cos(\frac{n\pi x}{l}) + \bar{B_n}\sin(\frac{n\pi x}{l})\}
    \end{equation}
    Such that $\frac{\bar{A_n}}{2}$ is zero and $\frac{\bar{B_n}}{2}$ is non zero and is the inner product with the odd extension of $\phi(x)$. 

    Taking an informal argument says that the integral
    \begin{equation}
        \bar{A_n} = \int_{-l}^{l}\phi_{odd}(x)\cos(\frac{n\pi x}{l})dx
    \end{equation}
    is nothing but an integral of an odd function, which by definition is zero. Similarly, for $B_n$, we see that we are taking the integral of an even function which is evidently non-zero.
    Thus, this concludes the proof. 
\end{proof}

The same argument can be made for the even extension as well, the proof follows similarly. 


\end{document}
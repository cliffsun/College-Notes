\documentclass{article}
\usepackage[left=3cm, right=3cm, top=2cm, bottom=2.5cm]{geometry}
\usepackage{graphicx}
\usepackage{amsmath}
\usepackage{amssymb}
\usepackage{amsthm}
\usepackage{fancyhdr}

\title{Exam 2 Review}
\author{Cliff Sun}

\newtheorem{theorem}{Theorem}[section]
\newtheorem{lemma}[theorem]{Lemma}
\newtheorem{definition}[theorem]{Definition}
\newtheorem{conjecture}[theorem]{Conjecture}
\newtheorem{corollary}[theorem]{Corollary}

\pagestyle{fancy}
\lhead{\textbf{\thepage}\ \ \nouppercase{\rightmark}}
\chead{Exam 2 Review}
\rhead{Cliff Sun}

\begin{document}

\maketitle

Topics covered on the exam:

\begin{enumerate}
    \item Separation of variables
    \item Fourier Series
    \item Harmonic Functions
\end{enumerate}

For (1), we first assume that 
\begin{equation}
    u(x,y) = f(x)g(y)
\end{equation}


For (2), we've discussed the full, sine, and cosine Fourier Series. For Full Fourier Series with period 2l, we have that
\begin{equation}
    \phi(x) = \frac{A_0}{2} + \sum_{n=1}^{\infty}(A_n\cos(\frac{n\pi x}{l}) + B_n\sin(\frac{n\pi x}{l}))
\end{equation}

Where 

\begin{equation}
    A_n = \frac{1}{l}\int_{-l}^{l}\phi(x)\cos(\frac{n\pi x}{l})dx
\end{equation}

and

\begin{equation}
    B_n = \frac{1}{l}\int_{-l}^{l}\phi(x)\sin(\frac{n\pi x}{l})dx
\end{equation}

For the Sine Fourier Series defined on the interval $[0,l]$ is the following:

\begin{equation}
    \phi(x) = \sum_{n=1}^{\infty}A_n\sin(\frac{n\pi x}{l})
\end{equation}

with

\begin{equation}
    A_n = \frac{2}{l}\int_{0}^{l}\phi(x)\sin(\frac{n\pi x}{l})dx
\end{equation}

And for Cosine defined on the interval $[0,l]$, we have the following

\begin{equation}
    \phi(x) = \frac{B_0}{2} + \sum_{n=1}^{\infty}B_n\cos(\frac{n\pi x}{l})
\end{equation}

and

\begin{equation}
    B_n = \frac{2}{l}\int_{0}^{l}\phi(x)\cos(\frac{n\pi x}{l})dx
\end{equation}

\begin{equation}
    B_0 = \frac{2}{l}\int_{0}^{l}\cos(\frac{n\pi x}{l})dx
\end{equation}

For full fourier series to work, the function must have a period of 2l. You can always redefine $l$ to fit the criteria. ]

\section*{Convergence}

Let $\Psi(x)$ be a function of period $2l$ defined by equation (2), then we have the following theorems for Convergence. 

\begin{theorem}
    If $\Psi(x)$ is differentiable at $x = x_0$, then the Fourier Series convergences to $\Psi(x)$ at $x = x_0$. (\underbar{Pointwise Convergence})
\end{theorem}

\begin{theorem}
    If $\Psi(x)$ is continuously differentiable, then the Fourier Series converges to $\Psi(x)$ on $\mathbb{R}$. (\underbar{Uniform Convergence})
\end{theorem}

\begin{theorem}
    The Fourier Series converges to $\Psi(x)$ on $L^2$ on $[-l, l]$. We also have Parseval's equality:
    \begin{equation}
        \int_{-l}^{l}|\Psi(x)|^2dx = l[\frac{A_0^2}{2} + \sum_{n=1}^{\infty}(A_n^2 + B_n^2)]
    \end{equation}
    (\underbar{$L^2$ convergence})
\end{theorem}

Here are Parseval's equalities for sine and cosine:

Sine:

\begin{equation}
    \int_{0}^{l}|\Psi(x)|^2dx = \frac{l}{2}[\sum_{n=1}^{\infty}A_n^2]
\end{equation}

Cosine:

\begin{equation}
    \int_{0}^{l}|\Psi(x)|^2dx = \frac{l}{2}[\frac{A_0^2}{2} + \sum_{n=1}^{\infty}B_n^2]
\end{equation}

\section*{Harmonic Functions}

Harmonic Functions in a Rectangle D. Assume that the edges are the following: top = $g_2(x)$, bottom = $g_1(x)$, left = $h_1(y)$, right = $h_2(y)$.
Then the unique function is as follows:

\begin{equation}
    u(x,y) = \sum_{}^{}\alpha_n e^{\frac{n\pi}{a}y}sin(\frac{n\pi x}{l}) + \beta_n e^{-\frac{n\pi}{a}y}sin(\frac{n\pi x}{l}) + \textrm{ the sine series for the other boundary. }
\end{equation}

An easy form of this formula will be on the test. The $\sin$'s variable depends on the location of the zeros. That is if the zeros go up and down, then $\sin$'s variable is y, and vice versa.
The length is the length of the rectangle in the direction of propagation.

\subsection*{Maximum Principle}

\begin{equation}
    max_{\bar{D}}u = max_{\partial D}u
\end{equation}

\subsection*{Invariance of Laplace Equation}

Laplace equation is invariant under rigid motion. \underline{But not covered on test}. 

\end{document}
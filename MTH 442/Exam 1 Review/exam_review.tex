\documentclass{article}
\usepackage[left=3cm, right=3cm, top=2cm, bottom=2.5cm]{geometry}
\usepackage{graphicx}
\usepackage{amsmath}
\usepackage{amssymb}
\usepackage{amsthm}
\usepackage{fancyhdr}

\title{Math 442 Exam \#1 Review}
\author{Cliff Sun}

\pagestyle{fancy}
\lhead{\textbf{\thepage}\ \ \nouppercase{\rightmark}}
\chead{Math 442 Exam \#1 Review}
\rhead{Cliff Sun}

\begin{document}

\maketitle
\section*{Topics:}
\begin{enumerate}
    \item What is a PDE?
    \item What is a degree of a PDE?
    \item Whether a PDE is linear or not
    \item First order linear equation - Method of characteristics, change of variables
    \item Homogenous and non-homogenous Wave equation 
    \item Heat Equation - maximum principle, diffusion on the half line
\end{enumerate}

\section*{Notes}

First order linear pde:

\begin{equation}
    a(x,y)u_{x}(x,y) + b(x,y)u_{y}(x,y) = 0
\end{equation}

Wave equation:

\begin{equation}
    u_{tt} - c^2u_{xx} = f(x,t)
\end{equation}

\begin{equation}
    u(x,0) = \phi(x)
\end{equation}

\begin{equation}
    u_t(x,0) = \Psi(x)
\end{equation}

The general solution is:

\begin{equation}
    u(x,t) = \frac{1}{2}(\phi(x - ct) + \phi(x + ct)) + \frac{1}{2c}\int_{x-ct}^{x+ct}\Psi(s)ds
\end{equation}

The heat equation is:

\begin{equation}
    u_t(x,t) - ku_{xx}(x,t) = f(x,t) \textrm{for} (x,t) \in \mathbb{R} \times (0,\infty)
\end{equation}

\begin{equation}
    u(x,0) = \psi(x)
\end{equation}

The general solution is:

\begin{equation}
    u(x,t) = \int_{-\infty}^{\infty}S(x-y,t)\psi(y)dy + \int_{0}^{t}\int_{-\infty}^{\infty}S(x-y,t-s)f(y,s)dyds
\end{equation}

Where 

\begin{equation}
    S(x,y) = \frac{1}{\sqrt{4\pi kt}}e^{-\frac{x^2}{4kt}}
\end{equation}

The error function is the following:

\begin{equation}
    Erf(x) = \frac{2}{\pi}\int_{0}^{x}e^{-p^2}dp
\end{equation}

Example of linearity within the heat equation:

\begin{equation}
    u_t - ku_{xx} = \cos(x) 
\end{equation}
\begin{equation}
    u(x,0) = \sin(x)
\end{equation}

We know that the solution to the homogenous heat equation with the same initial condition is just:

\begin{equation}
    v(x,t) = e^{-kt}\sin(x)
\end{equation}

Thus applying linearity, we have that:

\begin{equation}
    u(x,t) - v(x,t) = w(x,t)
\end{equation}

Applying this yields:

\begin{equation}
    w_t - kw_{xx} = \cos(x)
\end{equation}

Choosing $w_1(x,t)$ to be $\frac{1}{k}\cos(x)$, we see that now:

\begin{equation}
    w_2 = w - w_1
\end{equation}

Then, we have that the new heat equation becomes homogenous with an intial condition of $-\frac{1}{k}\cos(x)$. Now applying linearity, we can rederive the full solution. 


The maximum principle states that 

\begin{equation}
    \textrm{Max}_{\gamma}(u(x,t)) = \textrm{Max}_{\mathbb{R}}(u(x,t))
\end{equation}

Such that $\gamma \in \mathbb{R}$ and that $\gamma$ is the boundary of $\mathbb{R}$.


\end{document}
\documentclass{article}
\usepackage[left=3cm, right=3cm, top=2cm, bottom=2.5cm]{geometry}
\usepackage{graphicx}
\usepackage{amsmath}
\usepackage{amssymb}
\usepackage{amsthm}
\usepackage{fancyhdr}

\title{Fourier Transform and Complex Numbers}
\author{Cliff Sun}

\newtheorem{theorem}{Theorem}[section]
\newtheorem{lemma}[theorem]{Lemma}
\newtheorem{definition}[theorem]{Definition}
\newtheorem{conjecture}[theorem]{Conjecture}
\newtheorem{proposition}[theorem]{Proposition}
\newtheorem{corollary}[theorem]{Corollary}
\newtheorem{one minute paper}[theorem]{One Minute Paper}

\pagestyle{fancy}
\lhead{\textbf{\thepage}\ \ \nouppercase{\rightmark}}
\chead{Fourier Transform and Complex Numbers}
\rhead{Cliff Sun}

\begin{document}

\maketitle

\begin{one minute paper}
    Before: MTH 442, After: $i = e^{i\frac{\pi}{2}}$
\end{one minute paper}

\section*{Complex Numbers}

Magnitude of Complex Numbers

\begin{equation}
    |z| = \sqrt{\bar{z}z}
\end{equation}

Euler's Identity

\begin{equation}
    e^{i\theta} = \cos(\theta) + i\sin(\theta)
\end{equation}

\begin{equation}
    z = re^{i\theta} \textrm{ polar form in Complex Coordinates }
\end{equation}

Suppose the wave equation

\begin{equation}
    u_{tt} - v^2u_{xx} = 0
\end{equation}

Then we guess 

\begin{equation}
    f = Ae^{iwt}e^{ikx}
\end{equation}

We plug our guess into the wave equation

\begin{equation}
    A(iw)^2e^{iwt}e^{ikx} = v^2A(ik^2)e^{iwt}e^{ikx}
\end{equation}

\begin{equation}
    w^2 = v^2k^2
\end{equation}

This is called the \underline{dispersion relation}. 

In general, the law of superposition holds for linear PDE's. Thus, an integral also holds

\begin{equation}
    f(x,t) = \int_{-\infty}^{\infty}dk\tilde{f}(k)e^{ikx}e^{ivkt}
\end{equation}

This is called the fourier transform of $\widehat{f}$. 

\end{document}
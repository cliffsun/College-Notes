\documentclass{article}
\usepackage[left=3cm, right=3cm, top=2cm, bottom=2.5cm]{geometry}
\usepackage{graphicx}
\usepackage{amsmath}
\usepackage{amssymb}
\usepackage{amsthm}
\usepackage{fancyhdr}

\title{4-vectors and the Doppler effect}
\author{Cliff Sun}

\newtheorem{theorem}{Theorem}[section]
\newtheorem{lemma}[theorem]{Lemma}
\newtheorem{definition}[theorem]{Definition}
\newtheorem{conjecture}[theorem]{Conjecture}
\newtheorem{one minute paper}[theorem]{One Minute Paper}

\pagestyle{fancy}
\lhead{\textbf{\thepage}\ \ \nouppercase{\rightmark}}
\chead{4-vectors and the Doppler effect}
\rhead{Cliff Sun}

\begin{document}

\maketitle

\begin{one minute paper}
    Before: The frequency would increase when the truck drives towards me. 

    After: Moving away
\end{one minute paper}

\begin{definition}
    When choosing c = 1, v must be within the range of $[0, 1]$
\end{definition}

To recap, a vector in Newtonian Mechanics transforms the same way as coordinates under rotations. That is, the components of the vector must be conserved when taking the dot-product with itself. \\

Then, a 4-vector in special relativity is a vector that transforms the same way as coordinates under the Lorentz tranformations.
That is:
\begin{equation}
    x'^\mu = \Lambda^\mu_v x^v \iff v'^\mu = \Lambda^\mu_v v^v 
\end{equation}
That is, the components of this transformed vector must be conserved with respect to the relativistic dot-product. 
Then to take the time-derivative of the 4-vector displacement, we must take the time-derivative with respect to something more absolute. This is the proper time, or 
expressed by this equation:
\begin{equation}
    (d\tau)^2 = dx_\mu dx^\mu \iff (dt)^2 - (dx)^2 - (dy)^2 - (dz)^2
\end{equation}
There is an implied metric tensor that is operating this relativistic dot-product. This is Lorentz invariant under any Lorentz transformation frame. 
Then taking the time-derivative of the spatial 4-vector yields the following result:

\begin{equation}
    V^\mu = (\gamma, \gamma v)
\end{equation}
Where $v$ is the velocity vector, not the 4-vector. Then, 4-momentum is the following:
\begin{equation}
    P^\mu = (\gamma m, \gamma m v)
\end{equation}
In general, the energy of a particle can be reduced to $mc^2$ in a respective frame. Thus taking the relativistic dot-product, we have that 
\begin{equation}
    \gamma^2m^2c^4 - p^2 = m^2c^4  
\end{equation}
Thus, we are able to Lorentz transform the 4-momentum vector because this relativistic dot=product is conserved. This is something that we've already proved in class. Then, since momentum is a 4-vector, we can transform it using a Lorentz transformation, that is:
\begin{equation}
    P^\mu = \Lambda^\mu_v P^\mu
\end{equation}
Thus, we see that 
\begin{equation}
    F^\mu = \frac{dP^\mu}{d\tau} \iff mA^\mu
\end{equation}
Similarly, the Doppler effect states that the energy of a photon is transformed as the following:
\begin{equation}
    E' = E\gamma(1 + \beta) \iff E\sqrt{\frac{1 + \beta}{1 - \beta}}
\end{equation}
Such that $\beta \in [0,1]$.
\end{document}
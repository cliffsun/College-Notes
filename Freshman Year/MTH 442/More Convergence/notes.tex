\documentclass{article}
\usepackage[left=3cm, right=3cm, top=2cm, bottom=2.5cm]{geometry}
\usepackage{graphicx}
\usepackage{amsmath}
\usepackage{amssymb}
\usepackage{amsthm}
\usepackage{fancyhdr}

\title{Convergence of Sequences of Functions}
\author{Cliff Sun}

\newtheorem{theorem}{Theorem}[section]
\newtheorem{lemma}[theorem]{Lemma}
\newtheorem{definition}[theorem]{Definition}

\pagestyle{fancy}
\lhead{\textbf{\thepage}\ \ \nouppercase{\rightmark}}
\chead{Convergence of Sequences of Functions}
\rhead{Cliff Sun}

\begin{document}

\maketitle

Recap, we first have pointwise convergence which operates off the epsilon delta defintion. Next is uniform convergence which operates off the sup norm notation. Now, we define something called the $L^2$ convergence. 

\begin{definition}
    Let $I \in \mathbb{R}$ be an interval, and let $f(x)$ and $g(x)$ be functions. Then we define $L^{2}$ distance to be the following:
    \begin{equation}
        |f - g|_{2} = [ \int_{I}|f(x) - g(x)|^2 dx]^{1/2}
    \end{equation}
\end{definition}

\begin{definition}
    Let $I \in \mathbb{R}$ be an interval. Let $f(x)$ be a function on I and $f_m(x)$ be a sequence of functions on I. Then we say that $f_m(x)$ converges to $f(x)$ on I in $L^2$ if 
    \begin{equation}
        \lim_{m \rightarrow \infty} |f_m - f|_{2} = 0
    \end{equation}
    Where this subscript 2 represents the $L^2$ distance between the functions. 
\end{definition}

But $f_m$ can converge to multiple functions because the integral would get rid of discontinuities. So we define a notation $\sim$ such that 
\begin{equation}
    f \sim g \iff \textrm{$f(x) = g(x)$ almost everywhere}
\end{equation}

But $L^2$ convergence states that two different continuous functions cannot converge to each other. But a discontinous and a continuous function can converge to each other. 

One thing to note, uniform convergence implies pointwise convergence. 

We apply this to the Full Fourier Series:

\begin{theorem}
    Fix $l > 0$. Let $\Psi(x)$ be a function on $\mathbb{R}$ with a period of $2l$. That is $\Psi(x + 2l) = \Psi(x)$ for all x in $\mathbb{R}$. Then the full fourier series is the following:
    \begin{equation}
        \Psi(x) = \frac{A_0}{2} + \sum_{n =1}^{\infty} [A_n\cos(\frac{n\pi x}{l}) + B_n\sin(\frac{n\pi x}{l})]
    \end{equation}
    Where
    \begin{equation}
        A_n = \frac{1}{l}\int_{-l}^{l}\Psi(x)\cos(\frac{n\pi x}{l})dx
    \end{equation}
    and
    \begin{equation}
        B_n = \frac{1}{l}\int_{-l}^{l}\Psi(x)\sin(\frac{n\pi x}{l})dx
    \end{equation}
\end{theorem}

Then we can define convergence in the context of the Fourier Series as the following:

\begin{definition}
    \begin{enumerate}
        \item If $\Psi(x)$ is differentiable at $x_0$, then the Fourier Series converges pointwisely. 
        \item If $\Psi(x)$ is continuously differentiable, the the Fourier Series converges uniformly on $\mathbb{R}$
        \item If $\int_{-l}^{l}|\Psi(x)|^2dx < \infty$, then the full fourier series converges $L^2$ on $(-l, l)$
    \end{enumerate}

\end{definition}

\end{document}
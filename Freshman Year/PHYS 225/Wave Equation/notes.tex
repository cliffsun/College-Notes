\documentclass{article}
\usepackage[left=3cm, right=3cm, top=2cm, bottom=2.5cm]{geometry}
\usepackage{graphicx}
\usepackage{amsmath}
\usepackage{amssymb}
\usepackage{amsthm}
\usepackage{fancyhdr}

\title{Maxwell's Equations and Special Relativity}
\author{Cliff Sun}

\newtheorem{theorem}{Theorem}[section]
\newtheorem{lemma}[theorem]{Lemma}
\newtheorem{definition}[theorem]{Definition}
\newtheorem{conjecture}[theorem]{Conjecture}
\newtheorem{proposition}[theorem]{Proposition}
\newtheorem{corollary}[theorem]{Corollary}
\newtheorem{one minute paper}[theorem]{One Minute Paper}

\pagestyle{fancy}
\lhead{\textbf{\thepage}\ \ \nouppercase{\rightmark}}
\chead{Maxwell's Equations and Special Relativity}
\rhead{Cliff Sun}

\begin{document}

\maketitle

\begin{one minute paper}
    Before: Lorentz Transformations, identities, same physics in different reference frames, divergence of B = 0, fourier transform \\
    
    After: Fourier Transform
\end{one minute paper}

E\&M wave equation:

\begin{equation}
    \nabla^2E = \frac{1}{c^2}\frac{\partial^2E}{\partial t^2}
\end{equation}

We use natural units:

\begin{equation}
    \nabla^2E = \frac{\partial^2E}{\partial t^2}
\end{equation}

But in what frame? So we define a 4-gradient operator:

\begin{equation}
    \partial_\mu = (\partial_t, \partial_x, \cdots)
\end{equation}

This transforms like a 4-vector. Then we define a 4-laplacian called the d'Alembertian

\begin{equation}
    \partial_\mu \partial^\mu \equiv g_{\mu \mu} \partial_\mu \partial_\mu \iff \partial_t^2 - \nabla^2
\end{equation}

In natural units, the wave equation is the following:

\begin{equation}
    \partial_\mu\partial^\mu E  = 0
\end{equation}

and 

\begin{equation}
    \partial_\mu\partial^\mu B  = 0
\end{equation}

Thus this wave equation is invariant. Thus, the speed of the wave is invariant. Now we fourier transform the wave:

\begin{equation}
    E(t,x) = E_0e^{-ik_\mu x^\mu}
\end{equation}

Taking the derivative yields

\begin{equation}
    \partial_{\nu} (e^{-ik_\mu x^\mu}) = -k_{\nu}e^{-ik \cdot x}
\end{equation}

Similarly:

\begin{equation}
    \partial^\nu \partial_\nu = -k^2e^{-ik \cdot x}
\end{equation}

But since we have that 

\begin{equation}
    \partial^\nu \partial_\nu E = 0
\end{equation}

Thus

\begin{equation}
    k^2 = 0
\end{equation}

In other words, light is massless. Thus E\&M waves are described by massless photons. Thus 

\begin{center}
    k is proportional to the momentum 4-vector of the wave
\end{center}
\end{document}
\documentclass{article}
\usepackage[left=3cm, right=3cm, top=2cm, bottom=2.5cm]{geometry}
\usepackage{graphicx}
\usepackage{amsmath}
\usepackage{amssymb}
\usepackage{amsthm}
\usepackage{fancyhdr}

\title{Real numbers and least upper bounds}
\author{Cliff Sun}

\newtheorem{theorem}{Theorem}[section]
\newtheorem{lemma}[theorem]{Lemma}
\newtheorem{definition}[theorem]{Definition}
\newtheorem{conjecture}[theorem]{Conjecture}
\newtheorem{corollary}[theorem]{Corollary}
\newtheorem{axiom}[theorem]{Axiom}


\pagestyle{fancy}
\lhead{\textbf{\thepage}\ \ \nouppercase{\rightmark}}
\chead{Real numbers and least upper bounds}
\rhead{Cliff Sun}

\begin{document}

\maketitle

\begin{definition}
    Let $A \in \mathbb{R}$, then
    \begin{enumerate}
        \item If there is some real number $b$ such that $\forall a \in A$, $a < b$, then we say that b is an \underline{upper bound} of A and A is \underline{bounded above}.
        \item If $\exists b \in \mathbb{R}$, such that $b \leq a$, $\forall a \in A$, then we say that b is a \underline{lower bound} of A and A is \underline{bounded below}. 
        \item A is \underline{bounded} if it is bounded above and below. 
    \end{enumerate}
\end{definition}

\begin{axiom}

    Every nonempty $A \subseteq \mathbb{R}$ which is bounded above has a \underline{least} upper bound. 

    \begin{definition}
        A \underline{Least Upper Bound} is defined as b' such that for all $b \in B$ where $B$ is the set of all upperbounds, $b' \leq b$.
    \end{definition}

\end{axiom}

The reason why this axiom is called completeness is because it tells us that the real numbers has no "holes".

\section*{Notation}

We call the least upper bound of set $A$ its supremum, or $\sup A$. Similarly, the greatest lower bound of a set $A$ its infimum or $\inf A$. 
\end{document}
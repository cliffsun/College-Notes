\documentclass{article}
\usepackage[left=3cm, right=3cm, top=2cm, bottom=2.5cm]{geometry}
\usepackage{graphicx}
\usepackage{amsmath}
\usepackage{amssymb}
\usepackage{amsthm}
\usepackage{fancyhdr}

\title{7.5-7.6}
\author{Cliff Sun}

\newtheorem{theorem}{Theorem}[section]
\newtheorem{lemma}[theorem]{Lemma}
\newtheorem{definition}[theorem]{Definition}

\pagestyle{fancy}
\lhead{\textbf{\thepage}\ \ \nouppercase{\rightmark}}
\chead{Well-def of modular arithmetic, well-def of more generally, cardinalities of infinite sets}
\rhead{Cliff Sun}

\begin{document}

\maketitle

\section*{7.5}

\begin{definition}
    Let $\sim_n$ be the relation 
    \begin{equation}
        \equiv \mod n
    \end{equation}
    on $\mathbb{Z}$. Then
    \begin{equation}
        Z/\sim_n = \{[0], [1], \dots, [n-1]\}
    \end{equation}
\end{definition}

Our goal today is to view this as a number system, including the operations $+_n$ and $\cdot_n$.

We define these operations to be 

\begin{definition}
    For integers $a$ and $b$, we define the following:
    \begin{enumerate}
        \item $[a] +_n [b] = [a+b]$
        \item $[a] \cdot_n [b] = [a \cdot b]$
        \item The \underline{ring} of $\mathbb{Z}_n$ or $\mathbb{Z}/n$ is the set $\mathbb{Z}/\sim_n$ equipped with the operations stated above. The term \underline{Ring} is from abstract algebra and roughly means a set of elements which can be added, subtracted, and multiplied. 
\end{enumerate}
\end{definition}

To define the operation of $X \cdot_n Y$, we first choose two representatives from $X$ and $Y$, and multiply them together. Then the equivalence class of this new term $xy$ is the result of $X \cdot_n Y$. WLOG. 

\begin{theorem}
    $+_n$ and $\cdot_n$ are well-defined. That is if $x \equiv x' \mod n$ and $y \equiv y' \mod n$, then $x + y \equiv x' + y' \mod n$ and $x \cdot y \equiv x' \cdot y' \mod n$
\end{theorem}

Next, suppose $X$ and $Y$ are sets and $\sim$ is an equivalence relation. Then, let's try to define a function f such that 
\begin{equation}
    f: X/\sim \rightarrow Y
\end{equation}
\begin{enumerate}
    \item For each $x \in X$, we define $f([x]) \in Y$ by referring to x. 
    \item Prove that if $[x] = [x']$, then $f([x]) = f([x'])$, that is f is well-defined. 
\end{enumerate}

\end{document}
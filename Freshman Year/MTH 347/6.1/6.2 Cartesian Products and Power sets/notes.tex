\documentclass{article}
\usepackage[left=3cm, right=3cm, top=2cm, bottom=2.5cm]{geometry}
\usepackage{graphicx}
\usepackage{amsmath}
\usepackage{amssymb}
\usepackage{amsthm}
\usepackage{fancyhdr}

\title{6.1 \& 6.2: Cartesian Product and Power sets}
\author{Cliff Sun}

\newtheorem{theorem}{Theorem}[section]
\newtheorem{lemma}[theorem]{Lemma}
\newtheorem{definition}[theorem]{Definition}

\pagestyle{fancy}
\lhead{\textbf{\thepage}\ \ \nouppercase{\rightmark}}
\chead{Cartesian Product and Power sets}
\rhead{Cliff Sun}

\begin{document}

\maketitle

Previously, we showed that induction implies the well ordering of the natural numbers. 

\paragraph{Remark:} Strong induction is equivalent to this case as well. That is, the validity of Strong induction implies the well-ordering of the natural numbers. 

\section*{Cartesian Products}

\begin{theorem}
    Let $A$ and $B$ be sets. Then their Cartesian product is 
    \begin{equation}
        A \times B = \{(a,b), a \in A, b \in B\}
    \end{equation}
    You can also take the product of multiple sets $A$, $B$, $C$ where the definition would be:
    \begin{equation}
        A \times B \times C = \{(a,b,c), a \in \mathbb{A}, b \in \mathbb{B}, c \in \mathbb{C}\}
    \end{equation}
\end{theorem}

\paragraph{Remark:} $A \times B$ is not equal to $B \times A$ since the group $(a,b) \neq (b,a)$ \\

That means that 
\begin{equation}
    (A \times B) \times C \neq A \times (B \times C) \neq A \times B \times C
\end{equation}

However, there are bijections between them such that 

\begin{equation}
    f: A \times B \rightarrow B \times A
\end{equation}

or 

\begin{equation}
    (b,a) \rightarrow (a,b)
\end{equation}

Generally speaking the cardinality of $A \times B$ is the same as multiplying the cardinality of $A$ and $B$. 

\begin{proof}
    Let $A = \{a_1,a_2,\dots,a_m\}$ and $B = \{b_1,b_2,\dots,b_n\}$, then $A \times B$ can be written as: \\
    $\{(a_1,b_1), (a_1,b_2), \dots, (a_1,b_n) \dots (a_m, b_1)\}$. \\
    Since there are a total of $m \times n$ number of elements, it follows that the cardinality of $A \times B$ is just $|A||B|$.
\end{proof}

\begin{theorem}
    If $A \subseteq C$ and and $B \subseteq D$, then it follows that $A \times B \subseteq C \times D$
\end{theorem}

\begin{proof}
    Suppose that $A \subseteq C$ and $B \subseteq D$. We claim that $A \times B \subseteq C \times D$. To prove this, consider an element of $A \times B$. Then this element would be an ordered pair. 
    This has the form of $(a,b)$ for a in A and b in B. Since $A \subseteq C$, then a is an element in C. Similarly, we can apply the same argument for b. 
    Therefore, $(a,b)$ is an ordered pair where the first entry is an element in C and the second entry is in D. Thus, by definition, $(a,b)$ is an element in $C \times D$. 
    This concludes the proof. 
\end{proof}

\subsection*{What is the compliment of $A \times B$ in the context of $C \times D$?}

Suppose that $(x,y) \in C \times D$, and $(x,y) \not A \times B$. But that implies that 
$x \in C-A$ and $y \in D-B$. Then that implies that 
\begin{equation}
    (x,y) \in (C - A) \times D \cup (C) \times (D - B)
\end{equation}

\section*{Power Sets}

\begin{theorem}
    Let A be a set. Then the power set of A is 
    \begin{equation}
        P(A) = \{B: B \subseteq A\}
    \end{equation}
    Such that 
    \begin{equation}
        B \subseteq P(A) \iff B \subseteq A
    \end{equation}
    This can include vectors consisting of multiple elements. 
\end{theorem}

\begin{theorem}
    If |A| is n, then the cardinality of P(A) = $2^n$
\end{theorem}

\begin{proof}
    We will use induction on n. As a base case, $|A| = 0$, thus the power set of A has cardinality 1. 
    For the inductive step, suppose the theorem is true for all sets of cardinality n, and let $|A| = n+1$, 
    then B = set of all elements in A until $a_n$ and C = just the last element are subsets of A. 
\end{proof}



\end{document}
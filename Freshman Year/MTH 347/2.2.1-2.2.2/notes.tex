\documentclass{article}
\usepackage[left=3cm, right=3cm, top=2cm, bottom=2.5cm]{geometry}
\usepackage{graphicx}
\usepackage{amsmath}
\usepackage{amssymb}
\usepackage{amsthm}
\usepackage{fancyhdr}

\title{Squeeze Theorem and Continuity of Algebraic Operations}
\author{Cliff Sun}

\newtheorem{theorem}{Theorem}[section]
\newtheorem{lemma}[theorem]{Lemma}
\newtheorem{definition}[theorem]{Definition}
\newtheorem{conjecture}[theorem]{Conjecture}
\newtheorem{proposition}[theorem]{Proposition}
\newtheorem{corollary}[theorem]{Corollary}
\newtheorem{one minute paper}[theorem]{One Minute Paper}

\pagestyle{fancy}
\lhead{\textbf{\thepage}\ \ \nouppercase{\rightmark}}
\chead{Squeeze Theorem and Continuity of Algebraic Operations}
\rhead{Cliff Sun}

\begin{document}

\maketitle

\section*{Squeeze Theorem}

Claim:
\begin{equation}
    \lim_{n\rightarrow\infty}\frac{\cos n}{n} = 0
\end{equation}

\begin{proof}
    We use the squeeze theorem, with $a_n = \frac{-1}{n}$, $b_n = \frac{1}{n}$ and $x_n = \frac{\cos n}{n}$. Since we have that
    \begin{equation}
        a_n \leq x_n \leq b_n
    \end{equation}
    For all n, and that
    \begin{equation}
        \lim a_n = 0 \land \lim b_n = 0
    \end{equation}
    It follows that
    \begin{equation}
        \lim x_n = 0
    \end{equation}
\end{proof}

\begin{theorem}
    Suppose
    \begin{equation}
        a_n \leq x_n \leq b_n
    \end{equation}
    for all n. Then if
    \begin{equation}
        \lim a_n = a
    \end{equation}
    \begin{equation}
        \lim b_n = b
    \end{equation}
    and
    \begin{equation}
        a \neq b
    \end{equation}
    Then if $\lim x_n = x$, then we have that 
    \begin{equation}
        a \leq x \leq b
    \end{equation}
\end{theorem}

\begin{theorem}
    Suppose that $(x_n)$ and $(y_n)$ are sequences that converge to $x, y$ respectively. Then
    \begin{enumerate}
        \item $\lim (x_n + y_n) = x + y$
        \item $\lim (x_n - y_n) = x - y$
        \item $\lim (x_n \cdot y_n) = x \cdot y$
        \item If $y$ and all $y_n$ are not zero, then $\lim (\frac{x_n}{y_n}) = \frac{x}{y}$
    \end{enumerate}
    This tells us that addition, multiplication, subtraction, and division are all continuous functions. That is if $x_n$ is close to x and $y_n$ is close to y.
    Then $x_n + y_n$ is close to $x + y$. 
\end{theorem}

To begin, we prove statement (1). 

\begin{proof}
    Suppose $\lim x_n = x$ and $\lim y_n = y$, we claim that $\lim (x_n + y_n) = x + y$. Let $\epsilon > 0$, 
    plugging in $\frac{\epsilon}{2}$ for both $x_n$ and $y_n$, then we get $M_1$ and $M_2$. Choosing $M' = \max(M_1, M_2)$, we have that 
    \begin{equation}
        |x_n - x| < \frac{\epsilon}{2}
    \end{equation}
    and
    \begin{equation}
        |y_n - y| < \frac{\epsilon}{2}
    \end{equation}
    We rewrite this to be the following:
    \begin{equation}
        -\frac{\epsilon}{2} + x < x_n < \frac{\epsilon}{2} + x
    \end{equation}
    and
    \begin{equation}
        -\frac{\epsilon}{2} + y < y_n < \frac{\epsilon}{2} + y
    \end{equation}
    Adding the equations together yields
    \begin{equation}
        -\epsilon < x_n + y_n - (x + y) < \epsilon
    \end{equation}
    This concludes the proof. 
\end{proof}

\section*{lim sup and lim inf}

Recall that if $(x_n)$ converges, then $(x_n)$ is bounded. But the converse is clearly not true. Then what is the long-term behavior of a bounded divergent sequence?

Then we define lim inf to be the lower limit of the interval in which the sequence oscillates long term and lim sup similarly.

\begin{definition}
    Let $(x_n)$ be a bounded sequence such that
    \begin{enumerate}
        \item $a_n = \sup\{x_k : k \geq n\}$
        \item $b_n = \inf\{x_k : k \geq n\}$
        \item $\lim \sup(x_n) = \lim a_n$
        \item $\lim \inf(x_n) = \lim b_n$
    \end{enumerate}
\end{definition}

\end{document}
\documentclass{article}
\usepackage[left=3cm, right=3cm, top=2cm, bottom=2.5cm]{geometry}
\usepackage{graphicx}
\usepackage{amsmath}
\usepackage{amssymb}
\usepackage{amsthm}
\usepackage{fancyhdr}

\title{7.1-7.2 - Relations and Functions as Relations}
\author{Cliff Sun}

\newtheorem{theorem}{Theorem}[section]
\newtheorem{lemma}[theorem]{Lemma}
\newtheorem{definition}[theorem]{Definition}

\pagestyle{fancy}
\lhead{\textbf{\thepage}\ \ \nouppercase{\rightmark}}
\chead{7.1-7.2}
\rhead{Cliff Sun}

\begin{document}

\maketitle

\section*{Relations}

Idea: we're often interested in relationships between two elements of a set. In particular:

\begin{enumerate}
    \item $a < b$ for $a,b \in \mathbb{R}$
    \item $f(a) = b$ where $f: A \rightarrow B$
    \item $S \subseteq T$ like where $S, T \in P(U)$ where U is the universal set. 
\end{enumerate}

\begin{theorem}
    Let A and B be sets, then a relation from A to B is any subset of $A \times B$.
\end{theorem}

\begin{theorem}
    A relation on A is a relation from $A \rightarrow A$.
\end{theorem}

Notation: $(a,b) \in R \iff$ a R b. In this case, R is a noun and a verb. \\

Example: if $(2,3) \in R$, then 2 R 3.

\begin{theorem}
    On any set S, we have the relation R = "=" defined as 
    \begin{equation}
        R = \{(x,x): x \in S\} \iff x = y
    \end{equation}
    That is 
    \begin{equation}
        (x,y) \in R \iff x = y
    \end{equation}
\end{theorem}

\begin{theorem}
    Given any set S, we have that R = "$\subseteq$" on P(S) defined as 
    \begin{equation}
        R = \{(A,B) \subseteq P(S) \times P(S): A \subseteq B\}
    \end{equation}
    That is 
    \begin{equation}
        (A,B) \in R \iff A \subseteq B
    \end{equation}
\end{theorem}

Inverse relationships:

\begin{theorem}
    Given a relationship R from A to B, we have that the inverse relationship is 
    \begin{equation}
        R^{-1}=\{(x,y) \in B \times A: (y,x) \in R\}
    \end{equation}
    A concrete example would be that $<$'s inverse is $>$, etc. 
\end{theorem}

Functions as relations:

\begin{theorem}
    Say $f: A \rightarrow B$ is a function, then we have the relation
    \begin{equation}
        R = \{(a,f(a)): a \in A\}
    \end{equation}
\end{theorem}

This is a relation, but what properties does it have?

\begin{theorem}
    Given a relation R, the domain of R is 
    \begin{equation}
        R = \{a \in A: (a,b) \in R\}
    \end{equation}
    For values of a such that $f(a) = b$.
\end{theorem}

Now we can redefine functions to be 

\begin{theorem}
    A function $f: A \rightarrow B$ is a relation f from A to B such that 
    \begin{equation}
        dom(f) = A
    \end{equation}
    and $\forall a \in A, \land \forall b_1,b_2 \in B$ are if $(a,b_1)$ and $(a,b_2)$ are in R, then $b_1 = b_2$. 
    THIS IS NOT INJECTIVITY
\end{theorem}



\end{document}
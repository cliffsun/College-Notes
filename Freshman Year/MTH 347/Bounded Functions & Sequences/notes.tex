\documentclass{article}
\usepackage[left=3cm, right=3cm, top=2cm, bottom=2.5cm]{geometry}
\usepackage{graphicx}
\usepackage{amsmath}
\usepackage{amssymb}
\usepackage{amsthm}
\usepackage{fancyhdr}

\title{Bounded Functions and Sequences \& Limits}
\author{Cliff Sun}

\newtheorem{theorem}{Theorem}[section]
\newtheorem{lemma}[theorem]{Lemma}
\newtheorem{definition}[theorem]{Definition}
\newtheorem{conjecture}[theorem]{Conjecture}
\newtheorem{proposition}[theorem]{Proposition}
\newtheorem{corollary}[theorem]{Corollary}
\newtheorem{one minute paper}[theorem]{One Minute Paper}

\pagestyle{fancy}
\lhead{\textbf{\thepage}\ \ \nouppercase{\rightmark}}
\chead{Bounded Functions and Sequences \& Limits}
\rhead{Cliff Sun}

\begin{document}

\maketitle

\begin{proposition}

The sum of any two bounded functions on the same domain D is bounded. 

\end{proposition}

\begin{proof}
    Suppose that $f: D \rightarrow \mathbb{R}$ and $g: D \rightarrow \mathbb{R}$ are bounded. Then $|f(x)| \leq M$ and $|g(x)| \leq N$ for all $x \in D$. We then add them together
    \begin{equation}
        |f(x) + g(x)| \leq |f(x)| + |g(x)| \leq M + N
    \end{equation}
    Thus the function sum of both $f(x)$ and $g(x)$ is bounded. 
\end{proof}

\begin{definition}
    A sequence of numbers is a function from $f: \mathbb{N} \rightarrow \mathbb{R}$
\end{definition}

Note:
\begin{enumerate}
    \item All sequences are infinite
    \item We write $x_1, x_2, \cdots$ instead of $f(1), f(2), \cdots$
    \item The sequences as a whole is denoted as $(x_n)_{n=1}^{\infty}$ or $(x_n)$ for short. 
    \item An example would be $x_n = n$ or $(n)_{n=1}^{\infty} = (1,2,3,\cdots)$
\end{enumerate}

Boundedness, sup/inf all apply to sequences, in particular to the Im$(f) = \{x_n : x \in \mathbb{N}\}$

\begin{definition}
    Let $(x_n)$ be a sequence and $x \in \mathbb{R}$:
    \begin{enumerate}
        \item We say that $\lim_{n\rightarrow \infty}x_n = x$ if there exists some $\epsilon > 0$, there exists some $M \in \mathbb{N}$ such that for all $n \geq M$ such that $|x_n - n| < \epsilon$
        \item A sequence \underline{converges} if $\lim_{n\rightarrow \infty}x_n = x$ for some x, other it \underline{diverges}.
    \end{enumerate}
\end{definition}

Suppose that $x_n = \frac{\sin(n)}{n}$, claim is that it converges to 0.

\begin{proof}
    Let $\epsilon > 0$ be arbitrary, let $n \geq M$ be arbitrary, we must prove that
    \begin{equation}
        |\frac{\sin(n)}{n}| < \epsilon
    \end{equation}
    We choose $M = \frac{1}{\epsilon} + 1$, and using the fact that 
    \begin{equation}
        |\frac{\sin(n)}{n}| \leq \frac{1}{n}
    \end{equation}
    We can prove this. 
\end{proof}

\begin{proposition}
    If $(x_n)$ is a convergent sequence, then it is bounded. 
\end{proposition}

\begin{proof}
    Let $(x_n)$ be a covergent sequence, we claim that it is bounded. In other words, there exists some $B \in \mathbb{R}$ such that 
    \begin{equation}
        |x_n| \leq B \quad \forall n \in \mathbb{N}
    \end{equation}
    Let $\epsilon = 1$ since this a convergent sequence, we see that there exists some $M \in \mathbb{R}$ such that for all $n \geq M$,
    \begin{equation}
        |x_n - x| < 1
    \end{equation}
    We see that
    \begin{equation}
        |x_n| = |(x_n - x) + x|
    \end{equation}
    \begin{equation}
        \leq |x_n - x| + |x|
    \end{equation}
    \begin{equation}
        < 1 + |x|
    \end{equation}
    We now know that all $x_n$ are bounded between this value, except for $x_1, x_2, x_3, \dots$. Then we say that
    \begin{equation}
        B = \max(1 + |x|, |x_1|, |x_2|, |x_3|, \dots, |x_{M-1}|)
    \end{equation}
\end{proof}

\begin{proposition}
    A sequence can have, at most, 1 limit. 
\end{proposition}

\begin{proof}
    Suppose that $(x_n)$ is a sequence such that $x_n \rightarrow x$ and $x_n \rightarrow y$, then we claim that $x = y$. 
    To do this, let $\epsilon > 0$ be arbitrary, we claim that $|x - y| < \epsilon$. Plugging in $\frac{\epsilon}{2}$ to the definition, we see that there exists some $M_1$, there exists some $M_1 \in \mathbb{N}$ such that 
    for all $n \geq M_1$, we have that 
    \begin{equation}
        |x_n - x| < \frac{\epsilon}{2}
    \end{equation} 
    Similary, we have that for some $M_2 \in \mathbb{N}$, we have that 
    \begin{equation}
        |x_n - x| < \frac{\epsilon}{2}
    \end{equation}
    Choose $n = \max(M_1,M_2)$, then $n \geq M_1$ and $n \geq M_2$. Then,
    \begin{equation}
        |x-y| = |(x_n - y) - (x_n - x)| \leq |x_n - y| + |x_n - x| < \frac{\epsilon}{2} + \frac{\epsilon}{2} = \epsilon
    \end{equation}
\end{proof}

\end{document}
\documentclass{article}
\usepackage[left=3cm, right=3cm, top=2cm, bottom=2.5cm]{geometry}
\usepackage{graphicx}
\usepackage{amsmath}
\usepackage{amssymb}
\usepackage{amsthm}
\usepackage{fancyhdr}

\title{Mixer Notes}
\author{Cliff Sun}

\newtheorem{theorem}{Theorem}[section]
\newtheorem{lemma}[theorem]{Lemma}
\newtheorem{definition}[theorem]{Definition}

\pagestyle{fancy}
\lhead{\textbf{\thepage}\ \ \nouppercase{\rightmark}}
\chead{Mixer Notes}
\rhead{Cliff Sun}

\begin{document}

\maketitle

With a voltage square pulse of period 2 microseconds and Vpp = 2V, when applying a continuous microwave rf signal of power 9.8 dBm and frequency:

\begin{enumerate}
    \item 1.495 GHz, a strange oscillation forms between the two quasi sinusoidal peaks, and one with relatively high amplitude and appears well-ordered
    \item 3.8 GHz, a valley forms between the 2 quasi sinusoidal peaks, and it relatively maximized at this frequency. 
    \item 4.79Ghz, a hill forms between the 2 quasi sinusoidal peaks, when I flip the frequency up to 4.9 GHz, this hill immediately disappears.
\end{enumerate}

Terminology for signal processing:

\begin{enumerate}
    \item LO: Local oscillator port, the mixer uses the inputted signal in this port to decrease the primary signal (this is always an inputted signal)
    \item Rf: Radio frequency signal, used to inputs/outputs for signals like microwaves, etc.
    \item IF: Intermediate frequency, this is the frequency that the carrier signal (primary signal with the information) is turned into. 
\end{enumerate}

Generally speaking, if the desired frequency is greater than the inputted frequency, then RF is the output and IF is the input. Vice versa.

\end{document}